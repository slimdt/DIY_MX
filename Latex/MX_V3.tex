\documentclass[cn,11pt,chinese]{elegantbook}

\def\myformat#1{\hfil\hfil #1}


\title{毛泽东选集}
\subtitle{第三卷}

%\author{Ethan Deng \& Liam Huang}
%\institute{Elegant\LaTeX{} Program}
%\date{February 10, 2020}
%\version{3.10}
%\bioinfo{自定义}{信息}

\extrainfo{群众是真正的英雄,而我们自己则往往是幼稚可笑的,不了解这一点,就不可能得到起码的知识。}

%\logo{logo-blue.png}
\logo{mzd.jpg}
%\cover{cover.jpg}
\cover{jt.png}


% 本文档命令
\usepackage{array}
\newcommand{\ccr}[1]{\makecell{{\color{#1}\rule{1cm}{1cm}}}}
% 修改目录深度
\setcounter{tocdepth}{2}

\begin{document}

\maketitle
\frontmatter

\iffalse
\chapter*{特别声明}
\markboth{Introduction}{前言}

在过去的 2019 年,\href{https://elegantlatex.org/}{Elegant\LaTeX{}} 系列模板均逐步上线 \href{https://github.com/ElegantLaTeX}{GitHub}、\href{https://ctan.org/pkg/elegantbook}{CTAN}、\href{https://www.overleaf.com/latex/templates/elegantbook-template/zpsrbmdsxrgy}{Overleaf} 以及 \href{https://gitee.com/ElegantLaTeX/ElegantBook}{Gitee} 上。截止到 2019 年底,ElegantNote、ElegantBook、ElegantPaper 三个模板在 GitHub 上的收藏数达到了 194、333 和 220,从 2019 年 5 月开启捐赠之后收到了用户 33 笔合计超过 1500 元的捐赠,用户群人数也超过了 400 人。这些数字的背后,反映出 Elegant\LaTeX{} 越来越受用户的喜爱,在此非常感谢大家。

但是,我想声明的是:

\begin{center}
  由于某些原因,Elegant\LaTeX{} 项目 \underline{不再接受}\textbf{任何}非我本人预约的提交。
\end{center}

我是一个理想主义者,关于这个模板,我有自己的想法。我所关心的是,我周围的人能方便使用 \LaTeX{} 以及此模板,我自己会为自己的东西感到开心。如果维护模板让我不开心,那我就不会再维护了。诚然这个模板并不是完美的,但是相比 2.x 好很多了,这些改进离不开大家的反馈、China\TeX{} 和逐鹿人的鼓励以及支援人员的帮助!

\underline{如果你无法认同我的想法,建议直接删除本模板。}

\vskip 1.5cm

\begin{flushright}
Ethan Deng\\
February 10, 2020
\end{flushright}
\fi


\tableofcontents
%\listofchanges

\mainmatter


\chapter*{抗日战争时期(下)}\addcontentsline{toc}{chapter}{抗日战争时期(下)}\newpage\section*{\myformat{《农村调查》的序言和跋}\\\myformat{(一九四一年三月、四月)}}\addcontentsline{toc}{section}{《农村调查》的序言和跋}
\subsection*{\myformat{序}\\
\myformat{(一九四一年三月十七日)}}
现在党的农村政策,不是十年内战时期那样的土地革命政策,而是抗日民族统一战线的政策。全党应该执行一九四〇年七月七日和十二月二十五日的中央指示\footnote[1]{ 一九四〇年七月七日的中央指示,是指当时所发的《中共中央关于目前形势与党的政策的决定》。一九四〇年十二月二十五日的中央指示,见本书第二卷《论政策》。},应该执行即将到来的七次大会\footnote[2]{ 一九三七年十二月,中共中央政治局会议通过关于召集党的第七次全国代表大会的决议。一九三八年九月至十一月召开的中共六届六中全会批准了这项决议,并决定了七大的议事日程。这次大会曾准备在一九四一年五月一日举行,后来延至一九四五年才召开。}的指示。所以印这个材料,是为了帮助同志们找一个研究问题的方法。现在我们很多同志,还保存着一种粗枝大叶、不求甚解的作风,甚至全然不了解下情,却在那里担负指导工作,这是异常危险的现象。对于中国各个社会阶级的实际情况,没有真正具体的了解,真正好的领导是不会有的。\\
  要了解情况,唯一的方法是向社会作调查,调查社会各阶级的生动情况。对于担负指导工作的人来说,有计划地抓住几个城市、几个乡村,用马克思主义的基本观点,即阶级分析的方法,作几次周密的调查,乃是了解情况的最基本的方法。只有这样,才能使我们具有对中国社会问题的最基础的知识。\\
  要做这件事,第一是眼睛向下,不要只是昂首望天。没有眼睛向下的兴趣和决心,是一辈子也不会真正懂得中国的事情的。\\
  第二是开调查会。东张西望,道听途说,决然得不到什么完全的知识。我用开调查会的方法得来的材料,湖南的几个,井冈山的几个,都失掉了。这里印的,主要的是一个《兴国调查》,一个《长冈乡调查》和一个《才溪乡调查》。开调查会,是最简单易行又最忠实可靠的方法,我用这个方法得了很大的益处,这是比较什么大学还要高明的学校。到会的人,应是真正有经验的中级和下级的干部,或老百姓。我在湖南五县调查和井冈山两县调查,找的是各县中级负责干部;寻乌调查找的是一部分中级干部,一部分下级干部,一个穷秀才,一个破产了的商会会长,一个在知县衙门管钱粮的已经失了业的小官吏。他们都给了我很多闻所未闻的知识。使我第一次懂得中国监狱全部腐败情形的,是在湖南衡山县作调查时该县的一个小狱吏。兴国调查和长冈、才溪两乡调查,找的是乡级工作同志和普通农民。这些干部、农民、秀才、狱吏、商人和钱粮师爷,就是我的可敬爱的先生,我给他们当学生是必须恭谨勤劳和采取同志态度的,否则他们就不理我,知而不言,言而不尽。开调查会每次人不必多,三五个七八个人即够。必须给予时间,必须有调查纲目,还必须自己口问手写,并同到会人展开讨论。因此,没有满腔的热忱,没有眼睛向下的决心,没有求知的渴望,没有放下臭架子、甘当小学生的精神,是一定不能做,也一定做不好的。必须明白:群众是真正的英雄,而我们自己则往往是幼稚可笑的,不了解这一点,就不能得到起码的知识。\\
  我再度申明:出版这个参考材料的主要目的,在于指出一个如何了解下层情况的方法,而不是要同志们去记那些具体材料及其结论。一般地说,中国幼稚的资产阶级还没有来得及也永远不可能替我们预备关于社会情况的较完备的甚至起码的材料,如同欧美日本的资产阶级那样,所以我们自己非做搜集材料的工作不可。特殊地说,实际工作者须随时去了解变化着的情况,这是任何国家的共产党也不能依靠别人预备的。所以,一切实际工作者必须向下作调查。对于只懂得理论不懂得实际情况的人,这种调查工作尤有必要,否则他们就不能将理论和实际相联系。“没有调查就没有发言权”\footnote[3]{ 见本书第一卷《反对本本主义》。},这句话,虽然曾经被人讥为“狭隘经验论”的,我却至今不悔;不但不悔,我仍然坚持没有调查是不可能有发言权的。有许多人,“下车伊始”,就哇喇哇喇地发议论,提意见,这也批评,那也指责,其实这种人十个有十个要失败。因为这种议论或批评,没有经过周密调查,不过是无知妄说。我们党吃所谓“钦差大臣”的亏,是不可胜数的。而这种“钦差大臣”则是满天飞,几乎到处都有。斯大林的话说得对:“理论若不和革命实践联系起来,就会变成无对象的理论。”当然又是他的话对:“实践若不以革命理论为指南,就会变成盲目的实践。”\footnote[4]{ 见本书第一卷《实践论》注〔10〕。}除了盲目的、无前途的、无远见的实际家,是不能叫做“狭隘经验论”的。\\
  我现在还痛感有周密研究中国事情和国际事情的必要,这是和我自己对于中国事情和国际事情依然还只是一知半解这种事实相关联的,并非说我是什么都懂得了,只是人家不懂得。和全党同志共同一起向群众学习,继续当一个小学生,这就是我的志愿。\\
\subsection*{\myformat{跋}\\\myformat{(一九四一年四月十九日)}}
十年内战时期的经验,是现在抗日时期的最好的和最切近的参考。但这是指的关于如何联系群众和动员群众反对敌人这一方面,而不是指的策略路线这一方面。党的策略路线,在现在和过去是有原则区别的。在过去,是反对地主和反革命的资产阶级;在现在,是联合一切不反对抗日的地主和资产阶级。就是在十年内战的后期,对于向我们举行武装进攻的反动的政府和政党,和对于在我们政权管辖下一切带资本主义性的社会阶层,没有采取不同的政策,对于反动的政府和政党中各个不同的派别间,也没有采取不同的政策,这些也都是不正确的。那时,对于农民和城市下层小资产者以外的一切社会成分,执行了所谓“一切斗争”的政策,这个政策无疑是错误了。在土地政策方面,对于十年内战前期和中期\footnote[5]{ 这里所说的十年内战前期,是指一九二七年底至一九二八年底的时期,亦即人们通常所称的井冈山时期;中期是指一九二九年初至一九三一年秋的时期,即自中央革命根据地的创立至第三次反“围剿”战争胜利结束的时期。上文所说的十年内战后期,是指一九三一年底至一九三四年底的时期,即自第三次反“围剿”战争胜利结束后至中共中央在贵州遵义举行政治局扩大会议的时期。一九三五年一月的遵义会议,结束了一九三一年至一九三四年“左”倾机会主义路线在党内的统治,使党回到正确的路线上来。参见本卷《学习和时局》的附录《关于若干历史问题的决议》第三部分。}所采取的、也分配给地主一份和农民同样的土地、使他们从事耕种、以免流离失所或上山为匪破坏社会秩序,这样的正确的政策,加以否定,也是错误的。现在,党的政策必须与此不同,不是“一切斗争,否认联合”,也不是“一切联合,否认斗争”(如同一九二七年的陈独秀主义\footnote[6]{ 见本书第一卷《中国革命战争的战略问题》注〔4〕。}那样),而是联合一切反对日本帝国主义的社会阶层,同他们建立统一战线,但对他们中间存在着的投降敌人和反共反人民的动摇性反动性方面,又应按其不同程度,同他们作各种不同形式的斗争。现在的政策,是综合“联合”和“斗争”的两重性的政策。在劳动政策方面,是适当地改善工人生活和不妨碍资本主义经济正当发展的两重性的政策。在土地政策方面,是要求地主减租减息又规定农民部分地交租交息的两重性的政策。在政治权利方面,是一切抗日的地主资本家都有和工人农民一样的人身权利、政治权利和财产权利,但又防止他们可能的反革命行动的两重性的政策。国营经济和合作社经济是应该发展的,但在目前的农村根据地内,主要的经济成分,还不是国营的,而是私营的,而是让自由资本主义经济得着发展的机会,用以反对日本帝国主义和半封建制度。这是目前中国的最革命的政策,反对和阻碍这个政策的施行,无疑义地是错误的。严肃地坚决地保持共产党员的共产主义的纯洁性,和保护社会经济中的有益的资本主义成分,并使其有一个适当的发展,是我们在抗日和建设民主共和国时期不可缺一的任务。在这个时期内一部分共产党员被资产阶级所腐化,在党员中发生资本主义的思想,是可能的,我们必须和这种党内的腐化思想作斗争;但是不要把反对党内资本主义思想的斗争,错误地移到社会经济方面,去反对资本主义的经济成分。我们必须明确地分清这种界限。中国共产党是在复杂的环境中工作,每个党员,特别是干部,必须锻炼自己成为懂得马克思主义策略的战士,片面地简单地看问题,是无法使革命胜利的。\\
\newpage\section*{\myformat{改造我们的学习}\\\myformat{(一九四一年五月十九日)}}\addcontentsline{toc}{section}{改造我们的学习}
\begin{introduction}\item  这是毛泽东在延安干部会上所作的报告。这篇报告和《整顿党的作风》、《反对党八股》,是毛泽东关于整风运动的基本著作。在这些文章里,毛泽东进一步地从思想问题上总结了过去中国共产党内路线的分歧,分析了广泛存在于党内的非马克思列宁主义思想作风,主要是主观主义的倾向,宗派主义的倾向,和作为这两种倾向的表现形式的党八股。毛泽东号召开展全党范围的马克思列宁主义的教育运动,即按照马克思列宁主义的思想原则整顿作风的运动。毛泽东的这个号召,很快地在中国共产党内和党外引起了怎样以从实际出发的观点而不是以教条主义的观点来对待马克思列宁主义原理,怎样使马克思列宁主义的基本原理和中国革命的实际相结合,以及怎样对待一九三一年初至一九三四年底这段时期党内两条路线的斗争这样一些重大问题的大讨论,巩固了马克思列宁主义思想在党内外的阵地,使广大干部在思想上大大地提高了一步,使中国共产党达到了空前的团结。\end{introduction}
我主张将我们全党的学习方法和学习制度改造一下。其理由如次:\\
\subsubsection*{\myformat{一}}
中国共产党的二十年,就是马克思列宁主义的普遍真理和中国革命的具体实践日益结合的二十年。如果我们回想一下,我党在幼年时期,我们对于马克思列宁主义的认识和对于中国革命的认识是何等肤浅,何等贫乏,则现在我们对于这些的认识是深刻得多,丰富得多了。灾难深重的中华民族,一百年来,其优秀人物奋斗牺牲,前仆后继,摸索救国救民的真理,是可歌可泣的。但是直到第一次世界大战和俄国十月革命之后,才找到马克思列宁主义这个最好的真理,作为解放我们民族的最好的武器,而中国共产党则是拿起这个武器的倡导者、宣传者和组织者。马克思列宁主义的普遍真理一经和中国革命的具体实践相结合,就使中国革命的面目为之一新。抗日战争以来,我党根据马克思列宁主义的普遍真理研究抗日战争的具体实践,研究今天的中国和世界,是进一步了,研究中国历史也有某些开始。所有这些,都是很好的现象。\\
\subsubsection*{\myformat{二}}
但是我们还是有缺点的,而且还有很大的缺点。据我看来,如果不纠正这类缺点,就无法使我们的工作更进一步,就无法使我们在将马克思列宁主义的普遍真理和中国革命的具体实践互相结合的伟大事业中更进一步。\\
  首先来说研究现状。像我党这样一个大政党,虽则对于国内和国际的现状的研究有了某些成绩,但是对于国内和国际的各方面,对于国内和国际的政治、军事、经济、文化的任何一方面,我们所收集的材料还是零碎的,我们的研究工作还是没有系统的。二十年来,一般地说,我们并没有对于上述各方面作过系统的周密的收集材料加以研究的工作,缺乏调查研究客观实际状况的浓厚空气。“闭塞眼睛捉麻雀”,“瞎子摸鱼”,粗枝大叶,夸夸其谈,满足于一知半解,这种极坏的作风,这种完全违反马克思列宁主义基本精神的作风,还在我党许多同志中继续存在着。马克思、恩格斯、列宁、斯大林教导我们认真地研究情况,从客观的真实的情况出发,而不是从主观的愿望出发;我们的许多同志却直接违反这一真理。\\
  其次来说研究历史。虽则有少数党员和少数党的同情者曾经进行了这一工作,但是不曾有组织地进行过。不论是近百年的和古代的中国史,在许多党员的心目中还是漆黑一团。许多马克思列宁主义的学者也是言必称希腊,对于自己的祖宗,则对不住,忘记了。认真地研究现状的空气是不浓厚的,认真地研究历史的空气也是不浓厚的。\\
  其次说到学习国际的革命经验,学习马克思列宁主义的普遍真理。许多同志的学习马克思列宁主义似乎并不是为了革命实践的需要,而是为了单纯的学习。所以虽然读了,但是消化不了。只会片面地引用马克思、恩格斯、列宁、斯大林的个别词句,而不会运用他们的立场、观点和方法,来具体地研究中国的现状和中国的历史,具体地分析中国革命问题和解决中国革命问题。这种对待马克思列宁主义的态度是非常有害的,特别是对于中级以上的干部,害处更大。\\
  上面我说了三方面的情形:不注重研究现状,不注重研究历史,不注重马克思列宁主义的应用。这些都是极坏的作风。这种作风传播出去,害了我们的许多同志。\\
  确实的,现在我们队伍中确有许多同志被这种作风带坏了。对于国内外、省内外、县内外、区内外的具体情况,不愿作系统的周密的调查和研究,仅仅根据一知半解,根据“想当然”,就在那里发号施令,这种主观主义的作风,不是还在许多同志中间存在着吗?\\
  对于自己的历史一点不懂,或懂得甚少,不以为耻,反以为荣。特别重要的中国共产党的历史和鸦片战争以来的中国近百年史,真正懂得的很少。近百年的经济史,近百年的政治史,近百年的军事史,近百年的文化史,简直还没有人认真动手去研究。有些人对于自己的东西既无知识,于是剩下了希腊和外国故事,也是可怜得很,从外国故纸堆中零星地检来的。\\
  几十年来,很多留学生都犯过这种毛病。他们从欧美日本回来,只知生吞活剥地谈外国。他们起了留声机的作用,忘记了自己认识新鲜事物和创造新鲜事物的责任。这种毛病,也传染给了共产党。\\
  我们学的是马克思主义,但是我们中的许多人,他们学马克思主义的方法是直接违反马克思主义的。这就是说,他们违背了马克思、恩格斯、列宁、斯大林所谆谆告诫人们的一条基本原则:理论和实际统一。他们既然违背了这条原则,于是就自己造出了一条相反的原则:理论和实际分离。在学校的教育中,在在职干部的教育中,教哲学的不引导学生研究中国革命的逻辑,教经济学的不引导学生研究中国经济的特点,教政治学的不引导学生研究中国革命的策略,教军事学的不引导学生研究适合中国特点的战略和战术,诸如此类。其结果,谬种流传,误人不浅。在延安学了,到富县\footnote[1]{ 富县在延安南面约八十公里。}就不能应用。经济学教授不能解释边币和法币\footnote[2]{ 边币是一九四一年陕甘宁边区银行所发行的纸币。法币是一九三五年以后国民党官僚资本四大银行(中央、中国、交通、中国农民)依靠英美帝国主义支持所发行的纸币。毛泽东在本文中所说的,是指当时边币和法币之间所发生的兑换比价变化问题。},当然学生也不能解释。这样一来,就在许多学生中造成了一种反常的心理,对中国问题反而无兴趣,对党的指示反而不重视,他们一心向往的,就是从先生那里学来的据说是万古不变的教条。\\
  当然,上面我所说的是我们党里的极坏的典型,不是说普遍如此。但是确实存在着这种典型,而且为数相当地多,为害相当地大,不可等闲视之的。\\
\subsubsection*{\myformat{三}}
为了反复地说明这个意思,我想将两种互相对立的态度对照地讲一下。\\
  第一种:主观主义的态度。\\
  在这种态度下,就是对周围环境不作系统的周密的研究,单凭主观热情去工作,对于中国今天的面目若明若暗。在这种态度下,就是割断历史,只懂得希腊,不懂得中国,对于中国昨天和前天的面目漆黑一团。在这种态度下,就是抽象地无目的地去研究马克思列宁主义的理论。不是为了要解决中国革命的理论问题、策略问题而到马克思、恩格斯、列宁、斯大林那里找立场,找观点,找方法,而是为了单纯地学理论而去学理论。不是有的放矢,而是无的放矢。马克思、恩格斯、列宁、斯大林教导我们说:应当从客观存在着的实际事物出发,从其中引出规律,作为我们行动的向导。为此目的,就要像马克思所说的详细地占有材料,加以科学的分析和综合的研究\footnote[3]{ 参见马克思《资本论》第一卷第二版跋。马克思在这篇跋中说:“研究必须充分地占有材料,分析它的各种发展形式,探寻这些形式的内在联系。只有这项工作完成以后,现实的运动才能适当地叙述出来。”(《马克思恩格斯全集》第23卷,人民出版社1972年版,第23页)}。我们的许多人却是相反,不去这样做。其中许多人是做研究工作的,但是他们对于研究今天的中国和昨天的中国一概无兴趣,只把兴趣放在脱离实际的空洞的“理论”研究上。许多人是做实际工作的,他们也不注意客观情况的研究,往往单凭热情,把感想当政策。这两种人都凭主观,忽视客观实际事物的存在。或作讲演,则甲乙丙丁、一二三四的一大串;或作文章,则夸夸其谈的一大篇。无实事求是之意,有哗众取宠之心。华而不实,脆而不坚。自以为是,老子天下第一,“钦差大臣”满天飞。这就是我们队伍中若干同志的作风。这种作风,拿了律己,则害了自己;拿了教人,则害了别人;拿了指导革命,则害了革命。总之,这种反科学的反马克思列宁主义的主观主义的方法,是共产党的大敌,是工人阶级的大敌,是人民的大敌,是民族的大敌,是党性不纯的一种表现。大敌当前,我们有打倒它的必要。只有打倒了主观主义,马克思列宁主义的真理才会抬头,党性才会巩固,革命才会胜利。我们应当说,没有科学的态度,即没有马克思列宁主义的理论和实践统一的态度,就叫做没有党性,或叫做党性不完全。\\
  有一副对子,是替这种人画像的。那对子说:\\
  墙上芦苇,头重脚轻根底浅;\\
  山间竹笋,嘴尖皮厚腹中空。\\
  对于没有科学态度的人,对于只知背诵马克思、恩格斯、列宁、斯大林著作中的若干词句的人,对于徒有虚名并无实学的人,你们看,像不像?如果有人真正想诊治自己的毛病的话,我劝他把这副对子记下来;或者再勇敢一点,把它贴在自己房子里的墙壁上。马克思列宁主义是科学,科学是老老实实的学问,任何一点调皮都是不行的。我们还是老实一点吧!\\
  第二种:马克思列宁主义的态度。\\
  在这种态度下,就是应用马克思列宁主义的理论和方法,对周围环境作系统的周密的调查和研究。不是单凭热情去工作,而是如同斯大林所说的那样:把革命气概和实际精神结合起来\footnote[4]{ 参见斯大林《论列宁主义基础》第九部分《工作作风》(《斯大林选集》上卷,人民出版社1979年版,第272—275页)。}。在这种态度下,就是不要割断历史。不单是懂得希腊就行了,还要懂得中国;不但要懂得外国革命史,还要懂得中国革命史;不但要懂得中国的今天,还要懂得中国的昨天和前天。在这种态度下,就是要有目的地去研究马克思列宁主义的理论,要使马克思列宁主义的理论和中国革命的实际运动结合起来,是为着解决中国革命的理论问题和策略问题而去从它找立场,找观点,找方法的。这种态度,就是有的放矢的态度。“的”就是中国革命,“矢”就是马克思列宁主义。我们中国共产党人所以要找这根“矢”,就是为了要射中国革命和东方革命这个“的”的。这种态度,就是实事求是的态度。“实事”就是客观存在着的一切事物,“是”就是客观事物的内部联系,即规律性,“求”就是我们去研究。我们要从国内外、省内外、县内外、区内外的实际情况出发,从其中引出其固有的而不是臆造的规律性,即找出周围事变的内部联系,作为我们行动的向导。而要这样做,就须不凭主观想象,不凭一时的热情,不凭死的书本,而凭客观存在的事实,详细地占有材料,在马克思列宁主义一般原理的指导下,从这些材料中引出正确的结论。这种结论,不是甲乙丙丁的现象罗列,也不是夸夸其谈的滥调文章,而是科学的结论。这种态度,有实事求是之意,无哗众取宠之心。这种态度,就是党性的表现,就是理论和实际统一的马克思列宁主义的作风。这是一个共产党员起码应该具备的态度。如果有了这种态度,那就既不是“头重脚轻根底浅”,也不是“嘴尖皮厚腹中空”了。\\
\subsubsection*{\myformat{四}}
依据上述意见,我有下列提议:\\
  (一)向全党提出系统地周密地研究周围环境的任务。依据马克思列宁主义的理论和方法,对敌友我三方的经济、财政、政治、军事、文化、党务各方面的动态进行详细的调查和研究的工作,然后引出应有的和必要的结论。为此目的,就要引导同志们的眼光向着这种实际事物的调查和研究。就要使同志们懂得,共产党领导机关的基本任务,就在于了解情况和掌握政策两件大事,前一件事就是所谓认识世界,后一件事就是所谓改造世界。就要使同志们懂得,没有调查就没有发言权,夸夸其谈地乱说一顿和一二三四的现象罗列,都是无用的。例如关于宣传工作,如果不了解敌友我三方的宣传状况,我们就无法正确地决定我们的宣传政策。任何一个部门的工作,都必须先有情况的了解,然后才会有好的处理。在全党推行调查研究的计划,是转变党的作风的基础一环。\\
  (二)对于近百年的中国史,应聚集人材,分工合作地去做,克服无组织的状态。应先作经济史、政治史、军事史、文化史几个部门的分析的研究,然后才有可能作综合的研究。\\
  (三)对于在职干部的教育和干部学校的教育,应确立以研究中国革命实际问题为中心,以马克思列宁主义基本原则为指导的方针,废除静止地孤立地研究马克思列宁主义的方法。研究马克思列宁主义,又应以《苏联共产党(布)历史简要读本》为中心的材料。《苏联共产党(布)历史简要读本》是一百年来全世界共产主义运动的最高的综合和总结,是理论和实际结合的典型,在全世界还只有这一个完全的典型。我们看列宁、斯大林他们是如何把马克思主义的普遍真理和苏联革命的具体实践互相结合又从而发展马克思主义的,就可以知道我们在中国是应该如何地工作了。\\
  我们走过了许多弯路。但是错误常常是正确的先导。在如此生动丰富的中国革命环境和世界革命环境中,我们在学习问题上的这一改造,我相信一定会有好的结果。\\
\newpage\section*{\myformat{揭破远东慕尼黑的阴谋}\\\myformat{(一九四一年五月二十五日)}}\addcontentsline{toc}{section}{揭破远东慕尼黑的阴谋}
\begin{introduction}\item 这是毛泽东为中共中央写的对党内的指示。\end{introduction}
(一)日美妥协,牺牲中国,造成反共、反苏局面的东方慕尼黑的新阴谋,正在日美蒋之间酝酿着。我们必须揭穿它,反对它。\\
  (二)日本帝国主义以迫蒋投降为目的的军事进攻,现已告一段落,继之而来的必然是诱降活动。这是敌人一打一拉、又打又拉的老政策的重演。我们必须揭穿它,反对它。\\
  (三)日本和军事进攻同时发动了谣言攻势,例如所谓“八路军不愿和国民党中央军配合作战”,“八路军乘机扩大地盘”,“打通国际路线”,“另立中央政府”等等。这是日本挑拨国共关系以利诱降的诡计。国民党中央社和国民党报纸照抄散布,不惜和日本的反共宣传互相呼应,其用意所在,甚为可疑。我们也应揭穿它,反对它。\\
  (四)新四军虽被宣布为“叛变”,八路军虽没有领到一颗弹一文饷,然无一刻不与敌军搏斗。此次晋南战役\footnote[1]{ 晋南战役,即中条山战役。一九四一年五月,日本侵略军以约十万人的兵力进犯黄河以北位于晋南、豫北的中条山地区。集结在这个地区的国民党军队约十五六万人。这些国民党军队,本来以反共为主要任务,对日军缺乏作战准备,在日军进犯时,大部分采取避战方针。因此,虽然华北各地的八路军主动出击,截断了同蒲路、正太路、平汉路、白晋路等日军的交通线,给国民党军队以积极配合,国民党军仍然全部溃败,在三周之内损失兵力约七万余人,丧失了中条山及附近地区的大片国土。},八路军复自动配合国民党军队作战,两周以来在华北各线作全面出击,至今犹在酣战中。共产党领导的武力和民众已成了抗日战争中的中流砥柱。一切对于共产党的污蔑,其目的都在使抗战失败,以利投降。我们应发扬八路军新四军的战绩,反对一切失败主义者和投降主义者。\\
\newpage\section*{\myformat{关于反法西斯的国际统一战线}\\\myformat{(一九四一年六月二十三日)}}\addcontentsline{toc}{section}{关于反法西斯的国际统一战线}
\begin{introduction}\item  这是毛泽东为中共中央写的对党内的指示。\end{introduction}
德国法西斯统治者已于六月二十二日进攻苏联。此种背信弃义的侵略罪行,不仅是反对苏联的,而且也是反对一切民族的自由和独立的。苏联抵抗法西斯侵略的神圣战争,不仅是保卫苏联的,而且也是保卫正在进行反对法西斯奴役的解放斗争的一切民族的。\\
  目前共产党人在全世界的任务是动员各国人民组织国际统一战线,为着反对法西斯而斗争,为着保卫苏联、保卫中国、保卫一切民族的自由和独立而斗争。在目前时期,一切力量须集中于反对法西斯奴役。\\
  中国共产党在全中国的任务是:\\
  (一)坚持抗日民族统一战线,坚持国共合作,驱逐日本帝国主义出中国,即用此以援助苏联。\\
  (二)对于大资产阶级中的反动分子的任何反苏反共的活动,必须坚决反抗。\\
  (三)在外交上,同英美及其它国家一切反对德意日法西斯统治者的人们联合起来,反对共同的敌人。\\
\newpage\section*{\myformat{在陕甘宁边区参议会的演说}\\\myformat{(一九四一年十一月六日)}}\addcontentsline{toc}{section}{在陕甘宁边区参议会的演说}
各位参议员先生,各位同志:今天边区参议会开幕,是有重大意义的。参议会的目的,只有一个,就是要打倒日本帝国主义,建设新民主主义的中国,也就是革命的三民主义的中国。现在的中国不能有别的目的,只能有这个目的。因为现在我们的主要敌人不是国内的,而是日本和德意法西斯主义。现在苏联红军正在为苏联和全人类的命运奋斗,我们则在反对日本帝国主义。日本帝国主义还在继续侵略,它的目的是要灭亡中国。中国共产党的主张就是要团结全国一切抗日力量打倒日本帝国主义,要和全国一切抗日的党派、阶级、民族合作,只要不是汉奸,都要联合一致,共同奋斗。共产党的这种主张,是始终一致的。中国人民英勇抗战已有四年多,这个抗战是由国共两党的合作和各阶级各党派各民族的合作来支持的。但是还没有胜利;还要继续奋斗,还要使革命的三民主义见之实行,才能胜利。\\
  为什么我们要实行革命的三民主义?因为孙中山先生的革命的三民主义,直到现在还没有在全中国实现。为什么我们在现在不要求实行社会主义?社会主义当然是一个更好的制度,这个制度在苏联早已实行了,但是在今天的中国,还没有实行它的条件。陕甘宁边区所实行的是革命的三民主义。我们对于任何一个实际问题的解决,都没有超过革命的三民主义的范围。就目前来说,革命的三民主义中的民族主义,就是要打倒日本帝国主义;其民权主义和民生主义,就是要为全国一切抗日的人民谋利益,而不是只为一部分人谋利益。全国人民都要有人身自由的权利,参与政治的权利和保护财产的权利。全国人民都要有说话的机会,都要有衣穿,有饭吃,有事做,有书读,总之是要各得其所。中国社会是一个两头小中间大的社会,无产阶级和地主大资产阶级都只占少数,最广大的人民是农民、城市小资产阶级以及其它的中间阶级。任何政党的政策如果不顾到这些阶级的利益,如果这些阶级的人们不得其所,如果这些阶级的人们没有说话的权利,要想把国事弄好是不可能的。中国共产党提出的各项政策,都是为着团结一切抗日的人民,顾及一切抗日的阶级,而特别是顾及农民、城市小资产阶级以及其它中间阶级的。共产党提出的使各界人民都有说话机会、都有事做、都有饭吃的政策,是真正的革命三民主义的政策。在土地关系上,我们一方面实行减租减息,使农民有饭吃;另一方面又实行部分的交租交息,使地主也能过活。在劳资关系上,我们一方面扶助工人,使工人有工做,有饭吃;另一方面又实行发展实业的政策,使资本家也有利可图。所有这些,都是为了团结全国人民,合力抗日。这样的政策我们叫做新民主主义的政策。这是真正适合现在中国国情的政策;我们希望不但在陕甘宁边区实行,不但在敌后各抗日根据地实行,并且在全国也实行起来。\\
  我们实行这种政策是有成绩的,是得到全国人民赞成的。但是也有缺点。一部分共产党员,还不善于同党外人士实行民主合作,还保存一种狭隘的关门主义或宗派主义的作风。他们还不明白共产党员有义务同抗日的党外人士合作,无权利排斥这些党外人士的道理。这就是要倾听人民群众的意见,要联系人民群众,而不要脱离人民群众的道理。《陕甘宁边区施政纲领》上有一条,规定共产党员应当同党外人士实行民主合作,不得一意孤行,把持包办,就是针对着这一部分还不明白党的政策的同志而说的。共产党员必须倾听党外人士的意见,给别人以说话的机会。别人说得对的,我们应该欢迎,并要跟别人的长处学习;别人说得不对,也应该让别人说完,然后慢慢加以解释。共产党员决不可自以为是,盛气凌人,以为自己是什么都好,别人是什么都不好;决不可把自己关在小房子里,自吹自擂,称王称霸。除了勾结日寇汉奸以及破坏抗战和团结的反动的顽固派,这些人当然没有说话的资格以外,其它任何人,都有说话的自由,即使说错了也是不要紧的。国事是国家的公事,不是一党一派的私事。因此,共产党员只有对党外人士实行民主合作的义务,而无排斥别人、垄断一切的权利。共产党是为民族、为人民谋利益的政党,它本身决无私利可图。它应该受人民的监督,而决不应该违背人民的意旨。它的党员应该站在民众之中,而决不应该站在民众之上。各位代表先生们,各位同志们,共产党的这个同党外人士实行民主合作的原则,是固定不移的,是永远不变的。只要社会上还有党存在,加入党的人总是少数,党外的人总是多数,所以党员总是要和党外的人合作,现在就应在参议会中好好实行起来。我想,我们共产党的参议员,在我们这样的政策下面,可以在参议会中受到很好的锻炼,克服自己的关门主义和宗派主义。我们不是一个自以为是的小宗派,我们一定要学会打开大门和党外人士实行民主合作的方法,我们一定要学会善于同别人商量问题。也许到今天还有这样的共产党员,他们说,如果要和别人合作,我们就不干了。但是我相信,这样的人是极少的。我向各位保证,我党绝大多数的党员是一定能够执行我党中央的路线的。同时也要请各位党外同志了解我们的主张,了解共产党并不是一个只图私利的小宗派、小团体。不是的,共产党是真心实意想把国事办好的。但是我们的毛病还很多。我们不怕说出自己的毛病,我们一定要改正自己的毛病。我们要加强党内教育来清除这些毛病,我们还要经过和党外人士实行民主合作来清除这些毛病。这样的内外夹攻,才能把我们的毛病治好,才能把国事真正办好起来。\\
  各位参议员先生不辞辛勤,来此开会,我很高兴地庆祝这个盛会,庆祝这个盛会的成功。\\
\newpage\section*{\myformat{整顿党的作风}\\\myformat{(一九四二年二月一日)}}\addcontentsline{toc}{section}{整顿党的作风}
\begin{introduction}\item 这是毛泽东在中共中央党校开学典礼上的演说。\end{introduction}
党校今天开学,我庆祝这个学校的成功。\\
  今天我想讲一点关于我们的党的作风的问题。\\
  为什么要有革命党?因为世界上有压迫人民的敌人存在,人民要推翻敌人的压迫,所以要有革命党。就资本主义和帝国主义时代说来,就需要一个如共产党这样的革命党。如果没有共产党这样的革命党,人民要想推翻敌人的压迫,简直是不可能的。我们是共产党,我们要领导人民打倒敌人,我们的队伍就要整齐,我们的步调就要一致,兵要精,武器要好。如果不具备这些条件,那末,敌人就不会被我们打倒。\\
  现在我们的党还有什么问题呢?党的总路线是正确的,是没有问题的,党的工作也是有成绩的。党有几十万党员,他们在领导人民,向着敌人作艰苦卓绝的斗争。这是大家看见的,是不能怀疑的。\\
  那末,究竟我们的党还有什么问题没有呢?我讲,还是有问题的,而且就某种意义上讲,问题还相当严重。\\
  什么问题呢?就是有几样东西在一些同志的头脑中还显得不大正确,不大正派。\\
  这就是说,我们的学风还有些不正的地方,我们的党风还有些不正的地方,我们的文风也有些不正的地方。所谓学风有些不正,就是说有主观主义的毛病。所谓党风有些不正,就是说有宗派主义的毛病。所谓文风有些不正,就是说有党八股\footnote[1]{ 八股文是中国明、清封建皇朝考试制度所规定的一种特殊文体。它内容空洞,专讲形式,玩弄文字。这种文章的每一个段落都要死守在固定的格式里面,连字数都有一定的限制,人们只是按照题目的字义敷衍成文。党八股是指在革命队伍中某些人在写文章、发表演说或者做其它宣传工作的时候,对事物不加分析,只是搬用一些革命的名词和术语,言之无物,空话连篇,也和上述的八股文一样。}的毛病。这些作风不正,并不像冬天刮的北风那样,满天都是。主观主义、宗派主义、党八股,现在已不是占统治地位的作风了,这不过是一股逆风,一股歪风,是从防空洞里跑出来的。(笑声)但是我们党内还有这样的一种风,是不好的。我们要把产生这种歪风的洞塞死。我们全党都要来做这个塞洞工作,我们党校也要做这个工作。主观主义、宗派主义、党八股,这三股歪风,有它们的历史根源,现在虽然不是占全党统治地位的东西,但是它们还在经常作怪,还在袭击我们,因此,有加以抵制之必要,有加以研究分析说明之必要。\\
  反对主观主义以整顿学风,反对宗派主义以整顿党风,反对党八股以整顿文风,这就是我们的任务。\\
  我们要完成打倒敌人的任务,必须完成这个整顿党内作风的任务。学风和文风也都是党的作风,都是党风。只要我们党的作风完全正派了,全国人民就会跟我们学。党外有这种不良风气的人,只要他们是善良的,就会跟我们学,改正他们的错误,这样就会影响全民族。只要我们共产党的队伍是整齐的,步调是一致的,兵是精兵,武器是好武器,那末,任何强大的敌人都是能被我们打倒的。\\
  现在我来讲一讲主观主义。\\
  主观主义是一种不正派的学风,它是反对马克思列宁主义的,它是和共产党不能并存的。我们要的是马克思列宁主义的学风。所谓学风,不但是学校的学风,而且是全党的学风。学风问题是领导机关、全体干部、全体党员的思想方法问题,是我们对待马克思列宁主义的态度问题,是全党同志的工作态度问题。既然是这样,学风问题就是一个非常重要的问题,就是第一个重要的问题。\\
  现在有些糊涂观念,在许多人中间流行着,例如关于什么是理论家,什么是知识分子,什么是理论和实际联系等等问题的糊涂观念。\\
  我们首先要问,我们党的理论水平究竟是高还是低呢?近来马克思列宁主义的书籍翻译的多了,读的人也多了。这是很好的事。但是否就可以说我们党的理论水平已经是提得很高了呢?确实,我们的理论水平是比较过去高了一些。但是按照中国革命运动的丰富内容来说,理论战线就非常之不相称,二者比较起来,理论方面就显得非常之落后。一般地说来,我们的理论还不能够和革命实践相平行,更不去说理论应该跑到实践的前面去。我们还没有把丰富的实际提高到应有的理论程度。我们还没有对革命实践的一切问题,或重大问题,加以考察,使之上升到理论的阶段。你们看,中国的经济、政治、军事、文化,我们究有多少人创造了可以称为理论的理论,算得科学形态的、周密的而不是粗枝大叶的理论呢?特别是在经济理论方面,中国资本主义的发展,从鸦片战争到现在,已经一百年了,但是还没有产生一本合乎中国经济发展的实际的、真正科学的理论书。像在中国经济问题方面,能不能说理论水平已经高了呢?能不能说我党已经有了像样的经济理论家呢?实在不能说。我们读了许多马克思列宁主义的书籍,能不能就算是有了理论家呢?不能这样说。因为马克思列宁主义是马克思、恩格斯、列宁、斯大林他们根据实际创造出来的理论,从历史实际和革命实际中抽出来的总结论。我们如果仅仅读了他们的著作,但是没有进一步地根据他们的理论来研究中国的历史实际和革命实际,没有企图在理论上来思考中国的革命实践,我们就不能妄称为马克思主义的理论家。如果我们身为中国共产党员,却对于中国问题熟视无睹,只能记诵马克思主义书本上的个别的结论和个别的原理,那末,我们在理论战线上的成绩就未免太坏了。如果一个人只知背诵马克思主义的经济学或哲学,从第一章到第十章都背得烂熟了,但是完全不能应用,这样是不是就算得一个马克思主义的理论家呢?这还是不能算理论家的。我们所要的理论家是什么样的人呢?是要这样的理论家,他们能够依据马克思列宁主义的立场、观点和方法,正确地解释历史中和革命中所发生的实际问题,能够在中国的经济、政治、军事、文化种种问题上给予科学的解释,给予理论的说明。我们要的是这样的理论家。假如要作这样的理论家,那就要能够真正领会马克思列宁主义的实质,真正领会马克思列宁主义的立场、观点和方法,真正领会列宁斯大林关于殖民地革命和中国革命的学说,并且应用了它去深刻地、科学地分析中国的实际问题,找出它的发展规律,这样才是我们真正需要的理论家。\\
  现在我们党的中央做了决定\footnote[2]{ 指一九四一年八月一日《中央关于调查研究的决定》。这个决定要求全党采取具体措施,收集国内外政治、军事、经济、文化及社会阶级关系各方面的材料,加强对于历史,对于环境,对于国内外、省内外、县内外具体情况的调查研究,并将这种调查研究、了解情况的工作,同学习马克思列宁主义理论密切联系起来。},号召我们的同志学会应用马克思列宁主义的立场、观点和方法,认真地研究中国的历史,研究中国的经济、政治、军事和文化,对每一问题要根据详细的材料加以具体的分析,然后引出理论性的结论来。这个责任是担在我们的身上。\\
  我们党校的同志不应当把马克思主义的理论当成死的教条。对于马克思主义的理论,要能够精通它、应用它,精通的目的全在于应用。如果你能应用马克思列宁主义的观点,说明一个两个实际问题,那就要受到称赞,就算有了几分成绩。被你说明的东西越多,越普遍,越深刻,你的成绩就越大。现在我们的党校也要定这个规矩,看一个学生学了马克思列宁主义以后怎样看中国问题,有看得清楚的,有看不清楚的,有会看的,有不会看的,这样来分优劣,分好坏。\\
  其次讲一讲所谓“知识分子”的问题。因为我们中国是一个半殖民地半封建的国家,文化不发达,所以对于知识分子觉得特别宝贵。党中央在两年多以前作过一个关于知识分子问题的决定\footnote[3]{ 指一九三九年十二月一日中共中央关于吸收知识分子的决定,即本书第二卷《大量吸收知识分子》。},要争取广大的知识分子,只要他们是革命的,愿意参加抗日的,一概采取欢迎态度。我们尊重知识分子是完全应该的,没有革命知识分子,革命就不会胜利。但是我们晓得,有许多知识分子,他们自以为很有知识,大摆其知识架子,而不知道这种架子是不好的,是有害的,是阻碍他们前进的。他们应该知道一个真理,就是许多所谓知识分子,其实是比较地最无知识的,工农分子的知识有时倒比他们多一点。于是有人说:“哈!你弄颠倒了,乱说一顿。”(笑声)但是,同志,你别着急,我讲的多少有点道理。\\
  什么是知识?自从有阶级的社会存在以来,世界上的知识只有两门,一门叫做生产斗争知识,一门叫做阶级斗争知识。自然科学、社会科学,就是这两门知识的结晶,哲学则是关于自然知识和社会知识的概括和总结。此外还有什么知识呢?没有了。我们现在看看一些学生,看看那些同社会实际活动完全脱离关系的学校里面出身的学生,他们的状况是怎么样呢?一个人从那样的小学一直读到那样的大学,毕业了,算有知识了。但是他有的只是书本上的知识,还没有参加任何实际活动,还没有把自己学得的知识应用到生活的任何部门里去。像这样的人是否可以算得一个完全的知识分子呢?我以为很难,因为他的知识还不完全。什么是比较完全的知识呢?一切比较完全的知识都是由两个阶段构成的:第一阶段是感性知识,第二阶段是理性知识,理性知识是感性知识的高级发展阶段。学生们的书本知识是什么知识呢?假定他们的知识都是真理,也是他们的前人总结生产斗争和阶级斗争的经验写成的理论,不是他们自己亲身得来的知识。他们接受这种知识是完全必要的,但是必须知道,就一定的情况说来,这种知识对于他们还是片面性的,这种知识是人家证明了,而在他们则还没有证明的。最重要的,是善于将这些知识应用到生活和实际中去。所以我劝那些只有书本知识但还没有接触实际的人,或者实际经验尚少的人,应该明白自己的缺点,将自己的态度放谦虚一些。\\
  有什么办法使这种仅有书本知识的人变为名副其实的知识分子呢?唯一的办法就是使他们参加到实际工作中去,变为实际工作者,使从事理论工作的人去研究重要的实际问题。这样就可以达到目的。\\
  我这样说,难免有些人要发脾气。他们说:“照你这样解释,那末,马克思也算不得知识分子了。”我说:不对。马克思不但参加了革命的实际运动,而且进行了革命的理论创造。他从资本主义最单纯的因素——商品开始,周密地研究了资本主义社会的经济结构。商品这个东西,千百万人,天天看它,用它,但是熟视无睹。只有马克思科学地研究了它,他从商品的实际发展中作了巨大的研究工作,从普遍的存在中找出完全科学的理论来。他研究了自然,研究了历史,研究了无产阶级革命,创造了辩证唯物论、历史唯物论和无产阶级革命的理论。这样,马克思就成了一个代表人类最高智慧的最完全的知识分子,他和那些仅有书本知识的人有根本的区别。马克思在实际斗争中进行了详细的调查研究,概括了各种东西,得到的结论又拿到实际斗争中去加以证明,这样的工作就叫做理论工作。我们党内需要许多同志学做这样的工作。我们党内现在有大批的同志,可以学习从事于这样的理论研究工作,他们大都是聪明有为的人,我们要看重他们。但是他们的方针要对,过去犯过的错误他们不应重复。他们必须抛弃教条主义,必须不停止在现成书本的字句上。\\
  真正的理论在世界上只有一种,就是从客观实际抽出来又在客观实际中得到了证明的理论,没有任何别的东西可以称得起我们所讲的理论。斯大林曾经说过,脱离实际的理论是空洞的理论\footnote[4]{ 见本书第一卷《实践论》注〔10〕。}。空洞的理论是没有用的,不正确的,应该抛弃的。对于好谈这种空洞理论的人,应该伸出一个指头向他刮脸皮。马克思列宁主义是从客观实际产生出来又在客观实际中获得了证明的最正确最科学最革命的真理;但是许多学习马克思列宁主义的人却把它看成是死的教条,这样就阻碍了理论的发展,害了自己,也害了同志。\\
  另一方面,我们从事实际工作的同志,如果误用了他们的经验,也是要出毛病的。不错,这样的人往往经验很多,这是很可宝贵的;但是如果他们就以自己的经验为满足,那也很危险。他们须知自己的知识是偏于感性的或局部的,缺乏理性的知识和普遍的知识,就是说,缺乏理论,他们的知识也是比较地不完全。而要把革命事业做好,没有比较完全的知识是不行的。\\
  这样看来,有两种不完全的知识,一种是现成书本上的知识,一种是偏于感性和局部的知识,这二者都有片面性。只有使二者互相结合,才会产生好的比较完全的知识。\\
  但是,我们的工农干部要学理论,必须首先学文化。没有文化,马克思列宁主义的理论就学不进去。学好了文化,随时都可学习马克思列宁主义。我幼年没有进过马克思列宁主义的学校,学的是“子曰学而时习之,不亦说乎”\footnote[5]{ 这是孔子和他的弟子们的语录《论语》的开头一句话。}一套,这种学习的内容虽然陈旧了,但是对我也有好处,因为我识字便是从这里学来的。何况现在不是学的孔夫子,学的是新鲜的国语、历史、地理和自然常识,这些文化课学好了,到处有用。我们党中央现在着重要求工农干部学习文化,因为学了文化以后,政治、军事、经济哪一门都可学。否则工农干部虽有丰富经验,却没有学习理论的可能。\\
  由此看来,我们反对主观主义,必须使上述两种人各向自己缺乏的方面发展,必须使两种人互相结合。有书本知识的人向实际方面发展,然后才可以不停止在书本上,才可以不犯教条主义的错误。有工作经验的人,要向理论方面学习,要认真读书,然后才可以使经验带上条理性、综合性,上升成为理论,然后才可以不把局部经验误认为即是普遍真理,才可不犯经验主义的错误。教条主义、经验主义,两者都是主观主义,是从不同的两极发生的东西。\\
  所以,我们党内的主观主义有两种:一种是教条主义,一种是经验主义。他们都是只看到片面,没有看到全面。如果不注意,如果不知道这种片面性的缺点,并且力求改正,那就容易走上错误的道路。\\
  但是在这两种主观主义中,现在在我们党内还是教条主义更为危险。因为教条主义容易装出马克思主义的面孔,吓唬工农干部,把他们俘虏起来,充作自己的用人,而工农干部不易识破他们;也可以吓唬天真烂漫的青年,把他们充当俘虏。我们如果把教条主义克服了,就可以使有书本知识的干部,愿意和有经验的干部相结合,愿意从事实际事物的研究,可以产生许多理论和经验结合的良好的工作者,可以产生一些真正的理论家。我们如果把教条主义克服了,就可以使有经验的同志得着良好的先生,使他们的经验上升成为理论,而避免经验主义的错误。\\
  除了对于“理论家”和“知识分子”存在着糊涂观念而外,还有天天念的一句“理论和实际联系”,在许多同志中间也是一个糊涂观念。他们天天讲“联系”,实际上却是讲“隔离”,因为他们并不去联系。马克思列宁主义理论和中国革命实际,怎样互相联系呢?拿一句通俗的话来讲,就是“有的放矢”。“矢”就是箭,“的”就是靶,放箭要对准靶。马克思列宁主义和中国革命的关系,就是箭和靶的关系。有些同志却在那里“无的放矢”,乱放一通,这样的人就容易把革命弄坏。有些同志则仅仅把箭拿在手里搓来搓去,连声赞曰:“好箭!好箭!”却老是不愿意放出去。这样的人就是古董鉴赏家,几乎和革命不发生关系。马克思列宁主义之箭,必须用了去射中国革命之的。这个问题不讲明白,我们党的理论水平永远不会提高,中国革命也永远不会胜利。\\
  我们的同志必须明白,我们学马克思列宁主义不是为着好看,也不是因为它有什么神秘,只是因为它是领导无产阶级革命事业走向胜利的科学。直到现在,还有不少的人,把马克思列宁主义书本上的某些个别字句看作现成的灵丹圣药,似乎只要得了它,就可以不费气力地包医百病。这是一种幼稚者的蒙昧,我们对这些人应该作启蒙运动。那些将马克思列宁主义当宗教教条看待的人,就是这种蒙昧无知的人。对于这种人,应该老实地对他说,你的教条一点什么用处也没有。马克思、恩格斯、列宁、斯大林曾经反复地讲,我们的学说不是教条而是行动的指南。这些人偏偏忘记这句最重要最重要的话。中国共产党人只有在他们善于应用马克思列宁主义的立场、观点和方法,善于应用列宁斯大林关于中国革命的学说,进一步地从中国的历史实际和革命实际的认真研究中,在各方面作出合乎中国需要的理论性的创造,才叫做理论和实际相联系。如果只是口头上讲联系,行动上又不实行联系,那末,讲一百年也还是无益的。我们反对主观地片面地看问题,必须攻破教条主义的主观性和片面性。\\
  关于反对主观主义以整顿全党的学风的问题,今天讲的就是这些。\\
  现在我来讲一讲宗派主义的问题。\\
  由于二十年的锻炼,现在我们党内并没有占统治地位的宗派主义了。但是宗派主义的残余是还存在的,有对党内的宗派主义残余,也有对党外的宗派主义残余。对内的宗派主义倾向产生排内性,妨碍党内的统一和团结;对外的宗派主义倾向产生排外性,妨碍党团结全国人民的事业。铲除这两方面的祸根,才能使党在团结全党同志和团结全国人民的伟大事业中畅行无阻。\\
  什么是党内宗派主义的残余呢?主要的有下面几种:\\
  首先就是闹独立性。一部分同志,只看见局部利益,不看见全体利益,他们总是不适当地特别强调他们自己所管的局部工作,总希望使全体利益去服从他们的局部利益。他们不懂得党的民主集中制,他们不知道共产党不但要民主,尤其要集中。他们忘记了少数服从多数,下级服从上级,局部服从全体,全党服从中央的民主集中制。张国焘\footnote[6]{ 见本书第一卷《论反对日本帝国主义的策略》注〔24〕。}是向党中央闹独立性的,结果闹到叛党,做特务去了。现在讲的,虽然不是这种极端严重的宗派主义,但是这种现象必须预防,必须将各种不统一的现象完全除去。要提倡顾全大局。每一个党员,每一种局部工作,每一项言论或行动,都必须以全党利益为出发点,绝对不许可违反这个原则。\\
  闹这类独立性的人,常常跟他们的个人第一主义分不开,他们在个人和党的关系问题上,往往是不正确的。他们在口头上虽然也说尊重党,但他们在实际上却把个人放在第一位,把党放在第二位。刘少奇同志曾经说过,有一种人的手特别长,很会替自己个人打算,至于别人的利益和全党的利益,那是不大关心的。“我的就是我的,你的还是我的”。(大笑)这种人闹什么东西呢?闹名誉,闹地位,闹出风头。在他们掌管一部分事业的时候,就要闹独立性。为了这些,就要拉拢一些人,排挤一些人,在同志中吹吹拍拍,拉拉扯扯,把资产阶级政党的庸俗作风也搬进共产党里来了。这种人的吃亏在于不老实。我想,我们应该是老老实实地办事;在世界上要办成几件事,没有老实态度是根本不行的。什么人是老实人?马克思、恩格斯、列宁、斯大林是老实人,科学家是老实人。什么人是不老实的人?托洛茨基、布哈林、陈独秀、张国焘是大不老实的人,为个人利益为局部利益闹独立性的人也是不老实的人。一切狡猾的人,不照科学态度办事的人,自以为得计,自以为很聪明,其实都是最蠢的,都是没有好结果的。我们党校的学生一定要注意这个问题。我们一定要建设一个集中的统一的党,一切无原则的派别斗争,都要清除干净。要使我们全党的步调整齐一致,为一个共同目标而奋斗,我们一定要反对个人主义和宗派主义。\\
  外来干部和本地干部必须团结,必须反对宗派主义倾向。因为许多抗日根据地是八路军新四军到后才创立的,许多地方工作是外来干部去后才发展的,外来干部和本地干部的关系,必须加以很好的注意。我们的同志必须懂得,在这种条件下,只有外来干部和本地干部完全团结一致,只有本地干部大批地生长了,并提拔起来了,根据地才能巩固,我党在根据地内才能生根,否则是不可能的。外来干部和本地干部各有长处,也各有短处,必须互相取长补短,才能有进步。外来干部比较本地干部,对于熟悉情况和联系群众这些方面,总要差些。拿我来说,就是这样。我到陕北已经五六年了,可是对于陕北的情况的了解,对于和陕北人民的联系,和一些陕北同志比较起来就差得多。我们到山西、河北、山东以及其它抗日根据地的同志,一定要注意这个问题。不但如此,即在一个根据地内部,因为根据地内的各个区域有发展先后之不同,干部中也有外来本地之别。比较先进区域的干部到比较落后的区域去,对于当地,也是一种外来干部,也要十分注意扶助本地干部的问题。就一般情形说来,凡属外来干部负领导责任的地方,如果和本地干部的关系弄得不好,那末,这个责任主要地应该放在外来干部的身上。担负主要领导责任的同志,其责任就更大些。现在各地对这个问题的注意还很不够,有些人轻视本地干部,讥笑本地干部,他们说:“本地人懂得什么,土包子!”这种人完全不懂得本地干部的重要性,他们既不了解本地干部的长处,也不了解自己的短处,采取了不正确的宗派主义的态度。一切外来干部一定要爱护本地干部,经常帮助他们,而不许可讥笑他们,打击他们。自然,本地干部也必须学习外来干部的长处,必须去掉那些不适当的狭隘的观点,以求和外来干部完全不分彼此,打成一片,而避免宗派主义倾向。\\
  军队工作干部和地方工作干部的关系也是如此。两者必须完全团结一致,必须反对宗派主义的倾向。军队干部必须帮助地方干部,地方干部也必须帮助军队干部。如有纠纷,应该双方互相原谅,而各对自己作正确的自我批评。在军队干部事实上居于领导地位的地方,在一般的情形之下,如果和地方干部的关系弄不好,那末,主要的责任应该放在军队干部的身上。必须使军队干部首先懂得自己的责任,以谦虚的态度对待地方干部,才能使根据地的战争工作和建设工作得到顺利进行的条件。\\
  几部分军队之间、几个地方之间、几个工作部门之间的关系,也是如此。必须反对只顾自己不顾别人的本位主义的倾向。谁要是对别人的困难不管,别人要调他所属的干部不给,或以坏的送人,“以邻为壑”,全不为别部、别地、别人想一想,这样的人就叫做本位主义者,这就是完全失掉了共产主义的精神。不顾大局,对别部、别地、别人漠不关心,就是这种本位主义者的特点。对于这样的人,必须加重教育,使他们懂得这就是一种宗派主义的倾向,如果发展下去,是很危险的。\\
  还有一个问题,就是老干部和新干部的关系问题。抗战以来,我党有广大的发展,大批新干部产生了,这是很好的现象。斯大林同志在联共十八次代表大会上的报告中说:“老干部通常总是不多,比所需要的数量少,而且由于宇宙自然法则的关系,他们已部分地开始衰老死亡下去。”\footnote[7]{ 这段话的新译文是:“老干部总是少数,不能满足需要,而且由于自然界的天然规律,他们已经部分地开始丧失工作能力。”(《斯大林选集》下卷,人民出版社1979年版,第460页)}他在这里讲了干部状况,又讲了自然科学。我们党如果没有广大的新干部同老干部一致合作,我们的事业就会中断。所以一切老干部应该以极大的热忱欢迎新干部,关心新干部。不错,新干部是有缺点的,他们参加革命还不久,还缺乏经验,他们中的有些人还不免带来旧社会不良思想的尾巴,这就是小资产阶级个人主义思想的残余。但是这些缺点是可以从教育中从革命锻炼中逐渐地去掉的。他们的长处,正如斯大林说过的,是对于新鲜事物有锐敏的感觉,因而有高度的热情和积极性,而在这一点上,有些老干部则正是缺乏的\footnote[8]{ 参见斯大林《在党的第十八次代表大会上关于联共(布)中央工作的总结报告》第三部分第二节(《斯大林选集》下卷,人民出版社1979年版,第460页)。}。新老干部应该是彼此尊重,互相学习,取长补短,以便团结一致,进行共同的事业,而防止宗派主义的倾向。在老干部负主要领导责任的地方,在一般情形之下,如果老干部和新干部的关系弄得不好,那末,老干部就应该负主要的责任。\\
  以上所讲的局部和全体的关系,个人和党的关系,外来干部和本地干部的关系,军队干部和地方干部的关系,军队和军队、地方和地方、这一工作部门和那一工作部门之间的关系,老干部和新干部的关系,都是党内的相互关系。在这种种方面,都应该提高共产主义精神,防止宗派主义倾向,使我们的党达到队伍整齐,步调一致的目的,以利战斗。这是一个很重要的问题,我们整顿党的作风,必须彻底地解决这个问题。宗派主义是主观主义在组织关系上的一种表现;我们如果不要主观主义,要发展马克思列宁主义实事求是的精神,就必须扫除党内宗派主义的残余,以党的利益高于个人和局部的利益为出发点,使党达到完全团结统一的地步。\\
  宗派主义的残余,在党内关系上是应该消灭的,在党外关系上也是应该消灭的。其理由就是:单是团结全党同志还不能战胜敌人,必须团结全国人民才能战胜敌人。中国共产党在团结全国人民的事业上,二十年来做了艰苦的伟大的工作;抗战以来,这个工作的成绩更加伟大。但这并不是说,我们所有的同志对待人民群众都有了正确的作风,都没有了宗派主义的倾向。不是的。在一部分同志中,确实还有宗派主义的倾向,有些人并且很严重。我们的许多同志,喜欢对党外人员妄自尊大,看人家不起,藐视人家,而不愿尊重人家,不愿了解人家的长处。这就是宗派主义的倾向。这些同志,读了几本马克思主义的书籍之后,不是更谦虚,而是更骄傲了,总是说人家不行,而不知自己实在是一知半解。我们的同志必须懂得一条真理:共产党员和党外人员相比较,无论何时都是占少数。假定一百个人中有一个共产党员,全中国四亿五千万人中就有四百五十万共产党员。即使达到这样大的数目,共产党员也还是只占百分之一,百分之九十九都是非党员。我们有什么理由不和非党人员合作呢?对于一切愿意同我们合作以及可能同我们合作的人,我们只有同他们合作的义务,绝无排斥他们的权利。一部分党员却不懂得这个道理,看不起愿意同我们合作的人,甚至排斥他们。这是没有任何根据的。马克思、恩格斯、列宁、斯大林给了我们这样的根据吗?没有。相反地,他们总是谆谆告诫我们,要密切联系群众,而不要脱离群众。中国共产党中央给了我们这个根据吗?没有。中央的一切决议案中,没有一个决议说是我们可以脱离群众使自己孤立起来。相反地,中央总是叫我们密切联系群众,而不要脱离群众。所以,一切脱离群众的行为,并没有任何的根据,只是我们一部分同志自己造出来的宗派主义思想在那里作怪。因为这种宗派主义在一部分同志中还很严重,还在障碍党的路线的实行,所以我们要针对这个问题在党内进行广大的教育。首先要使我们的干部真正懂得这个问题的严重性,使他们懂得共产党员如果不同党外干部、党外人员互相联合,敌人就一定不能打倒,革命的目的就一定不能达到。\\
  一切宗派主义思想都是主观主义的,都和革命的实际需要不相符合,所以反对宗派主义和反对主观主义的斗争,应当同时并进。\\
  关于党八股的问题,今天不能讲了,准备在另外一个会议上来讨论。党八股是藏垢纳污的东西,是主观主义和宗派主义的一种表现形式。它是害人的,不利于革命的,我们必须肃清它。\\
  我们要反对主观主义,就要宣传唯物主义,就要宣传辩证法。但是我们党内还有许多同志,他们并不注重宣传唯物主义,也不注重宣传辩证法。有些同志听凭别人宣传主观主义,也安之若素。这些同志自以为相信马克思主义,但是,他们却不努力宣传唯物主义,听了或看了主观主义的东西也不想一想,也不发议论。这种态度不是共产党员的态度。这使得我们许多同志蒙受了主观主义思想的毒害,发生麻木的现象。所以我们要在党内发动一个启蒙运动,使我们同志的精神从主观主义、教条主义的蒙蔽中间解放出来,号召同志们对于主观主义、宗派主义、党八股加以抵制。这些东西好像日货,因为只有我们的敌人愿意我们保存这些坏东西,使我们继续受蒙蔽,所以我们应该提倡抵制,就像抵制日货\footnote[9]{ 抵制日货是中国人民在二十世纪上半叶所常常采取的反抗日本帝国主义侵略的一种斗争方法。例如,在一九一九年五四爱国运动时期,一九三一年九一八事变之后,中国人民都曾经进行过抵制日货的运动。}一样。一切主观主义、宗派主义、党八股的货色,我们都要抵制,使它们在市场上销售困难,不要让它们利用党内理论水平低,出卖自己那一套。为此目的,就要同志们提高嗅觉,就要同志们对于任何东西都用鼻子嗅一嗅,鉴别其好坏,然后才决定欢迎它,或者抵制它。共产党员对任何事情都要问一个为什么,都要经过自己头脑的周密思考,想一想它是否合乎实际,是否真有道理,绝对不应盲从,绝对不应提倡奴隶主义。\\
  最后,我们反对主观主义、宗派主义、党八股,有两条宗旨是必须注意的:第一是“惩前毖后”,第二是“治病救人”。对以前的错误一定要揭发,不讲情面,要以科学的态度来分析批判过去的坏东西,以便使后来的工作慎重些,做得好些。这就是“惩前毖后”的意思。但是我们揭发错误、批判缺点的目的,好像医生治病一样,完全是为了救人,而不是为了把人整死。一个人发了阑尾炎,医生把阑尾割了,这个人就救出来了。任何犯错误的人,只要他不讳疾忌医,不固执错误,以至于达到不可救药的地步,而是老老实实,真正愿意医治,愿意改正,我们就要欢迎他,把他的毛病治好,使他变为一个好同志。这个工作决不是痛快一时,乱打一顿,所能奏效的。对待思想上的毛病和政治上的毛病,决不能采用鲁莽的态度,必须采用“治病救人”的态度,才是正确有效的方法。\\
  趁着今天党校开学的机会,我讲了这许多话,希望同志们加以考虑。(热烈的鼓掌)\\
\newpage\section*{\myformat{反对党八股}\\\myformat{(一九四二年二月八日)}}\addcontentsline{toc}{section}{反对党八股}
\begin{introduction}\item  这是毛泽东在延安干部会上的讲演。\end{introduction}
刚才凯丰\footnote[1]{ 凯丰(一九〇六——一九五五),又名何克全,江西萍乡人。当时任中共中央宣传部代理部长。}同志讲了今天开会的宗旨。我现在想讲的是:主观主义和宗派主义怎样拿党八股\footnote[2]{ 见本卷《整顿党的作风》注〔1〕。}做它们的宣传工具,或表现形式。我们反对主观主义和宗派主义,如果不连党八股也给以清算,那它们就还有一个藏身的地方,它们还可以躲起来。如果我们连党八股也打倒了,那就算对于主观主义和宗派主义最后地“将一军”\footnote[3]{ “将一军”是中国象棋中的术语。中国象棋采取两军对战的形式,而以一方攻入对方堡垒捉住“将军”(主帅)作为赢棋。凡是一方走了一着棋,使对方的将军有立即被捉的危险时,就叫做向对方“将军”。},弄得这两个怪物原形毕露,“老鼠过街,人人喊打”,这两个怪物也就容易消灭了。\\
  一个人写党八股,如果只给自己看,那倒还不要紧。如果送给第二个人看,人数多了一倍,已属害人不浅。如果还要贴在墙上,或付油印,或登上报纸,或印成一本书,那问题可就大了,它就可以影响许多的人。而写党八股的人们,却总是想写给许多人看的。这就非加以揭穿,把它打倒不可。\\
  党八股也就是一种洋八股。这洋八股,鲁迅早就反对过的\footnote[4]{ 反对新旧八股是鲁迅作品里一贯的精神。鲁迅曾在《伪自由书•透底》一文中说:“八股原是蠢笨的产物。一来是考官嫌麻烦——他们的头脑大半是阴沉木做的,——什么代圣贤立言,什么起承转合,文章气韵,都没有一定的标准,难以捉摸,因此,一股一股地定出来,算是合于功令的格式,用这格式来‘衡文’,一眼就看得出多少轻重。二来,连应试的人也觉得又省力,又不费事了。这样的八股,无论新旧,都应当扫荡。”洋八股是五四运动以后一些浅薄的知识分子发展起来的东西,并经过他们的传播,长时期地在革命队伍中存在着。鲁迅在《透底》附录“回祝秀侠信”中批判这种洋八股说:“八股无论新旧,都在扫荡之列,……例如只会‘辱骂’‘恐吓’甚至于‘判决’,而不肯具体地切实地运用科学所求得的公式,去解释每天的新的事实,新的现象,而只抄一通公式,往一切事实上乱凑,这也是一种八股。”(《鲁迅全集》第5卷,人民文学出版社1981年版,第103—106页)}。我们为什么又叫它做党八股呢?这是因为它除了洋气之外,还有一点土气。也算一个创作吧!谁说我们的人一点创作也没有呢?这就是一个!(大笑)\\
  党八股在我们党内已经有了一个长久的历史;特别是在土地革命时期,有时竟闹得很严重。\\
  从历史来看,党八股是对于五四运动的一个反动。\\
  五四运动时期,一班新人物反对文言文,提倡白话文,反对旧教条,提倡科学和民主,这些都是很对的。在那时,这个运动是生动活泼的,前进的,革命的。那时的统治阶级都拿孔夫子的道理教学生,把孔夫子的一套当作宗教教条一样强迫人民信奉,做文章的人都用文言文。总之,那时统治阶级及其帮闲者们的文章和教育,不论它的内容和形式,都是八股式的,教条式的。这就是老八股、老教条。揭穿这种老八股、老教条的丑态给人民看,号召人民起来反对老八股、老教条,这就是五四运动时期的一个极大的功绩。五四运动还有和这相联系的反对帝国主义的大功绩;这个反对老八股、老教条的斗争,也是它的大功绩之一。但到后来就产生了洋八股、洋教条。我们党内的一些违反了马克思主义的人则发展这种洋八股、洋教条,成为主观主义、宗派主义和党八股的东西。这些就都是新八股、新教条。这种新八股、新教条,在我们许多同志的头脑中弄得根深蒂固,使我们今天要进行改造工作还要费很大的气力。这样看来,“五四”时期的生动活泼的、前进的、革命的、反对封建主义的老八股、老教条的运动,后来被一些人发展到了它的反对方面,产生了新八股、新教条。它们不是生动活泼的东西,而是死硬的东西了;不是前进的东西,而是后退的东西了;不是革命的东西,而是阻碍革命的东西了。这就是说,洋八股或党八股,是五四运动本来性质的反动。但五四运动本身也是有缺点的。那时的许多领导人物,还没有马克思主义的批判精神,他们使用的方法,一般地还是资产阶级的方法,即形式主义的方法。他们反对旧八股、旧教条,主张科学和民主,是很对的。但是他们对于现状,对于历史,对于外国事物,没有历史唯物主义的批判精神,所谓坏就是绝对的坏,一切皆坏;所谓好就是绝对的好,一切皆好。这种形式主义地看问题的方法,就影响了后来这个运动的发展。五四运动的发展,分成了两个潮流。一部分人继承了五四运动的科学和民主的精神,并在马克思主义的基础上加以改造,这就是共产党人和若干党外马克思主义者所做的工作。另一部分人则走到资产阶级的道路上去,是形式主义向右的发展。但在共产党内也不是一致的,其中也有一部分人发生偏向,马克思主义没有拿得稳,犯了形式主义的错误,这就是主观主义、宗派主义和党八股,这是形式主义向“左”的发展。这样看来,党八股这种东西,一方面是五四运动的积极因素的反动,一方面也是五四运动的消极因素的继承、继续或发展,并不是偶然的东西。我们懂得这一点是有好处的。如果“五四”时期反对老八股和老教条主义是革命的和必需的,那末,今天我们用马克思主义来批判新八股和新教条主义也是革命的和必需的。如果“五四”时期不反对老八股和老教条主义,中国人民的思想就不能从老八股和老教条主义的束缚下面获得解放,中国就不会有自由独立的希望。这个工作,五四运动时期还不过是一个开端,要使全国人民完全脱离老八股和老教条主义的统治,还须费很大的气力,还是今后革命改造路上的一个大工程。如果我们今天不反对新八股和新教条主义,则中国人民的思想又将受另一个形式主义的束缚。至于我们党内一部分(当然只是一部分)同志所中的党八股的毒,所犯的教条主义的错误,如果不除去,那末,生动活泼的革命精神就不能启发,拿不正确态度对待马克思主义的恶习就不能肃清,真正的马克思主义就不能得到广泛的传播和发展;而对于老八股和老教条在全国人民中间的影响,以及洋八股和洋教条在全国许多人中间的影响,也就不能进行有力的斗争,也就达不到加以摧毁廓清的目的。\\
  主观主义、宗派主义和党八股,这三种东西,都是反马克思主义的,都不是无产阶级所需要的,而是剥削阶级所需要的。这些东西在我们党内,是小资产阶级思想的反映。中国是一个小资产阶级成分极其广大的国家,我们党是处在这个广大阶级的包围中,我们又有很大数量的党员是出身于这个阶级的,他们都不免或长或短地拖着一条小资产阶级的尾巴进党来。小资产阶级革命分子的狂热性和片面性,如果不加以节制,不加以改造,就很容易产生主观主义、宗派主义,它的一种表现形式就是洋八股,或党八股。\\
  要做对于这些东西的肃清工作和打扫工作,是不容易的。做起来必须得当,就是说,要好好地说理。如果说理说得好,说得恰当,那是会有效力的。说理的首先一个方法,就是重重地给患病者一个刺激,向他们大喝一声,说:“你有病呀!”使患者为之一惊,出一身汗,然后好好地叫他们治疗。\\
  现在来分析一下党八股的坏处在什么地方。我们也仿照八股文章的笔法\footnote[5]{ 见本书第一卷《中国革命战争的战略问题》注〔52〕。}来一个“八股”,以毒攻毒,就叫做八大罪状吧。\\
  党八股的第一条罪状是:空话连篇,言之无物。我们有些同志欢喜写长文章,但是没有什么内容,真是“懒婆娘的裹脚,又长又臭”。为什么一定要写得那么长,又那么空空洞洞的呢?只有一种解释,就是下决心不要群众看。因为长而且空,群众见了就摇头,哪里还肯看下去呢?只好去欺负幼稚的人,在他们中间散布坏影响,造成坏习惯。去年六月二十二日,苏联进行那么大的反侵略战争,斯大林在七月三日发表了一篇演说,还只有我们《解放日报》\footnote[6]{ 《解放日报》是中共中央的机关报,一九四一年五月十六日在延安创刊,一九四七年三月二十七日终刊。}一篇社论那样长。要是我们的老爷写起来,那就不得了,起码得有几万字。现在是在战争的时期,我们应该研究一下文章怎样写得短些,写得精粹些。延安虽然还没有战争,但军队天天在前方打仗,后方也唤工作忙,文章太长了,有谁来看呢?有些同志在前方也喜欢写长报告。他们辛辛苦苦地写了,送来了,其目的是要我们看的。可是怎么敢看呢?长而空不好,短而空就好吗?也不好。我们应当禁绝一切空话。但是主要的和首先的任务,是把那些又长又臭的懒婆娘的裹脚,赶快扔到垃圾桶里去。或者有人要说:《资本论》不是很长的吗?那又怎么办?这是好办的,看下去就是了。俗话说:“到什么山上唱什么歌。”又说:“看菜吃饭,量体裁衣。”我们无论做什么事都要看情形办理,文章和演说也是这样。我们反对的是空话连篇言之无物的八股调,不是说任何东西都以短为好。战争时期固然需要短文章,但尤其需要有内容的文章。最不应该、最要反对的是言之无物的文章。演说也是一样,空话连篇言之无物的演说,是必须停止的。\\
  党八股的第二条罪状是:装腔作势,借以吓人。有些党八股,不只是空话连篇,而且装样子故意吓人,这里面包含着很坏的毒素。空话连篇,言之无物,还可以说是幼稚;装腔作势,借以吓人,则不但是幼稚,简直是无赖了。鲁迅曾经批评过这种人,他说:“辱骂和恐吓决不是战斗。”\footnote[7]{ 这是鲁迅《南腔北调集》中一篇文章的篇名,一九三二年作。(《鲁迅全集》第4卷,人民文学出版社1981年版,第451页)}科学的东西,随便什么时候都是不怕人家批评的,因为科学是真理,决不怕人家驳。主观主义和宗派主义的东西,表现在党八股式的文章和演说里面,却生怕人家驳,非常胆怯,于是就靠装样子吓人;以为这一吓,人家就会闭口,自己就可以“得胜回朝”了。这种装腔作势的东西,不能反映真理,而是妨害真理的。凡真理都不装样子吓人,它只是老老实实地说下去和做下去。从前许多同志的文章和演说里面,常常有两个名词:一个叫做“残酷斗争”,一个叫做“无情打击”。这种手段,用了对付敌人或敌对思想是完全必要的,用了对付自己的同志则是错误的。党内也常常有敌人和敌对思想混进来,如《苏联共产党(布)历史简要读本》结束语第四条所说的那样。对于这种人,毫无疑义地是应该采用残酷斗争或无情打击的手段的,因为那些坏人正在利用这种手段对付党,我们如果还对他们宽容,那就会正中坏人的奸计。但是不能用同一手段对付偶然犯错误的同志;对于这类同志,就须使用批评和自我批评的方法,这就是《苏联共产党(布)历史简要读本》结束语第五条所说的方法。从前我们那些同志之所以向这些同志也大讲其“残酷斗争”和“无情打击”,一方面是没有分析对象,一方面就是为着装腔作势,借以吓人。无论对什么人,装腔作势借以吓人的方法,都是要不得的。因为这种吓人战术,对敌人是毫无用处,对同志只有损害。这种吓人战术,是剥削阶级以及流氓无产者所惯用的手段,无产阶级不需要这类手段。无产阶级的最尖锐最有效的武器只有一个,那就是严肃的战斗的科学态度。共产党不靠吓人吃饭,而是靠马克思列宁主义的真理吃饭,靠实事求是吃饭,靠科学吃饭。至于以装腔作势来达到名誉和地位的目的,那更是卑劣的念头,不待说的了。总之,任何机关做决定,发指示,任何同志写文章,做演说,一概要靠马克思列宁主义的真理,要靠有用。只有靠了这个才能争取革命胜利,其它都是无益的。\\
  党八股的第三条罪状是:无的放矢,不看对象。早几年,在延安城墙上,曾经看见过这样一个标语:“工人农民联合起来争取抗日胜利。”这个标语的意思并不坏,可是那工人的工字第二笔不是写的一直,而是转了两个弯子,写成了“—ㄣ—”字。人字呢?在右边一笔加了三撇,写成了“[人彡]”字。这位同志是古代文人学士的学生是无疑的了,可是他却要写在抗日时期延安这地方的墙壁上,就有些莫名其妙了。大概他的意思也是发誓不要老百姓看,否则就很难得到解释。共产党员如果真想做宣传,就要看对象,就要想一想自己的文章、演说、谈话、写字是给什么人看、给什么人听的,否则就等于下决心不要人看,不要人听。许多人常常以为自己写的讲的人家都看得很懂,听得很懂,其实完全不是那么一回事,因为他写的和讲的是党八股,人家哪里会懂呢?“对牛弹琴”这句话,含有讥笑对象的意思。如果我们除去这个意思,放进尊重对象的意思去,那就只剩下讥笑弹琴者这个意思了。为什么不看对象乱弹一顿呢?何况这是党八股,简直是老鸦声调,却偏要向人民群众哇哇地叫。射箭要看靶子,弹琴要看听众,写文章做演说倒可以不看读者不看听众吗?我们和无论什么人做朋友,如果不懂得彼此的心,不知道彼此心里面想些什么东西,能够做成知心朋友吗?做宣传工作的人,对于自己的宣传对象没有调查,没有研究,没有分析,乱讲一顿,是万万不行的。\\
  党八股的第四条罪状是:语言无味,像个瘪三\footnote[8]{ 解放以前,上海人称城市中无正当职业而以乞讨为生的游民为瘪三,他们通常是极瘦的。}。上海人叫小瘪三的那批角色,也很像我们的党八股,干瘪得很,样子十分难看。如果一篇文章,一个演说,颠来倒去,总是那几个名词,一套“学生腔”,没有一点生动活泼的语言,这岂不是语言无味,面目可憎,像个瘪三吗?一个人七岁入小学,十几岁入中学,二十多岁在大学毕业,没有和人民群众接触过,语言不丰富,单纯得很,那是难怪的。但我们是革命党,是为群众办事的,如果也不学群众的语言,那就办不好。现在我们有许多做宣传工作的同志,也不学语言。他们的宣传,乏味得很;他们的文章,就没有多少人欢喜看;他们的演说,也没有多少人欢喜听。为什么语言要学,并且要用很大的气力去学呢?因为语言这东西,不是随便可以学好的,非下苦功不可。第一,要向人民群众学习语言。人民的语汇是很丰富的,生动活泼的,表现实际生活的。我们很多人没有学好语言,所以我们在写文章做演说时没有几句生动活泼切实有力的话,只有死板板的几条筋,像瘪三一样,瘦得难看,不像一个健康的人。第二,要从外国语言中吸收我们所需要的成分。我们不是硬搬或滥用外国语言,是要吸收外国语言中的好东西,于我们适用的东西。因为中国原有语汇不够用,现在我们的语汇中就有很多是从外国吸收来的。例如今天开的干部大会,这“干部”两个字,就是从外国学来的。我们还要多多吸收外国的新鲜东西,不但要吸收他们的进步道理,而且要吸收他们的新鲜用语。第三,我们还要学习古人语言中有生命的东西。由于我们没有努力学习语言,古人语言中的许多还有生气的东西我们就没有充分地合理地利用。当然我们坚决反对去用已经死了的语汇和典故,这是确定了的,但是好的仍然有用的东西还是应该继承。现在中党八股毒太深的人,对于民间的、外国的、古人的语言中有用的东西,不肯下苦功去学,因此,群众就不欢迎他们枯燥无味的宣传,我们也不需要这样蹩脚的不中用的宣传家。什么是宣传家?不但教员是宣传家,新闻记者是宣传家,文艺作者是宣传家,我们的一切工作干部也都是宣传家。比如军事指挥员,他们并不对外发宣言,但是他们要和士兵讲话,要和人民接洽,这不是宣传是什么?一个人只要他对别人讲话,他就是在做宣传工作。只要他不是哑巴,他就总有几句话要讲的。所以我们的同志都非学习语言不可。\\
  党八股的第五条罪状是:甲乙丙丁,开中药铺。你们去看一看中药铺,那里的药柜子上有许多抽屉格子,每个格子上面贴着药名,当归、熟地、大黄、芒硝,应有尽有。这个方法,也被我们的同志学到了。写文章,做演说,着书,写报告,第一是大壹贰叁肆,第二是小一二三四,第三是甲乙丙丁,第四是子丑寅卯,还有大ABCD,小abcd,还有阿拉伯数字,多得很!幸亏古人和外国人替我们造好了这许多符号,使我们开起中药铺来毫不费力。一篇文章充满了这些符号,不提出问题,不分析问题,不解决问题,不表示赞成什么,反对什么,说来说去还是一个中药铺,没有什么真切的内容。我不是说甲乙丙丁等字不能用,而是说那种对待问题的方法不对。现在许多同志津津有味于这个开中药铺的方法,实在是一种最低级、最幼稚、最庸俗的方法。这种方法就是形式主义的方法,是按照事物的外部标志来分类,不是按照事物的内部联系来分类的。单单按照事物的外部标志,使用一大堆互相没有内部联系的概念,排列成一篇文章、一篇演说或一个报告,这种办法,他自己是在做概念的游戏,也会引导人家都做这类游戏,使人不用脑筋想问题,不去思考事物的本质,而满足于甲乙丙丁的现象罗列。什么叫问题?问题就是事物的矛盾。哪里有没有解决的矛盾,哪里就有问题。既有问题,你总得赞成一方面,反对另一方面,你就得把问题提出来。提出问题,首先就要对于问题即矛盾的两个基本方面加以大略的调查和研究,才能懂得矛盾的性质是什么,这就是发现问题的过程。大略的调查和研究可以发现问题,提出问题,但是还不能解决问题。要解决问题,还须作系统的周密的调查工作和研究工作,这就是分析的过程。提出问题也要用分析,不然,对着模糊杂乱的一大堆事物的现象,你就不能知道问题即矛盾的所在。这里所讲的分析过程,是指系统的周密的分析过程。常常问题是提出了,但还不能解决,就是因为还没有暴露事物的内部联系,就是因为还没有经过这种系统的周密的分析过程,因而问题的面貌还不明晰,还不能做综合工作,也就不能好好地解决问题。一篇文章或一篇演说,如果是重要的带指导性质的,总得要提出一个什么问题,接着加以分析,然后综合起来,指明问题的性质,给以解决的办法,这样,就不是形式主义的方法所能济事。因为这种幼稚的、低级的、庸俗的、不用脑筋的形式主义的方法,在我们党内很流行,所以必须揭破它,才能使大家学会应用马克思主义的方法去观察问题、提出问题、分析问题和解决问题,我们所办的事才能办好,我们的革命事业才能胜利。\\
  党八股的第六条罪状是:不负责任,到处害人。上面所说的那些,一方面是由于幼稚而来,另一方面也是由于责任心不足而来的。拿洗脸作比方,我们每天都要洗脸,许多人并且不止洗一次,洗完之后还要拿镜子照一照,要调查研究一番,(大笑)生怕有什么不妥当的地方。你们看,这是何等地有责任心呀!我们写文章,做演说,只要像洗脸这样负责,就差不多了。拿不出来的东西就不要拿出来。须知这是要去影响别人的思想和行动的啊!一个人偶然一天两天不洗脸,固然也不好,洗后脸上还留着一个两个黑点,固然也不雅观,但倒并没有什么大危险。写文章做演说就不同了,这是专为影响人的,我们的同志反而随随便便,这就叫做轻重倒置。许多人写文章,做演说,可以不要预先研究,不要预先准备;文章写好之后,也不多看几遍,像洗脸之后再照照镜子一样,就马马虎虎地发表出去。其结果,往往是“下笔千言,离题万里”,仿佛像个才子,实则到处害人。这种责任心薄弱的坏习惯,必须改正才好。\\
  第七条罪状是:流毒全党,妨害革命。第八条罪状是:传播出去,祸国殃民。这两条意义自明,无须多说。这就是说,党八股如不改革,如果听其发展下去,其结果之严重,可以闹到很坏的地步。党八股里面藏的是主观主义、宗派主义的毒物,这个毒物传播出去,是要害党害国的。\\
  上面这八条,就是我们申讨党八股的檄文。\\
  党八股这个形式,不但不便于表现革命精神,而且非常容易使革命精神窒息。要使革命精神获得发展,必须抛弃党八股,采取生动活泼新鲜有力的马克思列宁主义的文风。这种文风,早已存在,但尚未充实,尚未得到普遍的发展。我们破坏了洋八股和党八股之后,新的文风就可以获得充实,获得普遍的发展,党的革命事业,也就可以向前推进了。\\
  不但文章里演说里有党八股,开会也有的。“一开会,二报告,三讨论,四结论,五散会”。假使每处每回无大无小都要按照这个死板的程序,不也就是党八股吗?在会场上做起“报告”来,则常常就是“一国际,二国内,三边区,四本部”,会是常常从早上开到晚上,没有话讲的人也要讲一顿,不讲好像对人不起。总之,不看实际情形,死守着呆板的旧形式、旧习惯,这种现象,不是也应该加以改革吗?\\
  现在许多人在提倡民族化、科学化、大众化了,这很好。但是“化”者,彻头彻尾彻里彻外之谓也;有些人则连“少许”还没有实行,却在那里提倡“化”呢!所以我劝这些同志先办“少许”,再去办“化”,不然,仍旧脱离不了教条主义和党八股,这叫做眼高手低,志大才疏,没有结果的。例如那些口讲大众化而实是小众化的人,就很要当心,如果有一天大众中间有一个什么人在路上碰到他,对他说:“先生,请你化一下给我看。”就会将起军的。如果是不但口头上提倡提倡而且自己真想实行大众化的人,那就要实地跟老百姓去学,否则仍然“化”不了的。有些天天喊大众化的人,连三句老百姓的话都讲不来,可见他就没有下过决心跟老百姓学,实在他的意思仍是小众化。\\
  今天会场上散发了一个题名《宣传指南》的小册子,里面包含四篇文章,我劝同志们多看几遍。\\
  第一篇,是从《苏联共产党(布)历史简要读本》上摘下来的,讲的是列宁怎样做宣传。其中讲到列宁写传单的情形:“在列宁领导下,彼得堡‘工人阶级解放斗争协会’第一次在俄国开始把社会主义与工人运动结合起来。当某一个工厂里爆发罢工时,‘斗争协会’因为经过自己小组中的参加者而很熟悉各企业中的情形,立刻就印发传单、印发社会主义的宣言来响应。在这些传单里,揭露出厂主虐待工人的事实,说明工人应如何为自身的利益而奋斗,载明工人群众的要求。这些传单把资本主义机体上的痈疽,工人的穷困生活,工人每日由十二小时至十四小时的过度沉重的劳动,工人之毫无权利等等真情实况,都揭露无余。同时,在这些传单里,又提出了相当的政治要求。”\\
  是“很熟悉”啊!是“揭露无余”啊!\\
  “一八九四年末,列宁在工人巴布石金参加下,写了第一个这样的鼓动传单和告彼得堡城塞棉尼可夫工厂罢工工人书。”\\
  写一个传单要和熟悉情况的同志商量。列宁就是根据这样的调查和研究来写文章做工作的。\\
  “每一个这样的传单,都大大提高了工人们的精神。工人们看见了,社会主义者是帮助他们、保护他们的。”\footnote[9]{ 以上三段引文见《联共(布)党史简明教程》第一章第三节(人民出版社1975年版,第18—19页)。}\\
  我们是赞成列宁的吗?如果是的话,就得依照列宁的精神去工作。不是空话连篇,言之无物;不是无的放矢,不看对象;也不是自以为是,夸夸其谈;而是要照着列宁那样地去做。\\
  第二篇,是从季米特洛夫\footnote[10]{ 季米特洛夫(一八八二——一九四九),保加利亚人。一九二一年任工会国际中央理事会理事,一九三五年至一九四三年任共产国际执行委员会总书记。一九四五年十一月回国后,任保加利亚共产党总书记和部长会议主席。}在共产国际第七次大会的报告中摘下来的。季米特洛夫说了些什么呢?他说:“应当学会不用书本上的公式而用为群众事业而奋斗的战士们的语言来和群众讲话,这些战士们的每一句话,每一个思想,都反映出千百万群众的思想和情绪。”\\
  “如果我们没有学会说群众懂得的话,那末广大群众是不能领会我们的决议的。我们远不是随时都善于简单地、具体地、用群众所熟悉和懂得的形象来讲话。我们还没有能够抛弃背得烂熟的抽象的公式。事实上,你们只要瞧一瞧我们的传单、报纸、决议和提纲,就可以看到:这些东西常常是用这样的语言写成的,写得这样地艰深,甚至于我们党的干部都难于懂得,更用不着说普通工人了。”\\
  怎么样?这不是把我们的毛病讲得一针见血吗?不错,党八股中国有,外国也有,可见是通病。(笑)但是我们总得照着季米特洛夫同志的指示把我们自己的毛病赶快治好才行。\\
  “我们每一个人,都应当切实领会下面这条起码的规则,把它当作定律,当作布尔什维克的定律:当你写东西或讲话的时候,始终要想到使每个普通工人都懂得,都相信你的号召,都决心跟着你走。要想到你究竟为什么人写东西,向什么人讲话。”\footnote[11]{ 以上三段引文见季米特洛夫一九三五年八月十三日在共产国际第七次代表大会上所作的结论《为工人阶级团结一致反对法西斯主义而斗争》的序言和第六部分《仅仅只有正确的路线还是不够的》。}\\
  这就是共产国际给我们治病的药方,是必须遵守的。这是“规则”啊!\\
  第三篇,是从《鲁迅全集》里选出的,是鲁迅复北斗杂志\footnote[12]{ 《北斗》杂志是中国左翼作家联盟在一九三一年至一九三二年间出版的文艺月刊。《答北斗杂志社问》载鲁迅《二心集》。(《鲁迅全集》第4卷,人民文学出版社1981年版,第364—365页)}社讨论怎样写文章的一封信。他说些什么呢?他一共列举了八条写文章的规则,我现在抽出几条来说一说。\\
  第一条:“留心各样的事情,多看看,不看到一点就写。”\\
  讲的是“留心各样的事情”,不是一样半样的事情。讲的是“多看看”,不是只看一眼半眼。我们怎么样?不是恰恰和他相反,只看到一点就写吗?\\
  第二条:“写不出的时候不硬写。”\\
  我们怎么样?不是明明脑子里没有什么东西硬要大写特写吗?不调查,不研究,提起笔来“硬写”,这就是不负责任的态度。\\
  第四条:“写完后至少看两遍,竭力将可有可无的字、句、段删去,毫不可惜。宁可将可作小说的材料缩成速写,决不将速写材料拉成小说。”\\
  孔夫子提倡“再思”\footnote[13]{ 参见《论语•公冶长》。},韩愈也说“行成于思”\footnote[14]{ 韩愈(七六八——八二四),中国唐代著名的大作家。他在《进学解》一文中说:“行成于思,毁于随。”意思是:作事成功由于思考,失败由于不思考。},那是古代的事情。现在的事情,问题很复杂,有些事情甚至想三四回还不够。鲁迅说“至少看两遍”,至多呢?他没有说,我看重要的文章不妨看它十多遍,认真地加以删改,然后发表。文章是客观事物的反映,而事物是曲折复杂的,必须反复研究,才能反映恰当;在这里粗心大意,就是不懂得做文章的起码知识。\\
  第六条:“不生造除自己之外,谁也不懂的形容词之类。”\\
  我们“生造”的东西太多了,总之是“谁也不懂”。句法有长到四五十个字一句的,其中堆满了“谁也不懂的形容词之类”。许多口口声声拥护鲁迅的人们,却正是违背鲁迅的啊!\\
  最后一篇文章,是中国共产党六届六中全会论宣传的民族化。六届六中全会是一九三八年开的,我们那时曾说:“离开中国特点来谈马克思主义,只是抽象的空洞的马克思主义。”这就是说,必须反对空谈马克思主义;在中国生活的共产党员,必须联系中国的革命实际来研究马克思主义。\\
  “洋八股必须废止,空洞抽象的调头必须少唱,教条主义必须休息,而代之以新鲜活泼的、为中国老百姓所喜闻乐见的中国作风和中国气派。把国际主义的内容和民族形式分离起来,是一点也不懂国际主义的人们的做法,我们则要把二者紧密地结合起来。在这个问题上,我们队伍中存在着的一些严重的错误,是应该认真地克服的。”\footnote[15]{ 以上两段引文见《中国共产党在民族战争中的地位》(本书第2卷第534页)。}\\
  这里叫洋八股废止,有些同志却实际上还在提倡。这里叫空洞抽象的调头少唱,有些同志却硬要多唱。这里叫教条主义休息,有些同志却叫它起床。总之,有许多人把六中全会通过的报告当做耳边风,好像是故意和它作对似的。\\
  中央现在做了决定,一定要把党八股和教条主义等类,彻底抛弃,所以我来讲了许多。希望同志们把我所讲的加以考虑,加以分析,同时也分析各人自己的情况。每个人应该把自己好好地想一想,并且把自己想清楚了的东西,跟知心的朋友们商量一下,跟周围的同志们商量一下,把自己的毛病切实改掉。\\
\newpage\section*{\myformat{在延安文艺座谈会上的讲话}\\\myformat{(一九四二年五月)}}\addcontentsline{toc}{section}{在延安文艺座谈会上的讲话}
\subsection*{\myformat{引  言}\\\myformat{(一九四二年五月二日)}}
同志们!今天邀集大家来开座谈会,目的是要和大家交换意见,研究文艺工作和一般革命工作的关系,求得革命文艺的正确发展,求得革命文艺对其它革命工作的更好的协助,借以打倒我们民族的敌人,完成民族解放的任务。\\
  在我们为中国人民解放的斗争中,有各种的战线,就中也可以说有文武两个战线,这就是文化战线和军事战线。我们要战胜敌人,首先要依靠手里拿枪的军队。但是仅仅有这种军队是不够的,我们还要有文化的军队,这是团结自己、战胜敌人必不可少的一支军队。“五四”\footnote[1]{ 见本书第一卷《实践论》注〔6〕。}以来,这支文化军队就在中国形成,帮助了中国革命,使中国的封建文化和适应帝国主义侵略的买办文化的地盘逐渐缩小,其力量逐渐削弱。到了现在,中国反动派只能提出所谓“以数量对质量”的办法来和新文化对抗,就是说,反动派有的是钱,虽然拿不出好东西,但是可以拚命出得多。在“五四”以来的文化战线上,文学和艺术是一个重要的有成绩的部门。革命的文学艺术运动,在十年内战时期有了大的发展。这个运动和当时的革命战争,在总的方向上是一致的,但在实际工作上却没有互相结合起来,这是因为当时的反动派把这两支兄弟军队从中隔断了的缘故。抗日战争爆发以后,革命的文艺工作者来到延安和各个抗日根据地的多起来了,这是很好的事。但是到了根据地,并不是说就已经和根据地的人民群众完全结合了。我们要把革命工作向前推进,就要使这两者完全结合起来。我们今天开会,就是要使文艺很好地成为整个革命机器的一个组成部分,作为团结人民、教育人民、打击敌人、消灭敌人的有力的武器,帮助人民同心同德地和敌人作斗争。为了这个目的,有些什么问题应该解决的呢?我以为有这样一些问题,即文艺工作者的立场问题,态度问题,工作对象问题,工作问题和学习问题。\\
  立场问题。我们是站在无产阶级的和人民大众的立场。对于共产党员来说,也就是要站在党的立场,站在党性和党的政策的立场。在这个问题上,我们的文艺工作者中是否还有认识不正确或者认识不明确的呢?我看是有的。许多同志常常失掉了自己的正确的立场。\\
  态度问题。随着立场,就发生我们对于各种具体事物所采取的具体态度。比如说,歌颂呢,还是暴露呢?这就是态度问题。究竟哪种态度是我们需要的?我说两种都需要,问题是在对什么人。有三种人,一种是敌人,一种是统一战线中的同盟者,一种是自己人,这第三种人就是人民群众及其先锋队。对于这三种人需要有三种态度。对于敌人,对于日本帝国主义和一切人民的敌人,革命文艺工作者的任务是在暴露他们的残暴和欺骗,并指出他们必然要失败的趋势,鼓励抗日军民同心同德,坚决地打倒他们。对于统一战线中各种不同的同盟者,我们的态度应该是有联合,有批评,有各种不同的联合,有各种不同的批评。他们的抗战,我们是赞成的;如果有成绩,我们也是赞扬的。但是如果抗战不积极,我们就应该批评。如果有人要反共反人民,要一天一天走上反动的道路,那我们就要坚决反对。至于对人民群众,对人民的劳动和斗争,对人民的军队,人民的政党,我们当然应该赞扬。人民也有缺点的。无产阶级中还有许多人保留着小资产阶级的思想,农民和城市小资产阶级都有落后的思想,这些就是他们在斗争中的负担。我们应该长期地耐心地教育他们,帮助他们摆脱背上的包袱,同自己的缺点错误作斗争,使他们能够大踏步地前进。他们在斗争中已经改造或正在改造自己,我们的文艺应该描写他们的这个改造过程。只要不是坚持错误的人,我们就不应该只看到片面就去错误地讥笑他们,甚至敌视他们。我们所写的东西,应该是使他们团结,使他们进步,使他们同心同德,向前奋斗,去掉落后的东西,发扬革命的东西,而决不是相反。\\
  工作对象问题,就是文艺作品给谁看的问题。在陕甘宁边区,在华北华中各抗日根据地,这个问题和在国民党统治区不同,和在抗战以前的上海更不同。在上海时期,革命文艺作品的接受者是以一部分学生、职员、店员为主。在抗战以后的国民党统治区,范围曾有过一些扩大,但基本上也还是以这些人为主,因为那里的政府把工农兵和革命文艺互相隔绝了。在我们的根据地就完全不同。文艺作品在根据地的接受者,是工农兵以及革命的干部。根据地也有学生,但这些学生和旧式学生也不相同,他们不是过去的干部,就是未来的干部。各种干部,部队的战士,工厂的工人,农村的农民,他们识了字,就要看书、看报,不识字的,也要看戏、看画、唱歌、听音乐,他们就是我们文艺作品的接受者。即拿干部说,你们不要以为这部分人数目少,这比在国民党统治区出一本书的读者多得多。在那里,一本书一版平常只有两千册,三版也才六千册;但是根据地的干部,单是在延安能看书的就有一万多。而且这些干部许多都是久经锻炼的革命家,他们是从全国各地来的,他们也要到各地去工作,所以对于这些人做教育工作,是有重大意义的。我们的文艺工作者,应该向他们好好做工作。\\
  既然文艺工作的对象是工农兵及其干部,就发生一个了解他们熟悉他们的问题。而为要了解他们,熟悉他们,为要在党政机关,在农村,在工厂,在八路军新四军里面,了解各种人,熟悉各种人,了解各种事情,熟悉各种事情,就需要做很多的工作。我们的文艺工作者需要做自己的文艺工作,但是这个了解人熟悉人的工作却是第一位的工作。我们的文艺工作者对于这些,以前是一种什么情形呢?我说以前是不熟,不懂,英雄无用武之地。什么是不熟?人不熟。文艺工作者同自己的描写对象和作品接受者不熟,或者简直生疏得很。我们的文艺工作者不熟悉工人,不熟悉农民,不熟悉士兵,也不熟悉他们的干部。什么是不懂?语言不懂,就是说,对于人民群众的丰富的生动的语言,缺乏充分的知识。许多文艺工作者由于自己脱离群众、生活空虚,当然也就不熟悉人民的语言,因此他们的作品不但显得语言无味,而且里面常常夹着一些生造出来的和人民的语言相对立的不三不四的词句。许多同志爱说“大众化”,但是什么叫做大众化呢?就是我们的文艺工作者的思想感情和工农兵大众的思想感情打成一片。而要打成一片,就应当认真学习群众的语言。如果连群众的语言都有许多不懂,还讲什么文艺创造呢?英雄无用武之地,就是说,你的一套大道理,群众不赏识。在群众面前把你的资格摆得越老,越像个“英雄”,越要出卖这一套,群众就越不买你的账。你要群众了解你,你要和群众打成一片,就得下决心,经过长期的甚至是痛苦的磨练。在这里,我可以说一说我自己感情变化的经验。我是个学生出身的人,在学校养成了一种学生习惯,在一大群肩不能挑手不能提的学生面前做一点劳动的事,比如自己挑行李吧,也觉得不像样子。那时,我觉得世界上干净的人只有知识分子,工人农民总是比较脏的。知识分子的衣服,别人的我可以穿,以为是干净的;工人农民的衣服,我就不愿意穿,以为是脏的。革命了,同工人农民和革命军的战士在一起了,我逐渐熟悉他们,他们也逐渐熟悉了我。这时,只是在这时,我才根本地改变了资产阶级学校所教给我的那种资产阶级的和小资产阶级的感情。这时,拿未曾改造的知识分子和工人农民比较,就觉得知识分子不干净了,最干净的还是工人农民,尽管他们手是黑的,脚上有牛屎,还是比资产阶级和小资产阶级知识分子都干净。这就叫做感情起了变化,由一个阶级变到另一个阶级。我们知识分子出身的文艺工作者,要使自己的作品为群众所欢迎,就得把自己的思想感情来一个变化,来一番改造。没有这个变化,没有这个改造,什么事情都是做不好的,都是格格不入的。\\
  最后一个问题是学习,我的意思是说学习马克思列宁主义和学习社会。一个自命为马克思主义的革命作家,尤其是党员作家,必须有马克思列宁主义的知识。但是现在有些同志,却缺少马克思主义的基本观点。比如说,马克思主义的一个基本观点,就是存在决定意识,就是阶级斗争和民族斗争的客观现实决定我们的思想感情。但是我们有些同志却把这个问题弄颠倒了,说什么一切应该从“爱”出发。就说爱吧,在阶级社会里,也只有阶级的爱,但是这些同志却要追求什么超阶级的爱,抽象的爱,以及抽象的自由、抽象的真理、抽象的人性等等。这是表明这些同志是受了资产阶级的很深的影响。应该很彻底地清算这种影响,很虚心地学习马克思列宁主义。文艺工作者应该学习文艺创作,这是对的,但是马克思列宁主义是一切革命者都应该学习的科学,文艺工作者不能是例外。文艺工作者要学习社会,这就是说,要研究社会上的各个阶级,研究它们的相互关系和各自状况,研究它们的面貌和它们的心理。只有把这些弄清楚了,我们的文艺才能有丰富的内容和正确的方向。\\
  今天我就只提出这几个问题,当作引子,希望大家在这些问题及其它有关的问题上发表意见。\\
\subsection*{\myformat{结  论}\\\myformat{(一九四二年五月二十三日)}}
同志们!我们这个会在一个月里开了三次。大家为了追求真理,进行了热烈的争论,有党的和非党的同志几十个人讲了话,把问题展开了,并且具体化了。我认为这是对整个文学艺术运动很有益处的。\\
  我们讨论问题,应当从实际出发,不是从定义出发。如果我们按照教科书,找到什么是文学、什么是艺术的定义,然后按照它们来规定今天文艺运动的方针,来评判今天所发生的各种见解和争论,这种方法是不正确的。我们是马克思主义者,马克思主义叫我们看问题不要从抽象的定义出发,而要从客观存在的事实出发,从分析这些事实中找出方针、政策、办法来。我们现在讨论文艺工作,也应该这样做。\\
  现在的事实是什么呢?事实就是:中国的已经进行了五年的抗日战争;全世界的反法西斯战争;中国大地主大资产阶级在抗日战争中的动摇和对于人民的高压政策;“五四”以来的革命文艺运动——这个运动在二十三年中对于革命的伟大贡献以及它的许多缺点;八路军新四军的抗日民主根据地,在这些根据地里面大批文艺工作者和八路军新四军以及工人农民的结合;根据地的文艺工作者和国民党统治区的文艺工作者的环境和任务的区别;目前在延安和各抗日根据地的文艺工作中已经发生的争论问题。——这些就是实际存在的不可否认的事实,我们就要在这些事实的基础上考虑我们的问题。\\
  那末,什么是我们的问题的中心呢?我以为,我们的问题基本上是一个为群众的问题和一个如何为群众的问题。不解决这两个问题,或这两个问题解决得不适当,就会使得我们的文艺工作者和自己的环境、任务不协调,就使得我们的文艺工作者从外部从内部碰到一连串的问题。我的结论,就以这两个问题为中心,同时也讲到一些与此有关的其它问题。\\
\subsubsection*{\myformat{一}}
第一个问题:我们的文艺是为什么人的?\\
  这个问题,本来是马克思主义者特别是列宁所早已解决了的。列宁还在一九〇五年就已着重指出过,我们的文艺应当“为千千万万劳动人民服务”\footnote[2]{ 见列宁《党的组织和党的出版物》。列宁在这篇论文中说:“这将是自由的写作,因为把一批又一批新生力量吸引到写作队伍中来的,不是私利贪欲,也不是名誉地位,而是社会主义思想和对劳动人民的同情。这将是自由的写作,因为它不是为饱食终日的贵妇人服务,不是为百无聊赖、胖得发愁的‘一万个上层分子’服务,而是为千千万万劳动人民,为这些国家的精华、国家的力量、国家的未来服务。这将是自由的写作,它要用社会主义无产阶级的经验和生气勃勃的工作去丰富人类革命思想的最新成就,它要使过去的经验(从原始空想的社会主义发展而成的科学社会主义)和现在的经验(工人同志们当前的斗争)之间经常发生相互作用。”(《列宁全集》第12卷,人民出版社1987年版,第96—97页)}。在我们各个抗日根据地从事文学艺术工作的同志中,这个问题似乎是已经解决了,不需要再讲的了。其实不然。很多同志对这个问题并没有得到明确的解决。因此,在他们的情绪中,在他们的作品中,在他们的行动中,在他们对于文艺方针问题的意见中,就不免或多或少地发生和群众的需要不相符合,和实际斗争的需要不相符合的情形。当然,现在和共产党、八路军、新四军在一起从事于伟大解放斗争的大批的文化人、文学家、艺术家以及一般文艺工作者,虽然其中也可能有些人是暂时的投机分子,但是绝大多数却都是在为着共同事业努力工作着。依靠这些同志,我们的整个文学工作,戏剧工作,音乐工作,美术工作,都有了很大的成绩。这些文艺工作者,有许多是抗战以后开始工作的;有许多在抗战以前就做了多时的革命工作,经历过许多辛苦,并用他们的工作和作品影响了广大群众的。但是为什么还说即使这些同志中也有对于文艺是为什么人的问题没有明确解决的呢?难道他们还有主张革命文艺不是为着人民大众而是为着剥削者压迫者的吗?\\
  诚然,为着剥削者压迫者的文艺是有的。文艺是为地主阶级的,这是封建主义的文艺。中国封建时代统治阶级的文学艺术,就是这种东西。直到今天,这种文艺在中国还有颇大的势力。文艺是为资产阶级的,这是资产阶级的文艺。像鲁迅所批评的梁实秋\footnote[3]{ 梁实秋(一九〇三——一九八七),北京人。新月社主要成员。先后在复旦大学、北京大学等校任教。曾写过一些文艺评论,长时期致力于文学翻译工作和散文的写作。鲁迅对梁实秋的批评,见《三闲集•新月社批评家的任务》、《二心集•“硬译”与“文学的阶级性”》等文。(《鲁迅全集》第4卷,人民文学出版社1981年版,第159、195—212页)}一类人,他们虽然在口头上提出什么文艺是超阶级的,但是他们在实际上是主张资产阶级的文艺,反对无产阶级的文艺的。文艺是为帝国主义者的,周作人、张资平\footnote[4]{ 周作人(一八八五——一九六七),浙江绍兴人。曾在北京大学、燕京大学等校任教。五四运动时从事新文学写作。他的著述很多,有大量的散文集、文学专着和翻译作品。张资平(一八九三——一九五九),广东梅县人。他写过很多小说,曾在暨南大学、大夏大学兼任教职。周作人、张资平于一九三八年和一九三九年先后在北平、上海依附侵略中国的日本占领者。}这批人就是这样,这叫做汉奸文艺。在我们,文艺不是为上述种种人,而是为人民的。我们曾说,现阶段的中国新文化,是无产阶级领导的人民大众的反帝反封建的文化。真正人民大众的东西,现在一定是无产阶级领导的。资产阶级领导的东西,不可能属于人民大众。新文化中的新文学新艺术,自然也是这样。对于中国和外国过去时代所遗留下来的丰富的文学艺术遗产和优良的文学艺术传统,我们是要继承的,但是目的仍然是为了人民大众。对于过去时代的文艺形式,我们也并不拒绝利用,但这些旧形式到了我们手里,给了改造,加进了新内容,也就变成革命的为人民服务的东西了。\\
  那末,什么是人民大众呢?最广大的人民,占全人口百分之九十以上的人民,是工人、农民、兵士和城市小资产阶级。所以我们的文艺,第一是为工人的,这是领导革命的阶级。第二是为农民的,他们是革命中最广大最坚决的同盟军。第三是为武装起来了的工人农民即八路军、新四军和其它人民武装队伍的,这是革命战争的主力。第四是为城市小资产阶级劳动群众和知识分子的,他们也是革命的同盟者,他们是能够长期地和我们合作的。这四种人,就是中华民族的最大部分,就是最广大的人民大众。\\
  我们的文艺,应该为着上面说的四种人。我们要为这四种人服务,就必须站在无产阶级的立场上,而不能站在小资产阶级的立场上。在今天,坚持个人主义的小资产阶级立场的作家是不可能真正地为革命的工农兵群众服务的,他们的兴趣,主要是放在少数小资产阶级知识分子上面。而我们现在有一部分同志对于文艺为什么人的问题不能正确解决的关键,正在这里。我这样说,不是说在理论上。在理论上,或者说在口头上,我们队伍中没有一个人把工农兵群众看得比小资产阶级知识分子还不重要的。我是说在实际上,在行动上。在实际上,在行动上,他们是否对小资产阶级知识分子比对工农兵还更看得重要些呢?我以为是这样。有许多同志比较地注重研究小资产阶级知识分子,分析他们的心理,着重地去表现他们,原谅并辩护他们的缺点,而不是引导他们和自己一道去接近工农兵群众,去参加工农兵群众的实际斗争,去表现工农兵群众,去教育工农兵群众。有许多同志,因为他们自己是从小资产阶级出身,自己是知识分子,于是就只在知识分子的队伍中找朋友,把自己的注意力放在研究和描写知识分子上面。这种研究和描写如果是站在无产阶级立场上的,那是应该的。但他们并不是,或者不完全是。他们是站在小资产阶级立场,他们是把自己的作品当作小资产阶级的自我表现来创作的,我们在相当多的文学艺术作品中看见这种东西。他们在许多时候,对于小资产阶级出身的知识分子寄予满腔的同情,连他们的缺点也给以同情甚至鼓吹。对于工农兵群众,则缺乏接近,缺乏了解,缺乏研究,缺乏知心朋友,不善于描写他们;倘若描写,也是衣服是劳动人民,面孔却是小资产阶级知识分子。他们在某些方面也爱工农兵,也爱工农兵出身的干部,但有些时候不爱,有些地方不爱,不爱他们的感情,不爱他们的姿态,不爱他们的萌芽状态的文艺(墙报、壁画、民歌、民间故事等)。他们有时也爱这些东西,那是为着猎奇,为着装饰自己的作品,甚至是为着追求其中落后的东西而爱的。有时就公开地鄙弃它们,而偏爱小资产阶级知识分子的乃至资产阶级的东西。这些同志的立足点还是在小资产阶级知识分子方面,或者换句文雅的话说,他们的灵魂深处还是一个小资产阶级知识分子的王国。这样,为什么人的问题他们就还是没有解决,或者没有明确地解决。这不光是讲初来延安不久的人,就是到过前方,在根据地、八路军、新四军做过几年工作的人,也有许多是没有彻底解决的。要彻底地解决这个问题,非有十年八年的长时间不可。但是时间无论怎样长,我们却必须解决它,必须明确地彻底地解决它。我们的文艺工作者一定要完成这个任务,一定要把立足点移过来,一定要在深入工农兵群众、深入实际斗争的过程中,在学习马克思主义和学习社会的过程中,逐渐地移过来,移到工农兵这方面来,移到无产阶级这方面来。只有这样,我们才能有真正为工农兵的文艺,真正无产阶级的文艺。\\
  为什么人的问题,是一个根本的问题,原则的问题。过去有些同志间的争论、分歧、对立和不团结,并不是在这个根本的原则的问题上,而是在一些比较次要的甚至是无原则的问题上。而对于这个原则问题,争论的双方倒是没有什么分歧,倒是几乎一致的,都有某种程度的轻视工农兵、脱离群众的倾向。我说某种程度,因为一般地说,这些同志的轻视工农兵、脱离群众,和国民党的轻视工农兵、脱离群众,是不同的;但是无论如何,这个倾向是有的。这个根本问题不解决,其它许多问题也就不易解决。比如说文艺界的宗派主义吧,这也是原则问题,但是要去掉宗派主义,也只有把为工农,为八路军、新四军,到群众中去的口号提出来,并加以切实的实行,才能达到目的,否则宗派主义问题是断然不能解决的。鲁迅曾说:“联合战线是以有共同目的为必要条件的。……我们战线不能统一,就证明我们的目的不能一致,或者只为了小团体,或者还其实只为了个人。如果目的都在工农大众,那当然战线也就统一了。”\footnote[5]{ 见鲁迅《二心集•对于左翼作家联盟的意见》(《鲁迅全集》第4卷,人民文学出版社1981年版,第237—238页)。}这个问题那时上海有,现在重庆也有。在那些地方,这个问题很难彻底解决,因为那些地方的统治者压迫革命文艺家,不让他们有到工农兵群众中去的自由。在我们这里,情形就完全两样。我们鼓励革命文艺家积极地亲近工农兵,给他们以到群众中去的完全自由,给他们以创作真正革命文艺的完全自由。所以这个问题在我们这里,是接近于解决的了。接近于解决不等于完全的彻底的解决;我们说要学习马克思主义和学习社会,就是为着完全地彻底地解决这个问题。我们说的马克思主义,是要在群众生活群众斗争里实际发生作用的活的马克思主义,不是口头上的马克思主义。把口头上的马克思主义变成为实际生活里的马克思主义,就不会有宗派主义了。不但宗派主义的问题可以解决,其它的许多问题也都可以解决了。\\
\subsubsection*{\myformat{二}}
为什么人服务的问题解决了,接着的问题就是如何去服务。用同志们的话来说,就是:努力于提高呢,还是努力于普及呢?\\
  有些同志,在过去,是相当地或是严重地轻视了和忽视了普及,他们不适当地太强调了提高。提高是应该强调的,但是片面地孤立地强调提高,强调到不适当的程度,那就错了。我在前面说的没有明确地解决为什么人的问题的事实,在这一点上也表现出来了。并且,因为没有弄清楚为什么人,他们所说的普及和提高就都没有正确的标准,当然更找不到两者的正确关系。我们的文艺,既然基本上是为工农兵,那末所谓普及,也就是向工农兵普及,所谓提高,也就是从工农兵提高。用什么东西向他们普及呢?用封建地主阶级所需要、所便于接受的东西吗?用资产阶级所需要、所便于接受的东西吗?用小资产阶级知识分子所需要、所便于接受的东西吗?都不行,只有用工农兵自己所需要、所便于接受的东西。因此在教育工农兵的任务之前,就先有一个学习工农兵的任务。提高的问题更是如此。提高要有一个基础。比如一桶水,不是从地上去提高,难道是从空中去提高吗?那末所谓文艺的提高,是从什么基础上去提高呢?从封建阶级的基础吗?从资产阶级的基础吗?从小资产阶级知识分子的基础吗?都不是,只能是从工农兵群众的基础上去提高。也不是把工农兵提到封建阶级、资产阶级、小资产阶级知识分子的“高度”去,而是沿着工农兵自己前进的方向去提高,沿着无产阶级前进的方向去提高。而这里也就提出了学习工农兵的任务。只有从工农兵出发,我们对于普及和提高才能有正确的了解,也才能找到普及和提高的正确关系。\\
  一切种类的文学艺术的源泉究竟是从何而来的呢?作为观念形态的文艺作品,都是一定的社会生活在人类头脑中的反映的产物。革命的文艺,则是人民生活在革命作家头脑中的反映的产物。人民生活中本来存在着文学艺术原料的矿藏,这是自然形态的东西,是粗糙的东西,但也是最生动、最丰富、最基本的东西;在这点上说,它们使一切文学艺术相形见绌,它们是一切文学艺术的取之不尽、用之不竭的唯一的源泉。这是唯一的源泉,因为只能有这样的源泉,此外不能有第二个源泉。有人说,书本上的文艺作品,古代的和外国的文艺作品,不也是源泉吗?实际上,过去的文艺作品不是源而是流,是古人和外国人根据他们彼时彼地所得到的人民生活中的文学艺术原料创造出来的东西。我们必须继承一切优秀的文学艺术遗产,批判地吸收其中一切有益的东西,作为我们从此时此地的人民生活中的文学艺术原料创造作品时候的借鉴。有这个借鉴和没有这个借鉴是不同的,这里有文野之分,粗细之分,高低之分,快慢之分。所以我们决不可拒绝继承和借鉴古人和外国人,哪怕是封建阶级和资产阶级的东西。但是继承和借鉴决不可以变成替代自己的创造,这是决不能替代的。文学艺术中对于古人和外国人的毫无批判的硬搬和模仿,乃是最没有出息的最害人的文学教条主义和艺术教条主义。中国的革命的文学家艺术家,有出息的文学家艺术家,必须到群众中去,必须长期地无条件地全心全意地到工农兵群众中去,到火热的斗争中去,到唯一的最广大最丰富的源泉中去,观察、体验、研究、分析一切人,一切阶级,一切群众,一切生动的生活形式和斗争形式,一切文学和艺术的原始材料,然后才有可能进入创作过程。否则你的劳动就没有对象,你就只能做鲁迅在他的遗嘱里所谆谆嘱咐他的儿子万不可做的那种空头文学家,或空头艺术家\footnote[6]{ 参见鲁迅《且介亭杂文末编•附集•死》(《鲁迅全集》第6卷,人民文学出版社1981年版,第612页)。}。\\
  人类的社会生活虽是文学艺术的唯一源泉,虽是较之后者有不可比拟的生动丰富的内容,但是人民还是不满足于前者而要求后者。这是为什么呢?因为虽然两者都是美,但是文艺作品中反映出来的生活却可以而且应该比普通的实际生活更高,更强烈,更有集中性,更典型,更理想,因此就更带普遍性。革命的文艺,应当根据实际生活创造出各种各样的人物来,帮助群众推动历史的前进。例如一方面是人们受饿、受冻、受压迫,一方面是人剥削人、人压迫人,这个事实到处存在着,人们也看得很平淡;文艺就把这种日常的现象集中起来,把其中的矛盾和斗争典型化,造成文学作品或艺术作品,就能使人民群众惊醒起来,感奋起来,推动人民群众走向团结和斗争,实行改造自己的环境。如果没有这样的文艺,那末这个任务就不能完成,或者不能有力地迅速地完成。\\
  什么是文艺工作中的普及和提高呢?这两种任务的关系是怎样的呢?普及的东西比较简单浅显,因此也比较容易为目前广大人民群众所迅速接受。高级的作品比较细致,因此也比较难于生产,并且往往比较难于在目前广大人民群众中迅速流传。现在工农兵面前的问题,是他们正在和敌人作残酷的流血斗争,而他们由于长时期的封建阶级和资产阶级的统治,不识字,无文化,所以他们迫切要求一个普遍的启蒙运动,迫切要求得到他们所急需的和容易接受的文化知识和文艺作品,去提高他们的斗争热情和胜利信心,加强他们的团结,便于他们同心同德地去和敌人作斗争。对于他们,第一步需要还不是“锦上添花”,而是“雪中送炭”。所以在目前条件下,普及工作的任务更为迫切。轻视和忽视普及工作的态度是错误的。\\
  但是,普及工作和提高工作是不能截然分开的。不但一部分优秀的作品现在也有普及的可能,而且广大群众的文化水平也是在不断地提高着。普及工作若是永远停止在一个水平上,一月两月三月,一年两年三年,总是一样的货色,一样的“小放牛”\footnote[7]{ “小放牛”是中国一出传统的小歌舞剧。全剧只有两个角色,男角是牧童,女角是乡村小姑娘,以互相对唱的方式表现剧的内容。抗日战争初期,革命的文艺工作者利用这个歌舞剧的形式,变动其原来的词句,宣传抗日,一时颇为流行。},一样的“人、手、口、刀、牛、羊”\footnote[8]{ “人、手、口、刀、牛、羊”是笔画比较简单的汉字,旧时一些小学国语读本把这几个字编在第一册的最初几课里。},那末,教育者和被教育者岂不都是半斤八两?这种普及工作还有什么意义呢?人民要求普及,跟着也就要求提高,要求逐年逐月地提高。在这里,普及是人民的普及,提高也是人民的提高。而这种提高,不是从空中提高,不是关门提高,而是在普及基础上的提高。这种提高,为普及所决定,同时又给普及以指导。就中国范围来说,革命和革命文化的发展不是平衡的,而是逐渐推广的。一处普及了,并且在普及的基础上提高了,别处还没有开始普及。因此一处由普及而提高的好经验可以应用于别处,使别处的普及工作和提高工作得到指导,少走许多弯路。就国际范围来说,外国的好经验,尤其是苏联的经验,也有指导我们的作用。所以,我们的提高,是在普及基础上的提高;我们的普及,是在提高指导下的普及。正因为这样,我们所说的普及工作不但不是妨碍提高,而且是给目前的范围有限的提高工作以基础,也是给将来的范围大为广阔的提高工作准备必要的条件。\\
  除了直接为群众所需要的提高以外,还有一种间接为群众所需要的提高,这就是干部所需要的提高。干部是群众中的先进分子,他们所受的教育一般都比群众所受的多些;比较高级的文学艺术,对于他们是完全必要的,忽视这一点是错误的。为干部,也完全是为群众,因为只有经过干部才能去教育群众、指导群众。如果违背了这个目的,如果我们给予干部的并不能帮助干部去教育群众、指导群众,那末,我们的提高工作就是无的放矢,就是离开了为人民大众的根本原则。\\
  总起来说,人民生活中的文学艺术的原料,经过革命作家的创造性的劳动而形成观念形态上的为人民大众的文学艺术。在这中间,既有从初级的文艺基础上发展起来的、为被提高了的群众所需要、或首先为群众中的干部所需要的高级的文艺,又有反转来在这种高级的文艺指导之下的、往往为今日最广大群众所最先需要的初级的文艺。无论高级的或初级的,我们的文学艺术都是为人民大众的,首先是为工农兵的,为工农兵而创作,为工农兵所利用的。\\
  我们既然解决了提高和普及的关系问题,则专门家和普及工作者的关系问题也就可以随着解决了。我们的专门家不但是为了干部,主要地还是为了群众。我们的文学专门家应该注意群众的墙报,注意军队和农村中的通讯文学。我们的戏剧专门家应该注意军队和农村中的小剧团。我们的音乐专门家应该注意群众的歌唱。我们的美术专门家应该注意群众的美术。一切这些同志都应该和在群众中做文艺普及工作的同志们发生密切的联系,一方面帮助他们,指导他们,一方面又向他们学习,从他们吸收由群众中来的养料,把自己充实起来,丰富起来,使自己的专门不致成为脱离群众、脱离实际、毫无内容、毫无生气的空中楼阁。我们应该尊重专门家,专门家对于我们的事业是很可宝贵的。但是我们应该告诉他们说,一切革命的文学家艺术家只有联系群众,表现群众,把自己当作群众的忠实的代言人,他们的工作才有意义。只有代表群众才能教育群众,只有做群众的学生才能做群众的先生。如果把自己看作群众的主人,看作高踞于“下等人”头上的贵族,那末,不管他们有多大的才能,也是群众所不需要的,他们的工作是没有前途的。\\
  我们的这种态度是不是功利主义的?唯物主义者并不一般地反对功利主义,但是反对封建阶级的、资产阶级的、小资产阶级的功利主义,反对那种口头上反对功利主义、实际上抱着最自私最短视的功利主义的伪善者。世界上没有什么超功利主义,在阶级社会里,不是这一阶级的功利主义,就是那一阶级的功利主义。我们是无产阶级的革命的功利主义者,我们是以占全人口百分之九十以上的最广大群众的目前利益和将来利益的统一为出发点的,所以我们是以最广和最远为目标的革命的功利主义者,而不是只看到局部和目前的狭隘的功利主义者。例如,某种作品,只为少数人所偏爱,而为多数人所不需要,甚至对多数人有害,硬要拿来上市,拿来向群众宣传,以求其个人的或狭隘集团的功利,还要责备群众的功利主义,这就不但侮辱群众,也太无自知之明了。任何一种东西,必须能使人民群众得到真实的利益,才是好的东西。就算你的是“阳春白雪”吧,这暂时既然是少数人享用的东西,群众还是在那里唱“下里巴人”,那末,你不去提高它,只顾骂人,那就怎样骂也是空的。现在是“阳春白雪”和“下里巴人”\footnote[9]{ “阳春白雪”和“下里巴人”,都是公元前三世纪楚国的歌曲。“阳春白雪”是供少数人欣赏的较高级的歌曲;“下里巴人”是流传很广的民间歌曲。《文选•宋玉对楚王问》记载一个故事,说有人在楚都唱歌,唱“阳春白雪”时,“国中属而和者(跟着唱的),不过数十人”;但唱“下里巴人”时,“国中属而和者数千人”。}统一的问题,是提高和普及统一的问题。不统一,任何专门家的最高级的艺术也不免成为最狭隘的功利主义;要说这也是清高,那只是自封为清高,群众是不会批准的。\\
  在为工农兵和怎样为工农兵的基本方针问题解决之后,其它的问题,例如,写光明和写黑暗的问题,团结问题等,便都一齐解决了。如果大家同意这个基本方针,则我们的文学艺术工作者,我们的文学艺术学校,文学艺术刊物,文学艺术团体和一切文学艺术活动,就应该依照这个方针去做。离开这个方针就是错误的;和这个方针有些不相符合的,就须加以适当的修正。\\
\subsubsection*{\myformat{三}}
我们的文艺既然是为人民大众的,那末,我们就可以进而讨论一个党内关系问题,党的文艺工作和党的整个工作的关系问题,和另一个党外关系的问题,党的文艺工作和非党的文艺工作的关系问题——文艺界统一战线问题。\\
  先说第一个问题。在现在世界上,一切文化或文学艺术都是属于一定的阶级,属于一定的政治路线的。为艺术的艺术,超阶级的艺术,和政治并行或互相独立的艺术,实际上是不存在的。无产阶级的文学艺术是无产阶级整个革命事业的一部分,如同列宁所说,是整个革命机器中的“齿轮和螺丝钉”\footnote[10]{ 见列宁《党的组织和党的出版物》。列宁在这篇论文中说:“写作事业应当成为整个无产阶级事业的一部分,成为由整个工人阶级的整个觉悟的先锋队所开动的一部巨大的社会民主主义机器的‘齿轮和螺丝钉’。”(《列宁全集》第12卷,人民出版社1987年版,第93页)}。因此,党的文艺工作,在党的整个革命工作中的位置,是确定了的,摆好了的;是服从党在一定革命时期内所规定的革命任务的。反对这种摆法,一定要走到二元论或多元论,而其实质就像托洛茨基那样:“政治——马克思主义的;艺术——资产阶级的。”我们不赞成把文艺的重要性过分强调到错误的程度,但也不赞成把文艺的重要性估计不足。文艺是从属于政治的,但又反转来给予伟大的影响于政治。革命文艺是整个革命事业的一部分,是齿轮和螺丝钉,和别的更重要的部分比较起来,自然有轻重缓急第一第二之分,但它是对于整个机器不可缺少的齿轮和螺丝钉,对于整个革命事业不可缺少的一部分。如果连最广义最普通的文学艺术也没有,那革命运动就不能进行,就不能胜利。不认识这一点,是不对的。还有,我们所说的文艺服从于政治,这政治是指阶级的政治、群众的政治,不是所谓少数政治家的政治。政治,不论革命的和反革命的,都是阶级对阶级的斗争,不是少数个人的行为。革命的思想斗争和艺术斗争,必须服从于政治的斗争,因为只有经过政治,阶级和群众的需要才能集中地表现出来。革命的政治家们,懂得革命的政治科学或政治艺术的政治专门家们,他们只是千千万万的群众政治家的领袖,他们的任务在于把群众政治家的意见集中起来,加以提炼,再使之回到群众中去,为群众所接受,所实践,而不是闭门造车,自作聪明,只此一家,别无分店的那种贵族式的所谓“政治家”,——这是无产阶级政治家同腐朽了的资产阶级政治家的原则区别。正因为这样,我们的文艺的政治性和真实性才能够完全一致。不认识这一点,把无产阶级的政治和政治家庸俗化,是不对的。\\
  再说文艺界的统一战线问题。文艺服从于政治,今天中国政治的第一个根本问题是抗日,因此党的文艺工作者首先应该在抗日这一点上和党外的一切文学家艺术家(从党的同情分子、小资产阶级的文艺家到一切赞成抗日的资产阶级地主阶级的文艺家)团结起来。其次,应该在民主一点上团结起来;在这一点上,有一部分抗日的文艺家就不赞成,因此团结的范围就不免要小一些。再其次,应该在文艺界的特殊问题——艺术方法艺术作风一点上团结起来;我们是主张社会主义的现实主义的,又有一部分人不赞成,这个团结的范围会更小些。在一个问题上有团结,在另一个问题上就有斗争,有批评。各个问题是彼此分开而又联系着的,因而就在产生团结的问题比如抗日的问题上也同时有斗争,有批评。在一个统一战线里面,只有团结而无斗争,或者只有斗争而无团结,实行如过去某些同志所实行过的右倾的投降主义、尾巴主义,或者“左”倾的排外主义、宗派主义,都是错误的政策。政治上如此,艺术上也是如此。\\
  在文艺界统一战线的各种力量里面,小资产阶级文艺家在中国是一个重要的力量。他们的思想和作品都有很多缺点,但是他们比较地倾向于革命,比较地接近于劳动人民。因此,帮助他们克服缺点,争取他们到为劳动人民服务的战线上来,是一个特别重要的任务。\\
\subsubsection*{\myformat{四}}
文艺界的主要的斗争方法之一,是文艺批评。文艺批评应该发展,过去在这方面工作做得很不够,同志们指出这一点是对的。文艺批评是一个复杂的问题,需要许多专门的研究。我这里只着重谈一个基本的批评标准问题。此外,对于有些同志所提出的一些个别的问题和一些不正确的观点,也来略为说一说我的意见。\\
  文艺批评有两个标准,一个是政治标准,一个是艺术标准。按照政治标准来说,一切利于抗日和团结的,鼓励群众同心同德的,反对倒退、促成进步的东西,便都是好的;而一切不利于抗日和团结的,鼓动群众离心离德的,反对进步、拉着人们倒退的东西,便都是坏的。这里所说的好坏,究竟是看动机(主观愿望),还是看效果(社会实践)呢?唯心论者是强调动机否认效果的,机械唯物论者是强调效果否认动机的,我们和这两者相反,我们是辩证唯物主义的动机和效果的统一论者。为大众的动机和被大众欢迎的效果,是分不开的,必须使二者统一起来。为个人的和狭隘集团的动机是不好的,有为大众的动机但无被大众欢迎、对大众有益的效果,也是不好的。检验一个作家的主观愿望即其动机是否正确,是否善良,不是看他的宣言,而是看他的行为(主要是作品)在社会大众中产生的效果。社会实践及其效果是检验主观愿望或动机的标准。我们的文艺批评是不要宗派主义的,在团结抗日的大原则下,我们应该容许包含各种各色政治态度的文艺作品的存在。但是我们的批评又是坚持原则立场的,对于一切包含反民族、反科学、反大众和反共的观点的文艺作品必须给以严格的批判和驳斥;因为这些所谓文艺,其动机,其效果,都是破坏团结抗日的。按着艺术标准来说,一切艺术性较高的,是好的,或较好的;艺术性较低的,则是坏的,或较坏的。这种分别,当然也要看社会效果。文艺家几乎没有不以为自己的作品是美的,我们的批评,也应该容许各种各色艺术品的自由竞争;但是按照艺术科学的标准给以正确的批判,使较低级的艺术逐渐提高成为较高级的艺术,使不适合广大群众斗争要求的艺术改变到适合广大群众斗争要求的艺术,也是完全必要的。\\
  又是政治标准,又是艺术标准,这两者的关系怎么样呢?政治并不等于艺术,一般的宇宙观也并不等于艺术创作和艺术批评的方法。我们不但否认抽象的绝对不变的政治标准,也否认抽象的绝对不变的艺术标准,各个阶级社会中的各个阶级都有不同的政治标准和不同的艺术标准。但是任何阶级社会中的任何阶级,总是以政治标准放在第一位,以艺术标准放在第二位的。资产阶级对于无产阶级的文学艺术作品,不管其艺术成就怎样高,总是排斥的。无产阶级对于过去时代的文学艺术作品,也必须首先检查它们对待人民的态度如何,在历史上有无进步意义,而分别采取不同态度。有些政治上根本反动的东西,也可能有某种艺术性。内容愈反动的作品而又愈带艺术性,就愈能毒害人民,就愈应该排斥。处于没落时期的一切剥削阶级的文艺的共同特点,就是其反动的政治内容和其艺术的形式之间所存在的矛盾。我们的要求则是政治和艺术的统一,内容和形式的统一,革命的政治内容和尽可能完美的艺术形式的统一。缺乏艺术性的艺术品,无论政治上怎样进步,也是没有力量的。因此,我们既反对政治观点错误的艺术品,也反对只有正确的政治观点而没有艺术力量的所谓“标语口号式”的倾向。我们应该进行文艺问题上的两条战线斗争。\\
  这两种倾向,在我们的许多同志的思想中是存在着的。许多同志有忽视艺术的倾向,因此应该注意艺术的提高。但是现在更成为问题的,我以为还是在政治方面。有些同志缺乏基本的政治常识,所以发生了各种糊涂观念。让我举一些延安的例子。\\
  “人性论”。有没有人性这种东西?当然有的。但是只有具体的人性,没有抽象的人性。在阶级社会里就是只有带着阶级性的人性,而没有什么超阶级的人性。我们主张无产阶级的人性,人民大众的人性,而地主阶级资产阶级则主张地主阶级资产阶级的人性,不过他们口头上不这样说,却说成为唯一的人性。有些小资产阶级知识分子所鼓吹的人性,也是脱离人民大众或者反对人民大众的,他们的所谓人性实质上不过是资产阶级的个人主义,因此在他们眼中,无产阶级的人性就不合于人性。现在延安有些人们所主张的作为所谓文艺理论基础的“人性论”,就是这样讲,这是完全错误的。\\
  “文艺的基本出发点是爱,是人类之爱。”爱可以是出发点,但是还有一个基本出发点。爱是观念的东西,是客观实践的产物。我们根本上不是从观念出发,而是从客观实践出发。我们的知识分子出身的文艺工作者爱无产阶级,是社会使他们感觉到和无产阶级有共同的命运的结果。我们恨日本帝国主义,是日本帝国主义压迫我们的结果。世上决没有无缘无故的爱,也没有无缘无故的恨。至于所谓“人类之爱”,自从人类分化成为阶级以后,就没有过这种统一的爱。过去的一切统治阶级喜欢提倡这个东西,许多所谓圣人贤人也喜欢提倡这个东西,但是无论谁都没有真正实行过,因为它在阶级社会里是不可能实行的。真正的人类之爱是会有的,那是在全世界消灭了阶级之后。阶级使社会分化为许多对立体,阶级消灭后,那时就有了整个的人类之爱,但是现在还没有。我们不能爱敌人,不能爱社会的丑恶现象,我们的目的是消灭这些东西。这是人们的常识,难道我们的文艺工作者还有不懂得的吗?\\
  “从来的文艺作品都是写光明和黑暗并重,一半对一半。”这里包含着许多糊涂观念。文艺作品并不是从来都这样。许多小资产阶级作家并没有找到过光明,他们的作品就只是暴露黑暗,被称为“暴露文学”,还有简直是专门宣传悲观厌世的。相反地,苏联在社会主义建设时期的文学就是以写光明为主。他们也写工作中的缺点,也写反面的人物,但是这种描写只能成为整个光明的陪衬,并不是所谓“一半对一半”。反动时期的资产阶级文艺家把革命群众写成暴徒,把他们自己写成神圣,所谓光明和黑暗是颠倒的。只有真正革命的文艺家才能正确地解决歌颂和暴露的问题。一切危害人民群众的黑暗势力必须暴露之,一切人民群众的革命斗争必须歌颂之,这就是革命文艺家的基本任务。\\
  “从来文艺的任务就在于暴露。”这种讲法和前一种一样,都是缺乏历史科学知识的见解。从来的文艺并不单在于暴露,前面已经讲过。对于革命的文艺家,暴露的对象,只能是侵略者、剥削者、压迫者及其在人民中所遗留的恶劣影响,而不能是人民大众。人民大众也是有缺点的,这些缺点应当用人民内部的批评和自我批评来克服,而进行这种批评和自我批评也是文艺的最重要任务之一。但这不应该说是什么“暴露人民”。对于人民,基本上是一个教育和提高他们的问题。除非是反革命文艺家,才有所谓人民是“天生愚蠢的”,革命群众是“专制暴徒”之类的描写。\\
  “还是杂文时代,还要鲁迅笔法。”鲁迅处在黑暗势力统治下面,没有言论自由,所以用冷嘲热讽的杂文形式作战,鲁迅是完全正确的。我们也需要尖锐地嘲笑法西斯主义、中国的反动派和一切危害人民的事物,但在给革命文艺家以充分民主自由、仅仅不给反革命分子以民主自由的陕甘宁边区和敌后的各抗日根据地,杂文形式就不应该简单地和鲁迅的一样。我们可以大声疾呼,而不要隐晦曲折,使人民大众不易看懂。如果不是对于人民的敌人,而是对于人民自己,那末,“杂文时代”的鲁迅,也不曾嘲笑和攻击革命人民和革命政党,杂文的写法也和对于敌人的完全两样。对于人民的缺点是需要批评的,我们在前面已经说过了,但必须是真正站在人民的立场上,用保护人民、教育人民的满腔热情来说话。如果把同志当作敌人来对待,就是使自己站在敌人的立场上去了。我们是否废除讽刺?不是的,讽刺是永远需要的。但是有几种讽刺:有对付敌人的,有对付同盟者的,有对付自己队伍的,态度各有不同。我们并不一般地反对讽刺,但是必须废除讽刺的乱用。\\
  “我是不歌功颂德的;歌颂光明者其作品未必伟大,刻画黑暗者其作品未必渺小。”你是资产阶级文艺家,你就不歌颂无产阶级而歌颂资产阶级;你是无产阶级文艺家,你就不歌颂资产阶级而歌颂无产阶级和劳动人民:二者必居其一。歌颂资产阶级光明者其作品未必伟大,刻画资产阶级黑暗者其作品未必渺小,歌颂无产阶级光明者其作品未必不伟大,刻画无产阶级所谓“黑暗”者其作品必定渺小,这难道不是文艺史上的事实吗?对于人民,这个人类世界历史的创造者,为什么不应该歌颂呢?无产阶级,共产党,新民主主义,社会主义,为什么不应该歌颂呢?也有这样的一种人,他们对于人民的事业并无热情,对于无产阶级及其先锋队的战斗和胜利,抱着冷眼旁观的态度,他们所感到兴趣而要不疲倦地歌颂的只有他自己,或者加上他所经营的小集团里的几个角色。这种小资产阶级的个人主义者,当然不愿意歌颂革命人民的功德,鼓舞革命人民的斗争勇气和胜利信心。这样的人不过是革命队伍中的蠹虫,革命人民实在不需要这样的“歌者”。\\
  “不是立场问题;立场是对的,心是好的,意思是懂得的,只是表现不好,结果反而起了坏作用。”关于动机和效果的辩证唯物主义观点,我在前面已经讲过了。现在要问:效果问题是不是立场问题?一个人做事只凭动机,不问效果,等于一个医生只顾开药方,病人吃死了多少他是不管的。又如一个党,只顾发宣言,实行不实行是不管的。试问这种立场也是正确的吗?这样的心,也是好的吗?事前顾及事后的效果,当然可能发生错误,但是已经有了事实证明效果坏,还是照老样子做,这样的心也是好的吗?我们判断一个党、一个医生,要看实践,要看效果;判断一个作家,也是这样。真正的好心,必须顾及效果,总结经验,研究方法,在创作上就叫做表现的手法。真正的好心,必须对于自己工作的缺点错误有完全诚意的自我批评,决心改正这些缺点错误。共产党人的自我批评方法,就是这样采取的。只有这种立场,才是正确的立场。同时也只有在这种严肃的负责的实践过程中,才能一步一步地懂得正确的立场是什么东西,才能一步一步地掌握正确的立场。如果不在实践中向这个方向前进,只是自以为是,说是“懂得”,其实并没有懂得。\\
  “提倡学习马克思主义就是重复辩证唯物论的创作方法的错误,就要妨害创作情绪。”学习马克思主义,是要我们用辩证唯物论和历史唯物论的观点去观察世界,观察社会,观察文学艺术,并不是要我们在文学艺术作品中写哲学讲义。马克思主义只能包括而不能代替文艺创作中的现实主义,正如它只能包括而不能代替物理科学中的原子论、电子论一样。空洞干燥的教条公式是要破坏创作情绪的,但是它不但破坏创作情绪,而且首先破坏了马克思主义。教条主义的“马克思主义”并不是马克思主义,而是反马克思主义的。那末,马克思主义就不破坏创作情绪了吗?要破坏的,它决定地要破坏那些封建的、资产阶级的、小资产阶级的、自由主义的、个人主义的、虚无主义的、为艺术而艺术的、贵族式的、颓废的、悲观的以及其它种种非人民大众非无产阶级的创作情绪。对于无产阶级文艺家,这些情绪应不应该破坏呢?我以为是应该的,应该彻底地破坏它们,而在破坏的同时,就可以建设起新东西来。\\
\subsubsection*{\myformat{五}}
我们延安文艺界中存在着上述种种问题,这是说明一个什么事实呢?说明这样一个事实,就是文艺界中还严重地存在着作风不正的东西,同志们中间还有很多的唯心论、教条主义、空想、空谈、轻视实践、脱离群众等等的缺点,需要有一个切实的严肃的整风运动。\\
  我们有许多同志还不大清楚无产阶级和小资产阶级的区别。有许多党员,在组织上入了党,思想上并没有完全入党,甚至完全没有入党。这种思想上没有入党的人,头脑里还装着许多剥削阶级的脏东西,根本不知道什么是无产阶级思想,什么是共产主义,什么是党。他们想:什么无产阶级思想,还不是那一套?他们哪里知道要得到这一套并不容易,有些人就是一辈子也没有共产党员的气味,只有离开党完事。因此我们的党,我们的队伍,虽然其中的大部分是纯洁的,但是为要领导革命运动更好地发展,更快地完成,就必须从思想上组织上认真地整顿一番。而为要从组织上整顿,首先需要在思想上整顿,需要展开一个无产阶级对非无产阶级的思想斗争。延安文艺界现在已经展开了思想斗争,这是很必要的。小资产阶级出身的人们总是经过种种方法,也经过文学艺术的方法,顽强地表现他们自己,宣传他们自己的主张,要求人们按照小资产阶级知识分子的面貌来改造党,改造世界。在这种情形下,我们的工作,就是要向他们大喝一声,说:“同志”们,你们那一套是不行的,无产阶级是不能迁就你们的,依了你们,实际上就是依了大地主大资产阶级,就有亡党亡国的危险。只能依谁呢?只能依照无产阶级先锋队的面貌改造党,改造世界。我们希望文艺界的同志们认识这一场大论战的严重性,积极起来参加这个斗争,使每个同志都健全起来,使我们的整个队伍在思想上和组织上都真正统一起来,巩固起来。\\
  因为思想上有许多问题,我们有许多同志也就不大能真正区别革命根据地和国民党统治区,并由此弄出许多错误。同志们很多是从上海亭子间\footnote[11]{ 亭子间是上海里弄房子中的一种小房间,位置在房子后部的楼梯中侧,狭小黑暗,因此租金比较低廉。解放以前,贫苦的作家、艺术家、知识分子和机关小职员,多半租这种房间居住。}来的;从亭子间到革命根据地,不但是经历了两种地区,而且是经历了两个历史时代。一个是大地主大资产阶级统治的半封建半殖民地的社会,一个是无产阶级领导的革命的新民主主义的社会。到了革命根据地,就是到了中国历史几千年来空前未有的人民大众当权的时代。我们周围的人物,我们宣传的对象,完全不同了。过去的时代,已经一去不复返了。因此,我们必须和新的群众相结合,不能有任何迟疑。如果同志们在新的群众中间,还是像我上次说的“不熟,不懂,英雄无用武之地”,那末,不但下乡要发生困难,不下乡,就在延安,也要发生困难的。有的同志想:我还是为“大后方”\footnote[12]{ 见本书第二卷《和中央社、扫荡报、新民报三记者的谈话》注〔3〕。}的读者写作吧,又熟悉,又有“全国意义”。这个想法,是完全不正确的。“大后方”也是要变的,“大后方”的读者,不需要从革命根据地的作家听那些早已听厌了的老故事,他们希望革命根据地的作家告诉他们新的人物,新的世界。所以愈是为革命根据地的群众而写的作品,才愈有全国意义。法捷耶夫的《毁灭》\footnote[13]{ 法捷耶夫(一九〇一——一九五六),苏联名作家。他所作的小说《毁灭》于一九二七年出版,内容是描写苏联国内战争时期由苏联远东滨海边区工人、农民和革命知识分子所组成的一支游击队同国内反革命白卫军以及日本武装干涉军进行斗争的故事。这部小说曾由鲁迅译为汉文。},只写了一支很小的游击队,它并没有想去投合旧世界读者的口味,但是却产生了全世界的影响,至少在中国,像大家所知道的,产生了很大的影响。中国是向前的,不是向后的,领导中国前进的是革命的根据地,不是任何落后倒退的地方。同志们在整风中间,首先要认识这一个根本问题。\\
  既然必须和新的群众的时代相结合,就必须彻底解决个人和群众的关系问题。鲁迅的两句诗,“横眉冷对千夫指,俯首甘为孺子牛”\footnote[14]{ 见鲁迅《集外集•自嘲》(《鲁迅全集》第7卷,人民文学出版社1981年版,第147页)。},应该成为我们的座右铭。“千夫”在这里就是说敌人,对于无论什么凶恶的敌人我们决不屈服。“孺子”在这里就是说无产阶级和人民大众。一切共产党员,一切革命家,一切革命的文艺工作者,都应该学鲁迅的榜样,做无产阶级和人民大众的“牛”,鞠躬尽瘁,死而后已。知识分子要和群众结合,要为群众服务,需要一个互相认识的过程。这个过程可能而且一定会发生许多痛苦,许多磨擦,但是只要大家有决心,这些要求是能够达到的。\\
  今天我所讲的,只是我们文艺运动中的一些根本方向问题,还有许多具体问题需要今后继续研究。我相信,同志们是有决心走这个方向的。我相信,同志们在整风过程中间,在今后长期的学习和工作中间,一定能够改造自己和自己作品的面貌,一定能够创造出许多为人民大众所热烈欢迎的优秀的作品,一定能够把革命根据地的文艺运动和全中国的文艺运动推进到一个光辉的新阶段。
\newpage\section*{\myformat{一个极其重要的政策}\\\myformat{(一九四二年九月七日)}}\addcontentsline{toc}{section}{一个极其重要的政策}
\begin{introduction}\item  这是毛泽东为延安《解放日报》写的社论。\end{introduction}
自从党中央提出精兵简政\footnote[1]{ 精兵简政,是一九四一年十一月李鼎铭等在陕甘宁边区第二届参议会上提出的。同年十二月,中共中央发出“精兵简政”的指示,要求切实整顿党、政、军各级组织机构,精简机关,充实连队,加强基层,提高效能,节约人力物力。这是在抗日根据地日益缩小的情况下,克服财政经济严重困难和休养生息民力的一项极其重要的政策。}这个政策以来,许多抗日根据地的党,都依照中央的指示,筹划和进行了这项工作。晋冀鲁豫边区的领导同志,对这项工作抓得很紧,做出了精兵简政的模范例子。但是还有若干根据地的同志们因为认识不够,没有认真地进行。这些地方的同志们还不理解精兵简政同当前形势和党的各项政策的关系,还没有把精兵简政当作一个极其重要的政策看待。关于这件事,《解放日报》曾多次讨论,今愿更有所说明。\\
  党的一切政策,都是为着战胜日寇。而第五年以后的抗战形势,实处于争取胜利的最后阶段。这个阶段,不但和抗日的第一第二年不相同,也和抗日的第三第四年不相同。抗日的第五第六年,包含着这样的情况,即接近着胜利,但又有极端的困难,也就是所谓“黎明前的黑暗”的情况。这种情况,整个反法西斯各国在目前阶段上都是有的,整个中国也是有的,不独八路军新四军的各个根据地为然,但是尤以我军的各个根据地表现得特别尖锐。我们要争取两年打败日寇。这两年将是极端困难的两年,它同抗日的开头两年和中间两年都有很大的不同。这种特点,革命政党和革命军队的领导人员必须事先看到。如果他们不能事先看到,那他们就只会跟着时间迁流,虽然也在努力工作,却不能取得胜利,反而有使革命事业受到损害的危险。敌后各抗日根据地的形势,截至今天为止,虽然已比过去增加了几倍的困难,但还不是极端的困难。如果现在没有正确的政策,那末极端的困难还在后头。普通的人,容易为过去和当前的情况所迷惑,以为今后也不过如此。他们缺乏事先看出航船将要遇到暗礁的能力,不能用清醒的头脑把握船舵,绕过暗礁。什么是抗日航船今后的暗礁呢?就是抗战最后阶段中的物质方面的极端严重的困难。党中央指出了这个困难,叫我们提起注意绕过这个暗礁。我们的许多同志已经懂得了,但是还有若干同志不懂得,这就是必须首先克服的障碍。抗战要有一个团结,在团结中有各种的困难。这个困难是政治上的困难,过去有,今后还可能有。五年以来,我党用了极大的力量逐步地克服着这个困难,我们的口号是增强团结,今后还要增强它。但是还有一个困难,就是物质方面的困难。这个困难,今后必然愈来愈厉害。目前还有若干同志处之泰然,不大觉得,我们就有唤起这些同志提起注意之必要。各抗日根据地的全体同志必须认识,今后的物质困难必然更甚于目前,我们必须克服这个困难,我们的重要的办法之一就是精兵简政。\\
  精兵简政何以是克服物质困难的一个重要的政策呢?很显然,目前的尤其是今后的根据地的战争情况,不容许我们停留在过去的观点上。我们的庞大的战争机构,是适应过去的情况的。那时的情况容许我们如此,也应该如此。但是现在不同了,根据地已经缩小,在今后的一个时期内还可能再缩小,我们便决然不能还像过去那样地维持着庞大的机构。在目前,战争的机构和战争的情况之间已经发生了矛盾,我们必须克服这个矛盾。敌人的方针是扩大我们这个矛盾,这就是他的“三光”政策\footnote[2]{ “三光”政策指日本帝国主义对抗日根据地实施的烧光、杀光、抢光的政策。}。假若我们还要维持庞大的机构,那就会正中敌人的奸计。假若我们缩小自己的机构,使兵精政简,我们的战争机构虽然小了,仍然是有力量的;而因克服了鱼大水小的矛盾,使我们的战争的机构适合战争的情况,我们就将显得越发有力量,我们就不会被敌人战胜,而要最后地战胜敌人。所以我们说,党中央提出的精兵简政的政策,是一个极其重要的政策。\\
  但是,现状和习惯往往容易把人们的头脑束缚得紧紧的,即使是革命者有时也不能免。庞大的机构是由自己亲手创造出来的,想不到又要由自己的手将它缩小,实行缩小时就感到很勉强,很困难。敌人以庞大的机构向我们压迫,难道我们还可以缩小吗?实行缩小就感到兵少不足以应敌。这些就是所谓为现状和习惯所束缚。气候变化了,衣服必须随着变化。每年的春夏之交,夏秋之交,秋冬之交和冬春之交,各要变换一次衣服。但是人们往往在那“之交”不会变换衣服,要闹出些毛病来,这就是由于习惯的力量。目前根据地的情况已经要求我们褪去冬衣,穿起夏服,以便轻轻快快地同敌人作斗争,我们却还是一身臃肿,头重脚轻,很不适于作战。若说:何以对付敌人的庞大机构呢?那就有孙行者对付铁扇公主为例。铁扇公主虽然是一个厉害的妖精,孙行者却化为一个小虫钻进铁扇公主的心脏里去把她战败了\footnote[3]{ 铁扇公主又名罗刹。孙行者变为小虫战败铁扇公主的故事,见明朝吴承恩着的神话小说《西游记》第五十九回。}。柳宗元曾经描写过的“黔驴之技”\footnote[4]{ 柳宗元(七七三——八一九),中国唐代的大作家之一。他写过一篇《三戒》,包括三段寓言,其中一段题为《黔之驴》,说:“黔无驴,有好事者船载以入。至则无可用,放之山下。虎见之,庞然大物也,以为神,蔽林间窥之。稍出近之,慭慭然,莫相知。他日,驴一鸣,虎大骇,远遁;以为且噬己也,甚恐。然往来视之,觉无异能者;益习其声,又近出前后,终不敢搏。稍近,益狎,荡倚冲冒。驴不胜怒,蹄之。虎因喜,计之曰:‘技止此耳!’因跳踉大�\,断其喉,尽其肉,乃去。”},也是一个很好的教训。一个庞然大物的驴子跑进贵州去了,贵州的小老虎见了很有些害怕。但到后来,大驴子还是被小老虎吃掉了。我们八路军新四军是孙行者和小老虎,是很有办法对付这个日本妖精或日本驴子的。目前我们须得变一变,把我们的身体变得小些,但是变得更加扎实些,我们就会变成无敌的了。\\
\newpage\section*{\myformat{第二次世界大战的转折点}\\\myformat{(一九四二年十月十二日)}}\addcontentsline{toc}{section}{第二次世界大战的转折点}
\begin{introduction}\item  这是毛泽东为延安《解放日报》写的社论。\end{introduction}
斯大林格勒之战,英美报纸比之为凡尔登战役,“红色凡尔登”之名已传遍于世界。这个比拟并不适当。今天的斯大林格勒之战,比起第一次世界大战时的凡尔登来,有性质的不同。但有一点是相同的,即有许多人在这种时候还被德国的攻势所迷惑,以为德国还有获胜的可能。第一次世界大战结束于一九一八年冬,在一九一六年,德军曾向法国要塞凡尔登举行数度的进攻。当时德军的战役统帅是德国皇太子,投入战斗的力量是德军的最精锐部分。当时的战斗是带决战性的。德军猛攻不克,整个德奥土保阵线再也找不到出路,从此日益困难,众叛亲离,土崩瓦解,走到了最后的崩溃。然而当时英美法阵线方面,还没有看出这种情况,以为德军仍极强大,不知道自己的胜利已经快到面前。在人类历史上,凡属将要灭亡的反动势力,总是要向革命势力进行最后挣扎的,而有些革命的人们也往往在一个期间内被这种外强中干的现象所迷惑,看不出敌人快要消灭,自己快要胜利的实质。整个法西斯势力的兴起及其进行了几年的侵略战争,正是这种最后挣扎的表现;而在战争中,又以攻击斯大林格勒表现它自己的最后挣扎。在这个历史的转折点面前,全世界反法西斯阵线内的人们也有许多被法西斯的凶恶面孔所迷惑,看不出它的实质。自从八月二十三日德军全部渡过顿河河曲,全面地开始攻击斯大林格勒,九月十五日德军一部打入该城西北部工业区,至十月九日苏联情报局宣布红军突破该区德军包围线为止,共计进行了四十八天人类历史上无与伦比的空前苦战。这一战终于胜利了。在这四十八天中,这个城市每天的胜负消息,紧系着无数千万、万万人民的呼吸,使他们忧愁,使他们欢乐。这一战,不但是苏德战争的转折点,甚至也不但是这次世界反法西斯战争的转折点,而且是整个人类历史的转折点。在这四十八天中,世界人民的注视斯大林格勒,和去年十月间世界人民的注视莫斯科,其关心程度,是有过之无不及的。\\
  希特勒在西线胜利以前,他似乎是谨慎的。攻波兰,攻挪威,攻荷、比、法,攻巴尔干,都是注全力于一处,不敢旁骛。西线胜利后,他就冲昏了头脑,企图在三个月内打败苏联。北起摩尔曼斯克,南至克里米亚,向这个庞大坚强的社会主义国家举行了全面的进攻,这样就分散了他的兵力。去年十月向莫斯科进攻的失败,结束了苏德战争的第一阶段;希特勒第一个战略计划破产了。红军制止了德军去年的进攻,并在冬季举行了全线的反攻,是为苏德战争的第二阶段;希特勒转到了退却和防御的地位。在此期间,希特勒撤消了他的前线总司令勃鲁齐区,自己充任总司令,决定放弃全面的进攻计划,搜索欧洲全力,准备向南线作局部的但被认为是打击苏联要害的最后进攻。因为这一进攻带着最后一次的性质,关系法西斯的存亡,希特勒就集中了极大的兵力,连在北非作战中的一部分飞机坦克都抽调过来了。从今年五月进攻刻赤和塞瓦斯托波尔起,进入战争的第三阶段。希特勒调动了一百五十万以上的兵力,附以飞机坦克的主力,向斯大林格勒和高加索作空前剧烈的进攻。他企图迅速攻下两处,达到切断伏尔加和夺取巴库两个目的,然后北攻莫斯科,南出波斯湾,并令日本法西斯集中兵力于满洲,准备在斯大林格勒攻下后进攻西伯利亚。希特勒妄想把苏联力量削弱到足以使德军主力从苏联战场上解脱出来,以便移到西线对付英美的进攻,并可掠取近东资源,打通德日联系,同时,日军主力也可从北面解脱出来,以便西进南进对付我国和英美,而无后顾之忧,这样来争取法西斯阵线的胜利。但是这个阶段的情况是怎样的呢?希特勒遇到了苏联制其死命的策略。苏联采取了先则诱敌深入、继则顽强抵抗的方针。五个月的战争,使德军既没有打进高加索油田,也没有打下斯大林格勒,迫使希特勒顿兵于高山与坚城之下,欲进不能,欲退不得,损失甚大,陷于僵局。现在已是十月,冬季就要到来,战争的第三阶段快要结束,第四阶段快要开始了。希特勒进攻苏联的战略企图没有一个不是失败的。在此期间,希特勒鉴于去夏分兵的失败,集中他的兵力向着南线。然而他尚欲东断伏尔加,南取高加索,一举达成两个目的,仍然分散了他的兵力。他尚未计算到他的实力和他的企图之间的不相称,以致“扁担没扎,两头打塌”,陷入目前的绝路。在相反方面,苏联则是越战越强。斯大林的英明战略指挥,完全站在主动的地位,处处把希特勒引向灭亡。今年冬季开始的第四个阶段,将是希特勒走向死亡的阶段。\\
  拿希特勒在第一阶段上的情况和第三阶段作比较,就可知希特勒是处在最后失败的门口了。目前红军在斯大林格勒和高加索两方面,实际上均已停止了德军的进攻,希特勒已到再衰三竭之时,他对斯大林格勒、高加索两处的进攻已经失败。他在去年十二月至今年五月整个冬季中所整备的一点兵力,已经耗竭了。在苏德战线,距冬季不到一个月了,他须赶快转入防御。整个顿河的以西以南是他的最危险的地带,红军将在这一带转入反攻。今年冬季,希特勒因被死亡所驱迫,将再一次整备他的军队。他或者还可能搜索他的一点残余力量装备出几个新的师团,此外则乞援于意、罗、匈三国法西斯伙伴,向他们勒索一些炮灰,以应付东西两线的危局。但是,他在东线须应付冬季战争的极大消耗,他在西线须准备对付第二条战线,而意、罗、匈等国则将在希特勒大势已去的这种悲观情绪中,一天一天变成离心离德。总之,十月九日以后的希特勒,将只有死路一条好走了。\\
  四十八天中,红军的保卫斯大林城,和去年保卫莫斯科市有某种相同。这就是说,它使得希特勒今年的计划也像他的去年计划一样,归于失败。其不同点,则在莫斯科保卫战之后,虽然接着举行了冬季反攻,可是还要遭到今年德军的一个夏季进攻,这是因为一则德国及其欧洲伙伴尚有余勇可贾,二则英美拖延开辟第二条战线的缘故。而在斯大林格勒保卫战之后,则形势将和去年完全两样。一方面苏联将举行极大规模的第二个冬季反攻,英美对第二条战线的开辟将无可拖延(虽然具体时间仍不能计算),欧洲人民也将准备着起义响应。另一方面,德国及其欧洲伙伴再也无力举行大规模的攻势了,希特勒只好把整个方针转入战略防御。只要迫使希特勒转入了战略防御,法西斯的命运就算完结了。因为像希特勒这样法西斯国家的政治生命和军事生命,从它出生的一天起,就是建立在进攻上面的,进攻一完结,它的生命也就完结了。斯大林格勒一战将停止法西斯的进攻,这一战是带着决定性的。这种决定性,是关系于整个世界战争的。\\
  希特勒面前遇着的,是三个强大敌人:苏联、英美及在其占领区的老百姓。在东线,是屹立不动的红军壁垒和整个第二冬季以及连续下去的红军反攻,这是整个战争和人类命运的决定的力量。在西线,即使英美还采取着观望和拖延的政策,但等到有死老虎可打的时候,第二条战线总是要建立的。希特勒还有一个内部战线,就是德国、法国及欧洲其它部分正在酝酿着的一个伟大的人民起义,只待苏联举行全面反攻和第二条战线炮响,他们将以第三条战线出来响应。这样,三条战线夹击希特勒,就将是斯大林格勒战役以后的伟大历史过程。\\
  拿破仑的政治生命,终结于滑铁卢,而其决定点,则是在莫斯科的失败\footnote[1]{ 一八一五年六月,拿破仑的军队同英普联军激战于比利时的滑铁卢。拿破仑战败,被流放于大西洋南部的圣赫勒拿岛,至一八二一年死于该岛。拿破仑一生征服过欧洲的许多国家,但是在一八一二年进攻俄国的战争中,在莫斯科遭到极大的失败,他的精锐部队几乎全部被消灭。拿破仑受到了这次打击,从此便一蹶不振。关于拿破仑在莫斯科的失败,见本书第二卷《论持久战》注〔41〕。}。希特勒今天正是走的拿破仑道路,斯大林格勒一役,是他的灭亡的决定点。\\
  这一形势,将直接影响到远东。明年也将不是日本法西斯的吉利年头。它将一天一天感到头痛,直至向它的墓门跨进。\\
  一切对世界形势作悲观观察的人们,应将自己的观点改变过来。\\
\newpage\section*{\myformat{祝十月革命二十五周年}\\\myformat{(一九四二年十一月六日)}}\addcontentsline{toc}{section}{祝十月革命二十五周年}
我们以最大的乐观来庆祝今年的十月革命节。我坚信,今年的十月革命节不但是苏德战争的转折点,而且是全世界反法西斯阵线战胜法西斯阵线的转折点。\\
  在过去时期内,因为红军单独抵抗法西斯德国及其欧洲伙伴,希特勒还能继续进攻,希特勒还没有被打败。现在,苏联的力量已经在战争中壮大起来了,希特勒的第二个夏季攻势已经破产了。从此以后,世界反法西斯阵线的任务,就是发动对法西斯阵线的进攻,最后地打败法西斯。\\
  斯大林格勒的红军战士做出了有关全人类命运的英雄事业。他们是十月革命的儿女。十月革命的旗帜是不可战胜的,而一切法西斯势力则必归于消灭。\\
  我们中国人民庆祝红军的胜利,同时也即是庆祝自己的胜利。我们的抗日战争已经进行五年多了,我们的前途虽然还有艰苦,但是胜利的曙光已经看得见了。战胜日本法西斯不但是确定的,而且是不远的了。\\
  一切努力集中于打击日本法西斯,这就是中国人民的任务。\\
\newpage\section*{\myformat{抗日时期的经济问题和财政问题}\\\myformat{(一九四二年十二月)}}\addcontentsline{toc}{section}{抗日时期的经济问题和财政问题}
\begin{introduction}\item  这是毛泽东在陕甘宁边区高级干部会议上所作的报告《经济问题与财政问题》的第一章,原题为《关于过去工作的基本总结》。一九四一年和一九四二年是抗日战争期间根据地最困难的时期。由于日本侵略军的野蛮进攻和国民党的包围封锁,根据地的财政发生了极大的困难。毛泽东指出党必须努力领导人民发展农业生产和其它生产事业,并号召根据地的机关、学校、部队尽可能地实行生产自给,以便克服财政和经济的困难。毛泽东的《经济问题与财政问题》一书,以及《开展根据地的减租、生产和拥政爱民运动》、《组织起来》等文,就是当时中国共产党领导根据地生产运动的基本纲领。在《经济问题与财政问题》一书里,毛泽东着重地批判了那种离开发展经济而单纯在财政收支问题上打主意的错误思想,和那种不注意动员人民帮助人民发展生产渡过困难而只注意向人民要东西的错误作风,提出了党的“发展经济,保障供给”的正确方针。在这个方针之下发展起来的陕甘宁边区和敌后各抗日根据地的生产运动,得到了巨大的成绩,不但使根据地军民胜利地渡过了抗日战争的最困难时期,而且给中国共产党在后来对于经济建设工作的领导积累了丰富的经验。\end{introduction}
发展经济,保障供给,是我们的经济工作和财政工作的总方针。但是有许多同志,片面地看重了财政,不懂得整个经济的重要性;他们的脑子终日只在单纯的财政收支问题上打圈子,打来打去,还是不能解决问题。这是一种陈旧的保守的观点在这些同志的头脑中作怪的缘故。他们不知道财政政策的好坏固然足以影响经济,但是决定财政的却是经济。未有经济无基础而可以解决财政困难的,未有经济不发展而可以使财政充裕的。陕甘宁边区的财政问题,就是几万军队和工作人员的生活费和事业费的供给问题,也就是抗日经费的供给问题。这些经费,都是由人民的赋税及几万军队和工作人员自己的生产来解决的。如果不发展人民经济和公营经济,我们就只有束手待毙。财政困难,只有从切切实实的有效的经济发展上才能解决。忘记发展经济,忘记开辟财源,而企图从收缩必不可少的财政开支去解决财政困难的保守观点,是不能解决任何问题的。\\
  五年以来,我们经过了几个阶段。最大的一次困难是在一九四〇年和一九四一年,国民党的两次反共磨擦\footnote[1]{ 这里指国民党发动的第一、第二次反共高潮。参见本卷《评国民党十一中全会和三届二次国民参政会》中关于这两次反共高潮的叙述。},都在这一时期。我们曾经弄到几乎没有衣穿,没有油吃,没有纸,没有菜,战士没有鞋袜,工作人员在冬天没有被盖。国民党用停发经费和经济封锁来对待我们,企图把我们困死,我们的困难真是大极了。但是我们渡过了困难。这不但是由于边区人民给了我们粮食吃,尤其是由于我们下决心自己动手,建立了自己的公营经济。边区政府办了许多的自给工业;军队进行了大规模的生产运动,发展了以自给为目标的农工商业;几万机关学校人员,也发展了同样的自给经济。军队和机关学校所发展的这种自给经济是目前这种特殊条件下的特殊产物,它在其它历史条件下是不合理的和不可理解的,但在目前却是完全合理并且完全必要的。我们就用这些办法战胜了困难。只有发展经济才能保障供给这一真理,不是被明白无疑的历史事实给我们证明了吗?到了现在,我们虽则还有很多的困难,但是我们的公营经济的基础,已经打下了。一九四三年再来一年,我们的基础就更加稳固了。\\
  发展经济的路线是正确的路线,但发展不是冒险的无根据的发展。有些同志不顾此时此地的具体条件,空嚷发展,例如要求建设重工业,提出大盐业计划、大军工计划等,都是不切实际的,不能采用的。党的路线是正确的发展路线,一方面要反对陈旧的保守的观点,另一方面又要反对空洞的不切实际的大计划。这就是党在财政经济工作中的两条战线上的斗争。\\
  我们要发展公营经济,但是我们不要忘记人民给我们帮助的重要性。人民给了我们粮食吃:一九四〇年的九万担,一九四一年的二十万担,一九四二年的十六万担\footnote[2]{ 毛泽东在这里所举的粮食数字,是一九四〇年至一九四二年陕甘宁边区农民各年所缴纳的公粮(即农业税)的总数。},保证了军队和工作人员的食粮。截至一九四一年,我们公营农业中的粮食生产一项,还是很微弱的,我们在粮食方面还是依靠老百姓。今后虽然一定要加重军队的粮食生产,但是暂时也还只能主要地依靠老百姓。陕甘宁边区虽然是没有直接遭受战争破坏的后方环境,但是地广人稀,只有一百五十万人口,供给这样多的粮食,是不容易的。老百姓为我们运公盐和出公盐代金,一九四一年还买了五百万元公债,也是不小的负担。为了抗日和建国的需要,人民是应该负担的,人民很知道这种必要性。在公家极端困难时,要人民多负担一点,也是必要的,也得到人民的谅解。但是我们一方面取之于民,一方面就要使人民经济有所增长,有所补充。这就是对人民的农业、畜牧业、手工业、盐业和商业,采取帮助其发展的适当步骤和办法,使人民有所失同时又有所得,并且使所得大于所失,才能支持长期的抗日战争。\\
  有些同志不顾战争的需要,单纯地强调政府应施“仁政”,这是错误的观点。因为抗日战争如果不胜利,所谓“仁政”不过是施在日本帝国主义身上,于人民是不相干的。反过来,人民负担虽然一时有些重,但是战胜了政府和军队的难关,支持了抗日战争,打败了敌人,人民就有好日子过,这个才是革命政府的大仁政。\\
  另外的错误观点,就是不顾人民困难,只顾政府和军队的需要,竭泽而渔,诛求无已。这是国民党的思想,我们决不能承袭。我们一时候加重了人民的负担,但是我们立即动手建设了公营经济。一九四一年和一九四二年两年中,军队和机关学校因自己动手而获得解决的部分,占了整个需要的大部分。这是中国历史上从来未有的奇迹,这是我们不可征服的物质基础。我们的自给经济愈发展,我们加在人民身上的赋税就可以愈减轻。一九三七年至一九三九年的第一个阶段中,我们取之于民是很少的;在这一阶段内,大大地休养了民力。一九四〇年至一九四二年为第二阶段,人民负担加重了。一九四三年以后,可以走上第三阶段。如果我们的公营经济在一九四三年和一九四四年两年内是继续发展的,如果我们在陕甘宁边区的军队在这两年内获得全部或大部屯田的机会,那末,在两年以后,人民负担又可减轻了,民力又可得到休养了。这个趋势是可能实现的,我们应该准备这样做。\\
  我们要批驳这样那样的偏见,而提出我们党的正确的口号,这就是“发展经济,保障供给”。在公私关系上,就是“公私兼顾”,或叫“军民兼顾”。我们认为只有这样的口号,才是正确的口号。只有实事求是地发展公营和民营的经济,才能保障财政的供给。虽在困难时期,我们仍要注意赋税的限度,使负担虽重而民不伤。而一经有了办法,就要减轻人民负担,借以休养民力。\\
  国民党的顽固分子觉得边区的建设是无希望的,边区的困难是不可克服的困难,他们每天都在等待着边区“塌台”。对于这种人,我们用不着和他们辩论,他们是永远也看不到我们“塌台”的日子的,我们只会兴盛起来。他们不知道在共产党和边区革命政府的领导下,人民群众总是拥护党和政府的。党和政府在经济和财政方面也一定有办法,足以渡过任何严重的困难。我们现在的困难,有的已经渡过,有的快要渡过。我们曾经历过比现在还要困难到多少倍的时候,那样的困难我们也渡过了。现在华北华中各根据地的困难,比陕甘宁边区要大得多,那里天天有严重的战争,那里已经支持了五年半,那里也一定能够继续支持,直到胜利。在我们面前是没有悲观的,我们能够战胜任何的困难。\\
  这次陕甘宁边区高级干部会议以后,我们就要实行“精兵简政”\footnote[3]{ 见本卷《一个极其重要的政策》注〔1〕。}。这一次精兵简政,必须是严格的、彻底的、普遍的,而不是敷衍的、不痛不痒的、局部的。在这次精兵简政中,必须达到精简、统一、效能、节约和反对官僚主义五项目的。这五项,对于我们的经济工作和财政工作,关系极大。精简之后,减少了消费性的支出,增加了生产的收入,不但直接给予财政以好影响,而且可以减少人民的负担,影响人民的经济。经济和财政工作机构中的不统一、闹独立性、各自为政等恶劣现象,必须克服,而建立统一的、指挥如意的、使政策和制度能贯彻到底的工作系统。这种统一的系统建立后,工作效能就可以增加。节约是一切工作机关都要注意的,经济和财政工作机关尤其要注意。实行节约的结果,可以节省一大批不必要的和浪费性的支出,其数目可以达到几千万元。从事经济和财政业务的工作人员,还必须克服存在着的有些还是很严重的官僚主义,例如贪污现象,摆空架子,无益的“正规化”,文牍主义等等。如果我们把这五项要求在党的、政府的、军队的各个系统中完全实行起来,那我们的这次精兵简政,就算达到了目的,我们的困难就一定能克服,那些笑我们会要“塌台”的人们的嘴巴也就可以被我们封住了。\\
\newpage\section*{\myformat{关于领导方法的若干问题}\\\myformat{(一九四三年六月一日)}}\addcontentsline{toc}{section}{关于领导方法的若干问题}
\begin{introduction}\item  这是毛泽东为中共中央所写的决定。\end{introduction}
(一)我们共产党人无论进行何项工作,有两个方法是必须采用的,一是一般和个别相结合,二是领导和群众相结合。\\
  (二)任何工作任务,如果没有一般的普遍的号召,就不能动员广大群众行动起来。但如果只限于一般号召,而领导人员没有具体地直接地从若干组织将所号召的工作深入实施,突破一点,取得经验,然后利用这种经验去指导其它单位,就无法考验自己提出的一般号召是否正确,也无法充实一般号召的内容,就有使一般号召归于落空的危险。例如一九四二年的各地整风,凡有成绩者,都是采用了一般号召和个别指导相结合的方法;凡无成绩者,都是没有采用此种方法。一九四三年的整风,各中央局、中央分局、区党委和地委,除提出一般号召(全年整风计划)外,必须在自己机关中和附近机关、学校、部队中,选择二、三单位(不要很多),深入研究,详细了解整风学习在这些单位的发展过程,详细了解这些单位中若干个(也不要很多)有代表性的工作人员的政治经历、思想特点、学习勤惰和工作优劣,并亲自指导这些单位的负责人具体地解决各该单位的实际问题,借以取得经验。一机关、一学校、一部队内部也有若干单位,该机关、该学校、该部队的领导人员也须这样去做。这又是领导人员指导和学习相结合的方法。任何领导人员,凡不从下级个别单位的个别人员、个别事件取得具体经验者,必不能向一切单位作普遍的指导。这一方法必须普遍地提倡,使各级领导干部都能学会使用。\\
  (三)一九四二年的整风经验又证明:每一单位的整风,必须在整风过程中形成一个以该单位的首要负责人为核心的少数积极分子的领导骨干,并使这一领导骨干和参加学习的广大群众密切结合,才能使整风完成任务。只有领导骨干的积极性,而无广大群众的积极性相结合,便将成为少数人的空忙。但如果只有广大群众的积极性,而无有力的领导骨干去恰当地组织群众的积极性,则群众积极性既不可能持久,也不可能走向正确的方向和提到高级的程度。任何有群众的地方,大致都有比较积极的、中间状态的和比较落后的三部分人。故领导者必须善于团结少数积极分子作为领导的骨干,并凭借这批骨干去提高中间分子,争取落后分子。凡属真正团结一致、联系群众的领导骨干,必须是从群众斗争中逐渐形成,而不是脱离群众斗争所能形成的。在多数情形下,一个伟大的斗争过程,其开始阶段、中间阶段和最后阶段的领导骨干,不应该是也不可能是完全同一的;必须不断地提拔在斗争中产生的积极分子,来替换原有骨干中相形见绌的分子,或腐化了的分子。许多地方和许多机关工作推不动的一个基本原因,就是缺乏这样一个团结一致、联系群众的经常健全的领导骨干。一个百人的学校,如果没有一个从教员中、职员中、学生中按照实际形成的(不是勉强凑集的)最积极最正派最机敏的几个人乃至十几个人的领导骨干,这个学校就一定办不好。斯大林论党的布尔什维克化的十二个条件的第九条中所说建立领导核心问题\footnote[1]{ 见斯大林《关于德国共产党的前途和布尔什维克化》(《斯大林选集》上卷,人民出版社1979年版,第313页)。},我们应该应用到一切大小机关、学校、部队、工厂和农村中去。这种领导骨干的标准,应当是季米特洛夫论干部政策中所举的四条干部标准(无限忠心,联系群众,有独立工作能力,遵守纪律)\footnote[2]{ 见季米特洛夫一九三五年八月十三日在共产国际第七次代表大会上所作的结论《为工人阶级团结一致反对法西斯主义而斗争》的第七部分《干部问题》。}。无论是执行战争、生产、教育(包括整风)等中心任务,或是执行检查工作、审查干部和其它工作,除采取一般号召和个别指导相结合的方法以外,都须采取领导骨干和广大群众相结合的方法。\\
  (四)在我党的一切实际工作中,凡属正确的领导,必须是从群众中来,到群众中去。这就是说,将群众的意见(分散的无系统的意见)集中起来(经过研究,化为集中的系统的意见),又到群众中去作宣传解释,化为群众的意见,使群众坚持下去,见之于行动,并在群众行动中考验这些意见是否正确。然后再从群众中集中起来,再到群众中坚持下去。如此无限循环,一次比一次地更正确、更生动、更丰富。这就是马克思主义的认识论。\\
  (五)领导骨干和广大群众在组织中在斗争行动中发生正确关系的思想,正确的领导意见只能从群众中集中起来又到群众中坚持下去的思想,在领导意见见之实行时要将一般号召和个别指导互相结合的思想,都必须在这次整风中普遍地加以宣传,借以纠正干部中在这个问题上的错误观点。许多同志,不注重和不善于团结积极分子组成领导核心,不注重和不善于使这种领导核心同广大群众密切地结合起来,因而使自己的领导变成脱离群众的官僚主义的领导。许多同志,不注重和不善于总结群众斗争的经验,而欢喜主观主义地自作聪明地发表许多意见,因而使自己的意见变成不切实际的空论。许多同志,满足于工作任务的一般号召,不注重和不善于在作了一般号召之后,紧紧地接着从事于个别的具体的指导,因而使自己的号召停止在嘴上、纸上或会议上,而变为官僚主义的领导。这次整风,必须纠正这些缺点,在整风学习、检查工作、审查干部中学会领导和群众相结合、一般和个别相结合的方法,并在以后应用此种方法于一切工作。\\
  (六)从群众中集中起来又到群众中坚持下去,以形成正确的领导意见,这是基本的领导方法。在集中和坚持过程中,必须采取一般号召和个别指导相结合的方法,这是前一个方法的组成部分。从许多个别指导中形成一般意见(一般号召),又拿这一般意见到许多个别单位中去考验(不但自己这样做,而且告诉别人也这样做),然后集中新的经验(总结经验),做成新的指示去普遍地指导群众。同志们在这次整风中应该这样去做,在任何工作中也应该这样去做。比较好的领导,就是从比较善于这样去做而得到的。\\
  (七)对于任何工作任务(革命战争、生产、教育,或整风学习、检查工作、审查干部,或宣传工作、组织工作、锄奸工作等等)的向下传达,上级领导机关及其个别部门都应当通过有关该项工作的下级机关的主要负责人,使他们负起责任来,达到分工而又统一的目的(一元化)。不应当只是由上级的个别部门去找下级的个别部门(例如上级组织部只找下级的组织部,上级宣传部只找下级的宣传部,上级锄奸部只找下级的锄奸部),而使下级机关的总负责人(例如书记、主席、主任、校长等)不知道,或不负责。应当使总负责人和分负责人都知道,都负责。这样分工而又统一的一元化的方法,使一件工作经过总负责人推动很多干部、有时甚至是全体人员去做,可以克服各单个部门干部不足的缺点,而使许多人都变为积极参加该项工作的干部。这也是领导和群众相结合的一种形式。例如审查干部,如果仅仅由组织部这个领导机关的少数人孤立地去做,必不可能做好;如果通过某一机关或某一学校的行政负责人,推动该机关该学校的许多人员、许多学生,有时甚至是全体人员、全体学生都参加审查,而上级组织部的领导人员则正确地指导这种审查,实行领导和群众相结合的原则,审查干部的目的就一定能完满地达到。\\
  (八)在任何一个地区内,不能同时有许多中心工作,在一定时间内只能有一个中心工作,辅以别的第二位、第三位的工作。因此,一个地区的总负责人,必须考虑到该处的斗争历史和斗争环境,将各项工作摆在适当的地位;而不是自己全无计划,只按上级指示来一件做一件,形成很多的“中心工作”和凌乱无秩序的状态。上级机关也不要不分轻重缓急地没有中心地同时指定下级机关做很多项工作,以致引起下级在工作步骤上的凌乱,而得不到确定的结果。领导人员依照每一具体地区的历史条件和环境条件,统筹全局,正确地决定每一时期的工作重心和工作秩序,并把这种决定坚持地贯彻下去,务必得到一定的结果,这是一种领导艺术。这也是在运用领导和群众相结合、一般和个别相结合这些原则时,必须注意解决的领导方法问题。\\
  (九)领导方法问题上的各个细节问题,这里不一一说到,希望各地同志根据这里所说的原则方针自己去用心思索,发扬自己的创造力。斗争愈是艰苦,就愈是需要共产党人的领导和广大群众的要求密切地相结合,愈是需要共产党人的一般号召和个别指导密切地相结合,而彻底粉碎主观主义的和官僚主义的领导方法。我党一切领导同志必须随时拿马克思主义的科学的领导方法去同主观主义的和官僚主义的领导方法相对立,而以前者去克服后者。主观主义者和官僚主义者不知道领导和群众相结合、一般和个别相结合的原则,极大地妨碍党的工作的发展。为了反对主观主义的和官僚主义的领导方法,必须广泛地深入地提倡马克思主义的科学的领导方法。\\
\newpage\section*{\myformat{质问国民党}\\\myformat{(一九四三年七月十二日)}}\addcontentsline{toc}{section}{质问国民党}
\begin{introduction}\item  这是毛泽东为延安《解放日报》写的社论。\end{introduction}
近月以来,中国抗日阵营内部,发生了一个很不经常很可骇怪的事实,这就是中国国民党领导的许多党政军机关发动了一个破坏团结抗战的运动。这个运动是以反对共产党的姿态出现,而其实际,则是反对中华民族和反对中国人民的。\\
  首先看国民党的军队。国民党领导的全国军队中,位置在西北方面的主力就有第三十四、第三十七、第三十八等三个集团军,都受第八战区副司令长官胡宗南指挥。其中有两个集团军用于包围陕甘宁边区,只有一个用于防守从宜川至潼关一段黄河沿岸,对付日寇。这种事实,已经是四年多了,只要不发生军事冲突,大家也就习以为常了。不料近日却发生了这样的变化,即担任河防的第一、第十六、第九十等三个军中,开动了两个军,第一军开到彬县、淳化一带,第九十军开到洛川一带,并积极准备进攻边区,而使对付日寇的河防,大部分空虚起来。\\
  这不能不使人们发生这样的疑问,这些国民党人同日本人之间的关系,究竟是怎样的呢?\\
  许多国民党人肆无忌惮地天天宣传共产党“破坏抗战”、“破坏团结”,难道尽撤河防主力,倒叫做增强抗战吗?难道进攻边区,倒叫做增强团结吗?\\
  请问干这些事的国民党人:你们拿背对着日本人,日本人却拿面对着你们,如果日本人向你们的背前进,那时你们怎么办呢?\\
  如果你们将大段的河防丢弃不管,而日本人却仍然静悄悄地在对岸望着不动,只是拿着望远镜兴高采烈地注视着你们愈走愈远的背影,那末,这其中又是一种什么缘故呢?为什么日本人这样欢喜你们的背,而你们丢了河防不管,让它大段地空着,你们的心就那么放得下去呢?\\
  在私有财产社会里,夜间睡觉总是要关门的。大家知道,这不是为了多事,而是为了防贼。现在你们将大门敞开,不怕贼来吗?假使敞开大门而贼竟不来,却是什么缘故呢?\\
  照你们的说法,中国境内只有共产党是“破坏抗战”的,你们则是如何如何的“民族至上”,那末,背向敌人,却是什么至上呢?\\
  照你们的说法,“破坏团结”的也是共产党,你们则是如何如何的“精诚团结”主义者,那末,你们以三个集团军(缺一个军)的大兵,手持刺刀,配以重炮,向着边区人民前进,这也可以算作“精诚团结”吗?\\
  或者照你们的另一种说法,你们并不爱好什么团结,而却十分爱好“统一”,因此就要荡平边区,消灭你们所说的“封建割据”,杀尽共产党。那末,好吧,为什么你们不怕日本人把中华民族“统一”了去,并且也把你们混在一起“统一”了去呢?\\
  如果事变的结果,只是你们旗开得胜地“统一”了边区,削平了共产党,而日本人却被你们的什么“蒙汗药”蒙住了,或被什么“定身法”定住了,动弹不得,因此民族以及你们都不曾被他们“统一”了去,那末,我们的亲爱的国民党先生们,可否把你们的这种什么“蒙汗药”或“定身法”给我们宣示一二呢?\\
  假如你们也没有什么对付日本人的“蒙汗药”、“定身法”,又没有和日本人订立默契,那就让我们正式告诉你们吧:你们不应该打边区,你们不可以打边区。“鹬蚌相持,渔人得利”,“螳螂捕蝉,黄雀在后”,这两个故事,是有道理的。你们应该和我们一道去把日本占领的地方统一起来,把鬼子赶出去才是正经,何必急急忙忙地要来“统一”这块巴掌大的边区呢?大好河山,沦于敌手,你们不急,你们不忙,而却急于进攻边区,忙于打倒共产党,可痛也夫!可耻也夫!\\
  其次看国民党的党务。国民党为了反对共产党,办了几百个特务大队,其中什么乌龟忘八也收了进去。即如中华民国三十二年,亦即公历一九四三年,七月六日,抗战六周年纪念的前夕,中国国民党的中央通讯社,发出了这样一个消息,说是陕西省的西安地方,有些什么“文化团体”开了一个会,决定打电报给毛泽东,叫他趁着第三国际\footnote[1]{ 见本书第一卷《中国社会各阶级的分析》注〔5〕。}解散的时机,将中国共产党也“解散”,还有一条是“取消边区割据”。读者定会觉得这是一条“新闻”吧,其实却是一条旧闻。\\
  原来这件事出于几百个特务大队中的一个大队。它受了特务总队部(即“国民政府军事委员会调查统计局”和“中国国民党中央执行委员会调查统计局”)的指令,叫一个以在国民党出钱的汉奸刊物《抗战与文化》上写反共文章出名、现充西安劳动营训导处长的托派汉奸张涤非,于六月十二日那天,就是说还在中央社发表消息这天以前二十五天,就召集了九个人开了十分钟的会,“通过”了一纸所谓电文。\\
  这个电文,延安到今天还没有收到,但其内容已经明白,据说是第三国际既已解散,中国共产党也应“解散”,还有“马列主义已经破产”云云。\\
  这也是国民党人说的话儿呢!我们常常觉得,这一类(物以类聚)国民党人的嘴里,是什么东西也放得出来的,果不其然,于今又放出了一通好家伙!\\
  现在中国境内党派甚多,单单国民党就有两个。其中有一个叫汪记国民党\footnote[2]{ 一九三八年十二月汪精卫公开投降日本帝国主义后,于次年八月在上海秘密召开伪“中国国民党第六次全国代表大会”。会议推选汪精卫为中央执行委员会主席。一九四〇年,汪精卫伪国民党中央移驻南京。}的,立在南京以及各地,打的也是青天白日旗,也有一个什么中央执行委员会,也有一批特务大队。此外,还有日本法西斯党遍于沦陷区。\\
  我们的亲爱的国民党先生们,你们在第三国际解散之后所忙得不可开交的,单单就在于图谋“解散”共产党,但是偏偏不肯多少用些力量去解散若干汉奸党和日本党,这是什么缘故呢?当你们指使张涤非写电文时,何以不于要求解散共产党之外,附带说一句还有汉奸党和日本党也值得解散呢?\\
  难道你们以为共产党太多了吗?全中国境内共产党只有一个,国民党却有两个,究竟谁是多了的呢?\\
  国民党先生们,你们也曾想一想这件事吗?为什么除了你们之外,还有日本人和汪精卫,一致下死劲地要打倒共产党,一致地宣称只有共产党是太多了,因此要打倒;而国民党呢,却总是不觉得多,只觉得少,到处扶植养育着汪记国民党,这是什么缘故呢?\\
  国民党先生们,让我们不厌麻烦地告诉你们吧:日本人和汪精卫之所以特别爱好国民党和三民主义者,就是因为这个党这个主义当中有可以给他们利用的地方。这个党在第一次世界大战后,只有在一九二四年至一九二七年时期,孙中山先生把它改组了,把共产党人接受进去,形成了国共合作式的民族联盟,才被一切帝国主义者和汉奸们所痛恨,所不敢爱好,所极力图谋打倒。这个主义,也只有在同一时期,经过孙中山的手加以改造,成为载在《中国国民党第一次全国代表大会宣言》中的三民主义,即革命的三民主义,才被一切帝国主义者和汉奸们所痛恨,所不敢爱好,所极力图谋打倒。除此而外,这个党,这个主义,在排除了共产党、排除了孙中山革命精神的条件下,就受到一切帝国主义者和汉奸们的爱好,因此日本法西斯和汉奸汪精卫也爱好起来,如获至宝地加以养育,加以扶植。从前汪记国民党的旗子左角上还有一块黄色符号,以示区别,于今索性不要这个区别了,一切改成一样,以免碍眼。其爱好之程度为何如?\\
  不但在沦陷区,而且在大后方\footnote[3]{ 见本书第二卷《和中央社、扫荡报、新民报三记者的谈话》注〔3〕。},汪记国民党也是林立的。有些是秘密的,这就是敌人的第五纵队。有些是公开的,这就是那些吃党饭,吃特务饭,但是毫不抗日,专门反共的人们。这些人,表皮上没有标出汪记,实际上是汪记。这些人也是敌人的第五纵队,不过比前一种稍具形式上的区别,借以伪装自己,迷人眼目而已。\\
  至此,问题就完全明白了。当你们指示张涤非写电文时,所以绝对不肯在要求“解散”共产党之外附带说一句还有日本党和汉奸党也值得解散者,是由于不论在思想上,在政策上,在组织上,你们和他们之间,都有许多共同的地方,其中最基本的共同思想,就是反共反人民。\\
  还有一条要质问国民党人的,世界上以及中国境内,“破产”的只有一种马克思列宁主义,别的都是好家伙吗?汪精卫的三民主义前面已经说过了,希特勒、墨索里尼、东条英机的法西斯主义怎么样呢?张涤非的托洛茨基主义又怎么样呢?中国境内不论张记李记的反革命特务机关的反革命主义又怎么样呢?\\
  我们的亲爱的国民党先生们,你们指示张涤非写电文时,何以对于这样许多像瘟疫一样、像臭虫一样、像狗屎一样的所谓“主义”,连一个附笔或一个但书也没有呢?难道在你们看来,一切这些反革命的东西,都是完好无缺,十全十美,惟独一个马克思列宁主义就是“破产”干净了的吗?\\
  老实说吧,我们很疑心你们同那些日本党、汉奸党互相勾结,所以如此和他们一个鼻孔出气,所以说出的一些话,做出的一些事,如此和敌人汉奸一模一样,毫无二致,毫无区别。敌人汉奸要解散新四军,你们就解散新四军;敌人汉奸要解散共产党,你们也要解散共产党;敌人汉奸要取消边区,你们也要取消边区;敌人汉奸不希望你们保卫河防,你们就丢弃河防;敌人汉奸攻打边区(六年以来,绥德、米脂、佳县、吴堡、清涧一线对岸的敌军,炮击八路军所守河防阵地没有断过),你们也想攻打边区;敌人汉奸反共,你们也反共;敌人汉奸痛骂共产主义和自由主义,你们也痛骂共产主义和自由主义\footnote[4]{ 毛泽东这里是指蒋介石在一九四三年三月发表的《中国之命运》一书中,大肆攻击共产党、共产主义和自由主义,竭力宣扬买办的封建的法西斯主义。};敌人汉奸捉了共产党员强迫他们登报自首,你们也是捉了共产党员强迫他们登报自首;敌人汉奸派遣反革命特务分子偷偷摸摸地钻入共产党、八路军、新四军内施行破坏工作,你们也派遣反革命特务分子偷偷摸摸地钻入共产党、八路军、新四军内施行破坏工作。何其一模一样,毫无二致,毫无区别至于此极呢?你们的这样许多言论行动,既然和敌人汉奸的所有这些言论行动一模一样,毫无二致,毫无区别,怎么能够不使人们疑心你们和敌人汉奸互相勾结,或订立了某种默契呢?\\
  我们正式向中国国民党中央提出抗议:撤退河防大军,准备进攻边区,发动内战,这是一种极端错误的行为,是不能容许的。中央社于七月六日发出破坏团结、侮辱共产党的消息,这是一种极端错误的言论,也是不能容许的。这两种错误,都是滔天大罪的性质,都是和敌人汉奸毫无区别的,你们必须纠正这些错误。\\
  我们正式向中国国民党总裁蒋介石先生提出要求:请你下令把胡宗南的军队撤回河防,请你取缔中央社,并惩办汉奸张涤非。我们向一切不愿撤离河防进攻边区和不愿要求解散共产党的真正的爱国的国民党人呼吁:请你们行动起来,制止这个内战危机。我们愿意和你们合作到底,共同挽救民族于危亡。\\
  我们认为这些要求是完全正当的。\\
\newpage\section*{\myformat{开展根据地的减租、生产和拥政爱民运动}\\\myformat{(一九四三年十月一日)}}\addcontentsline{toc}{section}{开展根据地的减租、生产和拥政爱民运动}
\begin{introduction}\item  这是毛泽东为中共中央写的对党内的指示。\end{introduction}
(一)秋收已到,各根据地的领导机关必须责成各级党政机关检查减租政策的实行情况。凡未认真实行减租的,必须于今年一律减租。减而不彻底的,必须于今年彻底减租。党委应即根据中央土地政策和当地情况发出指示,并亲手检查几个乡村,发现模范,推动他处。同时,应在报纸上发表关于减租的社论和关于减租的模范经验的报道。减租是农民的群众斗争,党的指示和政府的法令是领导和帮助这个群众斗争,而不是给群众以恩赐。凡不发动群众积极性的恩赐减租,是不正确的,其结果是不巩固的。在减租斗争中应当成立农民团体,或改造农民团体。政府应当站在执行减租法令和调节东佃利益的立场上。现在根据地已经缩小,我党在根据地内细心地认真地彻底地争取群众、和群众同生死共存亡的任务,较之过去六年有更加迫切的意义。今秋如能检查减租政策的实施程度,并实行彻底减租,就能发扬农民群众的积极性,加强明年的对敌斗争,推动明年的生产运动。\\
  (二)敌后各根据地的大多数干部,还没有学会推动党政机关人员、军队人员和人民群众(一切公私军民男女老少,绝无例外)实行大规模的生产。党委、政府和军队,必须于今年秋冬准备好明年在全根据地内实行自己动手、克服困难(除陕甘宁边区外,暂不提丰衣足食口号)的大规模生产运动,包括公私农业、工业、手工业、运输业、畜牧业和商业,而以农业为主体。实行按家计划,劳动互助(陕北称变工队\footnote[1]{ 参见本卷《组织起来》注〔4〕。},过去江西红色区域称耕田队或劳动互助社\footnote[2]{ 见本书第一卷《我们的经济政策》注〔2〕。}),奖励劳动英雄,举行生产竞赛,发展为群众服务的合作社。县区党政工作人员在财政经济问题上,应以百分之九十的精力帮助农民增加生产,然后以百分之十的精力从农民取得税收。对前者用了苦功,对后者便轻而易举。一切机关学校部队,必须于战争条件下厉行种菜、养猪、打柴、烧炭、发展手工业和部分种粮。除各大小单位应一律发展集体生产外,同时奖励一切个人(军队除外)从事小部分农业和手工业的个人业余生产(禁止做生意),以其收入归个人所有。各地应开办七天至十天为期的种菜训练班、养猪训练班和为着改善伙食的炊事人员训练班。在一切党政军机关中讲究节省,反对浪费,禁止贪污。各级党政军机关学校一切领导人员都须学会领导群众生产的一全套本领。凡不注重研究生产的人,不算好的领导者。一切军民人等凡不注意生产反而好吃懒做的,不算好军人、好公民。一切未脱离生产的农村党员,应以发展生产为自己充当群众模范的条件之一。在生产运动中,不注重发展经济,只片面地在开支问题上打算盘的保守的单纯的财政观点,是错误的。不注重组织党政军群众和人民群众的广大劳动力,以开展群众生产运动,只片面地注意少数政府人员忙于收粮收税弄钱弄饭的观点,是错误的。不知用全力帮助群众发展生产,只知向群众要粮要款的观点(国民党观点),是错误的。不注意全面地发动群众生产运动,只注意片面地以少数经济机关组织少数人从事生产的观点,是错误的。把共产党员为着供给家庭生活(农村党员)和改善自己生活(机关学校党员)以利革命事业,而从事家庭生产和个人业余生产,认为不光荣不道德的观点,是错误的。在有根据地的条件下,不提倡发展生产并在发展生产的条件下为改善物质生活而斗争,只是片面地提倡艰苦奋斗的观点,是错误的。不把合作社看作为群众服务的经济团体,而把合作社看作为少数工作人员赚钱牟利,或看作政府公营商店的观点,是错误的。不把陕甘宁边区一些农业劳动英雄的模范劳动方法(劳动互助,多犁多锄多上粪)推行于各地,而说这些方法不能在某些根据地推行的观点,是错误的。不在生产运动中实行首长负责,自己动手,领导骨干和广大群众相结合,一般号召和个别指导相结合,调查研究,分别缓急轻重,争取男女老幼和游民分子一律参加生产,培养干部,教育群众,只知把生产任务推给建设厅长、供给部长、总务处长的观点,是错误的。在目前条件下,发展生产的中心关节是组织劳动力。每一根据地,组织几万党政军的劳动力和几十万人民的劳动力(取按家计划、变工队、运输队、互助社、合作社等形式,在自愿和等价的原则下,把劳动力和半劳动力组织起来)以从事生产,即在现时战争情况下,都是可能的和完全必要的。共产党员必须学会组织劳动力的全部方针和方法。今年全部根据地的一律彻底减租,将是明年大规模发展生产的一个刺激。而明年不论党政军民男女老幼全体一律进行伟大的生产运动,增加粮食和日用品,准备同灾荒作斗争,将是继续坚持抗日根据地的物质基础。否则便将遇到极大的困难。\\
  (三)为了使党政军和人民打成一片,以利于开展明年的对敌斗争和生产运动,各根据地党委和军政领导机关,应准备于明年阴历正月普遍地、无例外地举行一次拥政爱民和拥军优抗\footnote[3]{ 拥政爱民,是抗日根据地的军队人员“拥护政府、爱护人民”的口号的简称。拥军优抗,是抗日根据地的党政机关、群众团体的工作人员和人民群众“拥护军队、优待抗日军人家属”的口号的简称。}的广大规模的群众运动。军队方面,重新宣布拥政爱民公约,自己开检讨会,召集居民开联欢会(当地党政参加),有损害群众利益者,实行赔偿、道歉。群众方面,由当地党政和群众团体领导,重新宣布拥军优抗公约,举行热烈的劳军运动。在拥政爱民和拥军优抗的运动中,彻底检查军队方面和党政方面各自在一九四三年的缺点错误,而于一九四四年坚决改正之。以后应于每年正月普遍举行一次,再三再四地宣读拥政爱民公约和拥军优抗公约,再三再四地将各根据地曾经发生的军队欺压党政民和党政民关心军队不足的缺点错误,实行公开的群众性的自我批评(各方面只批评自己,不批评对方),而彻底改正之。\\
\newpage\section*{\myformat{评国民党十一中全会和三届二次国民参政会}\\\myformat{(一九四三年十月五日)}}\addcontentsline{toc}{section}{评国民党十一中全会和三届二次国民参政会}
\begin{introduction}\item  这是毛泽东为延安《解放日报》写的社论。\end{introduction}
九月六日至十三日国民党召集了十一中全会,九月十八日至二十七日国民党政府召集了三届二次国民参政会,两个会议的全部材料现已收齐,我们可以作一总评。\\
  国际局势已到了大变化的前夜,现在无论何方均已感到了这一变化。欧洲轴心国是感到了这一变化的;希特勒采取了最后挣扎的政策。这一变化主要地是苏联造成的。苏联正在利用这一变化:红军已经用席卷之势打到了第聂伯河;再一个冬季攻势,不打到新国界,也要打到旧国界。英美也正在利用这个变化:罗斯福、丘吉尔正在等待希特勒摇摇欲坠时打进法国去。总之,德国法西斯战争机构快要土崩瓦解了,欧洲反法西斯战争的问题已处在总解决的前夜,而消灭法西斯的主力军是苏联。世界反法西斯战争的问题的枢纽在欧洲;欧洲问题解决,就决定了世界法西斯和反法西斯两大阵线的命运。日本帝国主义者已感到走投无路,它的政策也只能是集中一切力量准备作最后挣扎。它对于中国,则是对共产党“扫荡”,对国民党诱降。\\
  国民党人亦感到了这个变化。他们在这一形势面前,一则以喜,一则以惧。喜的是他们以为欧洲解决,英美可以腾出手来替他们打日本,他们可以不费气力地搬回南京。惧的是三个法西斯国家一齐垮台,世界成了自有人类历史以来未曾有过的伟大解放时代,国民党的买办封建法西斯独裁政治,成了世界自由民主汪洋大海中一个渺小的孤岛,他们惧怕自己“一个党,一个主义,一个领袖”的法西斯主义有灭顶之灾。\\
  本来,国民党人的主意是叫苏联独力去拚希特勒,并挑起日寇去攻苏联,把个社会主义国家拚死或拚坏,叫英美不要在欧洲闹什么第二第三战场,而把全力搬到东方先把日本打垮,再把中国共产党打掉,然后再说其它。国民党人起初大嚷“先亚后欧论”,后来又嚷“欧亚平分论”,就是为了这个不可告人的目的。今年八月魁北克会议\footnote[1]{ 一九四三年八月,美国总统罗斯福和英国首相丘吉尔在加拿大的魁北克举行会议。这次会议就盟军于一九四四年在法国北部登陆和在东南亚、太平洋地区加强对日本作战等军事问题,进行磋商并达成协议。在会议的最后两天,中国外长宋子文代表蒋介石,参与了有关对日作战和有效援助中国问题的讨论。}的末尾,罗斯福和丘吉尔叫了国民党政府外交部长宋子文去,讲了几句话,国民党人又嚷“罗丘视线移到东方了,先欧后亚计划改变了”,以及“魁北克会议是英美中三强会议”之类,还要自卖自夸地乐一阵。但这已是国民党人的最后一乐。自此以后,他们的情绪就有些变化了,“先亚后欧”或“欧亚平分”从此送入历史博物馆,他们可能要另打主意了。国民党的十一中全会和国民党操纵的这次参政会,可能就是这种另打主意的起点。\\
  国民党十一中全会污蔑共产党“破坏抗战,危害国家”,同时又声言“政治解决”和“准备实行宪政”。三届二次国民参政会,在大多数国民党员把持操纵之下,通过了和十一中全会大体相同的对共决议案。此外,十一中全会还“选举”了蒋介石作国民党政府的主席,加强独裁机构。\\
  十一中全会后国民党人可能打什么主意呢?不外三种:(一)投降日本帝国主义;(二)照老路拖下去;(三)改变政治方针。\\
  国民党内的失败主义者和投降主义者,适应日本帝国主义“对共产党打,对国民党拉”的要求,是一路来主张投降的。他们时刻企图策动反共内战,只要内战一开,抗日自然就不可能,只有投降一条路走。国民党在西北集中了四十至五十万大军,现在还在由其它战场把军队偷偷地集中到西北。据说将军们的胆气是很豪的,他们说:“打下延安是不成问题的问题。”这是他们在国民党十一中全会上听了蒋介石先生所谓共产党问题“为一个政治问题,应用政治方法解决”的演说,和全会作了与蒋所说大体相同的决议之后说的话。去年国民党十中全会亦作了与此相同的决议,可是墨汁未干,将军们即奉命作成消灭边区的军事计划;今年六、七两月实行调兵遣将,准备对边区发动闪击战,仅因国内外舆论的反对,才把这一阴谋暂时搁下。现在十一中全会决议的墨汁刚刚洒在白纸上,将军们的豪语和兵力的调动又见告了。“打下延安是不成问题的问题”,这是什么意思呢?就是说决定投降日本帝国主义。一切赞成“打延安”的国民党人,不一定都是主观上打定了主意的投降主义者。他们中间有些人也许是这样想:我们一面反共,一面还是要抗日的。许多黄埔系军人\footnote[2]{ 这里是指国民党军队中蒋介石的嫡系将领和军官。他们中的绝大多数曾经是黄埔军校的学生,也包括一部分曾经担任过黄埔军校教官的人。}可能就是这样想。但是我们共产党人要向这些先生们发出一些问题:你们忘了十年内战的经验吗?内战一开,那些打定了主意的投降主义者们容许你们再抗日吗?日本人和汪精卫\footnote[3]{ 见本书第一卷《论反对日本帝国主义的策略》注〔31〕。}容许你们再抗日吗?你们自己究有多大本领,能够对内对外两面作战吗?你们现在名曰有三百万兵,实际上士气颓丧已极,有人比做一担鸡蛋,碰一下就要垮。所有中条山战役,太行山战役,浙赣战役,鄂西战役,大别山战役,无不如此。其所以然,就是因为你们实行“积极反共”、“消极抗日”两个要命的政策而来的。一个民族敌人深入国土,你们越是积极反共和消极抗日,你们的士气就越发颓丧。你们对外敌如此,难道你们对共产党对人民就能忽然凶起来吗?不能的。只要你们内战一开,你们就只能一心一意打内战,什么“一面抗战”必然抛到九霄云外,结果必然要同日本帝国主义订立无条件投降的条约,只能有一个“降”字方针。国民党中一切不愿意真正投降的人们,只要你们积极地发动了或参加了内战,你们就不可避免地要变为投降主义者。如果你们听信投降派的策动,把国民党十一中全会的决议和参政会的决议当作动员舆论、准备发动反共内战的工具,其结果必然要走到此种地步。即使自己本来不愿意投降,但若听信了投降派的策动,采取了错误的步骤,结果就只好跟着投降派投降。这是十一中全会后国民党的第一种可能的方向,这个危机极端严重地存在着。在投降派看来,“政治解决”和“准备实行宪政”,正是准备内战亦即准备投降的最好的掩眼法,一切共产党人、爱国的国民党人、各个抗日党派和一切抗日同胞,都要睁起眼睛注视这个极端严重的时局,不要被投降派的掩眼法弄昏了头脑。须知正是在国民党十一中全会之后,内战危机是空前未有的。\\
  国民党十一中全会的决议和参政会的决议可以向另一个方向发展,这就是“暂时拖,将来打”。这个方向和投降派的方向有多少的差别,这是在表面上还要维持抗日的局面、但又绝对不愿放弃反共和独裁的人们的方向。这些人们是可能采取此种方向的,那是因为他们看见国际大变化不可避免,看见日本帝国主义必然要失败,看见内战就是投降,看见国内人心拥护抗日、反对内战,看见国民党脱离群众、丧失人心、自己已处于从来未有的孤立地位这种严重的危机,看见美国、英国、苏联一致反对中国政府发动内战,因此迫得他们把内战阴谋推迟下去,而以“政治解决”和“准备实行宪政”的空话,作为拖下去的工具。这些人们历来的手段就是善于“骗”和“拖”。这些人们之想“打下延安”和“消灭共产党”是做梦也不会忘记的。在这一点上,他们和投降派毫无二致。只是他们还想打着抗日的招牌,还不愿丧失国民党的国际地位,有时也还顾虑到国际国内的舆论指摘,所以他们可能暂时地拖一下,而以“政治解决”和“准备实行宪政”作为拖一下的幌子,等待将来的有利条件。他们并无真正“政治解决”和“实行宪政”的诚意,至少现时他们绝无此种诚意。去年国民党十中全会前,共产党中央派了林彪同志去重庆会见蒋介石先生,在重庆等候了十个月之久,但是蒋先生和国民党中央连一个具体问题也不愿意谈。今年三月,蒋先生发表《中国之命运》一书,强调反对共产主义和自由主义,把十年内战的责任推在共产党身上,污蔑共产党、八路军、新四军为“新式军阀”、“新式割据”,暗示两年内一定要解决共产党。今年六月二十八日,蒋先生允许周恩来、林彪等同志回延安\footnote[4]{ 一九四三年六月四日,周恩来根据中共中央指示,在同张治中谈话中提出,因国共谈判暂搁,林彪决定回延安,自己也拟同返。六月七日,蒋介石同周、林会面,表示允许他们回延安。六月二十八日,周、林等乘卡车离重庆,七月十六日抵达延安。},但他就在这时下令调动河防兵力向边区前进,下令叫全国各地以“民众团体”之名,乘第三国际\footnote[5]{ 见本书第一卷《中国社会各阶级的分析》注〔5〕。}解散机会,要求解散中国共产党。在这种情况之下,我们共产党人乃不得不向国民党和全国人民呼吁制止内战,不得不将国民党各种破坏抗战危害国家的阴谋黑幕加以揭发。我们已忍耐到了极点,有历史事实为证。武汉失守以来,华北华中的大小反共战斗没有断过。太平洋战争爆发亦已两年,国民党即在华中华北打了共产党两年,除原有国民党军队外,又复派遣王仲廉、李仙洲两个集团军到江苏、山东打共产党。太行山庞炳勋集团军是受命专门反共的,安徽和湖北的国民党军队亦是受命反共的。所有这些,我们过去长期内连事实都没有公布。国民党一切大小报纸刊物无时无刻不在辱骂共产党,我们在长期内一个字也没有回答。国民党毫无理由地解散了英勇抗日的新四军,歼灭新四军皖南部队九千余人,逮捕叶挺,打死项英,囚系新四军干部数百人,这是背叛人民、背叛民族的滔天罪行,我们除向国民党提出抗议和善后条件外,仍然相忍为国。陕甘宁边区是一九三七年六、七月间共产党代表周恩来同志和蒋介石先生在庐山会见时,经蒋先生允许发布命令、委任官吏、作为国民政府行政院直辖行政区域的。蒋先生不但食言而肥,而且派遣四五十万军队包围边区,实行军事封锁和经济封锁,必欲置边区人民和八路军后方留守机关于死地而后快。至于断绝八路军接济,称共产党为“奸党”,称新四军为“叛军”,称八路军为“奸军”等等事实,更是尽人皆知。总之,凡干这些事的国民党人,是把共产党当作敌人看待的。在国民党看来,共产党是比日本人更加十倍百倍地可恨的。国民党把最大的仇恨集中在共产党;对于日本人,如果说还有仇恨,也只剩下极小的一部分。这和日本法西斯对待国共两党的不同态度是一致的。日本法西斯把最大的仇恨集中在中国共产党,对于国民党则一天一天地心平气和了,“反共”、“灭党”两个口号,于今只剩下一个“反共”了。一切日本的和汪精卫的报纸刊物,再也不提“打倒国民党”、“推翻蒋介石”这类口号了。日本把其在华兵力百分之五十八压在共产党身上,只把百分之四十二监视国民党;近来在浙江、湖北又撤退了许多军队,减少监视兵力,以利诱降。日本帝国主义不敢向共产党说出半句诱降的话,对于国民党则敢于连篇累牍,呶呶不休,劝其降顺。国民党只在共产党和人民面前还有一股凶气,在日本面前则一点儿也凶不起来了。不但在行动上早已由抗战改为观战,就是在言论上也不敢对日本帝国主义的诱降和各种侮辱言论做出一点两点稍为尖锐的驳斥。日本人说:“蒋介石所著《中国之命运》的论述方向是没有错误的。”蒋先生及其党人曾经对这话提出过任何驳斥吗?没有,也不敢有。日本帝国主义看见蒋先生和国民党只对共产党提出所谓“军令政令”和“纪律”,但对二十个投敌的国民党中委,五十八个投敌的国民党将领,却不愿也不敢提出军令政令和纪律问题,这叫日本帝国主义如何不轻视国民党呢!在全国人民和全世界友邦面前,只看见蒋先生和国民党解散新四军,进攻八路军,包围边区,诬之为“奸党”、“奸军”、“新式军阀”、“新式割据”,诬之为“破坏抗战”、“危害国家”,经常不断地提出所谓“军令政令”和“纪律”,而对于二十个投敌的国民党中委,五十八个投敌的国民党将领,却不执行任何的军令政令,不执行任何的纪律处分。即在此次十一中全会和国民参政会,也是依然只有对付共产党的决议,没有任何一件对付国民党自己大批叛国投敌的中央委员和大批叛国投敌的军事将领的决议,这叫全国人民和全世界友邦又如何看待国民党呢!十一中全会果然又有“政治解决”和“准备实行宪政”的话头了,好得很,我们是欢迎这些话头的。但据国民党多年来一贯的政治路线看来,我们认为这不过是一堆骗人的空话,而其实是为着准备打内战和永不放弃反人民的独裁政治这一目的,争取其所必要的时间。\\
  时局的发展是否还可以有第三种方向呢?可以有的,这在一部分国民党员、全国人民和我们共产党人,都是希望如此的。什么是第三种方向?那就是公平合理地用政治方式解决国共关系问题,诚意实行真正民主自由的宪政,废除“一个党,一个主义,一个领袖”的法西斯独裁政治,并在抗战期内召集真正民意选举的国民大会。我们共产党人是自始至终主张这个方针的。一部分国民党人也会同意这个方针。就连蒋介石先生及其嫡系国民党,我们过去长期地也总是希望他们实行这个方针。但是依据几年的实际情形看来,依据目前事实看来,蒋先生和大部分当权的国民党人都没有任何事实表示他们愿意实行这种方针。\\
  实行这种方针,要有国际国内许多条件。目前国际条件(欧洲法西斯总崩溃的前夜)是有利于中国抗日的,但投降派却更想在这时策动内战以便投降,日本人和汪精卫却更想在这时策动内战以利招降。汪精卫说:“最亲善的兄弟终久还是兄弟,重庆将来一定和我们走同一道路,但我们希望这一日期愈快愈好。”(十月一日同盟社\footnote[6]{ 同盟社是当时日本的官方通讯社。}消息)何其亲昵、肯定和迫切乃尔!所以,目前的时局,最佳不过是拖一下,而突然恶化的危险是很严重的。第三个方向的条件还不具备,需要各党各派的爱国分子和全国人民进行各方面的努力,才能争取到。\\
  蒋介石先生在国民党十一中全会上宣称:“应宣明中央对于共产党并无其它任何要求,只望其放弃武装割据及停止其过去各地袭击国军破坏抗战之行为,并望其实践二十六年共赴国难之宣言,履行诺言中所举之四点。”\\
  蒋先生所谓“袭击国军破坏抗战之行为”,应该是讲的国民党,可惜他偏心地和忍心地污蔑了共产党。因为自武汉失守以来,国民党发动了三次反共高潮,在这三次反共高潮中都有国民党军队袭击共产党军队的事实。第一次是在一九三九年冬季至一九四〇年春季,那时国民党军队袭占了陕甘宁边区八路军驻防的淳化、旬邑、正宁、宁县、镇原五城,并且使用了飞机。在华北,派遣朱怀冰袭击太行区域的八路军,而八路军仅仅为自卫而作战。第二次是在一九四一年一月。先是何应钦白崇禧以《皓电》(一九四〇年十月十九日)送达朱、彭、叶、项,强迫命令黄河以南的八路军新四军限期一个月一律开赴黄河以北。我们答应将皖南部队北移,其它部队则事实上无法移动,但仍答应在抗战胜利后移向指定的地点。不料正当皖南部队九千余人于一月四日遵命移动之时,蒋先生早已下了“一网打尽”的命令。自六日起十四日止,所有皖南国民党军队果然将该部新四军实行“一网打尽”,蒋先生并于十七日下令解散新四军全军,审判叶挺。自此以后,华中华北一切有国民党军队存在的抗日根据地内,所有那里的八路军新四军无不遭受国民党军队的袭击,而八路军新四军则只是自卫。第三次,是从本年三月至现在。除国民党军队在华中华北继续袭击八路军新四军外,蒋先生又发表了反共反人民的《中国之命运》一书;调动了大量河防部队准备闪击边区;发动了全国各地所谓“民众团体”要求解散共产党;动员了在国民参政会内占大多数的国民党员,接受何应钦污蔑八路军的军事报告,通过反共决议案,把一个表示团结抗日的国民参政会,变成了制造反共舆论准备国内战争的国民党御用机关,以至共产党参政员董必武同志不得不声明退席,以示抗议。总此三次反共高潮,都是国民党有计划有准备地发动的。请问这不是“破坏抗战之行为”是什么?\\
  中国共产党中央在民国二十六年(一九三七年)九月二十二日发表共赴国难宣言。该宣言称:“为着取消敌人阴谋之借口,为着解除一切善意的怀疑者之误会,中国共产党中央委员会有披沥自己对于民族解放事业的赤忱之必要。因此,中共中央再郑重向全国宣言:一、孙中山先生的三民主义\footnote[7]{ 见本书第一卷《湖南农民运动考察报告》注〔8〕。}为中国今日之必需,本党愿为其彻底实现而奋斗;二、停止一切推翻国民党政权的暴动政策和以暴力没收地主土地的政策;三、改组现在的红色政府为特区民主政府,以期全国政权之统一;四、改变红军名义及番号,改编为国民革命军,受国民政府军事委员会之统辖,并待命出动,担任抗日前线之职责。”\\
  所有这四条诺言,我们是完全实践了的,蒋介石先生和任何国民党人也不能举出任何一条是我们没有实践的。第一,所有陕甘宁边区和敌后各抗日根据地内共产党所施行的政策都符合于孙中山三民主义的政策,绝对没有任何一项政策是违背孙中山三民主义的。第二,在国民党不投降民族敌人、不破裂国共合作、不发动反共内战的条件之下,我们始终遵守不以暴力政策推翻国民党政权和没收地主土地的诺言。过去如此,现在如此,将来亦准备如此。这就是说,仅仅在国民党投降敌人、破裂合作、举行内战的条件下,我们才被迫着无法继续实践自己的诺言,因而只有在这种条件下,我们才失去继续实践诺言的可能性。第三,原来的红色政权还在抗战第一年就改组了,“三三制”\footnote[8]{ 见本书第二卷《论政策》注〔7〕。}的民主政治也早已实现了,只是国民党至今没有实践他们承认陕甘宁边区的诺言,并且还骂我们做“封建割据”。蒋介石先生及国民党人须知,陕甘宁边区和各抗日根据地这种不被国民党政府承认的状态,这种你们所谓“割据”,不是我们所愿意的,完全是你们迫得我们这样做的。你们食言而肥,不承认这个原来答应承认了的区域,不承认这个民主政治,反而骂我们做“割据”,请问这是一种什么道理?我们天天请求你们承认,你们却老是不承认,这个责任究竟应该谁负呢?蒋介石先生以国民党总裁和国民党政府负责人的身份,在其自己的《中国之命运》一书中也是这样乱骂“割据”,自己不负一点责任,这有什么道理呢?现在乘着蒋先生又在十一中全会上要求我们实践诺言的机会,我们就要求蒋先生实践这个诺言:采取法令手续,承认早已实现民权主义的陕甘宁边区,并承认敌后各抗日民主根据地。若是你们依然采取不承认主义,那就是你们叫我们继续“割据”下去,其责任和过去一样,完全在你们而不在我们。第四,“红军名义及番号”早已改变了,早已“改编为国民革命军”了,早已“受国民政府军事委员会统辖”了,这条诺言早已实践了。只有国民革命军新编第四军现在是直接受共产党中央统辖,不受国民政府军事委员会统辖,这是因为国民政府军事委员会于一九四一年一月十七日发表了一个破坏抗战危害国家的反革命命令,宣布该军为“叛军”而“解散”之,并使该军天天挨到国民党军队的袭击。但是该军不但始终在华中抗日,而且始终实践四条诺言中第一至第三条诺言,并且愿意复受“国民政府军事委员会之统辖”,要求蒋先生取消解散命令,恢复该军番号,使该军获得实践第四条诺言之可能性。\\
  国民党十一中全会关于共产党问题的文件除上述各点外,又称:“至于其它问题,本会议已决议于战争结束后一年内召集国民大会,制颁宪法,尽可于国民大会中提出讨论解决。”所谓“其它问题”,就是取消国民党的独裁政治,取消法西斯特务机关,实行全国范围内的民主政治,取消妨碍民生的经济统制和苛捐杂税,实行全国范围内的减租减息的土地政策,和扶助中小工业、改善工人生活的经济政策。二十六年九月二十二日我党共赴国难宣言中曾称:“实现民权政治,召开国民大会,以制订宪法与规定救国方针。实现中国人民之幸福与愉快的生活,首先须切实救济灾荒,安定民生,发展国防经济,解除人民痛苦,与改善人民生活。”蒋介石先生既于这个宣言发表之第二日(九月二十三日)发表谈话,承认这个宣言的全部,就应该不但要求共产党实践这个宣言中的四条诺言,也应该要求蒋先生自己及国民党和国民党政府实践上述条文。蒋先生现在不但是国民党的总裁,又当了国民党政府(这个政府以“国民政府”为表面名称)的主席,应该把上述民主民生的条文和一切蒋先生自己许给我们共产党人和全国人民的无数诺言,认真地实践起来,不要还是把任何诺言都抛到九霄云外,只是一味高压,讲的是一套,做的又是一套。我们共产党人和全国人民要看事实,不愿再听骗人的空话。如有事实,我们是欢迎的;如无事实,则空话是不能长久骗人的。抗战到底,制止投降危险,继续合作,制止内战危机,承认边区和敌后各抗日根据地的民主政治,恢复新四军,制止反共运动,撤退包围陕甘宁边区的四五十万军队,不要再把国民参政会当作国民党制造反共舆论的御用机关,开放言论集会结社自由,废止国民党一党专政,减租减息,改善工人待遇,扶助中小工业,取消特务机关,取消特务教育,实行民主教育,这就是我们对蒋先生和国民党的要求。其中大多数,正是你们自己的诺言。你们如能实行这些要求和诺言,则我们向你们保证继续实践我们自己的诺言。在蒋先生和国民党愿意的条件之下,我们愿意随时恢复两党的谈判。\\
  总之,在国民党可能采取的三个方向中,第一个,投降和内战的方向,对蒋介石先生和国民党是死路。第二个,以空言骗人,把时间拖下去,而暗中念念不忘法西斯独裁和积极准备内战的方向,对蒋先生和国民党也不是生路。只有第三个方向,根本放弃法西斯独裁和内战的错误道路,实行民主和合作的正确道路,才是蒋先生和国民党的生路。但是走这个方向,在蒋先生和国民党今天尚无任何的事实表示,还不能使任何人相信,因此,全国人民仍然要警戒极端严重的投降危险和内战危险。\\
  一切爱国的国民党人应该团结起来,不许国民党当局走第一个方向,不让它继续走第二个方向,要求它走第三个方向。\\
  一切爱国的抗日党派、抗日人民应该团结起来,不许国民党当局走第一个方向,不让它继续走第二个方向,要求它走第三个方向。\\
  前所未有的世界大变化的局面很快就要到来了,我们希望蒋介石先生和国民党人对于这样一个伟大的时代关节有以善处,我们希望一切爱国党派和爱国人民对于这样一个伟大的时代关节有以善处。\\
\newpage\section*{\myformat{组织起来}\\\myformat{(一九四三年十一月二十九日)}}\addcontentsline{toc}{section}{组织起来}
\begin{introduction}\item  这是毛泽东在中共中央招待陕甘宁边区劳动英雄大会上的讲话。\end{introduction}
今天共产党中央招待陕甘宁边区从农民群众中、工厂中、部队中、机关学校中选举出来的男女劳动英雄,以及在生产中的模范工作者,我代表中央来讲几句话。我想讲的意思,拿几个字来概括,就是“组织起来”。边区的农民群众和部队、机关、学校、工厂中的群众,根据去年冬天中共中央西北局所召集的高级干部会议的决议,今年进行了一年的生产运动。这一年的生产,在各方面都有了很大的成绩和很大的进步,边区的面目为之一新。事实已经完全证明:高级干部会议的方针是正确的。高级干部会议方针的主要点,就是把群众组织起来,把一切老百姓的力量、一切部队机关学校的力量、一切男女老少的全劳动力半劳动力,只要是可能的,就要毫无例外地动员起来,组织起来,成为一支劳动大军。我们有打仗的军队,又有劳动的军队。打仗的军队,我们有八路军新四军;这支军队也要当两支用,一方面打仗,一方面生产。我们有了这两支军队,我们的军队有了这两套本领,再加上做群众工作一项本领,那末,我们就可以克服困难,把日本帝国主义打垮。如果边区去年以前的生产运动的成绩还不够大,还不够显着,还不足以完全证明这一点,那末今年的成绩,就完全证明了这一点,这是大家亲眼看见了的。\\
  边区的军队,今年凡有地的,做到每个战士平均种地十八亩,吃的菜、肉、油,穿的棉衣、毛衣、鞋袜,住的窑洞、房屋,开会的大小礼堂,日用的桌椅板凳、纸张笔墨,烧的柴火、木炭、石炭,差不多一切都可以自己造,自己办。我们用自己动手的方法,达到了丰衣足食的目的。每个战士,一年中只需花三个月工夫从事生产,其余九个月时间都可以从事训练和作战。我们的军队既不要国民党政府发饷,也不要边区政府发饷,也不要老百姓发饷,完全由军队自己供给;这一个创造,对于我们的民族解放事业,该有多么重大的意义啊!抗日战争六年半中,敌人在各抗日根据地内实行烧、杀、抢的“三光”政策,陕甘宁边区则遭受国民党的重重封锁,财政上经济上处于非常困难的地位,我们的军队如果只会打仗,那是不能解决问题的。现在我们边区的军队已经学会了生产;前方的军队,一部分也学会了,其它部分正在开始学习。只要我们全体英勇善战的八路军新四军,人人个个不但会打仗,会作群众工作,又会生产,我们就不怕任何困难,就会是孟夫子说过的:“无敌于天下。”\footnote[1]{ 见《孟子•公孙丑上》。}我们的机关学校,今年也大进了一步,向政府领款只占经费的一小部分,由自己生产解决的占了绝大部分;去年还只自给蔬菜百分之五十,今年就自给了百分之一百;喂猪养羊大大增加了肉食;又开设了许多作坊生产日用品。部队机关学校既然自己解决了全部或大部的物质问题,用税收方法从老百姓手中取给的部分就减少了,老百姓生产的结果归自己享受的部分就增多了。军民两方大家都发展生产,大家都做到丰衣足食,大家都欢喜。还有我们的工厂,发展了生产,清查了特务,生产效率也大大提高了。整个边区,产生了许多农业劳动英雄、工业劳动英雄、机关学校劳动英雄,军队中也出了许多劳动英雄,边区的生产,可以说是走上了轨道。凡此,都是实行把群众力量组织起来的结果。\\
  把群众力量组织起来,这是一种方针。还有什么与此相反的方针没有呢?有的。那就是缺乏群众观点,不依靠群众,不组织群众,不注意把农村、部队、机关、学校、工厂的广大群众组织起来,而只注意组织财政机关、供给机关、贸易机关的一小部分人;不把经济工作看作是一个广大的运动,一个广大的战线,而只看作是一个用以补救财政不足的临时手段。这就是另外一种方针,这就是错误的方针。陕甘宁边区过去是存在过这种方针的,经过历年的指正,特别是经过去年的高级干部会议和今年的群众运动,大概现在还作这样错误想法的人是少了。华北华中各个根据地,因为战争紧张,也因为领导机关注意不够,群众的生产运动还没有广大的开展。但是在中央今年十月一号的指示\footnote[2]{ 见本卷《开展根据地的减租、生产和拥政爱民运动》。}以后,各个地方也都在准备发动明年的生产运动了。前方的条件,比陕甘宁边区更困难,不但有严重的战争,有些地方还有严重的灾荒。但是为了支持战争,为了对付敌人的“三光”政策,为了救济灾荒,就不能不动员全体党政军民,一面打击敌人,一面实行生产。前方的生产,过去几年已经有了一些经验,加上今年冬天的思想准备、组织准备和物质准备,明年可能造成广大的运动,并且必须造成广大的运动。前方处于战争环境,还不能做到“丰衣足食”,但是“自己动手,克服困难”,则是完全可以做到,并且必须做到的。\\
  目前我们在经济上组织群众的最重要形式,就是合作社。我们部队机关学校的群众生产,虽不要硬安上合作社的名目,但是这种在集中领导下用互相帮助共同劳动的方法来解决各部门各单位各个人物质需要的群众的生产活动,是带有合作社性质的。这是一种合作社。\\
  在农民群众方面,几千年来都是个体经济,一家一户就是一个生产单位,这种分散的个体生产,就是封建统治的经济基础,而使农民自己陷于永远的穷苦。克服这种状况的唯一办法,就是逐渐地集体化;而达到集体化的唯一道路,依据列宁所说,就是经过合作社\footnote[3]{ 参见列宁《论合作社》(《列宁全集》第43卷,人民出版社1987年版,第361—368页)。}。在边区,我们现在已经组织了许多的农民合作社,不过这些在目前还是一种初级形式的合作社,还要经过若干发展阶段,才会在将来发展为苏联式的被称为集体农庄的那种合作社。我们的经济是新民主主义的,我们的合作社目前还是建立在个体经济基础上(私有财产基础上)的集体劳动组织。这又有几种样式。一种是“变工队”、“扎工队”这一类的农业劳动互助组织\footnote[4]{ “变工队”和“扎工队”,都是陕甘宁边区建立在个体经济基础上的农业劳动互助组织。“变工”即换工,是农民相互间调剂劳动力的方法,有人工换人工、畜工换畜工、人工换畜工等等。参加变工队的农民,各以自己的劳动力或者畜力,轮流地给本队各家耕种。结算时,多出了人工或者畜工的由少出了的补给工钱。“扎工队”一般是由土地不足的农民组成。参加扎工队的农民,除相互变工互助以外,主要是集体出雇于需要劳动力的人家。},从前江西红色区域叫做劳动互助社,又叫耕田队\footnote[5]{ 见本书第一卷《我们的经济政策》注〔2〕。},现在前方有些地方也叫互助社。无论叫什么名称,无论每一单位的人数是几个人的,几十个人的,几百个人的,又无论单是由全劳动力组成的,或有半劳动力参加的,又无论实行互助的是人力、畜力、工具,或者在农忙时竟至集体吃饭住宿,也无论是临时性的,还是永久性的,总之,只要是群众自愿参加(决不能强迫)的集体互助组织,就是好的。这种集体互助的办法是群众自己发明出来的。从前我们在江西综合了群众的经验,这次我们在陕北又综合了这样的经验。经过去年高级干部会议的提倡,今年一年的实行,边区的劳动互助就大为条理化和更加发展了。今年边区有许多变工队,实行集体的耕种、锄草、收割,收成比去年多了一倍。群众看见了这样大的实效,明年一定有更多的人实行这个办法。我们现在不希望在明年一年就把全边区的几十万个全劳动力和半劳动力都组织到合作社里去,但是在几年之内是可能达到这个目的的。妇女群众也要全部动员参加一定分量的生产。所有二流子都要受到改造,参加生产,变成好人。在华北华中各抗日根据地内,都应该在群众自愿的基础上,广泛组织这种集体互助的生产合作社。\\
  除了这种集体互助的农业生产合作社以外,还有三种形式的合作社,这就是延安南区合作社式的包括生产合作、消费合作、运输合作(运盐)、信用合作的综合性合作社,运输合作社(运盐队)以及手工业合作社。\\
  我们有了人民群众的这四种合作社,和部队机关学校集体劳动的合作社,我们就可以把群众的力量组织成为一支劳动大军。这是人民群众得到解放的必由之路,由穷苦变富裕的必由之路,也是抗战胜利的必由之路。每一个共产党员,必须学会组织群众的劳动。知识分子出身的党员,也必须学会;只要有决心,半年一年工夫就可以学好的。他们可以帮助群众组织生产,帮助群众总结经验。我们的同志学会了组织群众的劳动,学会了帮助农民做按家生产计划,组织变工队,组织运盐队,组织综合性合作社,组织军队的生产,组织机关学校的生产,组织工厂的生产,组织生产竞赛,奖励劳动英雄,组织生产展览会,发动群众的创造力和积极性,加上旁的各项本领,我们就一定可以把日本帝国主义打出去,一定可以协同全国人民,把一个新国家建立起来。\\
  我们共产党员,无论在什么问题上,一定要能够同群众相结合。如果我们的党员,一生一世坐在房子里不出去,不经风雨,不见世面,这种党员,对于中国人民究竟有什么好处没有呢?一点好处也没有的,我们不需要这样的人做党员。我们共产党员应该经风雨,见世面;这个风雨,就是群众斗争的大风雨,这个世面,就是群众斗争的大世面。“三个臭皮匠,合成一个诸葛亮”,这就是说,群众有伟大的创造力。中国人民中间,实在有成千成万的“诸葛亮”,每个乡村,每个市镇,都有那里的“诸葛亮”。我们应该走到群众中间去,向群众学习,把他们的经验综合起来,成为更好的有条理的道理和办法,然后再告诉群众(宣传),并号召群众实行起来,解决群众的问题,使群众得到解放和幸福。如果我们做地方工作的同志脱离了群众,不了解群众的情绪,不能够帮助群众组织生产,改善生活,只知道向他们要救国公粮,而不知道首先用百分之九十的精力去帮助群众解决他们“救民私粮”的问题,然后仅仅用百分之十的精力就可以解决救国公粮的问题,那末,这就是沾染了国民党的作风,沾染了官僚主义的灰尘。国民党就是只问老百姓要东西,而不给老百姓以任何一点什么东西的。如果我们共产党员也是这样,那末,这种党员的作风就是国民党的作风,这种党员的脸上就堆上了一层官僚主义的灰尘,就得用一盆热水好好洗干净。我觉得,在无论哪一个抗日根据地的地方工作中,都存在有这种官僚主义的作风,都有一部分缺乏群众观点因而脱离群众的工作同志。我们必须坚决地克服这种作风,才能和群众亲密地结合起来。\\
  此外,在我们的军队工作中,还存在有一种军阀主义作风,这也是一种国民党的作风,因为国民党军队是脱离群众的。我们的军队必须在军民关系上、军政关系上、军党关系上、官兵关系上、军事工作和政治工作关系上、干部相互关系上,遵守正确的原则,决不可犯军阀主义的毛病。官长必须爱护士兵,不能漠不关心,不能采取肉刑;军队必须爱护人民,不能损害人民利益;军队必须尊重政府,尊重党,不能闹独立性。我们的八路军新四军是人民的军队,历来是好的,现在也是好的,是全国军队中一支最好的军队。但是近年来确实生长了一种军阀主义的毛病,一部分军队工作同志养成了一种骄气,对士兵,对人民,对政府,对党,横蛮不讲理,只责备做地方工作的同志,不责备自己,只看见成绩,不看见缺点,只爱听恭维话,不爱听批评话。例如陕甘宁边区,就有这种现象。经过去年的高级干部会议和军政干部会,又经过今年春节的拥政爱民运动和拥军运动,这个倾向是根本地克服下去了,还有一些残余,还必须继续去克服。华北华中各根据地内,这种毛病都是有的,那里的党和军队必须注意克服这种毛病。\\
  无论在地方工作中,在军队工作中,无论是官僚主义倾向或军阀主义倾向,其毛病的性质都是一样,就是脱离群众。我们的同志,绝对大多数都是好同志。对于有了毛病的人,一经展开批评,揭发错误,也就可以改正。但是必须开展自我批评,正视错误倾向,认真实行改正。如果在地方工作中不批评官僚主义倾向,在军队工作中不批评军阀主义倾向,那就是愿意保存国民党作风,愿意保存官僚主义灰尘和军阀主义灰尘在自己清洁的脸上,那就不是一个好党员。如果我们在地方工作中去掉官僚主义倾向,在军队工作中去掉军阀主义倾向,那就一切工作都会顺利地开展,生产运动当然也是这样。\\
  我们边区的生产,无论在农民群众方面、机关学校方面、军队方面、工厂方面,都得到了很大的成绩,在军民关系上也有了很大进步,边区的面目,和以前大不相同了。所有这些,都是我们的同志的群众观点已经加强,同群众的结合大进一步的表现。但是我们不应该自满,我们还要继续作自我批评,还要继续求进步。我们的生产也要继续求进步。我们脸上有灰尘,就要天天洗脸,地上有灰尘,就要天天扫地。尽管我们在地方工作中的官僚主义倾向,在军队工作中的军阀主义倾向,已经根本上克服了,但是这些恶劣倾向又可以生长起来的。我们是处在日本帝国主义和中国反动势力的层层包围之中,我们是处在散漫的小资产阶级的包围之中,极端恶浊的官僚主义灰尘和军阀主义灰尘天天都向我们的脸上大批地扑来。因此,我们决不能一见成绩就自满自足起来。我们应该抑制自满,时时批评自己的缺点,好像我们为了清洁,为了去掉灰尘,天天要洗脸,天天要扫地一样。\\
  各位劳动英雄和模范生产工作者,你们是人民的领袖,你们的工作是很有成绩的,我希望你们也不要自满。我希望你们回到关中去,回到陇东去,回到三边去,回到绥德去,回到延属各县去,回到机关学校部队工厂去,领导人民,领导群众,把工作做得更好,首先是按自愿的原则把群众组织到合作社里来,组织得更多,更好。希望你们回去实行这一条,宣传这一条,使明年再开劳动英雄大会的时候,我们能够得到更大的成绩。\\
\newpage\section*{\myformat{学习和时局}\\\myformat{(一九四四年四月十二日)}\\\myformat{附录:关于若干历史问题的决议}}\addcontentsline{toc}{section}{学习和时局}
\begin{introduction}\item  中国共产党中央领导机关和高级干部在一九四一年到一九四四年间,对于党的历史特别是党在一九三一年初到一九三四年底这个时期的历史所进行的讨论,大大地帮助了党内思想在马克思列宁主义基础上的统一。一九三五年一月中共中央在贵州省遵义城所召集的扩大的政治局会议虽然纠正了从一九三一年初到一九三四年底的“左”倾错误路线,改变了中共中央的领导机关的成分,确立了以毛泽东为代表的领导,把党的路线转到马克思列宁主义的正确轨道上,但是在党的很多干部中间,对于过去的错误路线的性质却没有作过彻底的清算。为着进一步地提高党的干部的马克思列宁主义思想,中共中央政治局在一九四一年到一九四三年这个时间内,曾经几次进行了关于党的历史的讨论;随后又在一九四三年到一九四四年这个时间内,领导全党高级干部进行同样的讨论。这个讨论为一九四五年召集的中国共产党第七次全国代表大会作了重要的准备,使那次大会达到了中国共产党前所未有的思想上政治上的一致。《学习和时局》就是毛泽东一九四四年四月十二日在延安高级干部会议上和五月二十日在中央党校第一部对于这个讨论所作的讲演。关于中共中央对于一九三一年初到一九三四年底的“左”倾机会主义路线的错误所作的详细结论,参看本篇附录中国共产党第六届中央委员会扩大的第七次全体会议通过的《关于若干历史问题的决议》。 \end{introduction}
\subsubsection*{\myformat{一}}
去年冬季开始,我党高级干部学习了党史中的两条路线问题。这次学习使广大高级干部的政治水平大大地提高了。在这次学习中,同志们提出了许多问题,中央政治局曾对其中的几个重要问题作了结论。这些结论是:\\
  (一)关于研究历史经验应取何种态度问题。中央认为应使干部对于党内历史问题在思想上完全弄清楚,同时对于历史上犯过错误的同志在作结论时应取宽大的方针,以便一方面,彻底了解我党历史经验,避免重犯错误;又一方面,能够团结一切同志,共同工作。我党历史上,曾经有过反对陈独秀错误路线\footnote[1]{ 见本书第一卷《中国革命战争的战略问题》注〔4〕。}和李立三错误路线\footnote[2]{ 罗章龙,一八九六年生,湖南浏阳人。一九二一年加入中国共产党。曾被选为中共中央委员、中央候补委员。一九三一年一月中共六届四中全会后,组织“中央非常委员会”,进行分裂党的活动,被开除党籍。}的大斗争,这些斗争是完全应该的。但其方法有缺点:一方面,没有使干部在思想上彻底了解当时错误的原因、环境和改正此种错误的详细办法,以致后来又可能重犯同类性质的错误;另一方面,太着重了个人的责任,未能团结更多的人共同工作。这两个缺点,我们应引为鉴戒。这次处理历史问题,不应着重于一些个别同志的责任方面,而应着重于当时环境的分析,当时错误的内容,当时错误的社会根源、历史根源和思想根源,实行惩前毖后、治病救人的方针,借以达到既要弄清思想又要团结同志这样两个目的。对于人的处理问题取慎重态度,既不含糊敷衍,又不损害同志,这是我们的党兴旺发达的标志之一。\\
  (二)对于任何问题应取分析态度,不要否定一切。例如对于四中全会\footnote[3]{ 见本书第一卷《论反对日本帝国主义的策略》注〔23〕。}至遵义会议\footnote[4]{ 参见本书第一卷《论反对日本帝国主义的策略》注〔32〕。}时期中央的领导路线问题,应作两方面的分析:一方面,应指出那个时期中央领导机关所采取的政治策略、军事策略和干部政策在其主要方面都是错误的;另一方面,应指出当时犯错误的同志在反对蒋介石、主张土地革命和红军斗争这些基本问题上面,和我们之间是没有争论的。即在策略方面也要进行分析。例如在土地问题上,当时的错误是实行了地主不分田、富农分坏田的过左政策,但在没收地主土地分给无地和少地的农民这一点上,则是和我们一致的。列宁说,对于具体情况作具体的分析,是“马克思主义的最本质的东西、马克思主义的活的灵魂”\footnote[5]{ 一九三〇年八九月间,红军第一方面军进攻长沙。当时因国民党军筑垒死守,又有飞机和军舰的援助,红军久攻不克,而敌人的援军已日渐集中,形成不利形势。毛泽东说服了红一方面军中的干部,撤退围攻长沙的部队,接着又说服了干部放弃夺取中心城市九江和攻打其它大城市的意见,改变方针,分兵攻取茶陵、攸县、醴陵、萍乡、吉安等地,使红一方面军获得很大的发展。}。我们许多同志缺乏分析的头脑,对于复杂事物,不愿作反复深入的分析研究,而爱作绝对肯定或绝对否定的简单结论。我们报纸上分析文章的缺乏,党内分析习惯的还没有完全养成,都表示这个毛病的存在。今后应该改善这种状况。\\
  (三)关于党的第六次全国代表大会\footnote[6]{ 瞿秋白(一八九九——一九三五),江苏常州人。一九二二年加入中国共产党,是党的早期领导人之一。在一九二五年至一九二八年中国共产党的第四次至第六次全国代表大会上,都被选为中央委员。第一次国内革命战争时期,积极反对国民党右派反共反人民的“戴季陶主义”和中国共产党内陈独秀的右倾投降主义。一九二七年国民党叛变革命后,同李维汉主持召集八月七日的中共中央紧急会议,结束了陈独秀右倾投降主义在党内的统治。但是在一九二七年冬至一九二八年春,他在担任中央领导工作中曾经犯过“左”的盲动主义的错误。一九三〇年九月,他同周恩来主持召集中共六届三中全会,停止了危害党的李立三“左”倾路线的执行。但是在一九三一年一月的中共六届四中全会上,他却受到“左”的教条主义宗派主义分子的打击,被排斥于中央领导机关之外。从这时到一九三三年的一个时期,他在上海同鲁迅合作从事革命文化运动。一九三四年二月到中央革命根据地,担任中华苏维埃共和国中央政府教育人民委员(教育部长)。红军主力长征时,他被留在中央根据地。一九三五年二月在福建游击区被国民党政府逮捕,六月十八日就义于福建长汀。}文件的讨论。应该指出,第六次全国代表大会的路线是基本上正确的,因为它确定了现时革命的资产阶级民主主义性质,确定了当时形势是处在两个革命高潮之间,批判了机会主义和盲动主义,发布了十大纲领\footnote[7]{ 林育南(一八九八——一九三一),湖北黄冈人。中国共产党党员,中国早期职工和青年运动的领导者和组织者之一。曾经担任中国劳动组合书记部武汉分部主任,青年团中央委员,团中央秘书、组织部长,中华全国总工会执行委员兼秘书长等职。一九三一年在上海被国民党政府逮捕,牺牲于龙华。}等,这些都是正确的。第六次全国代表大会亦有缺点,例如没有指出中国革命的极大的长期性和农村根据地在中国革命中的极大的重要性,以及还有其它若干缺点或错误。但无论如何,第六次全国代表大会在我党历史上是起了进步作用的。\\
  (四)关于一九三一年上海临时中央及在其后由此临时中央召开的五中全会\footnote[8]{ 李求实(一九〇三——一九三一),湖北武昌人。中国共产党党员。在一九二三年和一九二七年青年团的第二、第四次全国代表大会上,先后被选为候补中央委员和中央委员,曾任团中央宣传部长、南方局书记和团中央机关刊物《中国青年》主编等职。一九二九年到中共中央宣传部工作,创办党报《上海报》。一九三一年在上海被国民党政府逮捕,牺牲于龙华。}是否合法的问题。中央认为是合法的,但应指出其选举手续不完备,并以此作为历史教训。\\
  (五)关于党内历史上的宗派问题。应该指出,我党历史上存在过并且起了不良作用的宗派,经过遵义会议以来的几次变化,现在已经不存在了。这次党内两条路线的学习,指出这种宗派曾经在历史上存在过并起了不良作用,这是完全必要的。但是如果以为经过一九三五年一月的遵义会议,一九三八年十月的第六届中央委员会第六次全体会议\footnote[9]{ 何孟雄(一八九八——一九三一),湖南酃县人。中国共产党党员,中国早期北方职工运动的组织者之一,曾创建京绥铁路工会。一九二七年国民党叛变革命以后,曾任中共江苏省委委员、省农民运动委员会秘书等职。一九三一年在上海被国民党政府逮捕,牺牲于龙华。},一九四一年九月的政治局扩大会议\footnote[10]{ 秦邦宪(一九〇七——一九四六),又名博古,江苏无锡人。一九三一年九月至一九三五年一月,曾是中共临时中央和中共六届五中全会后中央的主要负责人。在这期间,犯过严重的“左”倾冒险主义的错误。遵义会议后,被撤销了党和红军的最高领导职务。抗日战争初期,先后在中共中央长江局、南方局工作。一九四一年以后,在毛泽东的领导下,在延安创办和主持《解放日报》和新华通讯社。在这期间,对自己过去的错误作了自我批评。一九四五年在党的第七次全国代表大会上,继续当选为中央委员。一九四六年二月到重庆参加同国民党谈判。四月八日在返回延安的途中因飞机失事遇难。},一九四二年的全党整风和一九四三年冬季开始的对于党内历史上两条路线斗争的学习\footnote[11]{ 指一九三一年十一月一日至五日在江西瑞金召开的中央苏区党的第一次代表大会,又称赣南会议。},这样许多次党内斗争的变化之后,还有具备原来的错误的政治纲领和组织形态的那种宗派存在,则是不对的。过去的宗派现在已经没有了。目前剩下的,只是教条主义和经验主义思想形态的残余,我们继续深入地进行整风学习,就可以将它们克服过来。目前在我们党内严重地存在和几乎普遍地存在的乃是带着盲目性的山头主义倾向\footnote[12]{ 一九三五年秋,在陕北革命根据地(包括陕甘边和陕北),“左”倾机会主义路线被贯彻到政治、军事、组织各方面工作中去,使执行正确路线的、创造了陕北红军和革命根据地的刘志丹等遭到排斥。接着在肃清反革命的工作中,一大批执行正确路线的干部又被逮捕,从而造成陕北革命根据地的严重危机。同年十月中共中央经过长征到达陕北后,纠正了这个“左”倾错误,将刘志丹等从监狱中释放出来,因而挽救了陕北革命根据地的危险局面。}。例如由于斗争历史不同、工作地域不同(这一根据地和那一根据地的不同,敌占区、国民党统治区和革命根据地的不同)和工作部门不同(这一部分军队和那一部分军队的不同,这一种工作和那一种工作的不同)而产生的各部分同志间互相不了解、不尊重、不团结的现象,看来好似平常,实则严重地妨碍着党的统一和妨碍着党的战斗力的增强。山头主义的社会历史根源,是中国小资产阶级的特别广大和长期被敌人分割的农村根据地,而党内教育不足则是其主观原因。指出这些原因,说服同志们去掉盲目性,增加自觉性,打通同志间的思想,提倡同志间的互相了解、互相尊重,以实现全党大团结,是我们当前的重要任务。\\
  以上所说的问题,如能在全党获得明确的理解,则不但可以保证这次党内学习一定得到成功,而且将保证中国革命一定得到胜利。\\
\subsubsection*{\myformat{二}}
目前时局有两个特点,一是反法西斯阵线的增强和法西斯阵线的衰落;二是在反法西斯阵线内部人民势力的增强和反人民势力的衰落。前一个特点是很明显的,容易被人们看见。希特勒不久就会被打败,日寇也已处在衰败过程中。后一个特点,比较地还不明显,还不容易被一般人看见,但是它已在欧洲、在英美、在中国一天一天显露出来。\\
  中国人民势力的增强,要以我党为中心来说明。\\
  我党在抗日时期的发展,可分为三个阶段。一九三七年至一九四〇年为第一个阶段。在此阶段的头两年内,即在一九三七年和一九三八年,日本军阀重视国民党,轻视共产党,故用其主要力量向国民党战线进攻,对它采取以军事打击为主、以政治诱降为辅的政策,而对共产党领导的抗日根据地则不重视,以为不过是少数共产党人在那里打些游击仗罢了。但是自一九三八年十月日本帝国主义者占领武汉以后,他们即已开始改变这个政策,改为重视共产党,轻视国民党;改为以政治诱降为主、以军事打击为辅的政策去对付国民党,而逐渐转移其主力来对付共产党。因为这时日本帝国主义者感觉国民党已不可怕,共产党则是可怕的了。国民党在一九三七年和一九三八年内,抗战是比较努力的,同我党的关系也比较好,对于人民抗日运动虽有许多限制,但也允许有较多的自由。自从武汉失守以后,由于战争失败和仇视共产党这种情绪的发展,国民党就逐渐反动,反共活动逐渐积极,对日抗战逐渐消极。共产党在一九三七年,因为在内战时期受了挫折的结果,仅有四万左右有组织的党员和四万多人的军队,因此为日本军阀所轻视。但到一九四〇年,党员已发展到八十万,军队已发展到近五十万,根据地人口包括一面负担粮税和两面负担粮税的\footnote[13]{ 见本书第一卷《论反对日本帝国主义的策略》注〔8〕。},约达一万万。几年内,我党开辟了一个广大的解放区战场,以至于能够停止日寇主力向国民党战场作战略进攻至五年半之久,将日军主力吸引到自己周围,挽救了国民党战场的危机,支持了长期的抗战。但在此阶段内,我党一部分同志,犯了一种错误,这种错误就是轻视日本帝国主义(因此不注意战争的长期性和残酷性,主张以大兵团的运动战为主,而轻视游击战争),依赖国民党,缺乏清醒的头脑和缺乏独立的政策(因此产生对国民党的投降主义,对于放手发动群众建立敌后抗日民主根据地和大量扩大我党领导的军队等项政策,发生了动摇)。同时,我党吸收了广大数目的新党员,他们还没有经验;一切敌后根据地也都是新创的,还没有巩固起来。这一阶段内,由于时局开展和党与军队的发展,党内又生长了一种骄气,许多人以为自己了不得了。在这一阶段内,我们曾经克服了党内的右倾偏向,执行了独立政策,不但打击了日本帝国主义,创立了根据地,发展了八路军新四军,而且打退了国民党的第一次反共高潮。\\
  一九四一年和一九四二年为第二阶段。日本帝国主义者为准备和执行反英美的战争,将他们在武汉失守以后已经改变了的方针,即由对国民党为主的方针改为对共产党为主的方针,更加强调起来,更加集中其主力于共产党领导的一切根据地的周围,进行连续的“扫荡”战争,实行残酷的“三光”政策\footnote[14]{ 参见本书第一卷《关于蒋介石声明的声明》注〔1〕。},着重地打击我党,致使我党在一九四一年和一九四二年这两年内处于极端困难的地位。这一阶段内,我党根据地缩小了,人口降到五千万以下,八路军也缩小到三十多万,干部损失很多,财政经济极端困难。同时,国民党又认为他们已经闲出手来,千方百计地反对我党,发动了第二次反共高潮,和日本帝国主义配合着进攻我们。但是这种困难地位教育了共产党人,使我们学到了很多东西。我们学会了如何反对敌人的“扫荡”战争、“蚕食”政策\footnote[15]{ 参见斯大林《中国革命问题》、《中国革命和共产国际的任务》第二部分(《斯大林全集》第9卷,人民出版社1954年版,第199—207、259—267页)和《论中国革命的前途》(《斯大林选集》上卷,人民出版社1979年版,第483—495页)。}、“治安强化”运动\footnote[16]{ 见本书第一卷《湖南农民运动考察报告》等文。}、“三光”政策和自首政策;我们学会了或开始学会了统一战线政权的“三三制”\footnote[17]{ 见本书第一卷《井冈山的斗争》一文的《革命性质问题》部分。}政策、土地政策、整顿三风、精兵简政\footnote[18]{ 见《星星之火,可以燎原》(本书第1卷第103页)。}、统一领导、拥政爱民、发展生产等项工作,克服了许多缺点,并且把第一阶段内许多人自以为了不得的那股骄气也克服下去了。这一阶段内,我们虽然受了很大的损失,但是我们站住脚了,一方面打退了日寇的进攻,一方面又打退了国民党的第二次反共高潮。因为国民党的反共和我们不得不同国民党的反共政策作自卫斗争的这些情况,党内又生长了一种过左的偏向,例如以为国共合作就要破裂,因而过分地打击地主,不注意团结党外人士等。但是这些过左偏向,也被我们克服过来了。我们指出了在反磨擦斗争中的有理有利有节的原则,指出了在统一战线中又团结又斗争和以斗争求团结的必要,保持了国内和根据地内的抗日民族统一战线。\\
  一九四三年到现在为第三阶段。我们的各项政策更为见效,特别是整顿三风和发展生产这样两项工作,发生了根本性质的效果,使我党在思想基础和物质基础两方面,立于不败之地。此外,我们又在去年学会了或开始学会了审查干部和反对特务的政策。在这种情形下,我们根据地的面积又扩大了,根据地的人口,包括一面负担和两面负担的,又已上升到八千余万,军队又有了四十七万,民兵二百二十七万,党员发展到了九十多万。\\
  一九四三年,日本军阀对中国的政策没有什么变化,还是以打击共产党为主。从一九四一年至今这三年多以来,百分之六十以上的在华日军是压在我党领导的各个抗日根据地身上。三年多以来,国民党留在敌后的数十万军队经不起日本帝国主义的打击,约有一半投降了敌人,约有一半被敌人消灭,残存的和撤走的为数极少。这些投降敌人的国民党军队反过来进攻我党,我党又要担负抗击百分之九十以上的伪军。国民党只担负抗击不到百分之四十的日军和不到百分之十的伪军。从一九三八年十月武汉失守起,整整五年半时间,日本军阀没有举行过对国民党战场的战略进攻,只有几次较大的战役行动(浙赣、长沙、鄂西、豫南、常德),也是早出晚归,而集中其主要注意力于我党领导的抗日根据地。在此情况下,国民党采取上山政策和观战政策,敌人来了招架一下,敌人退了袖手旁观。一九四三年国民党的国内政策更加反动,发动了第三次反共高潮,但是又被我们打退了。\\
  一九四三年,直至今年春季,日寇在太平洋战线逐渐失利,美国的反攻增强了,西方的希特勒在苏联红军严重打击之下有摇摇欲倒之势。为救死计,日本帝国主义想到要打通平汉、粤汉两条铁路;又以其对重庆国民党的诱降政策还没有得到结果,有给它以再一次打击之必要,故有在今年大举进攻国民党战线的计划。河南战役\footnote[19]{ 见本书第一卷《中国的红色政权为什么能够存在?》、《井冈山的斗争》等文。}已打了一个多月。敌人不过几个师团,国民党几十万军队不战而溃,只有杂牌军还能打一下。汤恩伯部官脱离兵,军脱离民,混乱不堪,损失三分之二以上。胡宗南派到河南的几个师,也是一触即溃。这种情况,完全是几年来国民党厉行反动政策的结果。自武汉失守以来的五年半中,共产党领导的解放区战场担负了抗击日伪主力的重担,这在今后虽然可能发生某些变化,但这种变化也只能是暂时的,因为国民党在五年半以来消极抗日、积极反共的反动政策下所养成的极端腐化状态,今后必将遭到严重的挫败,到了那时,我党抗击敌伪的任务又将加重了。国民党以五年半的袖手旁观,得到了丧失战斗力的结果。共产党以五年半的苦战奋斗,得到了增强战斗力的结果。这一种情况,将决定今后中国的命运。\\
  同志们可以看见,一九三七年七月起至现在止,这七年时间内,在我党领导下的人民民主力量曾经历了上升、下降、再上升三种情况。我党抗击了日寇的残酷进攻,建立了广大的革命根据地,大大地发展了党和军队,打退了国民党的三次大规模的反共高潮,克服了党内发生的右的和“左”的错误思想,全党学得了许多可宝贵的经验。这些就是我们七年工作的总结。\\
  现在的任务是要准备担负比较过去更为重大的责任。我们要准备不论在何种情况下把日寇打出中国去。为使我党能够担负这种责任,就要使我党我军和我们的根据地更加发展和更加巩固起来,就要注意大城市和交通要道的工作,要把城市工作和根据地工作提到同等重要的地位。\\
  关于根据地工作,第一阶段内有大的发展,但是不巩固,因此在第二阶段内一受到敌人的严重打击,就缩小了。在第二阶段内,一切我党领导的抗日根据地都受到了严格的锻炼,比起第一阶段来好得多了;干部和党员的思想和政策的水平,大进一步,没有学会的东西,学会得更多了。但是思想的打通和政策的学习还需要时间,我们还有许多没有学会的东西。我党力量还不够强大,党内还不够统一,还不够巩固,因此还不能担负比较目前更为巨大的责任。今后的问题就是在继续抗战中使我党我军和我们的根据地更加发展和更加巩固,这就是为着将来担负巨大工作的第一个必要的思想准备和物质准备。没有这种准备,我们就不能把日寇赶出去,就不能解放全中国。\\
  关于大城市和交通要道的工作,我们一向是做得很差的。如果现在我们还不争取在大城市和交通要道中被日本帝国主义压迫的千百万劳动群众和市民群众围绕在我党的周围,并准备群众的武装起义,我们的军队和农村根据地就会得不到城市的配合而遇到种种困难。我们十多年来是处在农村中,提倡熟悉农村和建设农村根据地,这是必要的。党的第六次全国代表大会决定的准备城市起义的任务,没有也不可能在这十多年中间去实行。但是现在不同了,第六次全国代表大会的决议要在第七次全国代表大会以后实行了。我党的第七次全国代表大会不久就可能开会,这次代表大会将要讨论加强城市工作和争取全国胜利的问题。\\
  这几天陕甘宁边区召开工业会议,是有重大意义的。一九三七年边区还只有七百个工厂工人,一九四二年有了四千人,现在有了一万二千人。切不可轻视这样的数目字。我们要在根据地内学习好如何管理大城市的工商业和交通机关,否则到了那时将无所措手足。准备大城市和交通要道的武装起义,并且学习管理工商业,这是第二个必要的思想准备和物质准备。没有这种准备,我们也就不能把日寇赶出去,也就不能解放全中国。\\
\subsubsection*{\myformat{三}}
为了争取新的胜利,要在党的干部中间提倡放下包袱和开动机器。所谓放下包袱,就是说,我们精神上的许多负担应该加以解除。有许多的东西,只要我们对它们陷入盲目性,缺乏自觉性,就可能成为我们的包袱,成为我们的负担。例如:犯过错误,可以使人觉得自己反正是犯了错误的,从此萎靡不振;未犯错误,也可以使人觉得自己是未犯过错误的,从此骄傲起来。工作无成绩,可以使人悲观丧气;工作有成绩,又可以使人趾高气扬。斗争历史短的,可以因其短而不负责任;斗争历史长的,可以因其长而自以为是。工农分子,可以自己的光荣出身傲视知识分子;知识分子,又可以自己有某些知识傲视工农分子。各种业务专长,都可以成为高傲自大轻视旁人的资本。甚至年龄也可以成为骄傲的工具,青年人可以因为自己聪明能干而看不起老年人,老年人又可以因为自己富有经验而看不起青年人。对于诸如此类的东西,如果没有自觉性,那它们就会成为负担或包袱。有些同志高高在上,脱离群众,屡犯错误,背上了这类包袱是一个重要的原因。所以,检查自己背上的包袱,把它放下来,使自己的精神获得解放,实在是联系群众和少犯错误的必要前提之一。我党历史上曾经有过几次表现了大的骄傲,都是吃了亏的。第一次是在一九二七年上半年。那时北伐军到了武汉,一些同志骄傲起来,自以为了不得,忘记了国民党将要袭击我们。结果犯了陈独秀路线的错误,使这次革命归于失败。第二次是在一九三〇年。红军利用蒋冯阎大战\footnote[20]{ 一九三三年一月十七日,中华苏维埃临时中央政府、工农红军革命军事委员会发表宣言,向一切进攻革命根据地的国民党军队提议,在三个条件下订立停战协定,联合抗日。三个条件是:(一)停止进攻革命根据地,(二)保证民众的民主权利,(三)武装民众。}的条件,打了一些胜仗,又有一些同志骄傲起来,自以为了不得。结果犯了李立三路线的错误,也使革命力量遭到一些损失。第三次是在一九三一年。红军打破了第三次“围剿”,接着全国人民在日本进攻面前发动了轰轰烈烈的抗日运动,又有一些同志骄傲起来,自以为了不得。结果犯了更严重的路线错误,使辛苦地聚集起来的革命力量损失了百分之九十左右。第四次是在一九三八年。抗战起来了,统一战线建立了,又有一些同志骄傲起来,自以为了不得,结果犯了和陈独秀路线有某些相似的错误。这一次,又使得受这些同志的错误思想影响最大的那些地方的革命工作,遭到了很大的损失。全党同志对于这几次骄傲,几次错误,都要引为鉴戒。近日我们印了郭沫若论李自成的文章\footnote[21]{ 参见本书第一卷《论反对日本帝国主义的策略》注〔17〕。},也是叫同志们引为鉴戒,不要重犯胜利时骄傲的错误。\\
  所谓开动机器,就是说,要善于使用思想器官。有些人背上虽然没有包袱,有联系群众的长处,但是不善于思索,不愿用脑筋多想苦想,结果仍然做不成事业。再有一些人则因为自己背上有了包袱,就不肯使用脑筋,他们的聪明被包袱压缩了。列宁斯大林经常劝人要善于思索,我们也要这样劝人。脑筋这个机器的作用,是专门思想的。孟子说:“心之官则思。”\footnote[22]{ 见本书第一卷《中国共产党在抗日时期的任务》注〔2〕。}他对脑筋的作用下了正确的定义。凡事应该用脑筋好好想一想。俗话说:“眉头一皱,计上心来。”就是说多想出智慧。要去掉我们党内浓厚的盲目性,必须提倡思索,学会分析事物的方法,养成分析的习惯。这种习惯,在我们党内是太不够了。如果我们既放下了包袱,又开动了机器,既是轻装,又会思索,那我们就会胜利。\\
注释\\
\footnote[1]{ 见本书第一卷《中国革命战争的战略问题》注〔4〕。} 见本书第一卷《中国革命战争的战略问题》注〔4〕。\\
\footnote[2]{ 罗章龙,一八九六年生,湖南浏阳人。一九二一年加入中国共产党。曾被选为中共中央委员、中央候补委员。一九三一年一月中共六届四中全会后,组织“中央非常委员会”,进行分裂党的活动,被开除党籍。} 见本书第一卷《中国革命战争的战略问题》注〔5〕。\\
\footnote[3]{ 见本书第一卷《论反对日本帝国主义的策略》注〔23〕。} 四中全会指一九三一年一月七日在上海召开的中国共产党第六届中央委员会第四次全体会议。陈绍禹等人在共产国际及其代表米夫的支持下,通过这次会议取得了在中共中央的领导地位,开始了长达四年之久的“左”倾冒险主义在党内的统治。参见本文附录《关于若干历史问题的决议》第三部分。\\
\footnote[4]{ 参见本书第一卷《论反对日本帝国主义的策略》注〔32〕。} 见本书第一卷《中国革命战争的战略问题》注〔7〕。\\
\footnote[5]{ 一九三〇年八九月间,红军第一方面军进攻长沙。当时因国民党军筑垒死守,又有飞机和军舰的援助,红军久攻不克,而敌人的援军已日渐集中,形成不利形势。毛泽东说服了红一方面军中的干部,撤退围攻长沙的部队,接着又说服了干部放弃夺取中心城市九江和攻打其它大城市的意见,改变方针,分兵攻取茶陵、攸县、醴陵、萍乡、吉安等地,使红一方面军获得很大的发展。} 见本书第一卷《中国革命战争的战略问题》注〔11〕。\\
\footnote[6]{ 瞿秋白(一八九九——一九三五),江苏常州人。一九二二年加入中国共产党,是党的早期领导人之一。在一九二五年至一九二八年中国共产党的第四次至第六次全国代表大会上,都被选为中央委员。第一次国内革命战争时期,积极反对国民党右派反共反人民的“戴季陶主义”和中国共产党内陈独秀的右倾投降主义。一九二七年国民党叛变革命后,同李维汉主持召集八月七日的中共中央紧急会议,结束了陈独秀右倾投降主义在党内的统治。但是在一九二七年冬至一九二八年春,他在担任中央领导工作中曾经犯过“左”的盲动主义的错误。一九三〇年九月,他同周恩来主持召集中共六届三中全会,停止了危害党的李立三“左”倾路线的执行。但是在一九三一年一月的中共六届四中全会上,他却受到“左”的教条主义宗派主义分子的打击,被排斥于中央领导机关之外。从这时到一九三三年的一个时期,他在上海同鲁迅合作从事革命文化运动。一九三四年二月到中央革命根据地,担任中华苏维埃共和国中央政府教育人民委员(教育部长)。红军主力长征时,他被留在中央根据地。一九三五年二月在福建游击区被国民党政府逮捕,六月十八日就义于福建长汀。} 见本书第一卷《星星之火,可以燎原》注〔11〕。\\
\footnote[7]{ 林育南(一八九八——一九三一),湖北黄冈人。中国共产党党员,中国早期职工和青年运动的领导者和组织者之一。曾经担任中国劳动组合书记部武汉分部主任,青年团中央委员,团中央秘书、组织部长,中华全国总工会执行委员兼秘书长等职。一九三一年在上海被国民党政府逮捕,牺牲于龙华。} 参见本书第一卷《论反对日本帝国主义的策略》注〔32〕。\\
\footnote[8]{ 李求实(一九〇三——一九三一),湖北武昌人。中国共产党党员。在一九二三年和一九二七年青年团的第二、第四次全国代表大会上,先后被选为候补中央委员和中央委员,曾任团中央宣传部长、南方局书记和团中央机关刊物《中国青年》主编等职。一九二九年到中共中央宣传部工作,创办党报《上海报》。一九三一年在上海被国民党政府逮捕,牺牲于龙华。} 五中全会指一九三四年一月在江西瑞金召开的中国共产党第六届中央委员会第五次全体会议。这次会议错误地断定中国已存在“直接革命形势”,第五次反“围剿”“即是争取中国革命完全胜利的斗争”,使“左”倾错误发展到顶点。参见本文附录《关于若干历史问题的决议》第三部分。\\
\footnote[9]{ 何孟雄(一八九八——一九三一),湖南酃县人。中国共产党党员,中国早期北方职工运动的组织者之一,曾创建京绥铁路工会。一九二七年国民党叛变革命以后,曾任中共江苏省委委员、省农民运动委员会秘书等职。一九三一年在上海被国民党政府逮捕,牺牲于龙华。} 中国共产党第六届中央委员会第六次扩大的全体会议于一九三八年九月二十九日至十一月六日在延安举行。会上,毛泽东作了《论新阶段》的政治报告和会议结论,要求全党同志认真地负起领导抗日战争的重大历史责任。全会坚持抗日民族统一战线的政策,批判了关于统一战线问题上的右倾投降主义错误,确立了全党独立自主地领导抗日武装斗争的方针,把党的主要工作方面放在战区和敌后。会议强调全党必须自上而下地努力学习马克思列宁主义理论,善于把马克思列宁主义和国际经验应用于中国的具体环境,反对教条主义。\\
\footnote[10]{ 秦邦宪(一九〇七——一九四六),又名博古,江苏无锡人。一九三一年九月至一九三五年一月,曾是中共临时中央和中共六届五中全会后中央的主要负责人。在这期间,犯过严重的“左”倾冒险主义的错误。遵义会议后,被撤销了党和红军的最高领导职务。抗日战争初期,先后在中共中央长江局、南方局工作。一九四一年以后,在毛泽东的领导下,在延安创办和主持《解放日报》和新华通讯社。在这期间,对自己过去的错误作了自我批评。一九四五年在党的第七次全国代表大会上,继续当选为中央委员。一九四六年二月到重庆参加同国民党谈判。四月八日在返回延安的途中因飞机失事遇难。} 一九四一年九月至十月,中共中央政治局举行扩大会议,检讨了党的历史上特别是第二次国内革命战争后期的政治路线问题。毛泽东在会上作了重要讲话,明确提出反对主观主义和宗派主义。这次会议为一九四二年全党整风运动的开展,作了重要的准备。\\
\footnote[11]{ 指一九三一年十一月一日至五日在江西瑞金召开的中央苏区党的第一次代表大会,又称赣南会议。} 一九四二年的全党整风,指中国共产党自一九四二年起在全党范围内开展的一个马克思列宁主义的思想教育运动。主要内容是:反对主观主义以整顿学风,反对宗派主义以整顿党风,反对党八股以整顿文风。一九四三年十月,中共中央决定党的高级干部重新学习和研究党的历史和路线是非问题,使整风运动进入总结提高的阶段。经过这个运动,全党进一步地掌握了马克思列宁主义的普遍真理与中国革命的具体实践的统一这样一个基本方向。\\
\footnote[12]{ 一九三五年秋,在陕北革命根据地(包括陕甘边和陕北),“左”倾机会主义路线被贯彻到政治、军事、组织各方面工作中去,使执行正确路线的、创造了陕北红军和革命根据地的刘志丹等遭到排斥。接着在肃清反革命的工作中,一大批执行正确路线的干部又被逮捕,从而造成陕北革命根据地的严重危机。同年十月中共中央经过长征到达陕北后,纠正了这个“左”倾错误,将刘志丹等从监狱中释放出来,因而挽救了陕北革命根据地的危险局面。} 山头主义倾向是一种小团体主义的倾向,主要是在长期的游击战争中,农村革命根据地的分散和彼此间不相接触的情况下产生的。这些根据地开始多半是建立在山岳地区,一个集团好像一个山头,所以这种错误倾向被称为山头主义。\\
\footnote[13]{ 见本书第一卷《论反对日本帝国主义的策略》注〔8〕。} 这里所说的一面负担粮税的地区,是指根据地的比较巩固的地区,那里的人民只向抗日民主政府负担粮税;两面负担粮税的地区,是指根据地的边缘地区和游击区,在那些地区因为敌人可以经常来骚扰,人民除向抗日民主政府负担粮税外,还经常被迫向敌伪政权缴纳一些粮税。\\
\footnote[14]{ 参见本书第一卷《关于蒋介石声明的声明》注〔1〕。} “三光”政策指日本帝国主义对抗日根据地实施的烧光、杀光、抢光的政策。\\
\footnote[15]{ 参见斯大林《中国革命问题》、《中国革命和共产国际的任务》第二部分(《斯大林全集》第9卷,人民出版社1954年版,第199—207、259—267页)和《论中国革命的前途》(《斯大林选集》上卷,人民出版社1979年版,第483—495页)。} 日本侵略军在其妄想迅速“鲸吞”抗日根据地的计划破产后,于一九四一年初开始实行“蚕食”政策,即依托其所占领的交通线和据点,从抗日根据地边缘逐渐向内推进;或以“扫荡”为先导,深入抗日根据地内建立据点,并由这些据点逐步向外扩张。日本侵略军企图以长期的、逐步的、稳扎稳打的办法,达到缩小抗日根据地、扩大其占领区的目的。\\
\footnote[16]{ 见本书第一卷《湖南农民运动考察报告》等文。} 自一九四一年春至一九四二年冬,日本侵略军在华北地区连续进行了五次“治安强化”运动,对抗日根据地加紧军事“扫荡”和经济封锁;在游击区建立伪军,加强控制;在其占领区内实行保甲制度,调查户口,扩组伪军,进行奴化教育,以镇压抗日力量。\\
\footnote[17]{ 见本书第一卷《井冈山的斗争》一文的《革命性质问题》部分。} 见本书第二卷《论政策》注〔7〕。\\
\footnote[18]{ 见《星星之火,可以燎原》(本书第1卷第103页)。} 见本卷《一个极其重要的政策》注〔1〕。\\
\footnote[19]{ 见本书第一卷《中国的红色政权为什么能够存在?》、《井冈山的斗争》等文。} 一九四四年四月至五月,日本侵略军为打通平汉铁路南段的交通,以十余万人的兵力,发起河南战役。国民党军蒋鼎文、汤恩伯、胡宗南部约四十万人,在日本侵略军的进攻面前望风而逃,郑州、洛阳等三十八个县市相继陷落,汤恩伯部损失了二十多万人。\\
\footnote[20]{ 一九三三年一月十七日,中华苏维埃临时中央政府、工农红军革命军事委员会发表宣言,向一切进攻革命根据地的国民党军队提议,在三个条件下订立停战协定,联合抗日。三个条件是:(一)停止进攻革命根据地,(二)保证民众的民主权利,(三)武装民众。} 蒋冯阎大战指一九三〇年爆发的蒋介石同冯玉祥、阎锡山之间的大规模军阀战争。这次战争从五月正式开始,至十月基本上结束,历时半年,战区在河南、山东、安徽等省的陇海、津浦、平汉各铁路沿线,双方共死伤三十万人以上。\\
\footnote[21]{ 参见本书第一卷《论反对日本帝国主义的策略》注〔17〕。} 指郭沫若《甲申三百年祭》一文。该文作于一九四四年,纪念明朝末年李自成领导的农民起义军进入北京推翻明王朝三百周年。文中说明一六四四年李自成的农民起义军进入北京以后,它的一些首领因为胜利而骄傲起来,生活腐化,进行宗派斗争,以致这次起义在一六四五年陷于失败。这篇文章先在重庆《新华日报》发表,后来在延安《解放日报》转载,并且在各解放区印成单行本。\\
\footnote[22]{ 见本书第一卷《中国共产党在抗日时期的任务》注〔2〕。} 见《孟子•告子上》。\\
附录:关于若干历史问题的决议\\
(一九四五年四月二十日中国共产党第六届中央委员会扩大的第七次全体会议通过)\\
\subsection*{\myformat{(一)}}
中国共产党自一九二一年产生以来,就以马克思列宁主义的普遍真理和中国革命的具体实践相结合为自己一切工作的指针,毛泽东同志关于中国革命的理论和实践便是此种结合的代表。我们党一成立,就展开了中国革命的新阶段——毛泽东同志所指出的新民主主义革命的阶段。在为实现新民主主义而进行的二十四年(一九二一年至一九四五年)的奋斗中,在第一次大革命、土地革命和抗日战争的三个历史时期中,我们党始终一贯地领导了广大的中国人民,向中国人民的敌人——帝国主义和封建主义,进行了艰苦卓绝的革命斗争,取得了伟大的成绩和丰富的经验。党在奋斗的过程中产生了自己的领袖毛泽东同志。毛泽东同志代表中国无产阶级和中国人民,将人类最高智慧——马克思列宁主义的科学理论,创造地应用于中国这样的以农民为主要群众、以反帝反封建为直接任务而又地广人众、情况极复杂、斗争极困难的半封建半殖民地的大国,光辉地发展了列宁斯大林关于殖民地半殖民地问题的学说和斯大林关于中国革命问题的学说。由于坚持了正确的马克思列宁主义的路线,并向一切与之相反的错误思想作了胜利的斗争,党才在三个时期中取得了伟大的成绩,达到了今天这样在思想上、政治上、组织上的空前的巩固和统一,发展为今天这样强大的革命力量,有了一百二十余万党员,领导了拥有近一万万人民、近一百万军队的中国解放区,形成为全国人民抗日战争和解放事业的伟大的重心。\\
\subsection*{\myformat{(二)}}
在中国新民主主义革命的第一个时期中,在一九二一年至一九二七年,特别是在一九二四年至一九二七年,中国人民的反帝反封建的大革命,曾经在共产国际的正确指导之下,在中国共产党的正确领导的影响、推动和组织之下,得到了迅速的发展和伟大的胜利。中国共产党的全体同志,在这次大革命中,进行了轰轰烈烈的革命工作,发展了全国的工人运动、青年运动和农民运动,推进并帮助了国民党的改组和国民革命军的建立,形成了东征和北伐的政治上的骨干,领导了全国反帝反封建的伟大斗争,在中国革命史上写下了极光荣的一章。但是,由于当时的同盟者国民党内的反动集团在一九二七年叛变了这个革命,由于当时帝国主义和国民党反动集团的联合力量过于强大,特别是由于在这次革命的最后一个时期内(约有半年时间),党内以陈独秀为代表的右倾思想,发展为投降主义路线,在党的领导机关中占了统治地位,拒绝执行共产国际和斯大林同志的许多英明指示,拒绝接受毛泽东同志和其它同志的正确意见,以至于当国民党叛变革命,向人民突然袭击的时候,党和人民不能组织有效的抵抗,这次革命终于失败了。\\
  从一九二七年革命失败至一九三七年抗日战争爆发的十年间,中国共产党,并且只有中国共产党,在反革命的极端恐怖的统治下,全党团结一致地继续高举着反帝反封建的大旗,领导广大的工人、农民、士兵、革命知识分子和其它革命群众,作了政治上、军事上和思想上的伟大战斗。在这个战斗中,中国共产党创造了红军,建立了工农兵代表会议的政府,建立了革命根据地,分配了土地给贫苦的农民,抗击了当时国民党反动政府的进攻和一九三一年“九一八”以来的日本帝国主义的侵略,使中国人民的新民主主义的民族解放和社会解放的事业,取得了伟大的成绩。全党对于企图分裂党和实行叛党的托洛茨基陈独秀派\footnote[1]{ 见本书第一卷《中国革命战争的战略问题》注〔4〕。}和罗章龙\footnote[2]{ 罗章龙,一八九六年生,湖南浏阳人。一九二一年加入中国共产党。曾被选为中共中央委员、中央候补委员。一九三一年一月中共六届四中全会后,组织“中央非常委员会”,进行分裂党的活动,被开除党籍。}、张国焘\footnote[3]{ 见本书第一卷《论反对日本帝国主义的策略》注〔23〕。}等的反革命行为,也同样团结一致地进行了斗争,使党保证了在马克思列宁主义总原则下的统一。在这十年内,党的这个总方针和为实行这个总方针的英勇奋斗,完全是正确的和必要的。无数党员、无数人民和很多党外革命家,当时在各个战线上轰轰烈烈地进行革命斗争,他们的奋斗牺牲、不屈不挠、前仆后继的精神和功绩,在民族的历史上永垂不朽。假如没有这一切,则抗日战争即不能实现;即使实现,亦将因为没有一个积蓄了人民战争丰富经验的中国共产党作为骨干,而不能坚持和取得胜利。这是毫无疑义的。\\
  尤其值得我们庆幸的是,我们党以毛泽东同志为代表,创造性地把马克思、恩格斯、列宁、斯大林的革命学说应用于中国条件的工作,在这十年内有了很大的发展。我党终于在土地革命战争的最后时期,确立了毛泽东同志在中央和全党的领导。这是中国共产党在这一时期的最大成就,是中国人民获得解放的最大保证。\\
  但是我们必须指出,在这十年内,我党不仅有了伟大的成就,而且在某些时期中也犯过一些错误。其中以从党的一九三一年一月第六届中央委员会第四次全体会议(六届四中全会)到一九三五年一月扩大的中央政治局会议(遵义会议)这个时期内所犯政治路线、军事路线和组织路线上的“左”倾错误,最为严重。这个错误,曾经给了我党和中国革命以严重的损失。\\
  为了学习中国革命的历史教训,以便“惩前毖后,治病救人”,使“前车之覆”成为“后车之鉴”,在马克思列宁主义思想一致的基础上,团结全党同志如同一个和睦的家庭一样,如同一块坚固的钢铁一样,为着获得抗日战争的彻底胜利和中国人民的完全解放而奋斗,中国共产党第六届中央委员会扩大的第七次全体会议(扩大的六届七中全会)认为:对于这十年内若干党内历史问题,尤其是六届四中全会至遵义会议期间中央的领导路线问题,作出正式的结论,是有益的和必要的。\\
\subsection*{\myformat{(三)}}
一九二七年革命失败后,在党内曾经发生了“左”、右倾的偏向。\\
  以陈独秀为代表的一小部分第一次大革命时期的投降主义者,这时对于革命前途悲观失望,逐渐变成了取消主义者。他们采取了反动的托洛茨基主义立场,认为一九二七年革命后中国资产阶级对于帝国主义和封建势力已经取得了胜利,它对于人民的统治已趋稳定,中国社会已经是所谓资本主义占优势并将得到和平发展的社会;因此他们武断地说中国资产阶级民主革命已经完结,中国无产阶级只有待到将来再去举行“社会主义革命”,在当时就只能进行所谓以“国民会议”为中心口号的合法运动,而取消革命运动;因此他们反对党所进行的各种革命斗争,并污蔑当时的红军运动为所谓“流寇运动”。他们不但不肯接受党的意见,放弃这种机会主义的取消主义的反党观点,而且还同反动的托洛茨基分子相结合,成立了反党的小组织,因而不得不被驱逐出党,接着并堕落为反革命。\\
  另一方面,由于对国民党屠杀政策的仇恨和对陈独秀投降主义的愤怒而加强起来的小资产阶级革命急性病,也反映到党内,使党内的“左”倾情绪也很快地发展起来了。这种“左”倾情绪在一九二七年八月七日党中央的紧急会议(八七会议)上已经开端。八七会议在党的历史上是有功绩的。它在中国革命的危急关头坚决地纠正了和结束了陈独秀的投降主义,确定了土地革命和武装反抗国民党反动派屠杀政策的总方针,号召党和人民群众继续革命的战斗,这些都是正确的,是它的主要方面。但是八七会议在反对右倾错误的时候,却为“左”倾错误开辟了道路。它在政治上不认识当时应当根据各地不同情况,组织正确的反攻或必要的策略上的退却,借以有计划地保存革命阵地和收集革命力量,反而容许了和助长了冒险主义和命令主义(特别是强迫工人罢工)的倾向。它在组织上开始了宗派主义的过火的党内斗争,过分地或不适当地强调了领导干部的单纯的工人成分的意义,并造成了党内相当严重的极端民主化状态。这种“左”倾情绪在八七会议后继续生长,到了一九二七年十一月党中央的扩大会议,就形成为“左”倾的盲动主义(即冒险主义)路线,并使“左”倾路线第一次在党中央的领导机关内取得了统治地位。这时的盲动主义者认为,中国革命的性质是所谓“不断革命”(混淆民主革命和社会主义革命),中国革命的形势是所谓“不断高涨”(否认一九二七年革命的失败),因而他们仍然不但不去组织有秩序的退却,反而不顾敌人的强大和革命失败后的群众情况,命令少数党员和少数群众在全国组织毫无胜利希望的地方起义。和这种政治上的冒险主义同时,组织上的宗派主义的打击政策也发展了起来。但是由于这个错误路线一开始就引起了毛泽东同志和在白色区域工作的许多同志的正确的批评和非难,并在实际工作中招致了许多损失,到了一九二八年初,这个“左”倾路线的执行在许多地方已经停止,而到同年四月(距“左”倾路线的开始不到半年时间),就在全国范围的实际工作中基本上结束了。\\
  一九二八年六、七月间召开的党的第六次全国代表大会的路线,基本上是正确的。它正确地肯定了中国社会是半殖民地半封建社会,指出了引起现代中国革命的基本矛盾一个也没有解决,因此确定了中国现阶段的革命依然是资产阶级民主革命,并发布了民主革命的十大纲领\footnote[4]{ 参见本书第一卷《论反对日本帝国主义的策略》注〔32〕。}。它正确地指出了当时的政治形势是在两个革命高潮之间,指出了革命发展的不平衡,指出了党在当时的总任务不是进攻,不是组织起义,而是争取群众。它进行了两条战线的斗争,批判了右的陈独秀主义和“左”的盲动主义,特别指出了党内最主要的危险倾向是脱离群众的盲动主义、军事冒险主义和命令主义。这些都是完全必要的。另一方面,第六次大会也有其缺点和错误。它对于中间阶级的两面性和反动势力的内部矛盾,缺乏正确的估计和政策;对于大革命失败后党所需要的策略上的有秩序的退却,对于农村根据地的重要性和民主革命的长期性,也缺乏必要的认识。这些缺点和错误,虽然使得八七会议以来的“左”倾思想未能根本肃清,并被后来的“左”倾思想所片面发展和极端扩大,但仍然不足以掩盖第六次大会的主要方面的正确性。党在这次大会以后一个时期内的工作,是有成绩的。毛泽东同志在这个时期内,不但在实践上发展了第六次大会路线的正确方面,并正确地解决了许多为这次大会所不曾解决或不曾正确地解决的问题,而且在理论上更具体地和更完满地给了中国革命的方向以马克思列宁主义的科学根据。在他的指导和影响之下,红军运动已经逐渐发展成为国内政治的重要因素。党在白色区域的组织和工作,也有了相当的恢复。\\
  但是,在一九二九年下半年至一九三〇年上半年间,还在党内存在着的若干“左”倾思想和“左”倾政策,又有了某些发展。在这个基础上,遇着时局的对革命有利的变动,便发展成为第二次的“左”倾路线。在一九三〇年五月蒋冯阎战争爆发后的国内形势的刺激下,党中央政治局由李立三同志领导,在六月十一日通过了“左”倾的《新的革命高潮与一省或数省的首先胜利》决议案,使“左”倾路线第二次统治了中央的领导机关。产生这次错误路线(李立三路线)的原因,是由于李立三同志等不承认革命需要主观组织力量的充分准备,认为“群众只要大干,不要小干”,因而认为当时不断的军阀战争,加上红军运动的初步发展和白区工作的初步恢复,就已经是具备了可以在全国“大干”(武装起义)的条件;由于他们不承认中国革命的不平衡性,认为革命危机在全国各地都有同样的生长,全国各地都要准备马上起义,中心城市尤其要首先发动以形成全国革命高潮的中心,并污蔑毛泽东同志在长期中用主要力量去创造农村根据地,以农村来包围城市,以根据地来推动全国革命高潮的思想,是所谓“极端错误的”“农民意识的地方观念与保守观念”;由于他们不承认世界革命的不平衡性,认为中国革命的总爆发必将引起世界革命的总爆发,而中国革命又必须在世界革命的总爆发中才能成功;由于他们不承认中国资产阶级民主革命的长期性,认为一省数省首先胜利的开始即是向社会主义革命转变的开始,并因此规定了若干不适时宜的“左”倾政策。在这些错误认识下,立三路线的领导者定出了组织全国中心城市武装起义和集中全国红军进攻中心城市的冒险计划;随后又将党、青年团、工会的各级领导机关,合并为准备武装起义的各级行动委员会,使一切经常工作陷于停顿。在这些错误决定的形成和执行过程中,立三同志拒绝了许多同志的正确的批评和建议,并在党内强调地反对所谓“右倾”,在反“右倾”的口号下错误地打击了党内不同意他的主张的干部,因而又发展了党内的宗派主义。这样,立三路线的形态,就比第一次“左”倾路线更为完备。\\
  但是立三路线在党内的统治时间也很短(不到四个月时间)。因为凡实行立三路线的地方都使党和革命力量受到了损失,广大的干部和党员都要求纠正这一路线。特别是毛泽东同志,他不但始终没有赞成立三路线,而且以极大的忍耐心纠正了红一方面军中的“左”倾错误\footnote[5]{ 一九三〇年八九月间,红军第一方面军进攻长沙。当时因国民党军筑垒死守,又有飞机和军舰的援助,红军久攻不克,而敌人的援军已日渐集中,形成不利形势。毛泽东说服了红一方面军中的干部,撤退围攻长沙的部队,接着又说服了干部放弃夺取中心城市九江和攻打其它大城市的意见,改变方针,分兵攻取茶陵、攸县、醴陵、萍乡、吉安等地,使红一方面军获得很大的发展。},因而使江西革命根据地的红军在这个时期内不但没有受到损失,反而利用了当时蒋冯阎战争的有利形势而得到了发展,并在一九三〇年底至一九三一年初胜利地粉碎了敌人的第一次“围剿”。其它革命根据地的红军,除个别地区外,也得到了大体相同的结果。在白区,也有许多做实际工作的同志,经过党的组织起来反对立三路线。\\
  一九三〇年九月党的第六届中央委员会第三次全体会议(六届三中全会)及其后的中央,对于立三路线的停止执行是起了积极作用的。虽然六届三中全会的文件还表现了对立三路线调和妥协的精神(如否认它是路线错误,说它只是“策略上的错误”等),虽然六届三中全会在组织上还继续着宗派主义的错误,但是六届三中全会既然纠正了立三路线对于中国革命形势的极左估计,停止了组织全国总起义和集中全国红军进攻中心城市的计划,恢复了党、团、工会的独立组织和经常工作,因而它就结束了作为立三路线主要特征的那些错误。立三同志本人,在六届三中全会上也承认了被指出的错误,接着就离开了中央的领导地位。六届三中全会后的中央,又在同年十一月的补充决议和十二月的第九十六号通告中,进一步地指出了立三同志等的路线错误和六届三中全会的调和错误。当然,无论六届三中全会或其后的中央,对于立三路线的思想实质,都没有加以清算和纠正,因此一九二七年八七会议以来特别是一九二九年以来一直存在于党内的若干“左”倾思想和“左”倾政策,在六届三中全会上和六届三中全会后还是浓厚地存在着。但是六届三中全会及其后的中央既然对于停止立三路线作了上述有积极作用的措施,当时全党同志就应该在这些措施的基础上继续努力,以求反“左”倾错误的贯彻。\\
  但在这时,党内一部分没有实际革命斗争经验的犯“左”倾教条主义错误的同志,在陈绍禹(王明)同志的领导之下,却又在“反对立三路线”、“反对调和路线”的旗帜之下,以一种比立三路线更强烈的宗派主义的立场,起来反抗六届三中全会后的中央了。他们的斗争,并不是在帮助当时的中央彻底清算立三路线的思想实质,以及党内从八七会议以来特别是一九二九年以来就存在着而没有受到清算的若干“左”倾思想和“左”倾政策;在当时发表的陈绍禹同志的《两条路线》即《为中共更加布尔什维克化而斗争》的小册子中,实际上是提出了一个在新的形态下,继续、恢复或发展立三路线和其它“左”倾思想“左”倾政策的新的政治纲领。这样,“左”倾思想在党内就获得了新的滋长,而形成为新的“左”倾路线。\\
  陈绍禹同志领导的新的“左”倾路线虽然也批评了立三路线的“左”倾错误和六届三中全会的调和错误,但是它的特点,是它主要地反而批评了立三路线的“右”,是它指责六届三中全会“对立三路线的一贯右倾机会主义的理论与实际,未加以丝毫揭破和打击”,指责第九十六号通告没有看出“右倾依然是目前党内主要危险”。新的“左”倾路线在中国社会性质、阶级关系的问题上,夸大资本主义在中国经济中的比重,夸大中国现阶段革命中反资产阶级斗争、反富农斗争和所谓“社会主义革命成分”的意义,否认中间营垒和第三派的存在。在革命形势和党的任务问题上,它继续强调全国性的“革命高潮”和党在全国范围的“进攻路线”,认为所谓“直接革命形势”很快地即将包括一个或几个有中心城市在内的主要省份。它并从“左”的观点污蔑中国当时还没有“真正的”红军和工农兵代表会议政府,特别强调地宣称当时党内的主要危险是所谓“右倾机会主义”、“实际工作中的机会主义”和“富农路线”。在组织上,这条新的“左”倾路线的代表者们违反组织纪律,拒绝党所分配的工作,错误地结合一部分同志进行反中央的宗派活动,错误地在党员中号召成立临时的中央领导机关,要求以“积极拥护和执行”这一路线的“斗争干部”“来改造和充实各级的领导机关”等,因而造成了当时党内的严重危机。这样,虽然新的“左”倾路线并没有主张在中心城市组织起义,在一个时期内也没有主张集中红军进攻中心城市,但是整个地说来,它却比立三路线的“左”倾更坚决,更“有理论”,气焰更盛,形态也更完备了。\\
  一九三一年一月,党在这些以陈绍禹同志为首的“左”的教条主义宗派主义分子从各方面进行压迫的情势之下,也在当时中央一部分犯经验主义错误的同志对于他们实行妥协和支持的情势之下,召开了六届四中全会。这次会议的召开没有任何积极的建设的作用,其结果就是接受了新的“左”倾路线,使它在中央领导机关内取得胜利,而开始了土地革命战争时期“左”倾路线对党的第三次统治。六届四中全会直接实现了新的“左”倾路线的两项互相联系的错误纲领:反对所谓“目前党内主要危险”的“右倾”,和“改造充实各级领导机关”。尽管六届四中全会在形式上还是打着反立三路线、反“调和路线”的旗帜,它的主要政治纲领实质上却是“反右倾”。六届四中全会虽然在它自己的决议上没有作出关于当时政治形势的分析和党的具体政治任务的规定,而只是笼统地反对所谓“右倾”和所谓“实际工作中的机会主义”;但是在实际上,它是批准了那个代表着当时党内“左”倾思想,即在当时及其以后十多年内还继续被人们认为起过“正确的”“纲领作用”的陈绍禹同志的小册子——《两条路线》即《为中共更加布尔什维克化而斗争》;而这个小册子,如前面所分析的,基本上乃是一个完全错误的“反右倾”的“左”倾机会主义的总纲领。在这个纲领下面,六届四中全会及其后的中央,一方面提拔了那些“左”的教条主义和宗派主义的同志到中央的领导地位,另一方面过分地打击了犯立三路线错误的同志,错误地打击了以瞿秋白\footnote[6]{ 瞿秋白(一八九九——一九三五),江苏常州人。一九二二年加入中国共产党,是党的早期领导人之一。在一九二五年至一九二八年中国共产党的第四次至第六次全国代表大会上,都被选为中央委员。第一次国内革命战争时期,积极反对国民党右派反共反人民的“戴季陶主义”和中国共产党内陈独秀的右倾投降主义。一九二七年国民党叛变革命后,同李维汉主持召集八月七日的中共中央紧急会议,结束了陈独秀右倾投降主义在党内的统治。但是在一九二七年冬至一九二八年春,他在担任中央领导工作中曾经犯过“左”的盲动主义的错误。一九三〇年九月,他同周恩来主持召集中共六届三中全会,停止了危害党的李立三“左”倾路线的执行。但是在一九三一年一月的中共六届四中全会上,他却受到“左”的教条主义宗派主义分子的打击,被排斥于中央领导机关之外。从这时到一九三三年的一个时期,他在上海同鲁迅合作从事革命文化运动。一九三四年二月到中央革命根据地,担任中华苏维埃共和国中央政府教育人民委员(教育部长)。红军主力长征时,他被留在中央根据地。一九三五年二月在福建游击区被国民党政府逮捕,六月十八日就义于福建长汀。}同志为首的所谓犯“调和路线错误”的同志,并在六届四中全会后接着就错误地打击了当时所谓“右派”中的绝大多数同志。其实,当时的所谓“右派”,主要地是六届四中全会宗派主义的“反右倾”斗争的产物。这些人中间也有后来成为真正右派并堕落为反革命而被永远驱逐出党的以罗章龙为首的极少数的分裂主义者,对于他们,无疑地是应该坚决反对的;他们之成立并坚持第二党的组织,是党的纪律所绝不容许的。至于林育南\footnote[7]{ 林育南(一八九八——一九三一),湖北黄冈人。中国共产党党员,中国早期职工和青年运动的领导者和组织者之一。曾经担任中国劳动组合书记部武汉分部主任,青年团中央委员,团中央秘书、组织部长,中华全国总工会执行委员兼秘书长等职。一九三一年在上海被国民党政府逮捕,牺牲于龙华。}、李求实\footnote[8]{ 李求实(一九〇三——一九三一),湖北武昌人。中国共产党党员。在一九二三年和一九二七年青年团的第二、第四次全国代表大会上,先后被选为候补中央委员和中央委员,曾任团中央宣传部长、南方局书记和团中央机关刊物《中国青年》主编等职。一九二九年到中共中央宣传部工作,创办党报《上海报》。一九三一年在上海被国民党政府逮捕,牺牲于龙华。}、何孟雄\footnote[9]{ 何孟雄(一八九八——一九三一),湖南酃县人。中国共产党党员,中国早期北方职工运动的组织者之一,曾创建京绥铁路工会。一九二七年国民党叛变革命以后,曾任中共江苏省委委员、省农民运动委员会秘书等职。一九三一年在上海被国民党政府逮捕,牺牲于龙华。}等二十几个党的重要干部,他们为党和人民做过很多有益的工作,同群众有很好的联系,并且接着不久就被敌人逮捕,在敌人面前坚强不屈,慷慨就义。所谓犯“调和路线错误”的瞿秋白同志,是当时党内有威信的领导者之一,他在被打击以后仍继续做了许多有益的工作(主要是在文化方面),在一九三五年六月也英勇地牺牲在敌人的屠刀之下。所有这些同志的无产阶级英雄气概,乃是永远值得我们纪念的。六届四中全会这种对于中央机关的“改造”,同样被推广于各个革命根据地和白区地方组织。六届四中全会以后的中央,比六届三中全会及其以后的中央更着重地更有系统地向全国各地派遣中央代表、中央代表机关或新的领导干部,以此来贯彻其“反右倾”的斗争。\\
  在六届四中全会以后不久,一九三一年五月九日中央所发表的决议,表示新的“左”倾路线已经在实际工作中得到了具体的运用和发展。接着,中国连续发生了许多重大事变。江西中央区红军在毛泽东同志的正确领导和全体同志的积极努力之下,在六届四中全会后的中央还没有来得及贯彻其错误路线的情况之下,取得了粉碎敌人第二次和第三次“围剿”的巨大胜利;其它多数革命根据地和红军,在同一时期和同一情况下,也得到了很多的胜利和发展。另一方面,日本帝国主义在一九三一年“九一八”开始的进攻,又激起了全国民族民主运动的新的高涨。新的中央对于这些事变所造成的新形势,一开始就作了完全错误的估计。它过分地夸大了当时国民党统治的危机和革命力量的发展,忽视了“九一八”以后中日民族矛盾的上升和中间阶级的抗日民主要求,强调了日本帝国主义和其它帝国主义是要一致地进攻苏联的,各帝国主义和中国各反革命派别甚至中间派别是要一致地进攻中国革命的,并断定中间派别是所谓中国革命的最危险的敌人。因此它继续主张打倒一切,认为当时“中国政治形势的中心的中心,是反革命与革命的决死斗争”;因此它又提出了红军夺取中心城市以实现一省数省首先胜利,和在白区普遍地实行武装工农、各企业总罢工等许多冒险的主张。这些错误,最先表现于一九三一年九月二十日中央的《由于工农红军冲破敌人第三次“围剿”及革命危机逐渐成熟而产生的紧急任务决议》,并在后来临时中央的或在临时中央领导下作出的《关于日本帝国主义强占满洲事变的决议》(一九三一年九月二十二日)、《关于争取革命在一省与数省首先胜利的决议》(一九三二年一月九日)、《关于一二八事变的决议》(一九三二年二月二十六日)、《在争取中国革命在一省数省的首先胜利中中国共产党内机会主义的动摇》(一九三二年四月四日)、《中央区中央局关于领导和参加反对帝国主义进攻苏联瓜分中国与扩大民族革命战争运动周的决议》(一九三二年五月十一日)、《革命危机的增长与北方党的任务》(一九三二年六月二十四日)等文件中得到了继续和发挥。\\
  自一九三一年九月间以秦邦宪(博古)\footnote[10]{ 秦邦宪(一九〇七——一九四六),又名博古,江苏无锡人。一九三一年九月至一九三五年一月,曾是中共临时中央和中共六届五中全会后中央的主要负责人。在这期间,犯过严重的“左”倾冒险主义的错误。遵义会议后,被撤销了党和红军的最高领导职务。抗日战争初期,先后在中共中央长江局、南方局工作。一九四一年以后,在毛泽东的领导下,在延安创办和主持《解放日报》和新华通讯社。在这期间,对自己过去的错误作了自我批评。一九四五年在党的第七次全国代表大会上,继续当选为中央委员。一九四六年二月到重庆参加同国民党谈判。四月八日在返回延安的途中因飞机失事遇难。}同志为首的临时中央成立起,到一九三五年一月遵义会议止,是第三次“左”倾路线的继续发展的时期。其间,临时中央因为白区工作在错误路线的领导下遭受严重损失,在一九三三年初迁入江西南部根据地,更使他们的错误路线得以在中央所在的根据地和邻近各根据地进一步地贯彻执行。在这以前,一九三一年十一月的江西南部根据地党代表大会\footnote[11]{ 指一九三一年十一月一日至五日在江西瑞金召开的中央苏区党的第一次代表大会,又称赣南会议。}和一九三二年十月中央区中央局的宁都会议,虽然已经根据六届四中全会的“反右倾”和“改造各级领导机关”的错误纲领,污蔑过去江西南部和福建西部根据地的正确路线为“富农路线”和“极严重的一贯的右倾机会主义错误”,并改变了正确的党的领导和军事领导;但是因为毛泽东同志的正确战略方针在红军中有深刻影响,在临时中央的错误路线尚未完全贯彻到红军中去以前,一九三三年春的第四次反“围剿”战争仍然得到了胜利。而在一九三三年秋开始的第五次反“围剿”战争中,极端错误的战略就取得了完全的统治。在其它许多政策上,特别是对于福建事变的政策上,“左”倾路线的错误也得到了完全的贯彻。\\
  一九三四年一月,由临时中央召集的第六届中央委员会第五次全体会议(六届五中全会),是第三次“左”倾路线发展的顶点。六届五中全会不顾“左”倾路线所造成的中国革命运动的挫折和“九一八”“一二八”以来国民党统治区人民抗日民主运动的挫折,盲目地判断“中国的革命危机已到了新的尖锐的阶段——直接革命形势在中国存在着”;判断第五次反“围剿”的斗争“即是争取中国革命完全胜利的斗争”,说这一斗争将决定中国的“革命道路与殖民地道路之间谁战胜谁的问题”。它又重复立三路线的观点,宣称“在我们已将工农民主革命推广到中国重要部分的时候,实行社会主义革命将成为共产党的基本任务,只有在这个基础上,中国才会统一,中国民众才会完成民族的解放”等等。在反对“主要危险的右倾机会主义”、“反对对右倾机会主义的调和态度”和反对“用两面派的态度在实际工作中对党的路线怠工”等口号之下,它继续发展了宗派主义的过火斗争和打击政策。\\
  第三次“左”倾路线在革命根据地的最大恶果,就是中央所在地区第五次反“围剿”战争的失败和红军主力的退出中央所在地区。“左”倾路线在退出江西和长征的军事行动中又犯了逃跑主义的错误,使红军继续受到损失。党在其它绝大多数革命根据地(闽浙赣区、鄂豫皖区、湘鄂赣区、湘赣区、湘鄂西区、川陕区)和广大白区的工作,也同样由于“左”倾路线的统治而陷于失败。统治过鄂豫皖区和川陕区的张国焘路线,则除了一般的“左”倾路线之外,还表现为特别严重的军阀主义和在敌人进攻面前的逃跑主义。\\
  以上这些,就是第三次统治全党的、以教条主义分子陈绍禹秦邦宪二同志为首的、错误的“左”倾路线的主要内容。\\
  犯教条主义错误的同志们披着“马列主义理论”的外衣,仗着六届四中全会所造成的政治声势和组织声势,使第三次“左”倾路线在党内统治四年之久,使它在思想上、政治上、军事上、组织上表现得最为充分和完整,在全党影响最深,因而其危害也最大。但是犯这个路线错误的同志,在很长时期内,却在所谓“中共更加布尔什维克化”、“百分之百的布尔什维克”等武断词句下,竭力吹嘘同事实相反的六届四中全会以来中央领导路线之“正确性”及其所谓“不朽的成绩”,完全歪曲了党的历史。\\
  在第三次“左”倾路线时期中,以毛泽东同志为代表的主张正确路线的同志们,是同这条“左”倾路线完全对立的。他们不赞成并要求纠正这条“左”倾路线,因而他们在各地的正确领导,也就被六届四中全会以来的中央及其所派去的组织或人员所推翻了。但是“左”倾路线在实际工作中的不断碰壁,尤其是中央所在地区第五次反“围剿”中的不断失败,开始在更多的领导干部和党员群众面前暴露了这一路线的错误,引起了他们的怀疑和不满。在中央所在地区红军长征开始后,这种怀疑和不满更加增长,以至有些曾经犯过“左”倾错误的同志,这时也开始觉悟,站在反对“左”倾错误的立场上来了。于是广大的反对“左”倾路线的干部和党员,都在毛泽东同志的领导下团结起来,因而在一九三五年一月,在毛泽东同志所领导的在贵州省遵义城召开的扩大的中央政治局会议上,得以胜利地结束了“左”倾路线在党中央的统治,在最危急的关头挽救了党。\\
  遵义会议集中全力纠正了当时具有决定意义的军事上和组织上的错误,是完全正确的。这次会议开始了以毛泽东同志为首的中央的新的领导,是中国党内最有历史意义的转变。也正是由于这一转变,我们党才能够胜利地结束了长征,在长征的极端艰险的条件下保存了并锻炼了党和红军的基干,胜利地克服了坚持退却逃跑并实行成立第二党的张国焘路线,挽救了“左”倾路线所造成的陕北革命根据地的危机\footnote[12]{ 一九三五年秋,在陕北革命根据地(包括陕甘边和陕北),“左”倾机会主义路线被贯彻到政治、军事、组织各方面工作中去,使执行正确路线的、创造了陕北红军和革命根据地的刘志丹等遭到排斥。接着在肃清反革命的工作中,一大批执行正确路线的干部又被逮捕,从而造成陕北革命根据地的严重危机。同年十月中共中央经过长征到达陕北后,纠正了这个“左”倾错误,将刘志丹等从监狱中释放出来,因而挽救了陕北革命根据地的危险局面。},正确地领导了一九三五年的“一二九”救亡运动\footnote[13]{ 见本书第一卷《论反对日本帝国主义的策略》注〔8〕。},正确地解决了一九三六年的西安事变\footnote[14]{ 参见本书第一卷《关于蒋介石声明的声明》注〔1〕。},组织了抗日民族统一战线,推动了神圣的抗日战争的爆发。\\
  遵义会议后,党中央在毛泽东同志领导下的政治路线,是完全正确的。“左”倾路线在政治上、军事上、组织上都被逐渐地克服了。一九四二年以来,毛泽东同志所领导的全党反对主观主义、宗派主义、党八股的整风运动和党史学习,更从思想根源上纠正了党的历史上历次“左”倾以及右倾的错误。过去犯过“左”、右倾错误的同志,在长期体验中,绝大多数都有了很大的进步,做过了许多有益于党和人民的工作。这些同志,和其它广大同志在一起,在共同的政治认识上互相团结起来了。扩大的六届七中全会欣幸地指出:我党经过了自己的各种成功和挫折,终于在毛泽东同志领导下,在思想上、政治上、组织上、军事上,第一次达到了现在这样高度的巩固和统一。这是快要胜利了的党,这是任何力量也不能战胜了的党。\\
  扩大的六届七中全会认为:关于抗日时期党内的若干历史问题,因为抗日阶段尚未结束,留待将来做结论是适当的。\\
\subsection*{\myformat{(四)}}
为了使同志们进一步了解各次尤其是第三次“左”倾路线的错误,以利于“惩前毖后”,不在今后工作上重犯这类错误起见,特分别指出它们在政治上、军事上、组织上、思想上同正确路线相违抗的主要内容如下。\\
  (一)在政治上:\\
  如同斯大林同志所指出\footnote[15]{ 参见斯大林《中国革命问题》、《中国革命和共产国际的任务》第二部分(《斯大林全集》第9卷,人民出版社1954年版,第199—207、259—267页)和《论中国革命的前途》(《斯大林选集》上卷,人民出版社1979年版,第483—495页)。}和毛泽东同志所详细分析过的,现阶段的中国,是一个半殖民地半封建的国家(“九一八”以后部分地变为殖民地);这个国家的革命,在第一次世界大战之后,是国际无产阶级已在苏联胜利,中国无产阶级已有政治觉悟时代的民族民主革命。这就规定了中国现阶段革命的性质,是无产阶级领导的、以工人农民为主体而有其它广大社会阶层参加的、反帝反封建的革命,即是既区别于旧民主主义又区别于社会主义的新民主主义的革命。由于现阶段的中国是在强大而又内部互相矛盾的几个帝国主义国家和中国封建势力统治之下的半殖民地半封建的大国,其经济和政治的发展具有极大的不平衡性和不统一性,这就规定了中国新民主主义革命的发展之极大的不平衡性,使革命在全国的胜利不能不经历长期的曲折的斗争;同时又使这一斗争能广泛地利用敌人的矛盾,在敌人的统治比较薄弱的广大地区首先建立和保持武装的革命根据地。为中国革命实践所证明的中国革命的上述基本特点和基本规律,既为一切右倾路线所不了解和违抗,也为各次尤其是第三次“左”倾路线所不了解和违抗。“左”倾路线因此在政治上犯了三个主要方面的错误:\\
  第一,各次“左”倾路线首先在革命任务和阶级关系的问题上犯了错误。和斯大林同志一样,毛泽东同志还在第一次大革命时期,就不但指出中国现阶段革命的任务是反帝反封建,而且特别指出农民的土地斗争是中国反帝反封建的基本内容,中国的资产阶级民主革命实质上就是农民革命,因此对于农民斗争的领导是中国无产阶级在资产阶级民主革命中的基本任务\footnote[16]{ 见本书第一卷《湖南农民运动考察报告》等文。}。在土地革命战争初期,他又指出中国所需要的仍是资产阶级民主革命,“必定要经过这样的民权主义革命”,才谈得上社会主义的前途\footnote[17]{ 见本书第一卷《井冈山的斗争》一文的《革命性质问题》部分。};指出土地革命因为革命在城市的失败有了更大的意义,“半殖民地中国的革命,只有农民斗争得不到工人的领导而失败,没有农民斗争的发展超过工人的势力而不利于革命本身的”\footnote[18]{ 见《星星之火,可以燎原》(本书第1卷第103页)。};指出大资产阶级叛变革命之后,自由资产阶级仍然和买办资产阶级有区别,要求民主尤其是要求反帝国主义的阶层还是很广泛的,因此必须正确地对待和尽可能地联合或中立各种不同的中间阶级,而在乡村中则必须正确地对待中农和富农(“抽多补少,抽肥补瘦”,同时坚决地团结中农,保护富裕中农,给富农以经济的出路,也给一般地主以生活的出路)\footnote[19]{ 见本书第一卷《中国的红色政权为什么能够存在?》、《井冈山的斗争》等文。}。凡此都是新民主主义的基本思想,而“左”倾路线是不了解和反对这些思想的。虽然各次“左”倾路线所规定的革命任务,许多也还是民主主义的,但是它们都混淆了民主革命和社会主义革命的一定界限,并主观地急于要超过民主革命;都低估农民反封建斗争在中国革命中的决定作用;都主张整个地反对资产阶级以至上层小资产阶级。第三次“左”倾路线更把反资产阶级和反帝反封建并列,否认中间营垒和第三派的存在,尤其强调反对富农。特别是九一八事变发生以后,中国阶级关系有了一个明显的巨大的变化,但是当时的第三次“左”倾路线则不但不认识这个变化,反而把同国民党反动统治有矛盾而在当时积极活动起来的中间派别断定为所谓“最危险的敌人”。应当指出,第三次“左”倾路线的代表者也领导了农民分配土地,建立政权和武装反抗当时国民党政府的进攻,这些任务都是正确的;但是由于上述的“左”倾认识,他们就错误地害怕承认当时的红军运动是无产阶级领导的农民运动,错误地反对所谓“农民特殊革命性”、“农民的资本主义”和所谓“富农路线”,而实行了许多超民主主义的所谓“阶级路线”的政策,例如消灭富农经济及其它过左的经济政策、劳动政策,一切剥削者均无参政权的政权政策,强调以共产主义为内容的国民教育政策,对知识分子的过左政策,要兵不要官的兵运工作和过左的肃反政策等,而使当前的革命任务被歪曲,使革命势力被孤立,使红军运动受挫折。同样,应当指出,我党在一九二七年革命失败后的国民党统治区中,一贯坚持地领导了人民的民族民主运动,领导了工人及其它群众的经济斗争和革命的文化运动,反对了当时国民党政府出卖民族利益和压迫人民的政策;特别是“九一八”以后,我党领导了东北抗日联军,援助了“一二八”战争和察北抗日同盟军,和福建人民政府成立了抗日民主的同盟,提出了在三个条件下红军愿同国民党军队联合抗日\footnote[20]{ 一九三三年一月十七日,中华苏维埃临时中央政府、工农红军革命军事委员会发表宣言,向一切进攻革命根据地的国民党军队提议,在三个条件下订立停战协定,联合抗日。三个条件是:(一)停止进攻革命根据地,(二)保证民众的民主权利,(三)武装民众。},在六个条件下愿同各界人民建立民族武装自卫委员会\footnote[21]{ 参见本书第一卷《论反对日本帝国主义的策略》注〔17〕。},在一九三五年八月一日发表了《为抗日救国告全体同胞书》,号召成立国防政府和抗日联军等,这些也都是正确的。但是在各次尤其是第三次“左”倾路线统治时期,由于指导政策的错误,不能在实际上正确地解决问题,以致当时党在国民党统治区的工作也都没有得到应有的结果,或归于失败。当然,在抗日问题上,在当时还不能预料到代表中国大地主大资产阶级主要部分的国民党主要统治集团在一九三五年的华北事变\footnote[22]{ 见本书第一卷《中国共产党在抗日时期的任务》注〔2〕。}尤其是一九三六年的西安事变以后所起的变化,但是中间阶层和一部分大地主大资产阶级的地方集团已经发生了成为抗日同盟者的变化,这个变化是广大党员和人民都已经认识了的,却被第三次“左”倾路线所忽视或否认,形成了自己的严重的关门主义,使自己远落于中国人民的政治生活之后。这个关门主义错误所造成的孤立和落后的状况,在遵义会议以前,基本上是没有改变的。\\
  第二,各次“左”倾路线在革命战争和革命根据地的问题上,也犯了错误。斯大林同志说:“在中国,是武装的革命反对武装的反革命。这是中国革命的特点之一,也是中国革命的优点之一。”\footnote[23]{ 见斯大林《论中国革命的前途》(《斯大林选集》上卷,人民出版社1979年版,第487页)。}和斯大林同志一样,毛泽东同志在土地革命战争初期即已正确指出,由于半殖民地半封建的中国,是缺乏民主和工业的不统一的大国,武装斗争和以农民为主体的军队,是中国革命的主要斗争形式和组织形式。毛泽东同志又指出:广大农民所在的广大乡村,是中国革命必不可少的重要阵地(革命的乡村可以包围城市,而革命的城市不能脱离乡村);中国可以而且必须建立武装的革命根据地,以为全国胜利(全国的民主统一)的出发点\footnote[24]{ 见本书第一卷《中国的红色政权为什么能够存在?》、《星星之火,可以燎原》。}。在一九二四年至一九二七年革命时期,由于国共合作建立了联合政府,当时的根据地是以某些大城市为中心的,但是即在那个时期,也必须在无产阶级领导下建立以农民为主体的人民军队,并解决乡村土地问题,以巩固根据地的基础。而在土地革命战争时期,由于强大的反革命势力占据了全国的城市,这时的根据地就只能主要地依靠农民游击战争(而不是阵地战),在反革命统治薄弱的乡村(而不是中心城市)首先建立、发展和巩固起来。毛泽东同志指出这种武装的乡村革命根据地在中国存在的历史条件,是中国的“地方的农业经济(不是统一的资本主义经济)和帝国主义划分势力范围的分裂剥削政策”,是由此而来的“白色政权间的长期的分裂和战争”\footnote[25]{ 见本书第一卷《中国的红色政权为什么能够存在?》。}。他又指出这种根据地对于中国革命的历史意义,是“必须这样,才能树立全国革命群众的信仰,如苏联之于全世界然。必须这样,才能给反动统治阶级以甚大的困难,动摇其基础而促进其内部的分解。也必须这样,才能真正地创造红军,成为将来大革命的主要工具。总而言之,必须这样,才能促进革命的高潮”\footnote[26]{ 见《星星之火,可以燎原》(本书第1卷第98—99页)。}。至于这个时期的城市群众工作,则应如正确路线在白区工作中的代表刘少奇同志所主张的,采取以防御为主(不是以进攻为主),尽量利用合法的机会去工作(而不是拒绝利用合法),以便使党的组织深入群众,长期荫蔽,积蓄力量,并随时输送自己的力量到乡村去发展乡村武装斗争力量,借此以配合乡村斗争,推进革命形势,为其主要方针。因此,直至整个形势重新具有在城市中建立民主政府的条件时为止,中国革命运动应该以乡村工作为主,城市工作为辅;革命在乡村的胜利和在城市的暂时不能胜利,在乡村的进攻和在城市的一般处于防御,以至在这一乡村的胜利及进攻和在另一乡村的失败、退却和防御,就织成了在这一时期中全国的革命和反革命相交错的图画,也就铺成了在这一形势下革命由失败到胜利的必经道路。但是各次“左”倾路线的代表者,因为不了解半殖民地半封建的中国社会的特点,不了解中国资产阶级民主革命实质上是农民革命,不了解中国革命的不平衡性、曲折性和长期性,就从而低估了军事斗争特别是农民游击战争和乡村根据地的重要性,就从而反对所谓“枪杆子主义”和所谓“农民意识的地方观念与保守观念”,而总是梦想这时城市的工人斗争和其它群众斗争能突然冲破敌人的高压而勃兴,而发动中心城市的武装起义,而达到所谓“一省数省的首先胜利”,而形成所谓全国革命高潮和全国胜利,并以这种梦想作为一切工作布置的中心。但是实际上,在一九二七年革命失败后阶级力量对比的整个形势下,这种梦想的结果不是别的,首先就是造成了城市工作本身的失败。第一次“左”倾路线这样失败了,第二次“左”倾路线仍然继续同样的错误;所不同的,是要求红军的配合,因为这时红军已经逐渐长大了。第二次失败了,第三次“左”倾路线仍然要求在大城市“真正”准备武装起义;所不同的,是主要要求红军的占领,因为这时红军更大,城市工作更小了。这样不以当时的城市工作服从乡村工作,而以当时的乡村工作服从城市工作的结果,就是使城市工作失败以后,乡村工作的绝大部分也遭到失败。应当指出,在一九三二年以后,由于红军对中心城市的不能攻克或不能固守,特别是由于国民党的大举进攻,实际上已经停止了夺取中心城市的行动;而在一九三三年以后,又由于城市工作的更大破坏,临时中央也离开了城市而迁入了乡村根据地,实行了一个转变。但是这种转变,对于当时的“左”倾路线的同志们说来,不是自觉的,不是从研究中国革命特点得出正确结论的结果,因此,他们依然是以他们错误的城市观点,来指导红军和根据地的各项工作,并使这些工作受到破坏。例如,他们主张阵地战,而反对游击战和带游击性的运动战;他们错误地强调所谓“正规化”,而反对红军的所谓“游击主义”;他们不知道适应分散的乡村和长期的被敌人分割的游击战争,以节省使用根据地的人力物力,和采取其它必要的对策;他们在第五次反“围剿”中提出所谓“中国两条道路的决战”和所谓“不放弃根据地一寸土地”的错误口号,等等,就是明证。\\
  扩大的六届七中全会着重指出:我们上面所说的这一时期内乡村工作所应推进、城市工作所应等待的形势变化,现时已经迫近了。只有在现时,在抗日战争的最后阶段,在我党领导的军队已经壮大,并还将更加壮大的时候,将敌占区的城市工作提到和解放区工作并重的地位,积极地准备一切条件,以便里应外合地从中心城市消灭日本侵略者,然后把工作重心转到这些城市去,才是正确的。这一点,对于从一九二七年革命失败以来艰难地将工作重心转入乡村的我党,将是一个新的有历史意义的转变;全党同志都应充分自觉地准备这一转变,而不再重复“左”倾路线在土地革命战争时期由城市转入乡村问题上所表现的初则反对、违抗,继则勉强、被迫和不自觉的那种错误。至于国民党统治区域,则是另外一种情形;在那里,我们现时的任务是无论在乡村或城市,都应放手动员群众,坚决反对内战分裂,力争和平团结,要求加强对日作战,废止国民党一党专政,成立全国统一的民主的联合政府。当敌占城市在人民手中得到了解放,全国统一的民主的联合政府真正地实现了和巩固了的时候,就将是乡村根据地的历史任务完成的时候。\\
  第三,各次“左”倾路线在进攻和防御的策略指导上,也犯了错误。正确的策略指导,必须如斯大林同志所指出的,需要正确的形势分析(正确地估计阶级力量的对比,判断运动的来潮和退潮),需要由此而来的正确的斗争形式和组织形式,需要正确的“利用敌人阵营里的每一缝隙,善于给自己找寻同盟者”\footnote[27]{ 参见斯大林《论列宁主义基础》第七部分《战略和策略》(《斯大林全集》第6卷,人民出版社1956年版,第131—147页)和《时事问题简评》第二部分《关于中国》。这里的引语见《时事问题简评》(《斯大林全集》第9卷,人民出版社1954年版,第305页)。};而毛泽东同志对于中国革命运动的指导,正是一个最好的模范。毛泽东同志在一九二七年革命失败后,正确地指出全国革命潮流的低落,在全国范围内敌强于我,冒险的进攻必然要招致失败;但在反动政权内部不断分裂和战争,人民革命要求逐渐恢复和上升的一般条件下,和在群众经过第一次大革命斗争,并有相当力量的红军和有正确政策的共产党的特殊条件下,就可以“在四围白色政权的包围中间,产生一小块或若干小块的红色政权区域”\footnote[28]{ 见《井冈山的斗争》(本书第1卷第57页)。}。他又指出:在统治阶级破裂时期,红色政权的发展“可以比较地冒进,用军事发展割据的地方可以比较地广大”;若在统治阶级比较稳定时期,则这种发展“必须是逐渐地推进的。这时在军事上最忌分兵冒进,在地方工作方面(分配土地,建立政权,发展党,组织地方武装)最忌把人力分得四散,而不注意建立中心区域的坚实基础”\footnote[29]{ 见《井冈山的斗争》(本书第1卷第58页)。}。即在同一时期,由于敌人的强弱不同,我们的策略也应当不同,所以湘赣边的割据地区就“对统治势力比较强大的湖南取守势,对统治势力比较薄弱的江西取攻势”\footnote[30]{ 见《井冈山的斗争》(本书第1卷第59页)。}。湘赣边红军以后进入闽赣边,又提出“争取江西,同时兼及闽西、浙西”\footnote[31]{ 见《星星之火,可以燎原》(本书第1卷第105页)。}的计划。不同的敌人对革命的不同利害关系,是决定不同策略的重要根据。所以毛泽东同志始终主张“利用反革命内部的每一冲突,从积极方面扩大他们内部的裂痕”\footnote[32]{ 见《中共中央关于反对敌人五次“围剿”的总结决议》(《遵义会议文献》,人民出版社1985年版,第3—26页)。},“反对孤立政策,承认争取一切可能的同盟者”\footnote[33]{ 见《中国革命战争的战略问题》(本书第1卷第192页)。}。这些“利用矛盾,争取多数,反对少数,各个击破”\footnote[34]{ 见《论政策》(本书第2卷第764页)。}的策略原则的运用,在他所领导的历次反“围剿”战争中,尤其在遵义会议后的长征和抗日民族统一战线工作中,得到了光辉的发展。刘少奇同志在白区工作中的策略思想,同样是一个模范。刘少奇同志正确地估计到一九二七年革命失败后白区特别是城市敌我力量的悬殊,所以主张有系统地组织退却和防御,“在形势与条件不利于我们的时候,暂时避免和敌人决斗”,以“准备将来革命的进攻和决斗”\footnote[35]{ 见刘少奇《肃清关门主义与冒险主义》(《刘少奇选集》上卷,人民出版社1981年版,第25页)。};主张有计划地把一九二四年至一九二七年革命时期的党的公开组织严格地转变为秘密组织,而在群众工作中则“尽可能利用公开合法手段”,以便党的秘密组织能够在这种群众工作中长期地荫蔽力量,深入群众,“聚积与加强群众的力量,提高群众的觉悟”\footnote[36]{ 见刘少奇《关于过去白区工作给中央的信》。}。对于群众斗争的领导,刘少奇同志认为应当“根据当时当地的环境和条件,根据群众觉悟的程度,提出群众可能接受的部分的口号、要求和斗争的方式,去发动群众的斗争,并根据斗争过程中各种条件的变化,把群众的斗争逐渐提高到更高的阶段,或者‘适可而止’地暂时结束战斗,以准备下一次更高阶段和更大范围的战斗”。在利用敌人内部矛盾和争取暂时的同盟者的问题上,他认为应该“推动这些矛盾的爆发,与敌人营垒中可能和我们合作的成分,或者与今天还不是我们主要的敌人,建立暂时的联盟,去反对主要的敌人”;应该“向那些愿意同我们合作的同盟者作必要的让步,吸引他们同我们联合,参加共同的行动,再去影响他们,争取他们下层的群众”\footnote[37]{ 以上三段引文见刘少奇《肃清关门主义与冒险主义》(《刘少奇选集》上卷,人民出版社1981年版,第26、28、30页)。}。一二九运动的成功,证明了白区工作中这些策略原则的正确性。和这种正确的策略指导相反,各次“左”倾路线的同志们因为不知道客观地考察敌我力量的对比,不知道采取与此相当的斗争形式和组织形式,不承认或不重视敌人内部的矛盾,这样,他们在应当防御的时候,固然因为盲目地实行所谓“进攻路线”而失败,就在真正应当进攻的时候,也因为不会组织胜利的进攻而失败。他们“估计形势”的方法,是把对他们的观点有利的某些个别的、萌芽的、间接的、片面的和表面的现象,夸大为大量的、严重的、直接的、全面的和本质的东西,而对于不合他们的观点的一切实际(如敌人的强大和暂时胜利,我们的弱小和暂时失败,群众的觉悟不足,敌人的内部矛盾,中间派的进步方面等),则害怕承认,或熟视无睹。他们从不设想到可能的最困难和最复杂的情况,而只是梦想着不可能的最顺利和最简单的情况。在红军运动方面,他们总是把包围革命根据地的敌人描写为“十分动摇”、“恐慌万状”、“最后死亡”、“加速崩溃”、“总崩溃”等等。第三次“左”倾路线的代表者们甚至认为红军对于超过自己许多倍的整个的国民党军队还占优势,因此总是要求红军作无条件的甚至不休息的冒进。第三次“左”倾路线的代表者们否认一九二四年至一九二七年革命所造成的南方和北方革命发展的不平衡(这种情况只是到了抗日战争期间才起了一个相反的变化),错误地反对所谓“北方落后论”,要求在北方乡村中普遍地建立红色政权,在北方白色军队中普遍地组织哗变成立红军。他们也否认根据地的中心地区和边缘地区的不平衡,错误地反对所谓“罗明路线”\footnote[38]{ 罗明(一九〇一——一九八七),广东大埔人。一九三三年在担任中央革命根据地中共福建省委的代理书记时,曾经认为党在闽西上杭、永定等边缘地区的工作条件比较困难,党的政策应当不同于根据地的巩固地区,而受到党内“左”倾领导者的打击。当时这些领导者把他的意见错误地、夸大地说成是“悲观失望的”、“机会主义的、取消主义的逃跑退却路线”,并且开展了所谓“反对罗明路线的斗争”。}。他们拒绝利用进攻红军的各个军阀之间的矛盾,拒绝同愿意停止进攻红军的军队成立妥协。在白区工作方面,他们在革命已转入低潮而反革命的统治力量极为强大的城市,拒绝实行必要的退却和防御的步骤,拒绝利用一切合法的可能,而继续采取为当时情况所不允许的进攻的形式,组织庞大的没有掩护的党的机关和各种脱离广大群众的第二党式的所谓赤色群众团体,经常地无条件地号召和组织政治罢工、同盟罢工、罢课、罢市、罢操、罢岗、游行示威、飞行集会以至武装暴动等不易或不能得到群众参加和支持的行动,并曲解这一切行动的失败为“胜利”。总之,各次尤其是第三次“左”倾路线的同志们只知道关门主义和冒险主义,盲目地认为“斗争高于一切,一切为了斗争”,“不断地扩大和提高斗争”,因而不断地陷于不应有的和本可避免的失败。\\
  (二)在军事上:\\
  在中国革命的现阶段,军事斗争是政治斗争的主要形式。在土地革命战争时期,这一问题成为党的路线中的最迫切的问题。毛泽东同志不但运用马克思列宁主义规定了中国革命的正确的政治路线,而且从土地革命战争时期以来,也运用马克思列宁主义规定了服从于这一政治路线的正确的军事路线。毛泽东同志的军事路线从两个基本观点出发:第一,我们的军队不是也不能是其它样式的军队,它必须是服从于无产阶级思想领导的、服务于人民斗争和根据地建设的工具;第二,我们的战争不是也不能是其它样式的战争,它必须在承认敌强我弱、敌大我小的条件下,充分地利用敌之劣点与我之优点,充分地依靠人民群众的力量,以求得生存、胜利和发展。从第一个观点出发,红军(现在是八路军、新四军及其它人民军队)必须全心全意地为着党的路线、纲领和政策,也就是为着全国人民的各方面利益而奋斗,反对一切与此相反的军阀主义倾向。因此,红军必须反对军事不服从于政治或以军事来指挥政治的单纯军事观点和流寇思想;红军必须同时负起打仗、做群众工作和筹款(现在是生产)的三位一体的任务,而所谓做群众工作,就是要成为党和人民政权的宣传者和组织者,就是要帮助地方人民群众分配土地(现在是减租减息),建立武装,建立政权以至建立党的组织。因此,红军在军政关系和军民关系上,必须要求严格地尊重人民的政权机关和群众团体,巩固它们的威信,严格地执行“三大纪律”“八项注意”\footnote[39]{ “三大纪律”、“八项注意”,是毛泽东等在第二次国内革命战争时期为中国工农红军制订的纪律,后来成为八路军新四军的纪律,以后又成为人民解放军的纪律。其具体内容在不同时期和不同部队略有出入。一九四七年十月,中国人民解放军总部对其内容作了统一规定,并重新颁布。“三大纪律”是:(一)一切行动听指挥;(二)不拿群众一针一线;(三)一切缴获要归公。“八项注意”是:(一)说话和气;(二)买卖公平;(三)借东西要还;(四)损坏东西要赔;(五)不打人骂人;(六)不损坏庄稼;(七)不调戏妇女;(八)不虐待俘虏。};在军队的内部,必须建立正确的官兵关系,必须要有一定的民主生活和有威权的以自觉为基础的军事纪律;在对敌军的工作上,必须具有瓦解敌军和争取俘虏的正确政策。从第二个观点出发,红军必须承认游击战和带游击性的运动战是土地革命战争时期的主要战争形式,承认只有主力兵团和地方兵团相结合,正规军和游击队、民兵相结合,武装群众和非武装群众相结合的人民战争,才能够战胜比自己强大许多倍的敌人。因此,红军必须反对战略的速决战和战役的持久战,坚持战略的持久战和战役的速决战;反对战役战术的以少胜多,坚持战役战术的以多胜少。因此,红军必须实行“分兵以发动群众,集中以应付敌人”;“敌进我退,敌驻我扰,敌疲我打,敌退我追”;“固定区域的割据,用波浪式的推进政策;强敌跟追,用盘旋式的打圈子政策”\footnote[40]{ 见《星星之火,可以燎原》(本书第1卷第104页)。};“诱敌深入”\footnote[41]{ 见本书第一卷《中国革命战争的战略问题》第五章。};“集中优势兵力,选择敌人的弱点,在运动战中有把握地消灭敌人的一部或大部,以各个击破敌人”\footnote[42]{ 见一九三五年二月二十八日《中共中央关于冲破五次“围剿”的总结》。}等项战略战术的原则。各次“左”倾路线在军事上都是同毛泽东同志站在恰恰相反对的方面:第一次“左”倾路线的盲动主义,使红军脱离人民群众;第二次“左”倾路线,使红军实行冒险的进攻。但是这两次“左”倾路线在军事上都没有完整的体系。具有完整体系的是第三次。第三次“左”倾路线,在建军的问题上,把红军的三项任务缩小成为单纯的打仗一项,忽略正确的军民、军政、官兵关系的教育;要求不适当的正规化,把当时红军的正当的游击性当作所谓“游击主义”来反对;又发展了政治工作中的形式主义。在作战问题上,它否认了敌强我弱的前提;要求阵地战和单纯依靠主力军队的所谓“正规”战;要求战略的速决战和战役的持久战;要求“全线出击”和“两个拳头打人”;反对诱敌深入,把必要的转移当作所谓“退却逃跑主义”;要求固定的作战线和绝对的集中指挥等;总之是否定了游击战和带游击性的运动战,不了解正确的人民战争。在第五次反“围剿”作战中,他们始则实行进攻中的冒险主义,主张“御敌于国门之外”;继则实行防御中的保守主义,主张分兵防御,“短促突击”,同敌人“拚消耗”;最后,在不得不退出江西根据地时,又变为实行真正的逃跑主义。这些都是企图用阵地战代替游击战和运动战,用所谓“正规”战争代替正确的人民战争的结果。\\
  在抗日战争的战略退却和战略相持阶段中,因为敌我强弱相差更甚,八路军和新四军的正确方针是:“基本的是游击战,但不放松有利条件下的运动战。”\footnote[43]{ 见《论持久战》(本书第2卷第500页)。}强求过多的运动战是错误的。但在将要到来的战略反攻阶段,正如全党的工作重心需要由乡村转到城市一样,在我军获得新式装备的条件下,战略上也需要由以游击战为主变为以运动战和阵地战为主。对于这个即将到来的转变,也需要全党有充分的自觉来作准备。\\
  (三)在组织上:\\
  如毛泽东同志所说,正确的政治路线应该是“从群众中来,到群众中去”。而为使这个路线真正从群众中来,特别是真正能到群众中去,就不但需要党和党外群众(阶级和人民)有密切的联系,而且首先需要党的领导机关和党内群众(干部和党员)有密切的联系,也就是说,需要正确的组织路线。因此,毛泽东同志在党的各个时期既然规定了代表人民群众利益的政治路线,同时也就规定了服务于这一政治路线的联系党内党外群众的组织路线。这个工作,在土地革命战争时期也得到了重要的发展,其集中的表现,便是一九二九年红四军党的第九次代表大会的决议\footnote[44]{ 即古田会议决议。本书第一卷《关于纠正党内的错误思想》,是这个决议的第一部分。}。这个决议,一方面把党的建设提到了思想原则和政治原则的高度,坚持无产阶级思想的领导,正确地进行了反对单纯军事观点、主观主义、个人主义、平均主义、流寇思想、盲动主义等倾向的斗争,指出了这些倾向的根源、危害和纠正的办法;另一方面又坚持严格的民主集中制,既反对不正当地限制民主,也反对不正当地限制集中。毛泽东同志又从全党团结的利益出发,坚持局部服从全体,并根据中国革命的具体特点,规定了新干部和老干部、外来干部和本地干部、军队干部和地方干部、以及不同部门、不同地区的干部间的正确关系。这样,毛泽东同志就供给了一个坚持真理的原则性和服从组织的纪律性相结合的模范,供给了一个正确地进行党内斗争和正确地保持党内团结的模范。与此相反,在一切错误政治路线统治的同时,也就必然出现了错误的组织路线;这条错误的政治路线统治得愈久,则其错误的组织路线的为害也愈烈。因此,土地革命战争时期各次“左”倾路线,不但反对了毛泽东同志的政治路线,也反对了毛泽东同志的组织路线;不但形成了脱离党外群众的宗派主义(不把党当作人民群众利益的代表者和人民群众意志的集中者),也形成了脱离党内群众的宗派主义(不使党内一部分人的局部利益服从全党利益,不把党的领导机关当作全党意志的集中者)。尤其是第三次“左”倾路线的代表者,为贯彻其意旨起见,在党内曾经把一切因为错误路线行不通而对它采取怀疑、不同意、不满意、不积极拥护、不坚决执行的同志,不问其情况如何,一律错误地戴上“右倾机会主义”、“富农路线”、“罗明路线”、“调和路线”、“两面派”等大帽子,而加以“残酷斗争”和“无情打击”,甚至以对罪犯和敌人作斗争的方式来进行这种“党内斗争”。这种错误的党内斗争,成了领导或执行“左”倾路线的同志们提高其威信、实现其要求和吓唬党员干部的一种经常办法。它破坏了党内民主集中制的基本原则,取消了党内批评和自我批评的民主精神,使党内纪律成为机械的纪律,发展了党内盲目服从随声附和的倾向,因而使党内新鲜活泼的、创造的马克思主义之发展,受到打击和阻挠。同这种错误的党内斗争相结合的,则是宗派主义的干部政策。宗派主义者不把老干部看作党的宝贵的资本,大批地打击、处罚和撤换中央和地方一切同他们气味不相投的、不愿盲目服从随声附和的、有工作经验并联系群众的老干部。他们也不给新干部以正确的教育,不严肃地对待提拔新干部(特别是工人干部)的工作,而是轻率地提拔一切同他们气味相投的、只知盲目服从随声附和的、缺乏工作经验、不联系群众的新干部和外来干部,来代替中央和地方的老干部。这样,他们既打击了老干部,又损害了新干部。很多地区,更由于错误的肃反政策和干部政策中的宗派主义纠缠在一起,使大批优秀的同志受到了错误的处理而被诬害,造成了党内极可痛心的损失。这种宗派主义的错误,使党内发生了上下脱节和其它许多不正常现象,极大地削弱了党。\\
  扩大的六届七中全会在此宣布:对于一切被错误路线所错误地处罚了的同志,应该根据情形,撤消这种处分或其错误部分。一切经过调查确系因错误处理而被诬害的同志,应该得到昭雪,恢复党籍,并受到同志的纪念。\\
  (四)在思想上:\\
  一切政治路线、军事路线和组织路线之正确或错误,其思想根源都在于它们是否从马克思列宁主义的辩证唯物论和历史唯物论出发,是否从中国革命的客观实际和中国人民的客观需要出发。毛泽东同志从他进入中国革命事业的第一天起,就着重于应用马克思列宁主义的普遍真理以从事于对中国社会实际情况的调查研究,在土地革命战争时期,尤其再三再四地强调了“没有调查就没有发言权”的真理,再三再四地反对了教条主义和主观主义的危害。毛泽东同志在土地革命战争时期所规定的政治路线、军事路线和组织路线,正是他根据马克思列宁主义的普遍真理,根据辩证唯物论和历史唯物论,具体地分析了当时国内外党内外的现实情况及其特点,并具体地总结了中国革命的历史经验,特别是一九二四年至一九二七年革命的历史经验的光辉的成果。在中国生活和奋斗的中国共产党人学习辩证唯物论和历史唯物论,应该是为了用以研究和解决中国革命的各种实际问题,如同毛泽东同志所做的。但是一切犯“左”倾错误的同志们,在那时,当然是不能了解和接受毛泽东同志的做法的,第三次“左”倾路线的代表者更污蔑他是“狭隘经验主义者”;这是因为他们的思想根源乃是主观主义和形式主义,在第三次“左”倾路线统治时期更特别突出地表现为教条主义的缘故。教条主义的特点,是不从实际情况出发,而从书本上的个别词句出发。它不是根据马克思列宁主义的立场和方法来认真研究中国的政治、军事、经济、文化的过去和现在,认真研究中国革命的实际经验,得出结论,作为中国革命的行动指南,再在群众的实践中去考验这些结论是否正确;相反地,它抛弃了马克思列宁主义的实质,而把马克思列宁主义书本上的若干个别词句搬运到中国来当做教条,毫不研究这些词句是否合乎中国现时的实际情况。因此,他们的“理论”和实际脱离,他们的领导和群众脱离,他们不是实事求是,而是自以为是,他们自高自大,夸夸其谈,害怕正确的批评和自我批评,就是必然的了。\\
  在教条主义统治时期,同它合作并成为它的助手的经验主义的思想,也是主观主义和形式主义的一种表现形式。经验主义同教条主义的区别,是在于它不是从书本出发,而是从狭隘的经验出发。应当着重地指出:最广大的有实际工作经验的同志,他们的一切有益的经验,是极可宝贵的财产。科学地把这些经验总结起来,作为以后行动中的指导,这完全不是经验主义,而是马克思列宁主义;正像把马克思列宁主义的原理原则当做革命行动的指南,而不把它们当做教条,就完全不是教条主义,而是马克思列宁主义一样。但是,在一切有实际工作经验的同志中,如果有一些人满足于甚至仅仅满足于他们的局部经验,把它们当做到处可以使用的教条,不懂得而且不愿意承认“没有革命的理论,就不会有革命的运动”\footnote[45]{ 见列宁《俄国社会民主党人的任务》(《列宁全集》第2卷,人民出版社1984年版,第443页),并见列宁《怎么办?》第一章第四节(《列宁全集》第6卷,人民出版社1986年版,第23页)。}和“为着领导,必须预见”\footnote[46]{ 见一九二八年四月十三日斯大林在联共(布)莫斯科组织的积极分子会议上所作的报告《关于中央委员会和中央监察委员会四月联席全会的工作》(《斯大林选集》下卷,人民出版社1979年版,第12页)。}的真理,因而轻视从世界革命经验总结出来的马克思列宁主义的学习,并醉心于狭隘的无原则的所谓实际主义和无头脑无前途的事务主义,却坐在指挥台上,盲目地称英雄,摆老资格,不肯倾听同志们的批评和发展自我批评,这样,他们就成为经验主义者了。因此,经验主义和教条主义的出发点虽然不同,但是在思想方法的本质上,两者却是一致的。他们都是把马克思列宁主义的普遍真理和中国革命的具体实践分割开来;他们都违背辩证唯物论和历史唯物论,把片面的相对的真理夸大为普遍的绝对的真理;他们的思想都不符合于客观的全面的实际情况。因此,他们对于中国社会和中国革命,就有了许多共同的错误的认识(如错误的城市中心观点,白区工作中心观点,脱离实际情况的“正规”战观点等)。这就是这两部分同志能够互相合作的思想根源。虽然因为经验主义者的经验是局部的、狭隘的,他们中的多数对于全面性的问题往往缺乏独立的明确的完整的意见,因此,他们在和教条主义者相结合时,一般地是作为后者的附庸而出现;但是党的历史证明,教条主义者缺乏经验主义者的合作就不易“流毒全党”,而在教条主义被战胜以后,经验主义更成为党内的马克思列宁主义发展的主要障碍。因此,我们不但要克服主观主义的教条主义,而且也要克服主观主义的经验主义。必须彻底克服教条主义和经验主义的思想,马克思列宁主义的思想、路线和作风,才能普及和深入全党。\\
  以上所述政治、军事、组织和思想四方面的错误,实为各次尤其是第三次“左”倾路线的基本错误。而一切政治上、军事上和组织上的错误,都是从思想上违背马克思列宁主义的辩证唯物论和历史唯物论而来,都是从主观主义和形式主义、教条主义和经验主义而来。\\
  扩大的六届七中全会指出:我们在否定各次“左”倾路线的错误时,同时要牢记和实行毛泽东同志“对于任何问题应取分析态度,不要否定一切”\footnote[47]{ 见《学习和时局》(本卷第938页)。}的指示。应当指出:犯了这些错误的同志们的观点中,并不是一切都错了,他们在反帝反封建、土地革命、反蒋战争等问题上的若干观点,同主张正确路线的同志们仍然是一致的。还应指出,第三次“左”倾路线统治时间特别长久,所给党和革命的损失特别重大,但是这个时期的党,因为有广大的干部、党员群众和广大的军民群众在一起,进行了积极的工作和英勇的斗争,因而在许多地区和许多部门的实际工作中,仍然获得了很大的成绩(例如在战争中,在军事建设中,在战争动员中,在政权建设中,在白区工作中)。正是由于这种成绩,才能够支持反对敌人进攻的战争至数年之久,给了敌人以重大的打击;仅因错误路线的统治,这些成绩才终于受到了破坏。在各次错误路线统治时期,和党的任何其它历史时期一样,一切为人民利益而壮烈地牺牲了的党内党外的领袖、领导者、干部、党员和人民群众,都将永远被党和人民所崇敬。\\
\subsection*{\myformat{(五)}}
“左”倾路线的上述四方面错误的产生,不是偶然的,它有很深的社会根源。\\
  如同毛泽东同志所代表的正确路线反映了中国无产阶级先进分子的思想一样,“左”倾路线则反映了中国小资产阶级民主派的思想。半殖民地半封建的中国,是小资产阶级极其广大的国家。我们党不但从党外说是处在这个广大阶层的包围之中;而且在党内,由于十月革命以来马克思列宁主义在世界的伟大胜利,由于中国现时的社会政治情况,特别是国共两党的历史发展,决定了中国不能有强大的小资产阶级政党,因此就有大批的小资产阶级革命民主分子向无产阶级队伍寻求出路,使党内小资产阶级出身的分子也占了大多数。此外,即使工人群众和工人党员,在中国的经济条件下,也容易染有小资产阶级的色彩。因此,小资产阶级思想在我们党内常常有各色各样的反映,这是必然的,不足为怪的。\\
  党外的小资产阶级群众,除了农民是中国资产阶级民主革命的主要力量以外,城市小资产阶级大多数群众在中国也受着重重压迫,经常迅速大量地陷于贫困破产和失业的境地,其经济和政治的民主要求十分迫切,所以在现阶段的革命中,城市小资产阶级也是革命动力之一。但是小资产阶级由于是一个过渡的阶级,它是有两面性的:就其好的、革命的一面说来,是其大多数群众在政治上、组织上以至思想上能够接受无产阶级的影响,在目前要求民主革命,并能为此而团结奋斗,在将来也可能和无产阶级共同走向社会主义;而就其坏的、落后的一面说来,则不但有其各种区别于无产阶级的弱点,而且在失去无产阶级的领导时,还往往转而接受自由资产阶级以至大资产阶级的影响,成为他们的俘虏。因此,在现阶段上,无产阶级及其先进部队——中国共产党,对于党外的小资产阶级群众,应该在坚决地广泛地联合他们的基础上,一方面给以宽大的待遇,在不妨碍对敌斗争和共同的社会生活的条件下,容许其自由主义的思想和作风的存在;另一方面则给以适当的教育,以便巩固同他们的联合。\\
  至于由小资产阶级出身而自愿抛弃其原有立场、加入无产阶级政党的分子,则是完全另一种情形。党对于他们,和对于党外的小资产阶级群众,应该采取原则上不同的政策。由于他们本来和无产阶级相接近,又自愿地加入无产阶级政党,在党的马克思列宁主义教育和群众革命斗争的实际锻炼中,他们是可以逐渐在思想上无产阶级化,并给无产阶级队伍以重大利益的;而且在事实上,加入我党的小资产阶级出身的分子之绝大多数,也都为党和人民作了勇敢的奋斗和牺牲,他们的思想已经进步,很多人并已成为马克思列宁主义者了。但是,必须着重指出:任何没有无产阶级化的小资产阶级分子的革命性,在本质上和无产阶级革命性不相同,而且这种差别往往可能发展成为对抗状态。带着小资产阶级革命性的党员,虽然在组织上入了党,但是在思想上却还没有入党,或没有完全入党,他们往往是以马克思列宁主义者的面貌出现的自由主义者、改良主义者、无政府主义者、布朗基主义\footnote[48]{ 布朗基主义是指以法国布朗基(一八〇五——一八八一)为代表的一种革命冒险主义思想。布朗基主义否认阶级斗争,妄想不依靠无产阶级的阶级斗争,而用极少数知识分子的阴谋行动,就可以使人类摆脱资本主义的剥削制度。}者等等;在这种情况下,他们不但不能引导中国将来的共产主义运动达到胜利,而且也不能引导中国今天的新民主主义运动达到胜利。如果无产阶级先进分子不以马克思列宁主义的思想和这些小资产阶级出身的党员的旧有思想坚决地分清界限,严肃地、但是恰当地和耐心地进行教育和斗争,则他们的小资产阶级思想不但不能克服,而且必然力图以他们自己的本来面貌来代替党的无产阶级先进部队的面貌,实行篡党,使党和人民的事业蒙受损失。党外的小资产阶级愈是广大,党内的小资产阶级出身的党员愈是众多,则党便愈须严格地保持自己的无产阶级先进部队的纯洁性,否则小资产阶级思想向党的进攻必然愈是猛烈,而党所受的损失也必然愈是巨大。我党历史上各次错误路线和正确路线之间的斗争,实质上即是党外的阶级斗争在党内的表演;而上述“左”倾路线在政治上、军事上、组织上和思想上的错误,也即是这种小资产阶级思想在党内的反映。在这个问题上,可以从三个方面来加以分析:\\
  首先,在思想方法方面。小资产阶级的思想方法,基本上表现为观察问题时的主观性和片面性,即不从阶级力量对比之客观的全面的情况出发,而把自己主观的愿望、感想和空谈当做实际,把片面当成全面,局部当成全体,树木当做森林。脱离实际生产过程的小资产阶级知识分子,因为只有书本知识而缺乏感性知识,他们的思想方法就比较容易表现为我们前面所说的教条主义。联系生产的小资产阶级分子虽具有一定的感性知识,但是受着小生产的狭隘性、散漫性、孤立性和保守性的限制,他们的思想方法就比较容易表现为我们前面所说的经验主义。\\
  第二,在政治倾向方面。小资产阶级的政治倾向,因为他们的生活方式和由此而来的思想方法上的主观性片面性,一般地容易表现为左右摇摆。小资产阶级革命家的许多代表人物希望革命马上胜利,以求根本改变他们今天所处的地位;因而他们对于革命的长期努力缺乏忍耐心,他们对于“左”的革命词句和口号有很大的兴趣,他们容易发生关门主义和冒险主义的情绪和行动。小资产阶级的这种倾向,在党内反映出来,就构成了我们前面所说的“左”倾路线在革命任务问题、革命根据地问题、策略指导问题和军事路线问题上的各种错误。\\
  但是,这些小资产阶级革命家在另外一种情况下,或是另一部分小资产阶级革命家,也可以表现悲观失望,表现追随于资产阶级之后的右倾情绪和右倾观点。一九二四年至一九二七年革命后期的陈独秀主义,土地革命后期的张国焘主义和长征初期的逃跑主义,都是小资产阶级这种右倾思想在党内的反映。抗日时期,又曾发生过投降主义的思想。一般地说,在资产阶级和无产阶级分裂的时期,比较容易发生“左”倾错误(例如土地革命时期“左”倾路线统治党的领导机关至三次之多),而在资产阶级和无产阶级联合的时期,则比较容易发生右倾错误(例如一九二四年至一九二七年革命后期和抗日战争初期)。而无论是“左”倾或右倾,都是不利于革命而仅仅利于反革命的。由于各种情况的变化而产生的左右摇摆、好走极端、华而不实、投机取巧,是小资产阶级思想在坏的一面的特点。这是小资产阶级在经济上所处的不稳定地位在思想上的反映。\\
  第三,在组织生活方面。由于一般小资产阶级的生活方式和思想方法的限制,特别由于中国的落后的分散的宗法社会和帮口行会的社会环境,小资产阶级在组织生活上的倾向,容易表现为脱离群众的个人主义和宗派主义。这种倾向反映到党内,就造成我们前面所说的“左”倾路线的错误的组织路线。党长期地处在分散的乡村游击战争中的情况,更有利于这种倾向的发展。这种倾向,不是自我牺牲地为党和人民工作,而是利用党和人民的力量并破坏党和人民的利益来达到个人和宗派的目的,因此它是同党的联系群众的原则、党的民主集中制和党的纪律不兼容的。这种倾向,常常采取各种各样的形式,如官僚主义、家长制度、惩办主义、命令主义、个人英雄主义、半无政府主义、自由主义、极端民主主义、闹独立性、行会主义、山头主义、同乡同学观念、派别纠纷、耍流氓手腕等,破坏着党同人民群众的联系和党内的团结。\\
  这些就是小资产阶级思想的三个方面。我们党内历次发生的思想上的主观主义,政治上的“左”、右倾,组织上的宗派主义等项现象,无论其是否形成了路线,掌握了领导,显然都是小资产阶级思想之反马克思列宁主义、反无产阶级的表现。为了党和人民的利益,采取教育方法,将党内的小资产阶级思想加以分析和克服,促进其无产阶级化,是完全必要的。\\
\subsection*{\myformat{(六)}}
由上所述,可见各次尤其是第三次统治全党的“左”倾路线,不是偶然的产物,而是一定的社会历史条件的产物。因此,要克服错误的“左”倾思想或右倾思想,既不能草率从事,也不能操切从事,而必须深入马克思列宁主义的教育,提高全党对于无产阶级思想和小资产阶级思想的鉴别能力,并在党内发扬民主,展开批评和自我批评,进行耐心说服和教育的工作,具体地分析错误的内容及其危害,说明错误之历史的和思想的根源及其改正的办法。这是马克思列宁主义者克服党内错误的应有态度。扩大的六届七中全会指出:毛泽东同志在这次全党整风和党史学习中所采取的方针,即“惩前毖后,治病救人”,“既要弄清思想又要团结同志”\footnote[49]{ 见《学习和时局》(本卷第938页)。
}的方针,是马克思列宁主义者克服党内错误的正确态度的模范,因而取得了在思想上、政治上和组织上提高并团结全党的伟大成就。\\
  扩大的六届七中全会指出:在党的历史上,曾经有过反对陈独秀主义和李立三主义的斗争,这些斗争,是完全必要的。这些斗争的缺点,是没有自觉地作为改造在党内严重存在着的小资产阶级思想的严重步骤,因而没有在思想上彻底弄清错误的实质及其根源,也没有恰当地指出改正的方法,以致易于重犯错误;同时,又太着重了个人的责任,以为对于犯错误的人们一经给以简单的打击,问题就解决了。党在检讨了六届四中全会以来的错误以后,认为今后进行一切党内思想斗争时,应该避免这种缺点,而坚决执行毛泽东同志的方针。任何过去犯过错误的同志,只要他已经了解和开始改正自己的错误,就应该不存成见地欢迎他,团结他为党工作。即使还没有很好地了解和改正错误,但已不坚持错误的同志,也应该以恳切的同志的态度,帮助他去了解和改正错误。现在全党对于过去错误路线的认识,已经一致了,全党已经在以毛泽东同志为首的中央周围团结起来了。因此,全党今后的任务,就是在弄清思想、坚持原则的基础上加强团结,正像本决议的第二节上所说的:“团结全党同志如同一个和睦的家庭一样,如同一块坚固的钢铁一样,为着获得抗日战争的彻底胜利和中国人民的完全解放而奋斗”。我们党关于党内历史问题的一切分析、批判、争论,是应该从团结出发,而又达到团结的,如果违背了这个原则,那就是不正确的。但是鉴于党内小资产阶级思想的社会根源的存在以及党所处的长期分散的农村游击战争的环境,又鉴于教条主义和经验主义的思想残余还是存在着,尤其是对于经验主义还缺乏足够的批判,又鉴于党内严重的宗派主义虽然基本上已经被克服,而具有宗派主义倾向的山头主义则仍然相当普遍地存在着等项事实,全党应该警觉:要使党内思想完全统一于马克思列宁主义,还需要一个长时期的继续克服错误思想的斗争过程。因此,扩大的六届七中全会决定:全党必须加强马克思列宁主义的思想教育,并着重联系中国革命的实践,以达到进一步地养成正确的党风,彻底地克服教条主义、经验主义、宗派主义、山头主义等项倾向之目的。\\
\subsection*{\myformat{(七)}}
扩大的六届七中全会着重指出:二十四年来中国革命的实践证明了,并且还在证明着,毛泽东同志所代表的我们党和全国广大人民的奋斗方向是完全正确的。今天我党在抗日战争中所已经取得的伟大胜利及其所起的决定作用,就是这条正确路线的生动的证明。党在个别时期中所犯的“左”、右倾错误,对于二十四年来在我党领导之下的轰轰烈烈地发展着的、取得了伟大成绩和丰富经验的整个中国革命事业说来,不过是一些部分的现象。这些现象,在党还缺乏充分经验和充分自觉的时期内,是难于完全避免的;而且党正是在克服这些错误的斗争过程中而更加坚强起来,到了今天,全党已经空前一致地认识了毛泽东同志的路线的正确性,空前自觉地团结在毛泽东的旗帜下了。以毛泽东同志为代表的马克思列宁主义的思想更普遍地更深入地掌握干部、党员和人民群众的结果,必将给党和中国革命带来伟大的进步和不可战胜的力量。\\
  扩大的六届七中全会坚决相信:有了北伐战争、土地革命战争和抗日战争这样三次革命斗争的丰富经验的中国共产党,在以毛泽东同志为首的中央的正确领导之下,必将使中国革命达到彻底的胜利。\\
\newpage\section*{\myformat{为人民服务}\\\myformat{(一九四四年九月八日)}}\addcontentsline{toc}{section}{为人民服务}
\begin{introduction}\item  这是毛泽东在中共中央警备团追悼张思德的会上的讲演。\end{introduction}
我们的共产党和共产党所领导的八路军、新四军,是革命的队伍。我们这个队伍完全是为着解放人民的,是彻底地为人民的利益工作的。张思德\footnote[1]{ 张思德,四川仪陇人,中央警备团的战士。他在一九三三年参加红军,经历长征,负过伤,是一个忠实为人民服务的共产党员。一九四四年九月五日在陕北安塞县山中烧炭,因炭窑崩塌而牺牲。}同志就是我们这个队伍中的一个同志。\\
  人总是要死的,但死的意义有不同。中国古时候有个文学家叫做司马迁的说过:“人固有一死,或重于泰山,或轻于鸿毛。”\footnote[2]{ 司马迁,中国西汉时期著名的文学家和历史学家,着有《史记》一百三十篇。此处引语见《汉书•司马迁传》中的《报任少卿书》,原文是:“人固有一死,死有重于泰山,或轻于鸿毛。”}为人民利益而死,就比泰山还重;替法西斯卖力,替剥削人民和压迫人民的人去死,就比鸿毛还轻。张思德同志是为人民利益而死的,他的死是比泰山还要重的。\\
  因为我们是为人民服务的,所以,我们如果有缺点,就不怕别人批评指出。不管是什么人,谁向我们指出都行。只要你说得对,我们就改正。你说的办法对人民有好处,我们就照你的办。“精兵简政”这一条意见,就是党外人士李鼎铭\footnote[3]{ 李鼎铭(一八八一——一九四七),陕西米脂人,开明绅士。他在一九四一年十一月陕甘宁边区第二届参议会上提出“精兵简政”的提案,并在这次会议上当选为陕甘宁边区政府副主席。}先生提出来的;他提得好,对人民有好处,我们就采用了。只要我们为人民的利益坚持好的,为人民的利益改正错的,我们这个队伍就一定会兴旺起来。\\
  我们都是来自五湖四海,为了一个共同的革命目标,走到一起来了。我们还要和全国大多数人民走这一条路。我们今天已经领导着有九千一百万人口的根据地\footnote[4]{ 这是指当时陕甘宁边区和华北、华中、华南各抗日根据地所拥有的人口的总数。},但是还不够,还要更大些,才能取得全民族的解放。我们的同志在困难的时候,要看到成绩,要看到光明,要提高我们的勇气。中国人民正在受难,我们有责任解救他们,我们要努力奋斗。要奋斗就会有牺牲,死人的事是经常发生的。但是我们想到人民的利益,想到大多数人民的痛苦,我们为人民而死,就是死得其所。不过,我们应当尽量地减少那些不必要的牺牲。我们的干部要关心每一个战士,一切革命队伍的人都要互相关心,互相爱护,互相帮助。\\
  今后我们的队伍里,不管死了谁,不管是炊事员,是战士,只要他是做过一些有益的工作的,我们都要给他送葬,开追悼会。这要成为一个制度。这个方法也要介绍到老百姓那里去。村上的人死了,开个追悼会。用这样的方法,寄托我们的哀思,使整个人民团结起来。\\
\newpage\section*{\myformat{评蒋介石在双十节的演说}\\\myformat{(一九四四年十月十一日)}}\addcontentsline{toc}{section}{评蒋介石在双十节的演说}
\begin{introduction}\item  这是毛泽东为新华社写的评论。\end{introduction}
空洞无物,没有答复人民所关切的任何一个问题,是蒋介石双十演说的特色之一。蒋介石说,大后方尚有广大土地,不怕敌人。寡头专政的国民党领导者们,至今看不见他们有什么改革政治抗住敌人的意图和本领,只有“土地”一项现成资本可资抵挡。但是谁也明白,仅有这项资本是不够的,没有正确的政策和人的努力,日本帝国主义是天天在威胁这块剩余土地的。蒋介石大概是强烈地感到了敌人的这种威胁,只要看他向人民反复申述没有威胁,甚至说,“我自黄埔建军以来,二十年间,革命形势从来没有像今天这样的坚固”,就是他感到了这种威胁的反映。他又反复地说不要“丧失我们的自信”,就是在国民党队伍中,在国民党统治区的社会人士中,已有很多人丧失了信心的反映。蒋介石在寻找方法,以期重振这种信心。但是他不从政治军事经济文化的任何一个政策或工作方面去找振作的方法,他找到了拒谏饰非的方法。他说,“国际观察家”都是“莫明其妙”的,“外国舆论对我们军事政治纷纷议论”,都是相信了“敌寇汉奸造谣作祟”的缘故。说也奇怪,罗斯福一类的外国人,也和宋庆龄一类的国民党人、国民参政会的许多参政员以及一切有良心的中国人一样,都不相信蒋介石及其亲信们的好听的申辩,都“对我们军事政治纷纷议论”。蒋介石对于此种现象感到烦恼,一向没有找出一个在他认为理直气壮的论据,直到今年双十节才找到了,原来却是他们相信了“敌寇汉奸造谣作祟”。于是蒋介石在其演说中,用了极长的篇幅,痛骂这种所谓“敌寇汉奸造谣作祟”。他以为经他这一骂,一切中国人、外国人的嘴巴可被封住了。对于我的军事政治,有谁再来“纷纷议论”的么,谁就是相信“敌寇汉奸造谣作祟”!我们认为蒋介石的这种指摘,是十分可笑的。因为,对于国民党的寡头专政,抗战不力,腐败无能,对于国民党政府的法西斯主义的政令和失败主义的军令,敌寇汉奸从来没有批评过,倒是十分欢迎的。引起人们一致不满的蒋介石所著的《中国之命运》一书,日本帝国主义作过多次衷心的称赞。关于改组国民政府及其统帅部一事,也没有听见什么敌寇汉奸说过半句话,因为保存现在这样天天压迫人民和天天打败仗的政府和统帅部,正是敌寇汉奸的希望。蒋介石及其一群历来是日本帝国主义诱降的对象,难道不是事实吗?日本帝国主义原来提出的“反共”“灭党”两个口号,早已放弃了“灭党”,剩下一个“反共”,难道不是事实吗?日本帝国主义者至今还没有向国民党政府宣战,他们说,日本和国民党政府之间还没有战争状态存在呢!国民党的要人们在上海南京宁波一带的财产,至今被敌寇汉奸保存得好好的。敌酋畑俊六,派遣代表到奉化祭了蒋介石的祖坟。蒋介石的亲信们暗地里派遣使者,几乎经常不断地在上海等处和日寇保持联系,进行秘密谈判。特别是在日寇进攻紧急的时候,这种联系和谈判就来得越多。所有这些,难道不是事实吗?由此看来,对于蒋介石及其一群的军事政治发生“纷纷议论”的人们,究竟是“莫明其妙”呢,还是已明其妙呢?这个“妙”的出处,究竟是“敌寇汉奸造谣作祟”呢,还是在蒋介石自己及其一群的身上呢?\\
  在蒋介石的演说中,还有一项声明,就是他否认中国将有内战。但是他又说:“决没有人再敢背叛民国,破坏抗战,如汪精卫之流之所为。”蒋介石是在这里寻找内战的根据,并且他是找着了。有记性的中国人不会忘记,一九四一年,正当中国叛卖者们宣布解散新四军,中国人民起来制止内战危机的时候,在蒋介石的一次演说中,曾说:将来决不会有“剿共”战争,如果有的话,那就是讨伐叛逆的战争。读过《中国之命运》的人们也会记得,蒋介石在那里曾说:中共在一九二七年武汉政府时期“勾结”过汪精卫。一九四三年国民党十一中全会的决议上,又给中共安上了“破坏抗战危害国家”的八字由头。现在又读了蒋介石这篇演说,就使人们感觉内战危险不但存在,而且在发展着。中国人民现在就要牢牢地记着,不知哪一天的早晨,蒋介石会要下令讨伐所谓“叛逆”的,那时的罪状就是“背叛民国”,就是“破坏抗战”,就是“如汪精卫之流之所为”。蒋介石是擅长这一手的,他不擅长于宣布庞炳勋、孙良诚、陈孝强\footnote[1]{ 庞炳勋,曾任国民党河北省政府主席、冀察战区副总司令兼第二十四集团军总司令;孙良诚,曾任国民党山东省政府主席、第三十九集团军副总司令;陈孝强,曾任国民党第二十七军预备第八师师长。他们于一九四二年、一九四三年间公开投降日本侵略者。}一流人为叛逆,也不擅长于讨伐他们,但是他却擅长于宣布华中的新四军和山西的决死队\footnote[2]{ 山西的决死队,指山西青年抗敌决死队。它是抗日战争初期,由中国共产党人在与阎锡山建立统一战线的过程中组建和领导的山西人民的抗日武装。参见本书第二卷《团结一切抗日力量,反对反共顽固派》注〔4〕。}为“叛逆”,并且极擅长于讨伐他们。中国人民决不要忘记,当着蒋介石声言不打内战的时候,他已经派遣了七十七万五千人的军队,这些军队正在专门包围或攻打八路军、新四军和华南的人民游击队。\\
  蒋介石的演说在积极方面空洞无物,他没有替中国人民所热望的改善抗日阵线找出任何答案。在消极方面,这篇演说却充满了危险性。蒋介石的态度越变越反常了,他坚决地反对人民改革政治的要求,强烈地仇视中国共产党,暗示了他所准备的反共内战的借口。但是,蒋介石的这一切企图是不能成功的。如果他不愿意改变他自己的作法的话,他将搬起石头打他自己的脚。我们诚恳地希望他改变作法,因为他现在的作法是绝对地行不通的。他已宣布“放宽言论尺度”\footnote[3]{ 一九四四年以来,要求结束国民党的独裁统治、实行民主、保障言论自由,成为国民党统治区域的人民的普遍呼声。国民党为了搪塞人民的迫切要求,一九四四年四月,宣布所谓“放宽言论尺度”;五月,国民党五届十二中全会又宣言“保障言论自由”。但是国民党被迫所作的这些表示,事后一点也没有兑现,其压制人民言论的措施,随着人民民主运动的高涨而层出不穷。},就不应该以相信“敌寇汉奸造谣作祟”的污蔑之词来威胁和封闭人们“纷纷议论”之口。他已宣布“缩短训政时期”,就不应该拒绝人们改组政府和改组统帅部的要求。他已宣布“用政治方法解决共党问题”,就不应该又来寻找准备内战的理由。\\
\newpage\section*{\myformat{文化工作中的统一战线}\\\myformat{(一九四四年十月三十日)}}\addcontentsline{toc}{section}{文化工作中的统一战线}
\begin{introduction}\item  这是毛泽东在陕甘宁边区文教工作者会议上所作的讲演。\end{introduction}
我们的一切工作都是为了打倒日本帝国主义。日本帝国主义和希特勒一样,快要灭亡了。但是还须我们继续努力,才能最后地消灭它。我们的工作首先是战争,其次是生产,其次是文化。没有文化的军队是愚蠢的军队,而愚蠢的军队是不能战胜敌人的。\\
  解放区的文化已经有了它的进步的方面,但是还有它的落后的方面。解放区已有人民的新文化,但是还有广大的封建遗迹。在一百五十万人口的陕甘宁边区内,还有一百多万文盲,两千个巫神,迷信思想还在影响广大的群众。这些都是群众脑子里的敌人。我们反对群众脑子里的敌人,常常比反对日本帝国主义还要困难些。我们必须告诉群众,自己起来同自己的文盲、迷信和不卫生的习惯作斗争。为了进行这个斗争,不能不有广泛的统一战线。而在陕甘宁边区这样人口稀少、交通不便、原有文化水平很低的地方,加上在战争期间,这种统一战线就尤其要广泛。因此,在教育工作方面,不但要有集中的正规的小学、中学,而且要有分散的不正规的村学、读报组和识字组。不但要有新式学校,而且要利用旧的村塾加以改造。在艺术工作方面,不但要有话剧,而且要有秦腔\footnote[1]{ 秦腔,又名梆子腔,是流行于中国西北地区的具有悠久历史的地方戏曲。}和秧歌。不但要有新秦腔、新秧歌,而且要利用旧戏班,利用在秧歌队总数中占百分之九十的旧秧歌队,逐步地加以改造。在医药方面,更是如此。陕甘宁边区的人、畜死亡率都很高,许多人民还相信巫神。在这种情形之下,仅仅依靠新医是不可能解决问题的。新医当然比旧医高明,但是新医如果不关心人民的痛苦,不为人民训练医生,不联合边区现有的一千多个旧医和旧式兽医,并帮助他们进步,那就是实际上帮助巫神,实际上忍心看着大批人畜的死亡。统一战线的原则有两个:第一个是团结,第二个是批评、教育和改造。在统一战线中,投降主义是错误的,对别人采取排斥和鄙弃态度的宗派主义也是错误的。我们的任务是联合一切可用的旧知识分子、旧艺人、旧医生,而帮助、感化和改造他们。为了改造,先要团结。只要我们做得恰当,他们是会欢迎我们的帮助的。\\
  我们的文化是人民的文化,文化工作者必须有为人民服务的高度的热忱,必须联系群众,而不要脱离群众。要联系群众,就要按照群众的需要和自愿。一切为群众的工作都要从群众的需要出发,而不是从任何良好的个人愿望出发。有许多时候,群众在客观上虽然有了某种改革的需要,但在他们的主观上还没有这种觉悟,群众还没有决心,还不愿实行改革,我们就要耐心地等待;直到经过我们的工作,群众的多数有了觉悟,有了决心,自愿实行改革,才去实行这种改革,否则就会脱离群众。凡是需要群众参加的工作,如果没有群众的自觉和自愿,就会流于徒有形式而失败。“欲速则不达”\footnote[2]{ 见《论语•子路》。},这不是说不要速,而是说不要犯盲动主义,盲动主义是必然要失败的。在一切工作中都是如此;在改造群众思想的文化教育工作中尤其是如此。这里是两条原则:一条是群众的实际上的需要,而不是我们脑子里头幻想出来的需要;一条是群众的自愿,由群众自己下决心,而不是由我们代替群众下决心。\\
\newpage\section*{\myformat{必须学会做经济工作}\\\myformat{(一九四五年一月十日)}}\addcontentsline{toc}{section}{必须学会做经济工作}
\begin{introduction}\item  这是毛泽东在陕甘宁边区劳动英雄和模范工作者大会上的讲话。\end{introduction}
\\~\\
各位劳动英雄,各位模范工作者!\\
  你们开了会,总结了经验,大家欢迎你们,尊敬你们。你们有三种长处,起了三个作用。第一个,带头作用。这就是因为你们特别努力,有许多创造,你们的工作成了一般人的模范,提高了工作标准,引起了大家向你们学习。第二个,骨干作用。你们的大多数现在还不是干部,但是你们已经是群众中的骨干,群众中的核心,有了你们,工作就好推动了。到了将来,你们可能成为干部,你们现在是干部的后备军。第三个,桥梁作用。你们是上面的领导人员和下面的广大群众之间的桥梁,群众的意见经过你们传上来,上面的意见经过你们传下去。\\
  你们有许多的长处,有很大的功劳,但是你们切记不可以骄傲。你们被大家尊敬,是应当的,但是也容易因此引起骄傲。如果你们骄傲起来,不虚心,不再努力,不尊重人家,不尊重干部,不尊重群众,你们就会当不成英雄和模范了。过去已有一些这样的人,希望你们不要学他们。\\
  你们的经验,这次会议做了总结。这个总结文件说得很好,不但这里适用,各地也可以适用,我就不讲这些了。我想讲一点我们的经济工作。\\
  近几年中,我们开始学会了经济工作,我们在经济工作中有了很大的成绩,但这还只是开始。我们必须在两三年内,使陕甘宁边区和敌后各解放区,做到粮食和工业品的全部或大部的自给,并有盈余。我们必须使农业、工业、贸易三方面都比现在有更大的成绩。到了那时,才算学得更多,学得更好。如果哪一个地方的军民生活没有改善,为着反攻而准备的物质基础还不稳固,农业、工业、贸易不是一年一年地上涨,而是停止不进,甚至下降,便证明哪个地方的党政军工作人员还是没有学会经济工作,哪个地方就会遇到绝大的困难。\\
  有一个问题必须再一次引起大家注意的,就是我们的思想要适合于目前我们所处的环境。我们目前所处的环境是农村,这一点好像并没有什么问题,谁不知道我们是处在农村中呢?其实不然。我们有很多同志,虽然天天处在农村中,甚至自以为了解农村,但是他们并没有了解农村,至少是了解得不深刻。他们不从建立在个体经济基础上的、被敌人分割的、因而又是游击战争的农村环境这一点出发,他们就在政治问题上,军事问题上,经济问题上,文化问题上,党务问题上,工人运动、农民运动、青年运动、妇女运动等项的问题上,常常处理得不适当,或不大适当。他们带着城市观点去处理农村,主观地作出许多不适当的计划,强制施行,常常碰了壁。近几年来,由于整风,由于在工作中碰了钉子,我们的同志有了很多的进步。但是还须注意使我们的思想完全适合于我们所处的环境,然后才能使我们的工作样样见效,并迅速见效。如果我们真正了解了我们所处的环境是一个建立在个体经济基础上的、被敌人分割的、因而又是游击战争的农村根据地,如果我们所做的一切都是从这一点出发,看起来收效很慢,并不轰轰烈烈,但是在实际上,比较那种不从这一点出发而从别一点出发,例如说,从城市观点出发,其工作效果会怎么样呢?那就决不是很慢,反而是很快的。因为,如果我们从后一点出发,脱离今天的实际情况,做起来不是效率快慢的问题,而是老碰钉子,根本没有效果的问题。\\
  比如我们提倡采取现有样式的军民生产运动,发生了很大的效果,就是一个明显的证据。\\
  我们要打击日本侵略者,并且还要准备攻入城市,收复失地。然而我们是处在个体经济的被分割的游击战争的农村环境中,怎样能够达到这个目的呢?我们不能学国民党那样,自己不动手专靠外国人,连棉布这样的日用品也要依赖外国。我们是主张自力更生的。我们希望有外援,但是我们不能依赖它,我们依靠自己的努力,依靠全体军民的创造力。那末,有些什么办法呢?我们就用军民两方同时发动大规模生产运动这一种办法。\\
  由于是农村,人力物力都是分散的,我们的生产和供给就采取“统一领导,分散经营”的方针。\\
  由于是农村,农民都是分散的个体生产者,使用着落后的生产工具,而大部分土地又还为地主所有,农民受着封建的地租剥削,为了提高农民的生产兴趣和农业劳动的生产率,我们就采取减租减息和组织劳动互助这样两个方针。减租提高了农民的生产兴趣,劳动互助提高了农业劳动的生产率。我已得了华北华中各地的材料,这些材料都说:减租之后,农民生产兴趣大增,愿意组织如同我们这里的变工队一样的互助团体,三个人的劳动效率抵过四个人。如果是这样,九千万人就可以抵过一亿二千万人。还有两个人抵过三个人的。如果不是采取强迫命令、欲速不达的方针,而是采取耐心说服、典型示范的方针,那末,几年之内,就可能使大多数农民都组织在农业生产的和手工业生产的互助团体里面。这种生产团体,一经成为习惯,不但生产量大增,各种创造都出来了,政治也会进步,文化也会提高,卫生也会讲究,流氓也会改造,风俗也会改变;不要很久,生产工具也会有所改良。到了那时,我们的农村社会,就会一步一步地建立在新的基础的上面了。\\
  如果我们的工作人员用心地研究这项工作,用极大的精力帮助农村人民展开生产运动,几年之内,农村就会有丰富的粮食和日用品,不但可以坚持战斗,不但可以对付荒年,而且可以贮藏大批粮食和日用品,以为将来之用。\\
  不但要组织农民生产,而且要组织部队和机关一齐生产。\\
  由于是农村,由于是经常被敌人摧残的农村,由于是长期战争的农村,部队和机关就必须生产。由于是分散的游击战争,部队和机关也可能生产。在我们陕甘宁边区,则更由于部队和机关的人数和边区人口比较,所占比例数太大,如果不自己生产,则势将饿饭;如果取之于民太多,则人民负担不起,人民也势将饿饭。因此,我们决定开展大规模的生产运动。拿陕甘宁边区说,部队和机关每年需细粮(小米)二十六万担(每担三百斤),取之于民的占十六万担,自己生产的占十万担,如果不自己生产,则军民两方势必有一方要饿饭。由于展开了生产运动,现在我们不但不饿饭,而且军民两方面都吃得很好。\\
  我们边区的机关,除粮食被服两项之外,其它用费,大部自给,有些单位则全部自给。另有许多单位,并且自给一部分粮食,一部分被服。\\
  边区部队的功劳更大。许多部队,粮食被服和其它一切,全部自给,即自给百分之一百,不领政府一点东西。这是最高的标准,这是第一个标准,是在几年之内逐渐达到的。\\
  前方要作战,不能采取这个标准。前方可以设立第二、第三两个标准。第二个标准是除粮食被服两项由政府供给之外,其它如油(每人每日五钱)、盐(每人每日五钱)、菜(每人每日一斤至一斤半)、肉(每人每月一斤至二斤)、柴炭费、办公费、杂支费、教育费、保健费、擦枪费、旱烟、鞋子、袜子、手套、毛巾、牙刷等,一概生产自给,约占全部用费的百分之五十,可以在两年至三年内逐渐地做到。现在已有做到了的。这个标准,在巩固区内可以实行。\\
  第三个标准,是在边沿区和游击区内实行的,他们不可能自给百分之五十,但是可能自给百分之十五到二十五。能够这样,也就很好。\\
  总之,除有特殊情形者外,一切部队、机关,在战斗、训练和工作的间隙里,一律参加生产。部队和机关,除利用战斗、训练和工作的间隙,集体参加生产之外,应组织专门从事生产的人员,创办农场、菜园、牧场、作坊、小工厂、运输队、合作社,或者和农民伙种粮、菜。在目前条件下,为着渡过困难,任何机关、部队,都应建立起自己的家务。不愿建立家务的二流子习气,是可耻的。还应规定按质分等的个人分红制度,使直接从事生产的人员能够分得红利,借以刺激生产的发展。又须首长负责,自己动手,实行领导骨干和广大群众相结合、一般号召和具体指导相结合的办法,才能有效地推进生产工作。\\
  有人说:部队生产,就不能作战和训练了;机关生产,就不能工作了。这种说法是不对的。最近几年,我们边区部队从事大量的生产,衣食丰足,同时又进行练兵,又有政治和文化学习,这些都比从前有更大的成绩,军队内部的团结和军民之间的团结,也比从前更好了。在前方,去年一年进行了大规模的生产运动,可是去年一年作战方面有很大的成绩,并且普遍地开始了练兵运动。机关因为生产,工作人员生活改善了,工作更安心、更有效率,边区和前方都是这样。\\
  由此可见,处在农村游击战争环境中的机关和部队,如果有了生产自给运动,他们的战斗、训练和工作,就更加有劲,更加活跃了;他们的纪律,他们的内部的团结和外部的团结,也就更好了。这是我们中国长期游击战争的产物,这是我们的光荣。我们学会了这一条,我们就对一切物质困难都不怕了。我们将一年一年地更有生气,更有精力,愈战愈强,只有我们去压倒敌人,决不怕敌人来压倒我们。\\
  在这里,有一点还须引起我们前方同志的注意。我们有些地区开辟不久,还颇富足,但是那里的工作人员自恃富足,不肯节省,也不肯生产。这样就很不好,他们在将来一定会要吃亏的。任何地方必须十分爱惜人力物力,决不可只顾一时,滥用浪费。任何地方必须从开始工作的那一年起,就计算到将来的很多年,计算到长期坚持战争,计算到反攻,计算到赶走敌人之后的建设。一面决不滥用浪费,一面努力发展生产。过去有些地方缺少长期打算,既未注意节省人力物力,又未注意发展生产,吃了大亏。得了这个教训,现在必须引起注意。\\
  关于工业品,陕甘宁边区决定在两年内,做到花、纱、布、铁、纸及其它很多用品的完全自给。原来根本没有或者出产很少的,要一概自种自造自给,完全不靠外面。所有这些,由公营、私营和合作社经营三方面完成任务。一切产品,不但求数量多,而且求质量好,耐穿耐用。边区政府、八路军联防司令部、党中央西北局,对于这些抓得很紧,这是非常之对的。希望前方各地也是这样做。有许多地方已是这样做了,希望他们得到成功。\\
  我们边区和整个解放区,还要有两年至三年工夫,才能学会全部的经济工作。我们到了粮食和工业品全部或大部自种自造自给并有盈余的日子,就是我们全部学会在农村中如何做经济工作的日子。将来从城市赶跑敌人,我们也会做新的经济工作了。中国靠我们来建设,我们必须努力学习。\\
\newpage\section*{\myformat{游击区也能够进行生产}\\\myformat{(一九四五年一月三十一日)}}\addcontentsline{toc}{section}{游击区也能够进行生产}
\begin{introduction}\item  这是毛泽东为延安《解放日报》写的社论。\end{introduction}
我们在敌后解放区中那些比较巩固的根据地内,能够和必须发动军民的生产运动的问题,早已解决了,不成问题了。但是在游击区中,在敌后之敌后,是否也能够这样,在过去,在许多人的思想中,还是没有解决的,这是因为还缺少证明的缘故。\\
  可是现在有了证据了。根据一月二十八日《解放日报》所载张平凯同志关于晋察冀游击队的生产运动的报道,晋察冀边区的许多游击区内,已于一九四四年进行了大规模的生产,并且收到了极好的成绩。张同志报道中所提到的区域和部队,有冀中的第六分区,第二分区的第四区队,第四分区的第八区队,徐定支队,保满支队,云彪支队,有山西的代县和崞县\footnote[1]{ 崞县,今山西省原平县。}的部队。那些区域的环境是很恶劣的:“敌伪据点碉堡林立,沟墙公路如网,敌人利用它的军事上的优势和便利的交通条件,时常对我袭击,包围,‘清剿’;游击队为了应付环境,往往一日数处地转移。”然而他们仍然能够于战争的间隙,进行了生产。其结果:“使得大家的给养有了改善,每人每日增加到五钱油和盐,一斤菜,每月斤半肉。而且几年没有用过的牙刷、牙粉和识字本,现在也都齐全了。”大家看,谁说游击区不能生产呢!\\
  许多人说:人稠的地方没有土地。果真没有土地吗?请看晋察冀:“首先在农业为主的方针下,解决了土地问题。他们共有九种办法:第一,平毁封锁墙沟;第二,平毁可被敌人利用的汽车路,在其两旁种上庄稼;第三,利用小块荒地;第四,协助民兵,用武装掩护,月夜强种敌人堡垒底下的土地;第五,与缺乏劳动力的农民伙耕;第六,部队化装,用半公开的形式,耕种敌人据点碉堡旁边的土地;第七,利用河沿,筑堤修滩,起沙成地;第八,协助农民改旱地为水地;第九,利用自己活动的村庄,到处伴种。”\\
  农业生产是可以的,手工业及其它生产大概不能吧?果真不能吗?请看晋察冀:“沟线外部队的生产,不限于农业,而且也和巩固区一样,开展了手工业和运输业。第四区队开设了一个毡帽坊,一个油坊,一个面坊,七个月中盈利五十万元本币。不仅解决了本身困难,而且游击区群众的需要也解决了。毛衣毛袜等,战士们已能全部自给。”\\
  游击区战斗那样频繁,军队从事生产,恐怕要影响作战吧?果真如此吗?请看晋察冀:“实现了劳力和武力相结合的原则,把战斗任务和生产任务同样看重。”“以第二分区第四区队为例。当春耕开始时,就派有专门的部队去打击敌人,并进行强有力的政治攻势。正因为这样,军事动作也积极了,部队战斗力也提高了。这个小部队从二月至九月初,作了七十一次战斗,攻下了朱东社、上庄、野庄、凤家寨、崖头等据点,毙伤敌伪一百六十五名,俘伪军九十一名,缴了三挺轻机枪,一百零一枝长短枪。”“把军事动作和大生产运动的宣传配合起来,马上进行政治攻势:‘谁要破坏大生产运动就打击谁。’代、崞等县城内敌人问老百姓:‘为什么八路军近来这么厉害?’老百姓说:‘因为你们破坏边区的大生产运动。’伪军在下面纷纷议论:‘人家搞大生产运动,可不要出去。’”\\
  游击区人民群众是否也可以发动生产运动呢?那些地方,也许是还没有减租,或减租不彻底的,农民是否也有兴趣去增加生产呢?这一点,晋察冀那边也肯定地答复了。“沟线外部队生产运动的开展,还给了当地群众以直接的帮助。一方面,用武力掩护了群众的生产;另一方面,又用劳力进行了普遍的帮助。有的部队,规定了农忙时期以百分之五十的力量,无代价地帮助群众生产。群众生产情绪因此大大提高,军民关系更为融洽,群众都有了饭吃。游击区群众对共产党、八路军的同情和拥护,从此更增高一步。”\\
  游击区能够和必须进行军民的大规模的生产运动,一切问题都解决了。我们要求一切解放区党政军工作人员,特别是游击区工作人员,从思想上完全认识这一点,认识这个“能够”和“必须”,事情就可以普遍地办起来。晋察冀边区也正是从这里开始的:“在沟线外部队的生产运动中,由于干部的思想转变,重视生产,重视劳力和武力相结合,培养了群众中的英雄模范(初步总结中,有六十六个英雄模范),仅仅五个月中,我们沟线外的部队,不仅在生产任务上按时完成了计划,而且特别有了许多实事求是的新创造。”\\
  一九四五年,整个解放区,必须全体一致地从事一个比过去规模更大的军民生产运动,到今年冬季,我们来比较各区的成绩。\\
  战争不但是军事的和政治的竞赛,还是经济的竞赛。我们要战胜日本侵略者,除其它一切外,还必须努力于经济工作,必须于两三年内完全学会这一门;而在今年——一九四五年,必须收到较前更大的成绩。这是中共中央所殷殷盼望于整个解放区全体工作人员和全体人民的,我们希望这一计划能够完成。\\
\newpage\section*{\myformat{两个中国之命运}\\\myformat{(一九四五年四月二十三日)}}\addcontentsline{toc}{section}{两个中国之命运}
\begin{introduction}\item  这是毛泽东在中国共产党第七次全国代表大会上的开幕词。\end{introduction}
同志们!中国共产党第七次全国代表大会今天开幕了。\\
  我们这个大会有什么重要意义呢?我们应该讲,我们这次大会是关系全中国四亿五千万人民命运的一次大会。中国之命运有两种:一种是有人已经写了书的\footnote[1]{ 指一九四三年出版的蒋介石《中国之命运》一书。};我们这个大会是代表另一种中国之命运,我们也要写一本书出来\footnote[2]{ 指毛泽东准备在这次大会上作的《论联合政府》的报告。}。我们这个大会要打倒日本帝国主义,把全中国人民解放出来。这个大会是一个打败日本侵略者、建设新中国的大会,是一个团结全中国人民、团结全世界人民、争取最后胜利的大会。\\
  现在的时机很好。在欧洲,希特勒快要被打倒了。世界反法西斯战争的主要的一部分是在西方,那里的战争很快就要胜利了,这是苏联红军努力的结果。现在柏林已经听到红军的炮声,大概在不久就会打下来。在东方,打倒日本帝国主义的战争也接近着胜利的时节。我们的大会是处在反法西斯战争最后胜利的前夜。\\
  在中国人民面前摆着两条路,光明的路和黑暗的路。有两种中国之命运,光明的中国之命运和黑暗的中国之命运。现在日本帝国主义还没有被打败。即使把日本帝国主义打败了,也还是有这样两个前途。或者是一个独立、自由、民主、统一、富强的中国,就是说,光明的中国,中国人民得到解放的新中国;或者是另一个中国,半殖民地半封建的、分裂的、贫弱的中国,就是说,一个老中国。一个新中国还是一个老中国,两个前途,仍然存在于中国人民的面前,存在于中国共产党的面前,存在于我们这次代表大会的面前。\\
  既然日本现在还没有被打败,既然打败日本之后,还是存在着两个前途,那末,我们的工作应当怎样做呢?我们的任务是什么呢?我们的任务不是别的,就是放手发动群众,壮大人民力量,团结全国一切可能团结的力量,在我们党领导之下,为着打败日本侵略者,建设一个光明的新中国,建设一个独立的、自由的、民主的、统一的、富强的新中国而奋斗。我们应当用全力去争取光明的前途和光明的命运,反对另外一种黑暗的前途和黑暗的命运。我们的任务就是这一个!这就是我们大会的任务,这就是我们全党的任务,这就是全中国人民的任务。\\
  我们的希望能不能实现?我们认为是能够实现的。这个可能性是存在的,因为我们现在已经具备了这样几个条件:\\
  第一,有一个经验丰富和集合了一百二十一万党员的强大的中国共产党;\\
  第二,有一个强大的解放区,这个解放区包括九千五百五十万人口,九十一万军队,二百二十万民兵;\\
  第三,有全国广大人民的援助;\\
  第四,有全世界各国人民特别是苏联的援助。\\
  一个强大的中国共产党,一个强大的解放区,全国人民的援助,国际人民的援助,在这些条件下,我们的希望能不能实现呢?我们认为是能够实现的。这些条件,在中国是从来没有过的。多少年来虽然有了一些条件,但是没有现在这样完备。中国共产党从来没有现在这样强大过,革命根据地从来没有现在这样多的人口和这样大的军队,中国共产党在日本和国民党统治区域的人民中的威信也以现在为最高,苏联和各国人民的革命力量现在也是最大的。在这些条件下,打败侵略者,建设新中国,应当说是完全可能的。\\
  我们需要一个正确的政策。这个政策的基本点,就是放手发动群众,壮大人民的力量,在我们党领导之下,打败侵略者,建设新中国。\\
  中国共产党从一九二一年产生以来,已经二十四年了,其间经过了北伐战争、土地革命战争、抗日战争这样三个英勇奋斗的历史时期,积累了丰富的经验。到了现在,我们的党已经成了中国人民抗日救国的重心,已经成了中国人民解放的重心,已经成了打败侵略者、建设新中国的重心。中国的重心不在任何别的方面,而在我们这一方面。\\
  我们应该谦虚,谨慎,戒骄,戒躁,全心全意地为中国人民服务,在现时,为着团结全国人民战胜日本侵略者,在将来,为着团结全国人民建设新民主主义的国家。只要我们能够这样做,只要我们有正确的政策,只要我们一致努力,我们的任务是必能完成的。\\
  打倒日本帝国主义!\\
  中国人民解放万岁!\\
  中国共产党万岁!\\
  中国共产党第七次全国代表大会万岁!\\
\newpage\section*{\myformat{论联合政府}\\\myformat{(一九四五年四月二十四日)}}\addcontentsline{toc}{section}{论联合政府}
\begin{introduction}\item  这是毛泽东在中国共产党第七次全国代表大会上的政治报告。\end{introduction}
\subsection*{\myformat{一 中国人民的基本要求}}
我们的大会是在这种情况之下开会的:中国人民在其对于日本侵略者作了将近八年的坚决的英勇的不屈不挠的奋斗,经历了无数的艰难困苦和自我牺牲之后,出现了这样的新局面——整个世界上反对法西斯侵略者的神圣的正义的战争,已经取得了有决定意义的胜利,中国人民配合同盟国打败日本侵略者的时机,已经迫近了。但是中国现在仍然不团结,中国仍然存在着严重的危机。在这种情况下,我们应该怎样做呢?毫无疑义,中国急需把各党各派和无党无派的代表人物团结在一起,成立民主的临时的联合政府,以便实行民主的改革,克服目前的危机,动员和统一全中国的抗日力量,有力地和同盟国配合作战,打败日本侵略者,使中国人民从日本侵略者手中解放出来。然后,需要在广泛的民主基础之上,召开国民代表大会,成立包括更广大范围的各党各派和无党无派代表人物在内的同样是联合性质的民主的正式的政府,领导解放后的全国人民,将中国建设成为一个独立、自由、民主、统一和富强的新国家。一句话,走团结和民主的路线,打败侵略者,建设新中国。\\
  我们认为只有这样做,才是反映了中国人民的基本要求。因此,我的报告,主要地就是讨论这些要求。中国应否成立民主的联合政府,已成了中国人民和同盟国民主舆论界十分关心的问题。因此,我的报告将着重地说明这个问题。\\
  中国共产党在八年抗日战争中的工作,已经克服了很多的困难,获得了巨大的成绩;但是在目前形势下,在我们党和人民面前,还有严重的困难。目前的时局,要求我们党进一步地从事紧急的和更加切实的工作,继续地克服困难,为完成中国人民的基本要求而奋斗。\\
\subsection*{\myformat{二 国际形势和国内形势}}
中国人民能不能实现我们在上面提出的那些基本要求呢?这要依靠中国人民的觉悟、团结和努力的程度来决定。但是目前的国际国内形势,对中国人民提供了极其有利的条件。中国人民如果能很好地利用这些条件,积极地坚决地再接再厉地向前奋斗,战胜侵略者和建设新中国,是毫无疑义的。中国人民应当加倍努力,为完成自己的神圣任务而奋斗。\\
  目前的国际形势是怎样的呢?\\
  目前的军事形势是苏军已经攻击柏林,英美法联军也正在配合打击希特勒残军,意大利人民又已经发动了起义。这一切,将最后地消灭希特勒。希特勒被消灭以后,打败日本侵略者就为时不远了。和中外反动派的预料相反,法西斯侵略势力是一定要被打倒的,人民民主势力是一定要胜利的。世界将走向进步,决不是走向反动。当然应该提起充分的警觉,估计到历史的若干暂时的甚至是严重的曲折,可能还会发生;许多国家中不愿看见本国人民和外国人民获得团结、进步和解放的反动势力,还是强大的。谁要是忽视了这些,谁就将在政治上犯错误。但是,历史的总趋向已经确定,不能改变了。这种情况,仅仅不利于法西斯和实际上帮助法西斯的各国反动派,而对于一切国家的人民及其有组织的民主势力,则都是福音。人民,只有人民,才是创造世界历史的动力。苏联人民创造了强大力量,充当了打倒法西斯的主力军。苏联人民加上其它反法西斯同盟国的人民的努力,使打倒法西斯成为可能。战争教育了人民,人民将赢得战争,赢得和平,又赢得进步。\\
  这一新形势,和第一次世界大战时代的形势大不相同。在那时,还没有苏联,也没有现在许多国家的人民的觉悟程度。两次世界大战是两个完全不同的时代。\\
  法西斯侵略国家被打败、第二次世界大战结束、国际和平实现以后,并不是说就没有了斗争。广泛地散布着的法西斯残余势力,一定还要捣乱;反法西斯侵略战争的阵营中存在着反民主的和压迫其它民族的势力,他们仍然要压迫各国人民和各殖民地半殖民地。所以,国际和平实现以后,反法西斯的人民大众和法西斯残余势力之争,民主和反民主之争,民族解放和民族压迫之争仍将充满世界的大部分地方。只有经过长期的努力,克服了法西斯残余势力、反民主势力和一切帝国主义势力,才能有最广泛的人民的胜利。到达这一天,决不是很快和很容易的,但是必然要到达这一天。反法西斯的第二次世界大战的胜利,给这个战后人民斗争的胜利开辟了道路。也只有这后一种斗争胜利了,巩固的和持久的和平才有保障。\\
  目前的国内形势是怎样的呢?\\
  中国的长期战争,使中国人民付出了并且还将再付出重大的牺牲;但是同时,正是这个战争,锻炼了中国人民。这个战争促进中国人民的觉悟和团结的程度,是近百年来中国人民的一切伟大的斗争没有一次比得上的。在中国人民面前,不但存在着强大的民族敌人,而且存在着强大的实际上帮助民族敌人的国内反动势力,这是一方面。但是另一方面,中国人民不但已经有了比过去任何时候都高的觉悟程度,而且有了强大的中国解放区和日益高涨着的全国性的民主运动。这是国内的有利条件。如果说,中国近百年来一切人民斗争都遭到了失败或挫折,而这是因为缺乏国际的和国内的若干必要的条件,那末,这一次就不同了,比较以往历次,一切必要的条件都具备了。避免失败和取得胜利的可能性充分地存在着。如果我们能够团结全国人民,努力奋斗,并给以适当的指导,我们就能够胜利。\\
  中国人民团结起来打败侵略者和建设新中国的信心,现在是极大地增强了。中国人民克服一切困难,实现其具有伟大历史意义的基本要求的时机,已经到来了。这一点还有疑义吗?我以为没有疑义了。\\
  这些,就是目前国际和国内的一般形势。\\
\subsection*{\myformat{三 抗日战争中的两条路线}}
\subsubsection*{\myformat{中国问题的关键}}
谈到国内形势,我们还应对中国抗日战争加以具体的分析。\\
  中国是全世界参加反法西斯战争的五个最大的国家之一,是在亚洲大陆上反对日本侵略者的主要国家。中国人民不但在抗日战争中起了极大的作用,而且在保障战后世界和平上将起极大的作用,在保障东方和平上则将起决定的作用。中国在八年抗日战争中,为了自己的解放,为了帮助各同盟国,曾经作了伟大的努力。这种努力,主要地是属于中国人民方面的。中国军队的广大官兵,在前线流血战斗,中国的工人、农民、知识界、产业界,在后方努力工作,海外华侨输财助战,一切抗日政党,除了那些反人民分子外,都对战争有所尽力。总之,中国人民以自己的血和汗同日本侵略者英勇地奋战了八年之久。但是多年以来,中国反动分子造作谣言,蒙蔽舆论,不使中国人民在抗日战争中所起作用的真相为世人所知。同时,对于中国八年抗日战争的各项经验,也还没有人作出全面的总结来。因此,我们的大会,应当对这些经验作出适当的总结,借以教育人民,并为我党决定政策的根据。\\
  提到总结经验,那末,大家可以很清楚地看到,在中国有两条不同的指导路线,一条是能够打败日本侵略者的,一条是不但不能打败日本侵略者,而且在某些方面说来它是在实际上帮助日本侵略者危害抗日战争的。\\
  国民党政府所采取的对日消极作战的政策和对内积极摧残人民的反动政策,招致了战争的挫折,大部国土的沦陷,财政经济的危机,人民的被压迫,人民生活的痛苦,民族团结的破坏。这种反动政策妨碍了动员和统一一切中国人民的抗日力量进行有效的战争,妨碍了中国人民的觉醒和团结。但是,中国人民的觉醒和团结的运动并没有停止,它是在日本侵略者和国民党政府的双重压迫之下曲折地发展着。两条路线:国民党政府压迫中国人民实行消极抗战的路线和中国人民觉醒起来团结起来实行人民战争的路线,很久以来,就明显地在中国存在着。这就是一切中国问题的关键所在。\\
\subsubsection*{\myformat{走着曲折道路的历史}}
为了使大家明了何以这个两条路线问题是一切中国问题的关键所在,必须回溯一下我们抗日战争的历史。\\
  中国人民的抗日战争,是在曲折的道路上发展起来的。这个战争,还是在一九三一年就开始了。一九三一年九月十八日,日本侵略者占领沈阳\footnote[1]{ 一九三一年九月十八日,日本驻在中国东北境内的“关东军”进攻沈阳,九月十九日晨占领了沈阳城。},几个月内,就把东三省占领了。国民党政府采取了不抵抗政策。但是东三省的人民,东三省的一部分爱国军队,在中国共产党领导或协助之下,违反国民党政府的意志,组织了东三省的抗日义勇军和抗日联军,从事英勇的游击战争。这个英勇的游击战争,曾经发展到很大的规模,中间经过许多困难挫折,始终没有被敌人消灭。一九三二年,日本侵略者进攻上海,国民党内的一派爱国分子,又一次违反国民党政府的意志,率领十九路军,抵抗了日本侵略者的进攻。一九三三年,日本侵略者进攻热河、察哈尔\footnote[2]{ 热河,原来是一个省,一九五五年撤销,原辖地区划归河北、辽宁两省和内蒙古自治区。察哈尔,原来也是一个省,一九五二年撤销,原辖地区划归河北、山西两省。},国民党内的又一派爱国分子,第三次违反国民党政府的意志,并和共产党合作,组织了抗日同盟军,从事抵抗。但是一切这些抗日战争,除了中国人民、中国共产党、其它民主派别和海外爱国华侨给了援助之外,国民党政府根据其不抵抗政策,是没有给任何援助的。相反地,上海、察哈尔两次抗日行动,都被国民党政府一手破坏了。一九三三年,十九路军在福建成立的人民政府,也被国民党政府破坏了。\\
  那时的国民党政府为什么采取不抵抗政策呢?主要的原因,在于国民党在一九二七年破坏了国共两党的合作,破坏了中国人民的团结。\\
  一九二四年,孙中山先生接受了中国共产党的建议,召集了有共产党人参加的国民党第一次全国代表大会,订出了联俄、联共、扶助农工的三大政策\footnote[3]{ 参见本书第一卷《中国社会各阶级的分析》注〔4〕。},建立了黄埔军校\footnote[4]{ 见本书第二卷《战争和战略问题》注〔11〕。},实现了国共两党和各界人民的民族统一战线,因而在一九二四年至一九二五年,扫荡了广东的反动势力\footnote[5]{ 见本书第二卷《战争和战略问题》注〔2〕。},在一九二六年至一九二七年,举行了胜利的北伐战争,占领了长江流域和黄河流域的大部,打败了北洋军阀政府,发动了中国历史上空前广大的人民解放斗争。但是到了一九二七年春夏之交,正当北伐战争向前发展的紧要关头,这个代表中国人民解放事业的国共两党和各界人民的民族统一战线及其一切革命政策,就被国民党当局的叛卖性的反人民的“清党”政策和屠杀政策所破坏了。昨天的同盟者——中国共产党和中国人民,被看成了仇敌,昨天的敌人——帝国主义者和封建主义者,被看成了同盟者。就是这样,背信弃义地向着中国共产党和中国人民来一个突然的袭击;生气蓬勃的中国大革命就被葬送了。从此以后,内战代替了团结,独裁代替了民主,黑暗的中国代替了光明的中国。但是,中国共产党和中国人民并没有被吓倒,被征服,被杀绝。他们从地下爬起来,揩干净身上的血迹,掩埋好同伴的尸首,他们又继续战斗了。他们高举起革命的大旗,举行了武装的抵抗,在中国的广大区域内,组织了人民的政府,实行了土地制度的改革,创造了人民的军队——中国红军,保存了和发展了中国人民的革命力量。被国民党反动分子所抛弃的孙中山先生的革命的三民主义,由中国人民、中国共产党和其它民主分子继承下来了。\\
  到了日本侵略者打入东三省以后,中国共产党就在一九三三年,向一切进攻革命根据地和红军的国民党军队提议:在停止进攻、给予人民以自由权利和武装人民这样三个条件之下,订立停战协定,以便一致抗日。但是国民党当局拒绝了这个提议。\\
  从此以后,一方面,是国民党政府的内战政策越发猖狂;另一方面,是中国人民要求停止内战一致抗日的呼声越发高涨。各种人民爱国组织,在上海和其它许多地方建立起来。一九三四年至一九三六年,长江南北各地的红军主力,在我们党中央领导之下,经历了千辛万苦,移到了西北,并和西北红军汇合在一起。就在这两年,中国共产党适应新的情况,决定并执行了抗日民族统一战线的新的完整的政治路线,以团结抗日和建立新民主主义共和国为奋斗目标。一九三五年十二月九日,北平学生群众,在我们党领导之下,发动了英勇的爱国运动,成立了中华民族解放先锋队\footnote[6]{ 中华民族解放先锋队,简称“民先队”,是一二九运动中的先进青年在中国共产党领导下所组织的革命青年团体,成立于一九三六年二月。抗日战争爆发后,许多民先队员参加了战争和建立抗日根据地的工作。国民党统治地区的民先队组织,一九三八年被国民党政府强迫解散。在抗日根据地的民先队组织,后来并入更广泛的青年团体青年救国会。},并把这种爱国运动推广到了全国各大城市。一九三六年十二月十二日,国民党内部主张抗日的两派爱国分子——东北军和十七路军,联合起来,勇敢地反对国民党当局的对日妥协和对内屠杀的反动政策,举行了有名的西安事变。同时,国民党内的其它爱国分子,也不满意国民党当局的当时政策。在这种形势下,国民党当局被迫地放弃了内战政策,承认了人民的要求。西安事变的和平解决成了时局转换的枢纽:在新形势下的国内的合作形成了,全国的抗日战争发动了。在卢沟桥事变的前夜,即一九三七年五月,我们党召集了一个具有历史意义的全国代表会议,这个会议批准了党中央自一九三五年以来的新的政治路线。\\
  从一九三七年七月七日卢沟桥事变到一九三八年十月武汉失守这一个时期内,国民党政府的对日作战是比较努力的。在这个时期内,日本侵略者的大举进攻和全国人民民族义愤的高涨,使得国民党政府政策的重点还放在反对日本侵略者身上,这样就比较顺利地形成了全国军民抗日战争的高潮,一时出现了生气蓬勃的新气象。当时全国人民,我们共产党人,其它民主党派,都对国民党政府寄予极大的希望,就是说,希望它乘此民族艰危、人心振奋的时机,厉行民主改革,将孙中山先生的革命三民主义付诸实施。可是,这个希望是落空了。就在这两年,一方面,有比较积极的抗战;另一方面,国民党当局仍旧反对发动广大民众参加的人民战争,仍旧限制人民自动团结起来进行抗日和民主的活动。一方面,国民党政府对待中国共产党及其它抗日党派的态度比较过去有了一些改变;另一方面,仍旧不给各党派以平等地位,并多方限制它们的活动。许多爱国政治犯并没有释放。最主要的是国民党政府仍旧保持其自一九二七年发动内战以来的寡头专政制度,未能建立举国一致的民主的联合政府。\\
  还在这一时期的开始,我们共产党人就指出中国抗日战争的两条路线:或者是人民的全面的战争,这样就会胜利;或者是压迫人民的片面的战争,这样就会失败。我们又指出:战争将是长期的,必然要遇到许多艰难困苦;但是由于中国人民的努力,最后胜利必归于中国人民。\\
\subsubsection*{\myformat{人民战争}}
这一时期内,中国共产党领导的移到了西北的中国红军主力,改编为中国国民革命军第八路军,留在长江南北各地的中国红军游击部队,则改编为中国国民革命军新编第四军,相继开赴华北华中作战。内战时期的中国红军,保存了并发展了北伐时期黄埔军校和国民革命军的民主传统,曾经扩大到几十万人。由于国民党政府在南方各根据地内的残酷的摧毁、万里长征的消耗和其它原因,到抗日战争开始时,数量减少到只剩几万人。于是有些人就看不起这支军队,以为抗日主要地应当依靠国民党。但是人民是最好的鉴定人,他们知道八路军新四军这时数量虽小,质量却很高,只有它才能进行真正的人民战争,它一旦开到抗日的前线,和那里的广大人民相结合,其前途是无限的。人民是正确的,当我在这里做报告的时候,我们的军队已发展到了九十一万人,乡村中不脱离生产的民兵发展到了二百二十万人以上。不管现在我们的正式军队比起国民党现存的军队来(包括中央系和地方系)在数量上要少得多,但是按其所抗击的日军和伪军的数量及其所担负的战场的广大说来,按其战斗力说来,按其有广大的人民配合作战说来,按其政治质量及其内部统一团结等项情况说来,它已经成了中国抗日战争的主力军。\\
  这个军队之所以有力量,是因为所有参加这个军队的人,都具有自觉的纪律;他们不是为着少数人的或狭隘集团的私利,而是为着广大人民群众的利益,为着全民族的利益,而结合,而战斗的。紧紧地和中国人民站在一起,全心全意地为中国人民服务,就是这个军队的唯一的宗旨。\\
  在这个宗旨下面,这个军队具有一往无前的精神,它要压倒一切敌人,而决不被敌人所屈服。不论在任何艰难困苦的场合,只要还有一个人,这个人就要继续战斗下去。\\
  在这个宗旨下面,这个军队有一个很好的内部和外部的团结。在内部——官兵之间,上下级之间,军事工作、政治工作和后勤工作之间;在外部——军民之间,军政之间,我友之间,都是团结一致的。一切妨害团结的现象,都在必须克服之列。\\
  在这个宗旨下面,这个军队有一个正确的争取敌军官兵和处理俘虏的政策。对于敌方投诚的、反正的、或在放下武器后愿意参加反对共同敌人的人,一概表示欢迎,并给予适当的教育。对于一切俘虏,不许杀害、虐待和侮辱。\\
  在这个宗旨下面,这个军队形成了为人民战争所必需的一系列的战略战术。它善于按照变化着的具体条件从事机动灵活的游击战争,也善于作运动战。\\
  在这个宗旨下面,这个军队形成了为人民战争所必需的一系列的政治工作,其任务是为团结我军,团结友军,团结人民,瓦解敌军和保证战斗胜利而斗争。\\
  在这个宗旨下面,在游击战争的条件下,全军都可以并且已经是这样做了:利用战斗和训练的间隙,从事粮食和日用必需品的生产,达到军队自给、半自给或部分自给之目的,借以克服经济困难,改善军队生活和减轻人民负担。在各个军事根据地上,也利用了一切可能性,建立了许多小规模的军事工业。\\
  这个军队之所以有力量,还由于有人民自卫军和民兵这样广大的群众武装组织,和它一道配合作战。在中国解放区内,一切青年、壮年的男人和女人,都在自愿的民主的和不脱离生产的原则下,组织在抗日人民自卫军之中。自卫军中的精干分子,除加入军队和游击队者外,则组织在民兵的队伍中。没有这些群众武装力量的配合,要战胜敌人是不可能的。\\
  这个军队之所以有力量,还由于它将自己划分为主力兵团和地方兵团两部分,前者可以随时执行超地方的作战任务,后者的任务则固定在协同民兵、自卫军保卫地方和进攻当地敌人方面。这种划分,取得了人民的真心拥护。如果没有这种正确的划分,例如说,如果只注意主力兵团的作用,忽视地方兵团的作用,那末,在中国解放区的条件下,要战胜敌人也是不可能的。在地方兵团方面,组织了许多经过良好训练,在军事、政治、民运各项工作上说来都是比较地更健全的武装工作队,深入敌后之敌后,打击敌人,发动民众的抗日斗争,借以配合各个解放区正面战线的作战,收到了很大的成效。\\
  在中国解放区,在民主政府领导之下,号召一切抗日人民组织在工人的、农民的、青年的、妇女的、文化的和其它职业和工作的团体之中,热烈地从事援助军队的各项工作。这些工作不但包括动员人民参加军队,替军队运输粮食,优待抗日军人家属,帮助军队解决物质困难,而且包括动员游击队、民兵和自卫军,展开袭击运动和爆炸运动,侦察敌情,清除奸细,运送伤兵和保护伤兵,直接帮助军队的作战。同时,全解放区人民又热烈地从事政治、经济、文化、卫生各项建设工作。在这方面,最重要的是动员全体人民从事粮食和日用品的生产,并使一切机关、学校,除有特殊情形者外,一律于工作或学习之暇,从事生产自给,以配合人民和军队的生产自给,造成伟大的生产热潮,借以支持长期的抗日战争。在中国解放区,敌人的摧残是异常严重的;水、旱、虫灾,也时常发生。但是,解放区民主政府领导全体人民,有组织地克服了和正在克服着各种困难,灭蝗、治水、救灾的伟大群众运动,收到了史无前例的效果,使抗日战争能够长期地坚持下去。总之,一切为着前线,一切为着打倒日本侵略者和解放中国人民,这就是中国解放区全体军民的总口号、总方针。\\
  这就是真正的人民战争。只有这种人民战争,才能战胜民族敌人。国民党之所以失败,就是因为它拚命地反对人民战争。\\
  中国解放区的军队一旦得到新式武器的装备,它就会更加强大,就能够最后地打败日本侵略者了。\\
\subsubsection*{\myformat{两个战场}}
中国的抗日战争,一开始就分为两个战场:国民党战场和解放区战场。\\
  一九三八年十月武汉失守后,日本侵略者停止了向国民党战场的战略性的进攻,逐渐地将其主要军事力量移到了解放区战场;同时,针对着国民党政府的失败情绪,声言愿意和它谋取妥协的和平,并将卖国贼汪精卫诱出重庆,在南京成立伪政府,实施民族的欺骗政策。从这时起,国民党政府开始了它的政策上的变化,将其重点由抗日逐渐转移到反共反人民。这首先表现在军事方面。它采取了对日消极作战的政策,保存军事实力,而把作战的重担放在解放区战场上,让日寇大举进攻解放区,它自己则“坐山观虎斗”。\\
  一九三九年,国民党政府采取了反动的所谓《限制异党活动办法》\footnote[7]{ 见本书第二卷《必须制裁反动派》注〔5〕。},将抗战初期人民和各抗日党派争得的某些权利,一概取消。从此时起,在国民党统治区内,国民党政府将一切民主党派,首先和主要地是将中国共产党,打入地下。在国民党统治区各个省份的监狱和集中营内,充满了共产党人、爱国青年及其它民主战士。从一九三九年起直至一九四三年秋季为止的五年之内,国民党政府发动了三次大规模的“反共高潮”\footnote[8]{ 参见本卷《评国民党十一中全会和三届二次国民参政会》一文中关于国民党发动三次反共高潮的叙述。},分裂国内的团结,造成严重的内战危险。震动中外的“解散”新四军和歼灭皖南新四军部队九千余人的事变,就是发生在这个时期内。直到现时为止,国民党军队向解放区军队进攻的事件还未停止,并且看不出任何准备停止的征象。在这种情况下,一切污蔑和谩骂,都从国民党反动分子的嘴里喷出来。什么“奸党”、“奸军”、“奸区”,什么“破坏抗战、危害国家”等等污蔑共产党、八路军、新四军和解放区的称号和断语,都是这些反动分子制造出来的。一九三九年七月七日,中国共产党中央委员会发表宣言,针对着当时的危机,提出了这样的口号:“坚持抗战,反对投降;坚持团结,反对分裂;坚持进步,反对倒退。”按照这些适合时宜的口号,我们党在五年之内,有力地打退了三次反动的反人民的“反共高潮”,克服了当时的危机。\\
  在这几年内,国民党战场实际上没有严重的战争。日本侵略者的刀锋,主要地向着解放区。到一九四三年,侵华日军的百分之六十四和伪军的百分之九十五,为解放区军民所抗击;国民党战场所担负的,不过日军的百分之三十六和伪军的百分之五而已。\\
  一九四四年,日本侵略者举行打通大陆交通线的作战了,国民党军队表现了手足无措,毫无抵抗能力。几个月内,就将河南、湖南、广西、广东等省广大区域沦于敌手。仅在此时,两个战场分担的抗敌的比例,才起了一些变化。然而就在我做这个报告的时候,在侵华日军(满洲的未计在内)四十个师团,五十八万人中,解放区战场抗击的是二十二个半师团,三十二万人,占了百分之五十六;国民党战场抗击的,不过十七个半师团,二十六万人,仅占百分之四十四。抗击伪军的情况则完全无变化。\\
  还应指出,数达八十万以上的伪军(包括伪正规军和伪地方武装在内),大部分是国民党将领率部投敌,或由国民党投敌军官所组成的。国民党反动分子事先即供给这些伪军以所谓“曲线救国”\footnote[9]{ “曲线救国”,是抗日战争时期国民党内一些顽固派分子为实行降日反共而制造的一种叛国谬论。他们指使或支持一部分国民党军队和官员投降日本侵略者,变成伪军、伪官,和日军一起进攻抗日根据地,并将这种叛国投敌行为诡称为“曲线救国”。}的叛国谬论,事后又在精神上和组织上支持他们,使他们配合日本侵略者反对中国人民的解放区。此外,则动员大批军队封锁和进攻陕甘宁边区及各解放区,其数量达到了七十九万七千人之多。这种严重情形,在国民党政府的新闻封锁政策下,很多的中国人外国人都无法知道。\\
\subsubsection*{\myformat{中国解放区}}
中国共产党领导的中国解放区,现在有九千五百五十万人口。其地域,北起内蒙,南至海南岛,大部分敌人所到之处,都有八路军、新四军或其它人民军队的活动。这个广大的中国解放区,包括十九个大的解放区,其地域包括辽宁、热河、察哈尔、绥远\footnote[10]{ 绥远,原来是一个省,一九五四年撤销,原辖地区划归内蒙古自治区。}、陕西、甘肃、宁夏、山西、河北、河南、山东、江苏、浙江、安徽、江西、湖北、湖南、广东、福建等省的大部分或小部分。延安是所有解放区的指导中心。在这个广大的解放区内,黄河以西的陕甘宁边区,只有人口一百五十万,是十九个解放区中的一个;而且除了浙东、琼崖两区之外,按其人口说来,它是一个最小的。有些人不明了这种情形,以为所谓中国解放区,主要就是陕甘宁边区。这是国民党政府的封锁政策造成的一个误会。在所有这些解放区内,实行了抗日民族统一战线的全部必要的政策,建立了或正在建立民选的共产党人和各抗日党派及无党无派的代表人物合作的政府,亦即地方性的联合政府。解放区内全体人民的力量都动员起来了。所有这一切,使得中国解放区在强敌压迫之下,在国民党军队的封锁和进攻的情况之下,在毫无外援的情况之下,能够屹立不摇,并且一天一天发展,缩小敌占区,扩大自己的区域,成为民主中国的模型,成为配合同盟国作战、驱逐日本侵略者、解放中国人民的主要力量。中国解放区的军队——八路军、新四军和其它人民军队,不但在对日战争的作战上,起了英勇的模范的作用,在执行抗日民族统一战线的各项民主政策上,也是起了模范作用的。一九三七年九月二十二日,中国共产党中央委员会发表宣言,承认“孙中山先生的三民主义为中国今日之必需,本党愿为其彻底实现而奋斗”,这一宣言,在中国解放区是完全实践了。\\
\subsubsection*{\myformat{国民党统治区}}
国民党内的主要统治集团,坚持独裁统治,实行了消极的抗日政策和反人民的国内政策。这样,就使得它的军队缩小了一半以上,并且大部分几乎丧失了战斗力;使得它自己和广大人民之间发生了深刻的裂痕,造成了民生凋敝、民怨沸腾、民变蜂起的严重危机;使得它在抗日战争中的作用,不但是极大地减少了,并且变成了动员和统一中国人民一切抗日力量的障碍物。\\
  为什么在国民党主要统治集团领导下会产生这种严重情况呢?因为这个集团所代表的利益是中国的大地主、大银行家、大买办阶层的利益。这些极少数人所形成的反动阶层,垄断着国民党政府管辖之下的军事、政治、经济、文化的一切重要的机构。他们将保全自己少数人的利益放在第一位,而把抗日放在第二位。他们也说什么“民族至上”,但是他们的行为却不符合于民族中大多数人民的要求。他们也说什么“国家至上”,但是他们所指的国家,就是大地主、大银行家、大买办阶层的封建法西斯的独裁国家,并不是人民大众的民主国家。因此,他们惧怕人民起来,惧怕民主运动,惧怕认真地动员全民的抗日战争。这就是他们对日消极作战的政策,对内的反人民、反民主、反共的反动政策的总根源。他们在各方面都采取这样的两面政策。例如:一面虽在抗日,一面又采取消极的作战政策,并且还被日本侵略者经常选择为诱降的对象。一面在口头上宣称要发展中国经济,一面又在实际上积累官僚资本,亦即大地主、大银行家、大买办的资本,垄断中国的主要经济命脉,而残酷地压迫农民,压迫工人,压迫小资产阶级和自由资产阶级。一面在口头上宣称实行“民主”,“还政于民”,一面又在实际上残酷地压迫人民的民主运动,不愿实行丝毫的民主改革。一面在口头上宣称“共党问题为一政治问题,应用政治方法解决”,一面又在军事上、政治上、经济上残酷地压迫中国共产党,把共产党看成他们的所谓“第一个敌人”,而把日本侵略者看成“第二个敌人”,并且每天都在积极地准备内战,处心积虑地要消灭共产党。一面在口头上宣称要建立一个“近代国家”,一面又在实际上拚死命保持那个大地主、大银行家、大买办的封建法西斯独裁统治。一面和苏联在形式上保持外交关系,一面又在实际上采取仇视苏联的态度。一面同美国孤立派合唱“先亚后欧论”,借以延长法西斯德国也就是延长一切法西斯的寿命,延长自己对于中国人民的法西斯统治的寿命,一面又在外交上投机取巧,把自己打扮成为反法西斯的英雄。要问如此种种的自相矛盾的两面政策从何而来,就是来自大地主、大银行家、大买办社会阶层这一个总根源。\\
  但是,国民党是一个复杂的政党。它虽被这个代表大地主、大银行家、大买办阶层的反动集团所统治,所领导,却并不整个儿等于这个反动集团。它有一部分领袖人物不属于这个集团,而且被这个集团所打击、排斥或轻视。它有不少的干部、党员群众和三民主义青年团的团员群众并不满意这个集团的领导,而且有些甚至是反对它的领导的。在被这个反动集团所统制的国民党的军队、国民党的政府机关、国民党的经济机关和国民党的文化机关中,都存在着这种情形。在这些军队和机关里,包藏着不少的民主分子。这个反动集团,其中又分为几派,互相斗争,并不是一个严密的统一体。把国民党看成清一色的反动派,无疑是很不适当的。\\
\subsubsection*{\myformat{比  较}}
中国人民从中国解放区和国民党统治区,获得了明显的比较。\\
  难道还不明显吗?两条路线,人民战争的路线和反对人民战争的消极抗日的路线,其结果:一条是胜利的,即使处在中国解放区这种环境恶劣和毫无外援的地位;另一条是失败的,即使处在国民党统治区这种极端有利和取得外国接济的地位。\\
  国民党政府把自己的失败归咎于缺乏武器。但是试问:缺乏武器的是国民党的军队呢,还是解放区的军队?中国解放区的军队是中国军队中武器最缺乏的军队,他们只能从敌人手里夺取武器和在最恶劣条件下自己制造武器。\\
  国民党中央系军队的武器,不是比起地方系军队来要好得多吗?但是比起战斗力来,中央系却多数劣于地方系。\\
  国民党拥有广大的人力资源,但是在它的错误的兵役政策下,人力补充却极端困难。中国解放区处在被敌人分割和战斗频繁的情况之下,因为普遍实施了适合人民需要的民兵和自卫军制度,又防止了对于人力资源的滥用和浪费,人力动员却可以源源不竭。\\
  国民党拥有粮食丰富的广大地区,人民每年供给它七千万至一万万市担的粮食,但是大部分被经手人员中饱了,致使国民党的军队经常缺乏粮食,士兵饿得面黄肌瘦。中国解放区的主要部分隔在敌后,遭受敌人烧杀抢“三光”政策的摧残,其中有些是像陕北这样贫瘠的区域,但是却能用自己动手、发展农业生产的方法,很好地解决了粮食问题。\\
  国民党区域经济危机极端严重,工业大部分破产了,连布匹这样的日用品也要从美国运来。中国解放区却能用发展工业的方法,自己解决布匹和其它日用品的需要。\\
  在国民党区域,工人、农民、店员、公务人员、知识分子以及文化工作者,生活痛苦,达于极点。中国解放区的全体人民都有饭吃,有衣穿,有事做。\\
  利用抗战发国难财,官吏即商人,贪污成风,廉耻扫地,这是国民党区域的特色之一。艰苦奋斗,以身作则,工作之外,还要生产,奖励廉洁,禁绝贪污,这是中国解放区的特色之一。\\
  国民党区域剥夺人民的一切自由。中国解放区则给予人民以充分的自由。\\
  国民党统治者面前摆着这些反常的状况,怪谁呢?怪别人,还是怪他们自己呢?怪外国缺少援助,还是怪国民党政府的独裁统治和腐败无能呢?这难道还不明白吗?\\
\subsubsection*{\myformat{“破坏抗战、危害国家”的是谁?}}
真凭实据地破坏了中国人民的抗战和危害了中国人民的国家的,难道不正是国民党政府吗?这个政府一心一意地打了整十年的内战,将刀锋向着同胞,置一切国防事业于不顾,又用不抵抗政策送掉了东北四省\footnote[11]{ 见本书第一卷《论反对日本帝国主义的策略》注〔5〕。}。日本侵略者打进关内来了,仓皇应战,从卢沟桥退到了贵州省。但是国民党人却说:“共产党破坏抗战,危害国家。”(见一九四三年九月国民党十一中全会的决议案)唯一的证据,就是共产党联合了各界人民创造了英勇抗日的中国解放区。这些国民党人的逻辑,和中国人民的逻辑是这样的不相同,无怪乎很多问题都讲不通了。\\
  两个问题:\\
  第一个,究竟什么原因使得国民党政府抛弃了从黑龙江到卢沟桥,又从卢沟桥到贵州省这样广大的国土和这样众多的人民?难道不是由于国民党政府所采取的不抵抗政策、消极的抗日政策和反人民的国内政策吗?\\
  第二个,究竟什么原因使得中国解放区战胜了敌伪军长期的残酷的进攻,从民族敌人手里恢复了这样广大的国土,解放了这样众多的人民?难道不是由于人民战争的正确路线吗?\\
\subsubsection*{\myformat{所谓“不服从政令、军令”}}
国民党政府还经常以“不服从政令、军令”责备中国共产党。但是我们只能这样说:幸喜中国共产党人还保存了中国人民的普通常识,没有服从那些实际上是把中国人民艰难困苦地从日本侵略者手里夺回来的中国解放区再送交日本侵略者的这种所谓“政令、军令”,例如,一九三九年的所谓《限制异党活动办法》,一九四一年的所谓“解散新四军”和“退至旧黄河以北”,一九四三年的所谓“解散中国共产党”,一九四四年的所谓“限期取消十个师以外的全部军队”,以及在最近谈判中提出来的所谓将军队和地方政府移交给国民党,其交换条件是不许成立联合政府,只许收容几个共产党员到国民党独裁政府里去做官,并将这种办法称之为国民党政府的“让步”等等。幸喜我们没有服从这些东西,替中国人民保存了一片干净土,保存了一支英勇抗日的军队。难道中国人民不应该庆贺这一个“不服从”吗?难道国民党政府自己用自己的法西斯主义的政令和失败主义的军令,将黑龙江至贵州省的广大的土地、人民送交日本侵略者,还觉得不够吗?除了日本侵略者和反动派欢迎这些“政令、军令”之外,难道还有什么爱国的有良心的中国人欢迎这些东西吗?没有一个不是形式的而是实际的、不是法西斯独裁的而是民主的联合政府,能够设想中国人民会允许中国共产党人,擅自将这个获得了解放的中国解放区和抗日有功的人民军队,交给失败主义和法西斯主义的国民党法西斯独裁政府吗?假如没有中国解放区及其军队,中国人民的抗日事业还有今日吗?我们民族的前途还能设想吗?\\
\subsubsection*{\myformat{内战危险}}
迄今为止,国民党内的主要统治集团,坚持着独裁和内战的反动方针。有很多迹象表明,他们早已准备,尤其现在正在准备这样的行动:等候某一个同盟国的军队在中国大陆上驱逐日本侵略者到了某一程度时,他们就要发动内战。他们并且希望某些同盟国的将领们在中国境内执行英国斯科比\footnote[12]{ 斯科比是英国派驻希腊的英军司令。一九四四年十月,德国侵略军在希腊败退。斯科比率领英军,带着在伦敦的希腊流亡政府,进入希腊。同年十二月,斯科比指挥英军并协助希腊政府进攻长期英勇抵抗德军的希腊人民解放军,屠杀希腊爱国人民。}将军在希腊所执行的职务。他们对于斯科比和希腊反动政府的屠杀事业,表示欢呼。他们企图把中国抛回到一九二七年至一九三七年的国内战争的大海里去。国民党主要统治集团现在正在所谓“召开国民大会”和“政治解决”的烟幕之下,偷偷摸摸地进行其内战的准备工作。如果国人不加注意,不去揭露它的阴谋,阻止它的准备,那末,会有一个早上,要听到内战的炮声的。\\
谈  判\\
  为着打败日本侵略者和建设新中国,为着防止内战,中国共产党在取得了其它民主派别的同意之后,于一九四四年九月间的国民参政会上,提出了立即废止国民党一党专政、成立民主的联合政府一项要求。无疑地,这项要求是适合时宜的,几个月内,获得了广大人民的响应。\\
  关于如何废止一党专政、成立联合政府以及实行必要的民主改革等项问题,我们和国民党政府之间曾经有过多次谈判,但是我们的一切建议都遭到了国民党政府的拒绝。国民党不但对一党专政不愿废止,对联合政府不愿成立,即对任何迫切需要的民主改革,例如,取消特务机关,取消镇压人民自由的反动法令,释放政治犯,承认各党派的合法地位,承认解放区,撤退封锁和进攻解放区的军队等等,也一项不愿实行。就是这样,使得中国的政治关系处在非常严重的局面之下。\\
\subsubsection*{\myformat{两个前途}}
从整个形势看来,从上述一切国际国内的实际情况的分析看来,我请大家注意,不要以为我们的事业,一切都将是顺利的,美妙的。不,不是这样,事实是好坏两个可能性、好坏两个前途都存在着。继续法西斯独裁统治,不许民主改革;不是将重点放在反对日本侵略者方面,而是放在反对人民方面;即使日本侵略者被打败了,中国仍然可能发生内战,将中国拖回到痛苦重重的不独立、不自由、不民主、不统一、不富强的老状态里去。这是一个可能性,这是一个前途。这个可能性,这个前途,依然存在,并不因为国际形势好,国内人民觉悟程度增长和有组织的人民力量发展了,它就似乎没有了,或自然地消失了。希望中国实现这个可能性、实现这个前途的,在中国是国民党内的反人民集团,在外国是那些怀抱帝国主义思想的反动分子。这是一方面,这是必须注意的一方面。\\
  但是,另一方面,同样是从整个形势看来,从上述一切内外情况的分析看来,使我们更有信心地更有勇气地去争取第二个可能性,第二个前途。这就是克服一切困难,团结全国人民,废止国民党的法西斯独裁统治,实行民主改革,巩固和扩大抗日力量,彻底打败日本侵略者,将中国建设成为一个独立、自由、民主、统一和富强的新国家。希望中国实现这个可能性、实现这个前途的,在中国是广大的人民,中国共产党及其它民主派别,在外国是一切以平等地位待我的民族,外国的进步分子,外国的人民大众。\\
  我们清楚地懂得,在我们和中国人民面前,还有很大的困难,还有很多的障碍物,还要走很多的迂回路程。但是我们同样地懂得,任何困难和障碍物,我们和全国人民一道一定能够加以克服,而使中国的历史任务获得完成。竭尽全力地去反对第一个可能性,争取第二个可能性,反对第一个前途,争取第二个前途,是我们和全国人民的伟大任务。国际国内形势的主要方面,是有利于我们和全国人民的。这些,我在前面已经说得很清楚了。我们希望国民党当局,鉴于世界大势之所趋,中国人心之所向,毅然改变其错误的现行政策,使抗日战争获得胜利,使中国人民少受痛苦,使新中国早日诞生。须知不论怎样迂回曲折,中国人民独立解放的任务总是要完成的,而且这种时机已经到来了。一百多年来无数先烈所怀抱的宏大志愿,一定要由我们这一代人去实现,谁要阻止,到底是阻止不了的。\\
\subsection*{\myformat{四 中国共产党的政策}}
上面,我已将中国抗日战争中的两条路线,给了一个分析。这样的一个分析是完全必要的。因为在广大的中国人中间,至今还有很多人不明白中国抗日战争中的具体情况。在国民党统治区,在国外,由于国民党政府的封锁政策,很多人被蒙住了眼睛。在一九四四年中外新闻记者参观团来到中国解放区以前,那里的许多人对于解放区几乎是什么也不知道的。国民党政府非常害怕解放区的真实情况泄露出去,所以在一九四四年的一次新闻记者团回去之后,立即将大门堵上,不许一个新闻记者再来解放区。对于国民党区域的真相,国民党政府也是同样地加以封锁。因此,我感到我们有责任将“两个区域”的真相尽可能使人们弄清楚。只有在弄清中国的全部情况之后,才有可能了解中国的两个最大政党——中国共产党和中国国民党的政策何以有这样的不同,何以有这样的两条路线之争。只有这样,才会使人们了解,两党的争论,不是如有些人们所说不过是一些不必要的,不重要的,或甚至是意气用事的争论,而是关系着几万万人民生死问题的原则的争论。\\
  在目前中国时局的严重形势下,中国人民,中国一切民主党派和民主分子,一切关心中国时局的外国人民,都希望中国的分裂局面重趋于团结,都希望中国能实行民主改革,都愿意知道中国共产党对于解决当前许多重大问题上所持的政策。我们的党员对于这些,当然更加关心。\\
  我们的抗日民族统一战线的政策历来是明确的,八年的战争考验了这些政策。我们的大会应该对此作出结论,作为今后奋斗的指针。\\
  下面,我就来说明我们党在为解决中国问题而得出的关于重要政策方面的若干确定的结论。\\
\subsubsection*{\myformat{我们的一般纲领}}
为着动员和统一中国人民一切抗日力量,彻底消灭日本侵略者,并建立独立、自由、民主、统一和富强的新中国,中国人民,中国共产党和一切抗日的民主党派,迫切地需要一个互相同意的共同纲领。\\
  这种共同纲领,可以分为一般性的和具体性的两部分。我们先来说一般性的纲领,然后再说到具体性的纲领。\\
  在彻底消灭日本侵略者和建设新中国的大前提之下,在中国的现阶段,我们共产党人在这样一个基本点上是和中国人口中的最大多数相一致的。这就是说:第一,中国的国家制度不应该是一个由大地主大资产阶级专政的、封建的、法西斯的、反人民的国家制度,因为这种反人民的制度,已由国民党主要统治集团的十八年统治证明为完全破产了。第二,中国也不可能、因此就不应该企图建立一个纯粹民族资产阶级的旧式民主专政的国家,因为在中国,一方面,民族资产阶级在经济上和政治上都表现得很软弱;另一方面,中国早已产生了一个觉悟了的,在中国政治舞台上表现了强大能力的,领导了广大的农民阶级、城市小资产阶级、知识分子以及其它民主分子的中国无产阶级及其领袖——中国共产党这样的新条件。第三,在中国的现阶段,在中国人民的任务还是反对民族压迫和封建压迫,在中国社会经济的必要条件还不具备时,中国人民也不可能实现社会主义的国家制度。\\
  那末,我们的主张是什么呢?我们主张在彻底地打败日本侵略者之后,建立一个以全国绝对大多数人民为基础而在工人阶级领导之下的统一战线的民主联盟的国家制度,我们把这样的国家制度称之为新民主主义的国家制度。\\
  这是一个真正适合中国人口中最大多数的要求的国家制度,因为,第一,它取得了和可能取得数百万产业工人,数千万手工业工人和雇佣农民的同意;其次,也取得了和可能取得占中国人口百分之八十,即在四亿五千万人口中占了三亿六千万的农民阶级的同意;又其次,也取得了和可能取得广大的城市小资产阶级、民族资产阶级、开明士绅及其它爱国分子的同意。\\
  自然,这些阶级之间仍然是有矛盾的,例如劳资之间的矛盾,就是显着的一种;因此,这些阶级各有一些不同的要求。抹杀这种矛盾,抹杀这种不同要求,是虚伪的和错误的。但是,这种矛盾,这种不同的要求,在整个新民主主义的阶段上,不会也不应该使之发展到超过共同要求之上。这种矛盾和这种不同的要求,可以获得调节。在这种调节下,这些阶级可以共同完成新民主主义国家的政治、经济和文化的各项建设。\\
  我们主张的新民主主义的政治,就是推翻外来的民族压迫,废止国内的封建主义的和法西斯主义的压迫,并且主张在推翻和废止这些之后不是建立一个旧民主主义的政治制度,而是建立一个联合一切民主阶级的统一战线的政治制度。我们的这种主张,是和孙中山先生的革命主张完全一致的。孙先生在其所著《中国国民党第一次全国代表大会宣言》里说:“近世各国所谓民权制度,往往为资产阶级所专有,适成为压迫平民之工具。若国民党之民权主义,则为一般平民所共有,非少数人所得而私也。”这是孙先生的伟大的政治指示。中国人民,中国共产党及其它一切民主分子,必须尊重这个指示而坚决地实行之,并同一切违背和反对这个指示的任何人们和任何集团作坚决的斗争,借以保护和发扬这个完全正确的新民主主义的政治原则。\\
  新民主主义的政权组织,应该采取民主集中制,由各级人民代表大会决定大政方针,选举政府。它是民主的,又是集中的,就是说,在民主基础上的集中,在集中指导下的民主。只有这个制度,才既能表现广泛的民主,使各级人民代表大会有高度的权力;又能集中处理国事,使各级政府能集中地处理被各级人民代表大会所委托的一切事务,并保障人民的一切必要的民主活动。\\
  军队和其它武装力量,是新民主主义的国家权力机关的重要部分,没有它们,就不能保卫国家。新民主主义国家的一切武装力量,如同其它权力机关一样,是属于人民和保护人民的,它们和一切属于少数人、压迫人民的旧式军队、旧式警察等等,完全不同。\\
  我们主张的新民主主义的经济,也是符合于孙先生的原则的。在土地问题上,孙先生主张“耕者有其田”。在工商业问题上,孙先生在上述宣言里这样说:“凡本国人及外国人之企业,或有独占的性质,或规模过大为私人之力所不能办者,如银行、铁道、航路之属,由国家经营管理之,使私有资本制度不能操纵国民之生计,此则节制资本之要旨也。”在现阶段上,对于经济问题,我们完全同意孙先生的这些主张。\\
  有些人怀疑中国共产党人不赞成发展个性,不赞成发展私人资本主义,不赞成保护私有财产,其实是不对的。民族压迫和封建压迫残酷地束缚着中国人民的个性发展,束缚着私人资本主义的发展和破坏着广大人民的财产。我们主张的新民主主义制度的任务,则正是解除这些束缚和停止这种破坏,保障广大人民能够自由发展其在共同生活中的个性,能够自由发展那些不是“操纵国民生计”而是有益于国民生计的私人资本主义经济,保障一切正当的私有财产。\\
  按照孙先生的原则和中国革命的经验,在现阶段上,中国的经济,必须是由国家经营、私人经营和合作社经营三者组成的。而这个国家经营的所谓国家,一定要不是“少数人所得而私”的国家,一定要是在无产阶级领导下而“为一般平民所共有”的新民主主义的国家。\\
  新民主主义的文化,同样应该是“为一般平民所共有”的,即是说,民族的、科学的、大众的文化,决不应该是“少数人所得而私”的文化。\\
  上述一切,就是我们共产党人在现阶段上,在整个资产阶级民主革命的阶段上所主张的一般纲领,或基本纲领。对于我们的社会主义和共产主义制度的将来纲领或最高纲领说来,这是我们的最低纲领。实行这个纲领,可以把中国从现在的国家状况和社会状况向前推进一步,即是说,从殖民地、半殖民地和半封建的国家和社会状况,推进到新民主主义的国家和社会。\\
  这个纲领所规定的无产阶级在政治上的领导权,无产阶级领导下的国营经济和合作社经济,是社会主义的因素。但是这个纲领的实行,还没有使中国成为社会主义社会。\\
  我们共产党人从来不隐瞒自己的政治主张。我们的将来纲领或最高纲领,是要将中国推进到社会主义社会和共产主义社会去的,这是确定的和毫无疑义的。我们的党的名称和我们的马克思主义的宇宙观,明确地指明了这个将来的、无限光明的、无限美妙的最高理想。每个共产党员入党的时候,心目中就悬着为现在的新民主主义革命而奋斗和为将来的社会主义和共产主义而奋斗这样两个明确的目标,而不顾那些共产主义敌人的无知的和卑劣的敌视、污蔑、谩骂或讥笑;对于这些,我们必须给以坚决的排击。对于那些善意的怀疑者,则不是给以排击而是给以善意的和耐心的解释。所有这些,都是异常清楚、异常确定和毫不含糊的。\\
  但是,一切中国共产党人,一切中国共产主义的同情者,必须为着现阶段的目标而奋斗,为着反对民族压迫和封建压迫,为着使中国人民脱离殖民地、半殖民地、半封建的悲惨命运,和建立一个在无产阶级领导下的以农民解放为主要内容的新民主主义性质的,亦即孙中山先生革命三民主义性质的独立、自由、民主、统一和富强的中国而奋斗。我们果然是这样做了,我们共产党人,协同广大的中国人民,曾为此而英勇奋斗了二十四年。\\
  对于任何一个共产党人及其同情者,如果不为这个目标奋斗,如果看不起这个资产阶级民主革命而对它稍许放松,稍许怠工,稍许表现不忠诚、不热情,不准备付出自己的鲜血和生命,而空谈什么社会主义和共产主义,那就是有意无意地、或多或少地背叛了社会主义和共产主义,就不是一个自觉的和忠诚的共产主义者。只有经过民主主义,才能到达社会主义,这是马克思主义的天经地义。而在中国,为民主主义奋斗的时间还是长期的。没有一个新民主主义的联合统一的国家,没有新民主主义的国家经济的发展,没有私人资本主义经济和合作社经济的发展,没有民族的科学的大众的文化即新民主主义文化的发展,没有几万万人民的个性的解放和个性的发展,一句话,没有一个由共产党领导的新式的资产阶级性质的彻底的民主革命,要想在殖民地半殖民地半封建的废墟上建立起社会主义社会来,那只是完全的空想。\\
  有些人不了解共产党人为什么不但不怕资本主义,反而在一定的条件下提倡它的发展。我们的回答是这样简单:拿资本主义的某种发展去代替外国帝国主义和本国封建主义的压迫,不但是一个进步,而且是一个不可避免的过程。它不但有利于资产阶级,同时也有利于无产阶级,或者说更有利于无产阶级。现在的中国是多了一个外国的帝国主义和一个本国的封建主义,而不是多了一个本国的资本主义,相反地,我们的资本主义是太少了。说也奇怪,有些中国资产阶级代言人不敢正面地提出发展资本主义的主张,而要转弯抹角地来说这个问题。另外有些人,则甚至一口否认中国应该让资本主义有一个必要的发展,而说什么一下就可以到达社会主义社会,什么要将三民主义和社会主义“毕其功于一役”。很明显地,这类现象,有些是反映着中国民族资产阶级的软弱性,有些则是大地主大资产阶级对于民众的欺骗手段。我们共产党人根据自己对于马克思主义的社会发展规律的认识,明确地知道,在中国的条件下,在新民主主义的国家制度下,除了国家自己的经济、劳动人民的个体经济和合作社经济之外,一定要让私人资本主义经济在不能操纵国民生计的范围内获得发展的便利,才能有益于社会的向前发展。对于中国共产党人,任何的空谈和欺骗,是不会让它迷惑我们的清醒头脑的。\\
  有些人怀疑共产党人承认“三民主义为中国今日之必需,本党愿为其彻底实现而奋斗”,似乎不是忠诚的。这是由于不了解我们所承认的孙中山先生在一九二四年《中国国民党第一次全国代表大会宣言》里所解释的三民主义的基本原则,同我党在现阶段的纲领即最低纲领里的若干基本原则,是互相一致的。应当指出,孙先生的这种三民主义,和我党在现阶段上的纲领,只是在若干基本原则上是一致的东西,并不是完全一致的东西。我党的新民主主义纲领,比之孙先生的,当然要完备得多;特别是孙先生死后这二十年中中国革命的发展,使我党新民主主义的理论、纲领及其实践,有了一个极大的发展,今后还将有更大的发展。但是,孙先生的这种三民主义,按其基本性质说来,是一个和在此以前的旧三民主义相区别的新民主主义的纲领,当然这是“中国今日之必需”,当然“本党愿为其彻底实现而奋斗”。对于中国共产党人,为本党的最低纲领而奋斗和为孙先生的革命三民主义即新三民主义而奋斗,在基本上(不是在一切方面)是一件事情,并不是两件事情。因此,不但在过去和现在已经证明,而且在将来还要证明:中国共产党人是革命三民主义的最忠诚最彻底的实现者。\\
  有些人怀疑共产党得势之后,是否会学俄国那样,来一个无产阶级专政和一党制度。我们的答复是:几个民主阶级联盟的新民主主义国家,和无产阶级专政的社会主义国家,是有原则上的不同的。毫无疑义,我们这个新民主主义制度是在无产阶级的领导之下,在共产党的领导之下建立起来的,但是中国在整个新民主主义制度期间,不可能、因此就不应该是一个阶级专政和一党独占政府机构的制度。只要共产党以外的其它任何政党,任何社会集团或个人,对于共产党是采取合作的而不是采取敌对的态度,我们是没有理由不和他们合作的。俄国的历史形成了俄国的制度,在那里,废除了人剥削人的社会制度,实现了最新式的民主主义即社会主义的政治、经济、文化制度,一切反对社会主义的政党都被人民抛弃了,人民仅仅拥护布尔什维克党,因此形成了俄国的局面,这在他们是完全必要和完全合理的。但是在俄国的政权机关中,即使是处在除了布尔什维克党以外没有其它政党的条件下,实行的还是工人、农民和知识分子联盟,或党和非党联盟的制度,也不是只有工人阶级或只有布尔什维克党人才可以在政权机关中工作。中国现阶段的历史将形成中国现阶段的制度,在一个长时期中,将产生一个对于我们是完全必要和完全合理同时又区别于俄国制度的特殊形态,即几个民主阶级联盟的新民主主义的国家形态和政权形态。\\
\subsubsection*{\myformat{我们的具体纲领}}
根据上述一般纲领,我们党在各个时期中还应当有具体的纲领。在整个资产阶级民主革命阶段中,在几十年中,我们的新民主主义的一般纲领是不变的。但是在这个大阶段的各个小的阶段中,情形是变化了和变化着的,我们的具体纲领便不能不有所改变,这是当然的事情。例如,在北伐战争时期,在土地革命战争时期和在抗日战争时期,我们的新民主主义的一般纲领并没有变化,但其具体纲领,三个时期中是有了变化的,这是因为我们的敌军和友军在三个时期中发生了变化的缘故。\\
  目前中国人民是处在这样的情况中:(一)日本侵略者还未被打败;(二)中国人民迫切地需要团结起来,实现一个民主的改革,以便造成民族团结,迅速地动员和统一一切抗日力量,配合同盟国打败日本侵略者;(三)国民党政府分裂民族团结,阻碍这种民主的改革。在这些情况下,我们的具体纲领即中国人民的现时要求是什么呢?\\
  我们认为下面这些要求是适当的,并且是最低限度的。\\
  动员一切力量,配合同盟国,彻底打败日本侵略者,并建立国际和平;要求废止国民党一党专政,建立民主的联合政府和联合统帅部;要求惩办那些分裂民族团结和反对人民的亲日分子、法西斯主义分子和失败主义分子,造成民族团结;要求惩办那些制造内战危机的反动分子,保障国内和平;要求惩办汉奸,讨伐降敌军官,惩办日本间谍;要求取消一切镇压人民的反动的特务机关和特务活动,取消集中营;要求取消一切镇压人民的言论、出版、集会、结社、思想、信仰和身体等项自由的反动法令,使人民获得充分的自由权利;要求承认一切民主党派的合法地位;要求释放一切爱国政治犯;要求撤退一切包围和进攻中国解放区的军队,并将这些军队使用于抗日前线;要求承认中国解放区的一切抗日军队和民选政府;要求巩固和扩大解放区及其军队,收复一切失地;要求帮助沦陷区人民组织地下军,准备武装起义;要求允许中国人民自动武装起来,保乡卫国;要求从政治上军事上改造那些由国民党统帅部直接领导的经常打败仗、经常压迫人民和经常排斥异己的军队,惩办那些应对溃败负责的将领;要求改善兵役制度和改善官兵生活;要求优待抗日军人家属,使前线官兵安心作战;要求优待殉国战士的遗族,优待残废军人,帮助退伍军人解决生活和就业问题;要求发展军事工业,以利作战;要求将同盟国的武器和财政援助公平地分配给抗战各军;要求惩办贪官污吏,实现廉洁政治;要求改善中下级公务员的待遇;要求给予中国人民以民主的权利;要求取消压迫人民的保甲制度\footnote[13]{ 保甲制度是国民党政府实行法西斯统治的基层政治制度。一九三二年八月,蒋介石在河南、湖北、安徽三省颁布《各县编查保甲户口条例》,其中规定“保甲之编组以户为单位,户设户长,十户为甲,甲设甲长,十甲为保,保设保长”,实行各户互相监视和互相告发的联保连坐法,以及各项反革命的强迫劳役办法。一九三四年十一月七日,国民党政府正式决定在它所统治的各省市一律推行这种保甲制度。};要求救济难民和救济灾荒;要求设立大量的救济基金,在国土收复后,广泛地救济沦陷区的受难人民;要求取消苛捐杂税,实行统一的累进税;要求实行农村改革,减租减息,适当地保证佃权,对贫苦农民给予低利贷款,并使农民组织起来,以利于发展农业生产;要求取缔官僚资本;要求废止现行的经济统制政策;要求制止无限制的通货膨胀和无限制的物价高涨;要求扶助民间工业,给予民间工业以借贷资本、购买原料和推销产品的便利;要求改善工人生活,救济失业工人,并使工人组织起来,以利于发展工业生产;要求取消国民党的党化教育\footnote[14]{  这里是指国民党政府所实行的封建的买办的法西斯教育。},发展民族的科学的大众的文化教育;要求保障教职员生活和学术自由;要求保护青年、妇女、儿童的利益,救济失学青年,并使青年、妇女组织起来,以平等地位参加有益于抗日战争和社会进步的各项工作,实现婚姻自由,男女平等,使青年和儿童得到有益的学习;要求改善国内少数民族的待遇,允许各少数民族有民族自治的权利;要求保护华侨利益,扶助回国的华侨;要求保护因被日本侵略者压迫而逃来中国的外国人民,并扶助其反对日本侵略者的斗争;要求改善中苏邦交,等等。而要做到这一切,最重要的是要求立即取消国民党一党专政,建立一个包括一切抗日党派和无党派的代表人物在内的举国一致的民主的联合的临时的中央政府。没有这个前提条件,要想在全国范围内,就是说,在国民党统治区域进行稍为认真的改革,是不可能的。\\
  这些都是中国广大人民的呼声,也是各同盟国广大民主舆论界的呼声。\\
  一个为各个抗日民主党派互相同意的最低限度的具体纲领,是完全必要的,我们准备以上述纲领为基础和他们进行协商。各党可以有不同的要求,但是各党之间应该协定一个共同的纲领。\\
  这样的纲领,对于国民党统治区,暂时还是一个要求的纲领;对于沦陷区,除组织地下军准备武装起义一项外,是一个要等到收复后才能实施的纲领;对于解放区,则是一个早已实施并应当继续实施的纲领。\\
  在上述中国人民的目前要求或具体纲领中,包含着许多战时和战后的重大问题,需要在下面加以说明。在说明这些问题时,我们将批评国民党主要统治集团的一些错误观点,同时也将回答其它人们的一些疑问。\\
\subsubsection*{\myformat{第一 彻底消灭日本侵略者,不许中途妥协}}
开罗会议\footnote[15]{ 开罗会议是一九四三年十一月中、美、英三国首脑在埃及首都开罗举行的一次国际会议。这次会议发表了中、美、英三国开罗宣言,明确规定日本必须无条件投降,并将侵占的中国领土台湾等地归还中国。}决定应使日本侵略者无条件投降,这是正确的。但是,现在日本侵略者正在暗地里进行活动,企图获得妥协的和平;国民党政府中的亲日分子,经过南京傀儡政府,也正在和日本密使勾勾搭搭,并未遇到制止。因此,中途妥协的危险并未完全过去。开罗会议又决定将东北四省、台湾、澎湖列岛归还中国,这是很好的。但是根据国民党政府的现行政策,要想依靠它打到鸭绿江边,收复一切失地,是不可能的。在这种情形下,中国人民应该怎么办呢?中国人民应该要求国民党政府彻底消灭日本侵略者,不许中途妥协。一切妥协的阴谋活动,必须立刻制止。中国人民应该要求国民党政府改变现在的消极的抗日政策,将其一切军事力量用于积极对日作战。中国人民应该扩大自己的军队——八路军、新四军及其它人民军队,并在一切敌人所到之处,广泛地自动地发展抗日武装,准备直接配合同盟国作战,收复一切失地,决不要单纯地依靠国民党。打败日本侵略者,是中国人民的神圣的权利。如果反动分子要想剥夺中国人民的这种神圣的权利,要想压制中国人民的抗日活动,要想破坏中国人民的抗日力量,那末,中国人民在向他们劝说无效之后,应该站在自卫立场上给以坚决的回击。因为中国反动分子的这种背叛民族利益的反动行为,完全是帮助日本侵略者的。\\
\subsubsection*{\myformat{第二 废止国民党一党专政,建立民主的联合政府}}
为着彻底消灭日本侵略者,必须在全国范围内实行民主改革。而要这样做,不废止国民党的一党专政,建立民主的联合政府,是不可能的。\\
  所谓国民党的一党专政,实际上是国民党内反人民集团的专政,它是中国民族团结的破坏者,是国民党战场抗日失败的负责者,是动员和统一中国人民抗日力量的根本障碍物。从八年抗日战争的惨痛经验中,中国人民已经深刻地认识了它的罪恶,很自然地要求立即废止它。这个反人民的专政,又是内战的祸胎,如不立即废止,内战惨祸又将降临。\\
  中国人民要求废止这个反人民专政的呼声是如此普遍而响亮了,使得国民党当局自己也不能不公开承认“提早结束训政”,可见这个所谓“训政”或“一党专政”的丧失人心,威信扫地,到了何种地步了。在中国,已经没有一个人还敢说“训政”或“一党专政”有什么好处,不应该废止或“结束”了,这是当前时局的一大变化。\\
  应该“结束”是确定的了,毫无疑义的了。但是如何结束呢,可就意见分歧了。一个说:立即结束,成立民主的临时的联合政府。一个说:等一会再结束,召开“国民大会”,“还政于民”,却不能还政于联合政府。\\
  这是什么意思呢?\\
  这是两种做法的表现:真做和假做。\\
  第一种,真做。这就是立即宣布废止国民党一党专政,成立一个由国民党、共产党、民主同盟\footnote[16]{  民主同盟成立于一九四一年,当时名中国民主政团同盟,一九四四年改组为中国民主同盟。}和无党无派分子的代表人物联合组成的临时的中央政府,发布一个民主的施政纲领,如同我们在前面提出的那些中国人民的现时要求,以便恢复民族团结,打败日本侵略者。为着讨论这些事情,召集一个各党派和无党派的代表人物的圆桌会议,成立协议,动手去做。这是一个团结的方针,中国人民是坚决拥护这个方针的。\\
  第二种,假做。不顾广大人民和一切民主党派的要求,一意孤行地召开一个由国民党反人民集团一手包办的所谓“国民大会”,在这个会上通过一个实际上维持独裁反对民主的所谓“宪法”,使那个仅仅由几十个国民党人私自委任的、完全没有民意基础的、强安在人民头上的、不合法的所谓国民政府,披上合法的外衣,装模作样地“还政于民”,实际上,依然是“还政”于国民党内的反人民集团。谁要不赞成,就说他是破坏“民主”,破坏“统一”,就有“理由”向他宣布讨伐令。这是一个分裂的方针,中国人民是坚决反对这个方针的。\\
  我们的反人民的英雄们根据这种分裂方针所准备采取的步骤,有把他们自己推到绝路上去的危险。他们准备把一条绳索套在自己的脖子上,并且让它永远也解不开,这条绳索的名称就叫做“国民大会”。他们的原意是想把所谓“国民大会”当作法宝,祭起来,一则抵制联合政府,二则维持独裁统治,三则准备内战理由。可是,历史的逻辑将向他们所设想的反面走去,“搬起石头打自己的脚”。因为现在谁也明白,在国民党统治区域,人民没有自由,在日寇占领区域,人民不能参加选举,有了自由的中国解放区,国民党政府又不承认它,在这种情况下,哪里来的国民代表?哪里来的“国民大会”?现在叫着要开的,是那个还在内战时期,还在八年以前,由国民党独裁政府一手伪造的所谓国民大会。如果这个会开成了,势必闹到全国人民群起反对,请问我们的反人民的英雄们如何下台?归根结底,伪造国民大会如果开成了,不过将他们自己推到绝路上。\\
  我们共产党人提出结束国民党一党专政的两个步骤:第一个步骤,目前时期,经过各党各派和无党无派代表人物的协议,成立临时的联合政府;第二个步骤,将来时期,经过自由的无拘束的选举,召开国民大会,成立正式的联合政府。总之,都是联合政府,团结一切愿意参加的阶级和政党的代表在一起,在一个民主的共同纲领之下,为现在的抗日和将来的建国而奋斗。\\
  不管国民党人或任何其它党派、集团和个人如何设想,愿意或不愿意,自觉或不自觉,中国只能走这条路。这是一个历史法则,是一个必然的、不可避免的趋势,任何力量,都是扭转不过来的。\\
  在这个问题和其它任何有关民主改革的问题上,我们共产党人声明:不管国民党当局现在还是怎样坚持其错误政策和怎样借谈判为拖延时间、搪塞舆论的手段,只要他们一旦愿意放弃其错误的现行政策,同意民主改革,我们是愿意和他们恢复谈判的。但是谈判的基础必须放在抗日、团结和民主的总方针上,一切离开这个总方针的所谓办法、方案,或其它空话,不管它怎样说得好听,我们是不能赞成的。\\
\subsubsection*{\myformat{第三 人民的自由}}
目前中国人民争自由的目标,首先地和主要地是向着日本侵略者。但是国民党政府剥夺人民的自由,捆起人民的手足,使他们不能反对日本侵略者。不解决这个问题,就不能在全国范围内动员和统一一切抗日的力量。我们在纲领中提出了废止一党专政,成立联合政府,取消特务,取消镇压自由的法令,惩办汉奸、间谍、亲日分子、法西斯分子和贪官污吏,释放政治犯,承认各民主党派的合法地位,撤退包围和进攻解放区的军队,承认解放区,废止保甲制度,以及其它许多经济的文化的和民众运动的要求,就是为着解开套在人民身上的绳索,使人民获得抗日、团结和民主的自由。\\
  自由是人民争来的,不是什么人恩赐的。中国解放区的人民已经争得了自由,其它地方的人民也可能和应该争得这种自由。中国人民争得的自由越多,有组织的民主力量越大,一个统一的临时的联合政府便越有成立的可能。这种联合政府一经成立,它将转过来给予人民以充分的自由,巩固联合政府的基础。然后才有可能,在日本侵略者被打倒之后,在全部国土上进行自由的无拘束的选举,产生民主的国民大会,成立统一的正式的联合政府。没有人民的自由,就没有真正民选的国民大会,就没有真正民选的政府。难道还不清楚吗?\\
  人民的言论、出版、集会、结社、思想、信仰和身体这几项自由,是最重要的自由。在中国境内,只有解放区是彻底地实现了。\\
  一九二五年,孙中山先生在其临终的遗嘱上说:“余致力国民革命凡四十年,其目的在求中国之自由平等。积四十年之经验,深知欲达到此目的,必须唤起民众及联合世界上以平等待我之民族共同奋斗。”背叛孙先生的不肖子孙,不是唤起民众,而是压迫民众,将民众的言论、出版、集会、结社、思想、信仰和身体等项自由权利剥夺得干干净净;对于认真唤起民众、认真保护民众自由权利的共产党、八路军、新四军和解放区,则称之为“奸党”、“奸军”、“奸区”。我们希望这种颠倒是非的时代快些过去。如果要延长这种颠倒是非的时间,中国人民将不能忍耐了。\\
\subsubsection*{\myformat{第四 人民的统一}}
为着消灭日本侵略者,为着防止内战,为着建设新中国,必须将分裂的中国变为统一的中国,这是中国人民的历史任务。\\
  但是如何统一呢?独裁者的专制的统一,还是人民的民主的统一呢?从袁世凯\footnote[17]{  见本书第一卷《论反对日本帝国主义的策略》注〔1〕。}以来,北洋军阀强调专制的统一。但是结果怎么样呢?和这些军阀的志愿相反,所得的不是统一而是分裂,最后是他们自己从台上滚下去。国民党反人民集团抄袭袁世凯的老路,追求专制的统一,打了整整十年的内战,结果把一个日本侵略者打了进来,自己也缩上了峨眉山\footnote[18]{  峨眉山是四川省西南部的名山。毛泽东的这句话是说国民党统治集团在抗日战争中最后撤退到四川山地。}。现在又在山上大叫其专制统一论,这是叫给谁听呢?难道还有什么爱国的有良心的中国人愿意听它吗?经过了十六年的北洋军阀的统治,又经过了十八年的国民党的独裁统治,人民已经有了充分的经验,有了明亮的眼睛。他们要一个人民大众的民主的统一,不要独裁者的专制的统一。我们共产党人还在一九三五年就提出了抗日民族统一战线的方针,没有一天不为此而奋斗。一九三九年国民党推行其反动的《限制异党活动办法》,造成投降、分裂、倒退的危机,国民党人大叫其专制统一论的时候,我们又说:非统一于投降而统一于抗战,非统一于分裂而统一于团结,非统一于倒退而统一于进步。只有这后一种统一才是真统一,其它一切都是假统一\footnote[19]{  参见本书第二卷《必须制裁反动派》、《团结一切抗日力量,反对反共顽固派》、《向国民党的十点要求》等文。}。又过了六年了,问题还是一样。\\
  没有人民的自由,没有人民的民主政治,能够统一吗?有了这些,立刻就统一了。中国人民争自由、争民主、争联合政府的运动,同时就是争统一的运动。我们在具体纲领中提出了许多争自由争民主的要求,提出了联合政府的要求,同时就是为了这个目的。不废止国民党内反人民集团的专政,成立民主的联合政府,不但在国民党统治区不能实行任何民主的改革,不能动员那里的全体军民打倒日本侵略者,而且还将发展为内战的惨祸,这是很多人都明白的常识了。为什么如此众多的有党有派无党无派的民主分子,包括国民党内的许多民主分子在内,一致要求成立联合政府?就因为他们看清楚了时局的危机,非如此不能克服这种危机,不能达到团结对敌和团结建国的目的。\\
\subsubsection*{\myformat{第五 人民的军队}}
中国人民要自由,要统一,要联合政府,要彻底地打倒日本侵略者和建设新中国,没有一支站在人民立场上的军队,那是不行的。彻底地站在人民立场的军队,现在还只有解放区的不很大的八路军和新四军,还很不够。可是,国民党内的反人民集团却处心积虑地要破坏和消灭解放区的军队。一九四四年,国民党政府提出了一个所谓“提示案”,叫共产党“限期取消”解放区军队的五分之四。一九四五年,即最近的一次谈判,又叫共产党将解放区军队全部交给它,然后它给共产党以“合法地位”。\\
  这些人们向共产党人说:你交出军队,我给你自由。根据这个学说,没有军队的党派该有自由了。但是一九二四年至一九二七年,中国共产党只有很少一点军队,国民党政府的“清党”政策和屠杀政策一来,自由也光了。现在的中国民主同盟和中国国民党的民主分子并没有军队,同时也没有自由。十八年中,在国民党政府统治下的工人、农民、学生以及一切要求进步的文化界、教育界、产业界,他们一概没有军队,同时也一概没有自由。难道是由于上述这些民主党派和人民组织了什么军队,实行了什么“封建割据”,成立了什么“奸区”,违反了什么“政令军令”,因此才不给自由的吗?完全不是。恰恰相反,正是因为他们没有这样做。\\
  “军队是国家的”,非常之正确,世界上没有一个军队不是属于国家的。但是什么国家呢?大地主、大银行家、大买办的封建法西斯独裁的国家,还是人民大众的新民主主义的国家?中国只应该建立新民主主义的国家,并在这个基础之上建立新民主主义的联合政府;中国的一切军队都应该属于这个国家的这个政府,借以保障人民的自由,有效地反对外国侵略者。什么时候中国有一个新民主主义的联合政府出现了,中国解放区的军队将立即交给它。但是一切国民党的军队也必须同时交给它。\\
  一九二四年,孙中山先生说:“今日以后,当划一国民革命之新时代。……第一步使武力与国民相结合;第二步使武力为国民之武力。”\footnote[20]{  见一九二四年十一月十日孙中山的《北上宣言》(《孙中山全集》第11卷,中华书局1986年版,第296—297页)。}八路军、新四军正是因为实行了这种方针,成了“国民之武力”,就是说,成了人民的军队,所以能打胜仗。国民党军队在北伐战争的前期,做到了孙先生所说的“第一步”,所以打了胜仗。从北伐战争后期直至现在,连“第一步”也丢了,站在反人民的立场上,所以一天一天腐败堕落,除了“内战内行”之外,对于“外战”,就不能不是一个“外行”。国民党军队中一切爱国的有良心的军官们,应该起来恢复孙先生的精神,改造自己的军队。\\
  在改造旧军队的工作中,对于一切可以教育的军官,应当给予适当的教育,帮助他们学得正确观点,清除陈旧观点,为人民的军队而继续服务。\\
  为创造中国人民的军队而奋斗,是全国人民的责任。没有一个人民的军队,便没有人民的一切。对于这个问题,切不可只发空论。\\
  我们共产党人愿意赞助改革中国军队的事业。八路军、新四军对于一切愿意团结人民、反对日本侵略者而不反对中国解放区的军队,都应该看作自己的友军,给以适当的协助。\\
\subsubsection*{\myformat{第六 土地问题}}
为着消灭日本侵略者和建设新中国,必须实行土地制度的改革,解放农民。孙中山先生的“耕者有其田”的主张,是目前资产阶级民主主义性质的革命时代的正确的主张。\\
  为什么把目前时代的革命叫做“资产阶级民主主义性质的革命”?这就是说,这个革命的对象不是一般的资产阶级,而是民族压迫和封建压迫;这个革命的措施,不是一般地废除私有财产,而是一般地保护私有财产;这个革命的结果,将使工人阶级有可能聚集力量因而引导中国向社会主义方向发展,但在一个相当长的时期内仍将使资本主义获得适当的发展。“耕者有其田”,是把土地从封建剥削者手里转移到农民手里,把封建地主的私有财产变为农民的私有财产,使农民从封建的土地关系中获得解放,从而造成将农业国转变为工业国的可能性。因此,“耕者有其田”的主张,是一种资产阶级民主主义性质的主张,并不是无产阶级社会主义性质的主张,是一切革命民主派的主张,并不单是我们共产党人的主张。所不同的,在中国条件下,只有我们共产党人把这项主张看得特别认真,不但口讲,而且实做。哪些人们是革命民主派呢?除了无产阶级是最彻底的革命民主派之外,农民是最大的革命民主派。农民的绝对大多数,就是说,除开那些带上了封建尾巴的富农之外,无不积极地要求“耕者有其田”。城市小资产阶级也是革命民主派,“耕者有其田”使农业生产力获得发展,对于他们是有利的。民族资产阶级是一个动摇的阶级,他们需要市场,他们也赞成“耕者有其田”;他们又多半和土地联系着,他们中的许多人就又惧怕“耕者有其田”。孙中山是中国最早的革命民主派,他代表民族资产阶级的革命派、城市小资产阶级和乡村农民,实行武装革命,提出了“平均地权”和“耕者有其田”的主张。但是可惜,在他掌握政权的时候并没有主动地实行过土地制度的改革。自国民党反人民集团掌握政权以后,便完全背叛了孙中山的主张。现在坚决地反对“耕者有其田”的,正是这个反人民集团,因为他们是代表大地主、大银行家、大买办阶层的。中国没有单独代表农民的政党,民族资产阶级的政党没有坚决的土地纲领,因此,只有制订和执行了坚决的土地纲领、为农民利益而认真奋斗、因而获得最广大农民群众作为自己伟大同盟军的中国共产党,成了农民和一切革命民主派的领导者。\\
  一九二七年至一九三六年,中国共产党实行了彻底改革土地制度的办法,实现了孙先生的“耕者有其田”的主张。出而张牙舞爪,进行了十年反人民战争,亦即反“耕者有其田”的战争的,就是那个集中了孙中山一切不肖子孙在内的团体——国民党内的反人民集团。\\
  抗日期间,中国共产党让了一大步,将“耕者有其田”的政策,改为减租减息的政策。这个让步是正确的,推动了国民党参加抗日,又使解放区的地主减少其对于我们发动农民抗日的阻力。这个政策,如果没有特殊阻碍,我们准备在战后继续实行下去,首先在全国范围内实现减租减息,然后采取适当方法,有步骤地达到“耕者有其田”。\\
  但是背叛孙先生的人们不但反对“耕者有其田”,连减租减息也反对。国民党政府自己颁布的“二五减租”一类的法令,自己不实行,仅仅我们在解放区实行了,因此也就成立了罪状:名之曰“奸区”。\\
  在抗日期间,出现了所谓民族革命阶段和民主民生革命阶段的两阶段论,这是错误的。\\
  大敌当前,民主民生改革的问题不应该提起,等日本人走了再提好了。——这是国民党反人民集团的谬论,其目的是不愿抗日战争获得彻底胜利。有些人居然随声附和,作了这种谬论的尾巴。\\
  大敌当前,不解决民主民生问题,就不能建立抗日根据地抵抗日本的进攻。——这是中国共产党的主张,并且已经这样作了,收到了很好的效果。\\
  在抗日期间,减租减息及其它一切民主改革是为着抗日的。为了减少地主对于抗日的阻力,只实行减租减息,不取消地主的土地所有权,同时又奖励地主的资财向工业方面转移,并使开明士绅和其它人民的代表一道参加抗日的社会工作和政府工作。对于富农,则鼓励其发展生产。所有这些,是在坚决执行农村民主改革的路线里包含着的,是完全必要的。\\
  两条路线:或者坚决反对中国农民解决民主民生问题,而使自己腐败无能,无力抗日;或者坚决赞助中国农民解决民主民生问题,而使自己获得占全人口百分之八十的最伟大的同盟军,借以组织雄厚的战斗力量。前者就是国民党政府的路线,后者就是中国解放区的路线。\\
  动摇于两者之间,口称赞助农民,但不坚决实行减租减息、武装农民和建立农村民主政权,这是机会主义者的路线。\\
  国民党反人民集团动员一切力量,向着中国共产党放出了一切恶毒的箭:明的和暗的,军事的和政治的,流血的和不流血的。两党的争论,就其社会性质说来,实质上是在农村关系的问题上。我们究竟在哪一点上触怒了国民党反人民集团呢?难道不正是在这个问题上面吗?国民党反人民集团之所以受到日本侵略者的欢迎和鼓励,难道不正是在这个问题上面,给日本侵略者帮了大忙吗?所谓“共产党破坏抗战、危害国家”,所谓“奸党”、“奸军”、“奸区”,所谓“不服从政令、军令”,难道不正是因为中国共产党在这个问题上做了真正符合于民族利益的认真的事业吗?\\
  农民——这是中国工人的前身。将来还要有几千万农民进入城市,进入工厂。如果中国需要建设强大的民族工业,建设很多的近代的大城市,就要有一个变农村人口为城市人口的长过程。\\
  农民——这是中国工业市场的主体。只有他们能够供给最丰富的粮食和原料,并吸收最大量的工业品。\\
  农民——这是中国军队的来源。士兵就是穿起军服的农民,他们是日本侵略者的死敌。\\
  农民——这是现阶段中国民主政治的主要力量。中国的民主主义者如不依靠三亿六千万农民群众的援助,他们就将一事无成。\\
  农民——这是现阶段中国文化运动的主要对象。所谓扫除文盲,所谓普及教育,所谓大众文艺,所谓国民卫生,离开了三亿六千万农民,岂非大半成了空话?\\
  我这样说,当然不是忽视其它约占人口九千万的人民在政治上经济上文化上的重要性,尤其不是忽视在政治上最觉悟因而具有领导整个革命运动的资格的工人阶级,这是不应该发生误会的。\\
  认识这一切,不但中国共产党人,而且一切民主派,都是完全必要的。\\
  土地制度获得改革,甚至仅获得初步的改革,例如减租减息之后,农民的生产兴趣就增加了。然后帮助农民在自愿原则下,逐渐地组织在农业生产合作社及其它合作社之中,生产力就会发展起来。这种农业生产合作社,现时还只能是建立在农民个体经济基础上的(农民私有财产基础上的)集体的互助的劳动组织,例如变工队、互助组、换工班之类,但是劳动生产率的提高和生产量的增加,已属惊人。这种制度,已在中国解放区大大发展起来,今后应当尽量推广。\\
  这里应当指出一点,就是说,变工队一类的合作组织,原来在农民中就有了的,但在那时,不过是农民救济自己悲惨生活的一种方法。现在中国解放区的变工队,其形式和内容都起了变化;它成了农民群众为着发展自己的生产,争取富裕生活的一种方法。\\
  中国一切政党的政策及其实践在中国人民中所表现的作用的好坏、大小,归根到底,看它对于中国人民的生产力的发展是否有帮助及其帮助之大小,看它是束缚生产力的,还是解放生产力的。消灭日本侵略者,实行土地改革,解放农民,发展现代工业,建立独立、自由、民主、统一和富强的新中国,只有这一切,才能使中国社会生产力获得解放,才是中国人民所欢迎的。\\
  这里还要指出一点,就是说,从城市到农村工作的知识分子,不容易了解农村现时还是以分散的落后的个体经济为基础的这种特点;在解放区,则还要加上暂时还是被敌人分割的和游击战争的环境的特点。因为不了解这些特点,他们就往往不适当地带着他们在城市里生活或工作的观点去观察农村问题,去处理农村工作,因而脱离农村的实际情况,不能和农民打成一片。这种现象,必须用教育的方法加以克服。\\
  中国广大的革命知识分子应该觉悟到将自己和农民结合起来的必要。农民正需要他们,等待他们的援助。他们应该热情地跑到农村中去,脱下学生装,穿起粗布衣,不惜从任何小事情做起,在那里了解农民的要求,帮助农民觉悟起来,组织起来,为着完成中国民主革命中一项极其重要的工作,即农村民主革命而奋斗。\\
  在日本侵略者被消灭以后,对于日本侵略者和重要汉奸分子的土地应当没收,并分配给无地和少地的农民。\\
\subsubsection*{\myformat{第七 工业问题}}
为着打败日本侵略者和建设新中国,必须发展工业。但是,在国民党政府统治之下,一切依赖外国,它的财政经济政策是破坏人民的一切经济生活的。国民党统治区内仅有的一点小型工业,也不能不处于大部分破产的状态中。政治不改革,一切生产力都遭到破坏的命运,农业如此,工业也是如此。\\
  就整个来说,没有一个独立、自由、民主和统一的中国,不可能发展工业。消灭日本侵略者,这是谋独立。废止国民党一党专政,成立民主的统一的联合政府,使全国军队成为人民的武力,实现土地改革,解放农民,这是谋自由、民主和统一。没有独立、自由、民主和统一,不可能建设真正大规模的工业。没有工业,便没有巩固的国防,便没有人民的福利,便没有国家的富强。一八四〇年鸦片战争\footnote[21]{  见本书第一卷《论反对日本帝国主义的策略》注〔35〕。}以来的一百零五年的历史,特别是国民党当政以来的十八年的历史,清楚地把这个要点告诉了中国人民。一个不是贫弱的而是富强的中国,是和一个不是殖民地半殖民地的而是独立的,不是半封建的而是自由的、民主的,不是分裂的而是统一的中国,相联结的。在一个半殖民地的、半封建的、分裂的中国里,要想发展工业,建设国防,福利人民,求得国家的富强,多少年来多少人做过这种梦,但是一概幻灭了。许多好心的教育家、科学家和学生们,他们埋头于自己的工作或学习,不问政治,自以为可以所学为国家服务,结果也化成了梦,一概幻灭了。这是好消息,这种幼稚的梦的幻灭,正是中国富强的起点。中国人民在抗日战争中学得了许多东西,知道在日本侵略者被打败以后,有建立一个新民主主义的独立、自由、民主、统一、富强的中国之必要,而这些条件是互相关联的,不可缺一的。果然如此,中国就有希望了。解放中国人民的生产力,使之获得充分发展的可能性,有待于新民主主义的政治条件在全中国境内的实现。这一点,懂得的人已一天一天地多起来了。\\
  在新民主主义的政治条件获得之后,中国人民及其政府必须采取切实的步骤,在若干年内逐步地建立重工业和轻工业,使中国由农业国变为工业国。新民主主义的国家,如无巩固的经济做它的基础,如无进步的比较现时发达得多的农业,如无大规模的在全国经济比重上占极大优势的工业以及与此相适应的交通、贸易、金融等事业做它的基础,是不能巩固的。\\
  我们共产党人愿意协同全国各民主党派,各部分产业界,为上述目标而奋斗。中国工人阶级在这个任务中将起伟大的作用。\\
  中国工人阶级,自第一次世界大战以来,就开始以自觉的姿态,为中国的独立、解放而斗争。一九二一年,产生了它的先锋队——中国共产党,从此以后,使中国的解放斗争进入了新阶段。在北伐战争、土地革命战争和抗日战争三个时期中,中国工人阶级和中国共产党,对于中国人民的解放事业,作了极大的努力和极有价值的贡献。在最后打败日本侵略者的斗争中,特别是在收复大城市和交通要道的斗争中,中国工人阶级将起着极大的作用。在抗日结束以后,可以预断,中国工人阶级的努力和贡献将会是更大的。中国工人阶级的任务,不但是为着建立新民主主义的国家而斗争,而且是为着中国的工业化和农业近代化而斗争。\\
  在新民主主义的国家制度下,将采取调节劳资间利害关系的政策。一方面,保护工人利益,根据情况的不同,实行八小时到十小时的工作制以及适当的失业救济和社会保险,保障工会的权利;另一方面,保证国家企业、私人企业和合作社企业在合理经营下的正当的赢利;使公私、劳资双方共同为发展工业生产而努力。\\
  日本侵略者被打败以后,日本侵略者和重要汉奸分子的企业和财产,应当没收,归政府处理。\\
\subsubsection*{\myformat{第八 文化、教育、知识分子问题}}
从百分之八十的人口中扫除文盲,是新中国的一项重要工作。\\
  一切奴化的、封建主义的和法西斯主义的文化和教育,应当采取适当的坚决的步骤,加以扫除。\\
  应当积极地预防和医治人民的疾病,推广人民的医药卫生事业。\\
  对于旧文化工作者、旧教育工作者和旧医生们的态度,是采取适当的方法教育他们,使他们获得新观点、新方法,为人民服务。\\
  中国国民文化和国民教育的宗旨,应当是新民主主义的;就是说,中国应当建立自己的民族的、科学的、人民大众的新文化和新教育。\\
  对于外国文化,排外主义的方针是错误的,应当尽量吸收进步的外国文化,以为发展中国新文化的借镜;盲目搬用的方针也是错误的,应当以中国人民的实际需要为基础,批判地吸收外国文化。苏联所创造的新文化,应当成为我们建设人民文化的范例。对于中国古代文化,同样,既不是一概排斥,也不是盲目搬用,而是批判地接收它,以利于推进中国的新文化。\\
\subsubsection*{\myformat{第九 少数民族问题}}
  国民党反人民集团否认中国有多民族存在,而把汉族以外的各少数民族称之为“宗族”\footnote[23]{  这是指蒋介石在《中国之命运》中的一种错误说法。}。他们对于各少数民族,完全继承清朝政府和北洋军阀政府的反动政策,压迫剥削,无所不至。一九四三年对于伊克昭盟蒙族人民的屠杀事件\footnote[24]{ 一九四二年冬至一九四三年春,驻内蒙伊克昭盟的国民党反动军队,强行霸占蒙族人民的牧地,并且向当地人民勒索大量粮食、牲畜。一九四三年三月二十六日,伊克昭盟的蒙族保安队和人民群众被迫发动了武装反抗。四月,国民党军队前往镇压,对当地的蒙族人民进行了血腥的屠杀。},一九四四年直至现在对于新疆少数民族的武力镇压事件\footnote[25]{ 一九四四年九月,新疆北部伊犁地区的少数民族人民,为了反对国民党反动派的民族压迫和经济掠夺,发动了声势浩大的武装起义。这次起义先后同阿山、塔城一带的起义武装联合,占领了新疆北部的大部分地区,所以又被称为“三区革命”。国民党反动派从甘肃和新疆各地调集了大批军队,对起义军实行长期的大规模的武力镇压。起义军在新疆各族广大人民的积极支持下,进行了英勇的抵抗,一直坚持到一九四九年新疆解放。},以及近几年对于甘肃回民的屠杀事件\footnote[26]{ 这里是指一九四二年、一九四三年国民党反动派对甘肃省南部回、汉、藏等族起义农民的屠杀事件。一九四二年冬,甘肃省南部临洮、康乐一带的农民,为了反对国民党反动派的横征暴敛、抓兵抓夫等反动措施,在回民马福善等率领下,发动了大规模的武装起义。到一九四三年四月以后,起义地区发展到二十多县,参加人数最多时达到十万多人。国民党反动派先后调动了七个师以上的军队,甚至出动飞机,配合地方武装,对起义的群众进行残酷的屠杀。},就是明证。这是大汉族主义的错误的民族思想和错误的民族政策。\\
  一九二四年,孙中山先生在其所著的《中国国民党第一次全国代表大会宣言》里说:“国民党之民族主义,有两方面之意义:一则中国民族自求解放;二则中国境内各民族一律平等。”“国民党敢郑重宣言,承认中国以内各民族之自决权,于反对帝国主义及军阀之革命获得胜利以后,当组织自由统一的(各民族自由联合的)中华民国。”\\
  中国共产党完全同意上述孙先生的民族政策。共产党人必须积极地帮助各少数民族的广大人民群众为实现这个政策而奋斗;必须帮助各少数民族的广大人民群众,包括一切联系群众的领袖人物在内,争取他们在政治上、经济上、文化上的解放和发展,并成立维护群众利益的少数民族自己的军队。他们的言语、文字、风俗、习惯和宗教信仰,应被尊重。\\
  多年以来,陕甘宁边区和华北各解放区对待蒙回两民族的态度是正确的,其工作是有成绩的。\\
\subsubsection*{\myformat{第十 外交问题}}
中国共产党同意大西洋宪章和莫斯科、开罗、德黑兰、克里米亚各次国际会议\footnote[27]{ 大西洋宪章是一九四一年八月美英大西洋会议结束时联合发表的一个文件。莫斯科会议是一九四三年十月苏、美、英三国外长在莫斯科举行的会议。德黑兰会议是一九四三年十一月至十二月苏、美、英三国首脑在伊朗首都德黑兰举行的会议。克里米亚会议是一九四五年二月苏、美、英三国首脑在苏联南部克里米亚半岛雅尔塔举行的会议。当时所有这些国际会议都决定以联合的力量击败法西斯德国和日本,并在战后防止侵略势力和法西斯残余的再起,维护世界和平,赞助各国人民的独立民主的愿望。}的决议,因为这些国际会议的决议都是有利于打败法西斯侵略者和维持世界和平的。\\
  中国共产党的外交政策的基本原则,是在彻底打倒日本侵略者,保持世界和平,互相尊重国家的独立和平等地位,互相增进国家和人民的利益及友谊这些基础之上,同各国建立并巩固邦交,解决一切相互关系问题,例如配合作战、和平会议、通商、投资等等。\\
  中国共产党对于保障战后国际和平安全的机构之建立,完全同意敦巴顿橡树林会议所作的建议和克里米亚会议对这个问题所作的决定。中国共产党欢迎旧金山联合国代表大会。中国共产党已经派遣自己的代表加入中国代表团出席旧金山会议,借以表达中国人民的意志\footnote[28]{ 一九四四年八月至十月,苏、美、英、中四国代表按照莫斯科会议和德黑兰会议的决定,在美国首都华盛顿郊区的敦巴顿橡树园举行会议,草拟了联合国机构的组织草案。一九四五年四月至六月,在美国旧金山召开了有五十个国家代表参加的联合国成立大会。当时中国解放区派遣董必武为代表加入中国代表团,参加了这次会议。}。\\
  我们认为国民党政府必须停止对于苏联的仇视态度,迅速地改善中苏邦交。苏联是第一个废除不平等条约并和中国订立平等新约的国家。在一九二四年孙中山先生召集的国民党第一次全国代表大会时和在其后进行北伐战争时,苏联是当时唯一援助中国解放战争的国家。在一九三七年抗日战争开始以后,苏联又是第一个援助中国反对日本侵略者的国家。中国人民对于苏联政府和苏联人民的这些援助,表示感谢。我们认为太平洋问题的最后的彻底的解决,没有苏联参加是不可能的。\\
  我们要求各同盟国政府,首先是美英两国政府,对于中国最广大人民的呼声,加以严重的注意,不要使他们自己的外交政策违反中国人民的意志,因而损害同中国人民之间的友谊。我们认为任何外国政府,如果援助中国反动分子而反对中国人民的民主事业,那就将要犯下绝大的错误。\\
  中国人民欢迎许多外国政府宣布废除对于中国的不平等条约,并和中国订立平等新约的措施。但是,我们认为平等条约的订立,并不就表示中国在实际上已经取得真正的平等地位。这种实际上的真正的平等地位,决不能单靠外国政府的给予,主要地应靠中国人民自己努力争取,而努力之道就是把中国在政治上经济上文化上建设成为一个新民主主义的国家,否则便只会有形式上的独立、平等,在实际上是不会有的。就是说,依据国民党政府的现行政策,决不会使中国获得真正的独立和平等。\\
  我们认为在日本侵略者被打败并无条件投降之后,为着彻底消灭日本的法西斯主义、军国主义及其所由产生的政治、经济、社会的原因,必须帮助一切日本人民的民主力量建立日本人民的民主制度。没有日本人民的民主制度,便不能彻底地消灭日本法西斯主义和军国主义,便不能保证太平洋的和平。\\
  我们认为开罗会议关于朝鲜独立的决定是正确的,中国人民应当帮助朝鲜人民获得解放。\\
  我们希望印度独立。因为一个独立的民主的印度,不但是印度人民的需要,也是世界和平的需要。\\
  对于南洋各国——缅甸、马来亚、印度尼西亚、越南、菲律宾,我们希望这些国家的人民在日本侵略者被打败以后,能够得到建立独立的民主的国家制度的权利。对于泰国,应当仿照对待欧洲法西斯附属国的方法去处理。\\
  关于具体纲领的说明,主要的就是这样。\\
  再说一遍,一切这些具体纲领,如果没有一个举国一致的民主的联合政府,就不可能顺利地在全中国实现。\\
  中国共产党在其为中国人民的解放事业而奋斗的二十四年中,创造了这样的地位,就是说,不论什么政党或社会集团,也不论是中国人或外国人,在有关中国的问题上,如果采取不尊重中国共产党的意见的态度,那是极其错误而且必然要失败的。过去和现在都有这样的人,企图孤行己见,不尊重我们的意见,但是结果都行不通。这是什么缘故呢?不是别的,就是因为我们的意见,符合于最广大的中国人民的利益。中国共产党是中国人民的最忠实的代言人,谁要是不尊重中国共产党,谁就是在实际上不尊重最广大的中国人民,谁就一定要失败。\\
\subsubsection*{\myformat{中国国民党统治区的任务}}
关于我党的一般纲领和具体纲领,我已在上面作了充分的说明。无疑地,这些纲领是要在全中国实行的;整个国际国内的形势,给中国人民展开了这种想望。但是,目前在国民党统治区,在沦陷区,在解放区,这三种地方互不相同的情势,不能不使我们在实行时要有所区别。不同的情形,产生不同的任务。这些任务,有些我已经在前面说到了,有些还须在下面加以补充。\\
  在国民党统治区,人民没有爱国活动的自由,民主运动被认为非法,但是包括许多阶层、许多民主党派和民主分子的积极活动是在发展中。中国民主同盟,在今年一月发表了要求结束国民党一党专政和成立联合政府的宣言。社会各界发表同类性质的宣言的,还有许多。国民党内也有许多人,对于他们自己的领导机关的政策,日益表示怀疑和不满,日益感觉他们的党在广大人民中孤立起来的危险性,而要求有一种适合时宜的民主的改革。重庆等地的工人、农民、文化界、学生界、教育界、妇女界、工商界、公务人员乃至一部分军人的民主运动,正在发展。所有这些,预示着一切受压迫阶层的民主运动正在逐渐地向着同一的目标而汇合起来。目前运动的弱点,在于社会的基层分子还没有广泛地参加,地位非常重要而生活痛苦不堪的农民、工人、士兵和下层公教人员,还没有组织起来。目前运动的另一弱点,是参加运动的民主分子中,还有许多人对于根据民主原则发动斗争以求转变时局这一个基本方针,还缺乏明确的和坚决的精神。但是客观形势,正在迫着一切受压迫的阶层、党派和社会集团,逐渐地觉悟和团结起来。不管国民党政府如何镇压,也不能阻止这一运动的发展。\\
  国民党统治区内被压迫的一切阶层、党派和集团的民主运动,应当有一个广大的发展,并把分散的力量逐渐统一起来,为着实现民族团结,建立联合政府,打败日本侵略者和建设新中国而斗争。中国共产党和解放区人民,应当给予他们以一切可能的援助。\\
  在国民党统治区,共产党人应当继续执行广泛的抗日民族统一战线政策。不管什么人,哪怕昨天还是反对我们的,只要他今天不反对了,就应该同他合作,为共同的目标而奋斗。\\
\subsubsection*{\myformat{中国沦陷区的任务}}
在沦陷区,共产党人应当号召一切抗日人民,学习法国和意大利的榜样,将自己组织于各种团体中,组织地下军,准备武装起义,一俟时机成熟,配合从外部进攻的军队,里应外合地消灭日本侵略者。日本侵略者及其忠实走狗,对于我沦陷区内的兄弟姊妹们的摧残、掠夺、奸淫和侮辱,激起了一切中国人的火一样的愤怒,报仇雪耻的时机快要到来了。沦陷区的人民,在欧洲战场的胜利和八路军新四军的胜利的鼓舞之下,极大地增高了他们的抗日情绪。他们迫切地需要组织起来,以便尽可能迅速地获得解放。因此,我们必须将沦陷区的工作提到和解放区的工作同等重要的地位上。必须有大批工作人员到沦陷区去工作。必须就沦陷区人民中训练和提拔大批的积极分子,参加当地的工作。在沦陷区中,东北四省沦陷最久,又是日本侵略者的产业中心和屯兵要地,我们应当加紧那里的地下工作。对于流亡到关内的东北人民,应当加紧团结他们,准备收复失地。\\
  在一切沦陷区,共产党人应当执行最广泛的抗日民族统一战线政策。不管什么人,只要是反对日本侵略者及其忠实走狗的,就要联合起来,为打倒共同敌人而斗争。\\
  应当向一切帮助敌人反对同胞的伪军伪警及其它人员提出警告:他们必须赶快认识自己的罪恶行为,及时回头,帮助同胞反对敌人,借以赎回自己的罪恶。否则,敌人崩溃之日,民族纪律是不会宽容他们的。\\
  共产党人应当向一切有群众的伪组织进行争取说服工作,使被欺骗的群众站到反对民族敌人的战线上来。同时,对于那些罪大恶极不愿改悔的汉奸分子进行调查工作,以便在国土收复之后,依法惩治他们。\\
  对于国民党内组织汉奸反对中国人民、中国共产党、八路军、新四军和其它人民军队的背叛民族的反动分子,必须向他们提出警告,叫他们早日悔罪。否则,在国土收复之后,必然要将他们和汉奸一体治罪,决不宽饶。\\
\subsubsection*{\myformat{中国解放区的任务}}
我党的全部新民主主义的纲领已经在解放区实行了并且有了显着的成绩,聚集了巨大的抗日力量,今后应当从各方面发展和巩固这种力量。\\
  在目前条件下,解放区的军队应向一切被敌伪占领而又可能攻克的地方,发动广泛的进攻,借以扩大解放区,缩小沦陷区。\\
  但是同时应当注意,敌人在目前还是有力量的,它还可能向解放区发动进攻。解放区军民必须随时准备粉碎敌人的进攻,并注意解放区的各项巩固工作。\\
  应当扩大解放区的军队、游击队、民兵和自卫军,并加紧整训,增强战斗力,为最后打败侵略者准备充分的力量。\\
  在解放区,一方面,军队应实行拥政爱民的工作,另一方面,民主政府应领导人民实行拥军优抗的工作,更大地改善军民关系。\\
  共产党人在各个地方性的联合政府的工作中,在社会工作中,应当继续同一切抗日民主分子,在新民主主义纲领的基础上,进行很好的合作。\\
  同样,在军事工作中,共产党人应当同一切愿意和我们合作的抗日民主分子,在解放区军队的内部和外部,很好地合作。\\
  为了提高工农劳动群众在抗日和生产中的积极性,减租减息和改善工人、职员待遇的政策,必须充分地执行。解放区的工作人员,必须努力学会做经济工作。必须动员一切可能的力量,大规模地发展解放区的农业、工业和贸易,改善军民生活。为此目的,必须实行劳动竞赛,奖励劳动英雄和模范工作者。在城市驱逐日本侵略者以后,我们的工作人员,必须迅速学会做城市的经济工作。\\
  为着提高解放区人民大众首先是广大的工人、农民、士兵群众的觉悟程度和培养大批工作干部,必须发展解放区的文化教育事业。解放区的文化工作者和教育工作者在推进他们的工作时,应当根据目前的农村特点,根据农村人民的需要和自愿的原则,采用适宜的内容和形式。\\
  在推进解放区的各项工作时,必须十分爱惜当地的人力物力,任何地方都要作长期打算,避免滥用和浪费。这不但是为着打败日本侵略者,而且是为着建设新中国。\\
  在推进解放区的各项工作时,必须十分注意扶助本地人管理本地的事业,必须十分注意从本地人民优秀分子中大批地培养本地的工作干部。一切从外地去的人,如果不和本地人打成一片,如果不是满腔热情地勤勤恳恳地并适合情况地去帮助本地干部,爱惜他们,如同爱惜自己的兄弟姊妹一样,那就不能完成农村民主革命这个伟大的任务。\\
  八路军、新四军及其它人民军队,每到一地,就应立即帮助本地人民,不但要组织以本地人民的干部为领导的民兵和自卫军,而且要组织以本地人民的干部为领导的地方部队和地方兵团。然后,就可以产生有本地人领导的主力部队和主力兵团。这是一项非常重要的任务。如果不能完成此项任务,就不能建立巩固的抗日根据地,也不能发展人民的军队。\\
  当然,一切本地人,应当热烈地欢迎和帮助从外地去的革命工作人员和人民军队。\\
  关于对待暗藏的民族破坏分子的问题,必须提起大家的注意。因为公开的敌人,公开的民族破坏分子,容易识别,也容易处置;暗藏的敌人,暗藏的民族破坏分子,就不容易识别,也就不容易处置。因此,对于这后一种人必须采取严肃态度,而在处理时又要采取谨慎态度。\\
  根据信教自由的原则,中国解放区容许各派宗教存在。不论是基督教、天主教、回教、佛教及其它宗教,只要教徒们遵守人民政府法律,人民政府就给以保护。信教的和不信教的各有他们的自由,不许加以强迫或歧视。\\
  我们的大会应向各解放区人民提议,尽可能迅速地在延安召开中国解放区人民代表会议\footnote[29]{ 中国共产党第七次全国代表大会以后,一九四五年七月十三日,各解放区、各人民团体以及八路军、新四军等各方面的代表,曾在延安开会,成立“中国解放区人民代表会议筹备委员会”。日本投降以后,因为时局变化,中国解放区人民代表会议没有召开。},以便讨论统一各解放区的行动,加强各解放区的抗日工作,援助国民党统治区人民的抗日民主运动,援助沦陷区人民的地下军运动,促进全国人民的团结和联合政府的成立。中国解放区现在已经成了全国广大人民抗日救国的重心,全国广大人民的希望寄托在我们身上,我们有责任不要使他们失望。中国解放区人民代表会议的召集,将对中国人民的民族解放事业起一个巨大的推进作用。\\
\subsection*{\myformat{五 全党团结起来,为实现党的任务而斗争}}
同志们,我们已经了解了我们的任务和我们为完成这些任务所采取的政策,那末,我们应该用怎样的工作态度去执行这些政策和完成这些任务呢?\\
  目前国际国内的形势,在我们和中国人民面前显示了光明的前途,具备了前所未有的有利条件,这是显然的,毫无疑义的。但是同时,依然存在着严重的困难条件。谁要是只看见光明一面,不看见困难一面,谁就会不能很好地为实现党的任务而斗争。\\
  我们的党和中国人民一道,不论在整个党的二十四年历史中,在八年抗日战争中,为中国人民创造了巨大的力量,我们的工作成绩是很显然的,毫无疑义的。但是同时,我们的工作中依然存在着缺点。谁要是只看见成绩一面,不看见缺点一面,谁也就不会很好地为实现党的任务而斗争。\\
  中国共产党自从一九二一年诞生以来,在其二十四年的历史中,经历了三次的伟大斗争,这就是北伐战争、土地革命战争和现在还在进行中的抗日战争。我们的党从它一开始,就是一个以马克思列宁主义的理论为基础的党,这是因为这个主义是全世界无产阶级的最正确最革命的科学思想的结晶。马克思列宁主义的普遍真理一经和中国革命的具体实践相结合,就使中国革命的面目为之一新,产生了新民主主义的整个历史阶段。以马克思列宁主义的理论思想武装起来的中国共产党,在中国人民中产生了新的工作作风,这主要的就是理论和实践相结合的作风,和人民群众紧密地联系在一起的作风以及自我批评的作风。\\
  反映了全世界无产阶级实践斗争的马克思列宁主义的普遍真理,在它同中国无产阶级和广大人民群众的革命斗争的具体实践相结合的时候,就成为中国人民百战百胜的武器。中国共产党正是这样做了。我们党的发展和进步,是从同一切违反这个真理的教条主义和经验主义作坚决斗争的过程中发展和进步起来的。教条主义脱离具体的实践,经验主义把局部经验误认为普遍真理,这两种机会主义的思想都是违背马克思主义的。我们党在自己的二十四年奋斗中,克服了和正在克服着这些错误思想,使得我们的党在思想上极大地巩固了。我们党现在已有了一百二十一万党员。其中绝大多数是在抗日时期入党的,在他们之中存在着各种不纯正的思想。在抗日以前入党的党员中,也有这种情形。几年来的整风工作收到了巨大的成效,使这些不纯正的思想受到了很多的纠正。今后应当继续这种工作,以“惩前毖后、治病救人”的精神,更大地展开党内的思想教育。必须使各级党的领导骨干都懂得,理论和实践这样密切地相结合,是我们共产党人区别于其它任何政党的显着标志之一。因此,掌握思想教育,是团结全党进行伟大政治斗争的中心环节。如果这个任务不解决,党的一切政治任务是不能完成的。\\
  我们共产党人区别于其它任何政党的又一个显着的标志,就是和最广大的人民群众取得最密切的联系。全心全意地为人民服务,一刻也不脱离群众;一切从人民的利益出发,而不是从个人或小集团的利益出发;向人民负责和向党的领导机关负责的一致性;这些就是我们的出发点。共产党人必须随时准备坚持真理,因为任何真理都是符合于人民利益的;共产党人必须随时准备修正错误,因为任何错误都是不符合于人民利益的。二十四年的经验告诉我们,凡属正确的任务、政策和工作作风,都是和当时当地的群众要求相适合,都是联系群众的;凡属错误的任务、政策和工作作风,都是和当时当地的群众要求不相适合,都是脱离群众的。教条主义、经验主义、命令主义、尾巴主义、宗派主义、官僚主义、骄傲自大的工作态度等项弊病之所以一定不好,一定要不得,如果什么人有了这类弊病一定要改正,就是因为它们脱离群众。我们的代表大会应该号召全党提起警觉,注意每一个工作环节上的每一个同志,不要让他脱离群众。教育每一个同志热爱人民群众,细心地倾听群众的呼声;每到一地,就和那里的群众打成一片,不是高踞于群众之上,而是深入于群众之中;根据群众的觉悟程度,去启发和提高群众的觉悟,在群众出于内心自愿的原则之下,帮助群众逐步地组织起来,逐步地展开为当时当地内外环境所许可的一切必要的斗争。在一切工作中,命令主义是错误的,因为它超过群众的觉悟程度,违反了群众的自愿原则,害了急性病。我们的同志不要以为自己了解了的东西,广大群众也和自己一样都了解了。群众是否已经了解并且是否愿意行动起来,要到群众中去考察才会知道。如果我们这样做,就可以避免命令主义。在一切工作中,尾巴主义也是错误的,因为它落后于群众的觉悟程度,违反了领导群众前进一步的原则,害了慢性病。我们的同志不要以为自己还不了解的东西,群众也一概不了解。许多时候,广大群众跑到我们的前头去了,迫切地需要前进一步了,我们的同志不能做广大群众的领导者,却反映了一部分落后分子的意见,并且将这种落后分子的意见误认为广大群众的意见,做了落后分子的尾巴。总之,应该使每个同志明了,共产党人的一切言论行动,必须以合乎最广大人民群众的最大利益,为最广大人民群众所拥护为最高标准。应该使每一个同志懂得,只要我们依靠人民,坚决地相信人民群众的创造力是无穷无尽的,因而信任人民,和人民打成一片,那就任何困难也能克服,任何敌人也不能压倒我们,而只会被我们所压倒。\\
  有无认真的自我批评,也是我们和其它政党互相区别的显着的标志之一。我们曾经说过,房子是应该经常打扫的,不打扫就会积满了灰尘;脸是应该经常洗的,不洗也就会灰尘满面。我们同志的思想,我们党的工作,也会沾染灰尘的,也应该打扫和洗涤。“流水不腐,户枢不蠹”,是说它们在不停的运动中抵抗了微生物或其它生物的侵蚀。对于我们,经常地检讨工作,在检讨中推广民主作风,不惧怕批评和自我批评,实行“知无不言,言无不尽”,“言者无罪,闻者足戒”,“有则改之,无则加勉”这些中国人民的有益的格言,正是抵抗各种政治灰尘和政治微生物侵蚀我们同志的思想和我们党的肌体的唯一有效的方法。以“惩前毖后,治病救人”为宗旨的整风运动之所以发生了很大的效力,就是因为我们在这个运动中展开了正确的而不是歪曲的、认真的而不是敷衍的批评和自我批评。以中国最广大人民的最大利益为出发点的中国共产党人,相信自己的事业是完全合乎正义的,不惜牺牲自己个人的一切,随时准备拿出自己的生命去殉我们的事业,难道还有什么不适合人民需要的思想、观点、意见、办法,舍不得丢掉的吗?难道我们还欢迎任何政治的灰尘、政治的微生物来玷污我们的清洁的面貌和侵蚀我们的健全的肌体吗?无数革命先烈为了人民的利益牺牲了他们的生命,使我们每个活着的人想起他们就心里难过,难道我们还有什么个人利益不能牺牲,还有什么错误不能抛弃吗?\\
  同志们,我们的大会闭幕之后,我们就要上战场去,根据大会的决议,为着最后地打败日本侵略者和建设新中国而奋斗。为达此目的,我们要和全国人民团结起来。我重说一遍,不管什么阶级,什么政党,什么社会集团或个人,只要是赞成打败日本侵略者和建设新中国的,我们就要加以联合。为达此目的,我们要把我们党的一切力量在民主集中制的组织和纪律的原则之下,坚强地团结起来。不论什么同志,只要他是愿意服从党纲、党章和党的决议的,我们就要和他团结。我们的党,在北伐战争时期,不超过六万党员,后来大部分被当时的敌人打散了;在土地革命战争时期,不超过三十万党员,后来大部分也被当时的敌人打散了。现在我们有了一百二十多万党员,这一回无论如何不要被敌人打散。只要我们能吸取三个时期的经验,采取谦虚态度,防止骄傲态度,在党内,和全体同志更好地团结起来,在党外,和全国人民更好地团结起来,就可以保证,不但不会被敌人打散,相反地,一定要把日本侵略者及其忠实走狗坚决、彻底、干净、全部地消灭掉,并且在消灭他们之后,把一个新民主主义的中国建设起来。\\
  三次革命的经验,尤其是抗日战争的经验,给了我们和中国人民这样一种信心:没有中国共产党的努力,没有中国共产党人做中国人民的中流砥柱,中国的独立和解放是不可能的,中国的工业化和农业近代化也是不可能的。\\
  同志们,有了三次革命经验的中国共产党,我坚决相信,我们是能够完成我们的伟大政治任务的。\\
  成千成万的先烈,为着人民的利益,在我们的前头英勇地牺牲了,让我们高举起他们的旗帜,踏着他们的血迹前进吧!\\
  一个新民主主义的中国不久就要诞生了,让我们迎接这个伟大的日子吧!\\
\newpage\section*{\myformat{论联合政府(未删节版)}\\\myformat{(一九四五年四月二四日在中国共产党第七次全国代表大会上的政治报告)}}\addcontentsline{toc}{section}{论联合政府(未删节版)}
\begin{introduction}\item 【编者按:这篇文章在收入毛选的时候作了非常大幅度的修改。在第二节《国际形势与国内形势》,有很多段落是大力称许英国和美国在反法西斯战争中的作用,几乎全部删去。在第四节《中国共产党的政策》,有一小段主张建立中华民主共和国联邦,还有“在这个联邦基础上组织中央政府”的话,亦全部删去。在同一节,删去一小段,这段强调新民主主义并不是要实行社会主义﹔不仅如此,整节在很多地方都在事后补加上“无产阶级领导”的话。窜改的目的是为了使原文中那种强调发展资本主义,强调社会主义是另一个历史时期的立场,变成为处处强调在“无产阶级领导下”发展新民主主义经济,使人觉得毛泽东会迅速从新民主主义转变为社会主义。此外删去的还有很多的,例如有一大段保证共产党即使在国民大会中得到多数,也不会组织一党政府;在发展工业的那部分,把欢迎外资的话删去了。】\end{introduction}
\subsection*{\myformat{一 中国人民的基本要求}}
同志们! 盼望很久的我们党的第七次全国代表大会,现在开会了,我代表中央委员会向你们作报告。目前的时局,要求我们的大会讨论与决定许多重大问题。然后,我们将向中国人民说明我们的意见。如果他们同意我们的意见,我们就协同他们动手去做。\\
  我们的大会是在这种情况之下开会的:中国人民在其对于日本侵略者作了将近八年的坚决的英勇的不屈不挠的奋斗,经历了无数的艰难困苦与自我牺牲之后,出现了这样的新局面:整个世界上反对法西斯侵略者的神圣的正义的战争,已经取得了有决定意义的胜利,中国人民配合同盟国打败日本侵略者的时机,己经迫近了。但是中国现在仍然不团结, 日本侵略者仍然在压迫我们, 中国仍然存在着严重的危机。在此种情况下,我们应该怎样做呢? 毫无疑义,中国急需团结各党各派及无党无派的代表人物在一起,成立民主的临时的联合政府,以便实行民主的改革,克服目前的危机,动员与统一全中国的抗日力量,有力地和同盟国配合作战,打败日本侵略者,使中国人民从日本侵略者手中解放出来。然后,在广泛的民主基础之上,召开国民代表大会,成立包括更广大范围的各党各派与无党无派代表人物在内的同样是联合性质的民主的正式政府,领导解放后的全国人民,将中国建设成为一个独立、自由、民主、统一与富强的新国家。一句话,走团结与民主的路线,打败侵略者,建设新中国。\\
  我们认为只有这样做,才是反映了中国人民的基本要求。因此,我的报告,主要地就是讨论这些要求。中国应否成立民主的联合政府,已成了中国人民及同盟国舆论界十分关心的问题,因此,我的报告,将着重地说明联合政府问题。\\
  中国共产党在八年抗日战争中的工作,己经克服了很多的困难,获得了巨大的成绩。但是在目前形势下,在我党与人民面前,尚存在着严重的困难。目前的时局,要求我党进一步从事紧急的与更加切实的工作,继续地克服困难,为完成中国人民的基本要求而奋斗。\\
\subsection*{\myformat{二 国际形势与国内形势}}
中国人民能不能实现我们在上面提出的那些基本要求呢? 我们认为两种可能性都存在,依靠中国人民觉悟、团结与努力的程度来决定。但是目前的国际国内形势,都对中国人民提供了有利的条件。中国人民如能很好地利用这些条件,积极地坚决地再接再厉地向前奋斗,战胜侵略者与建设新中国,是毫无疑义的。中国人民应当加倍努力,为完成自己的神圣任务而奋斗。\\
  目前的国际形势是怎样的呢?\\
  和中国及外国一切反动派的预料相反,英美苏三大民主国一直是团结的。她们之间,过去存在过,将来还可能发生某些争议,但是团结终究是统治一切的。这是一个决定一切的条件,克里米亚会议最后地证明了这一点。这个条件是在世界历史的重大转变关节——二十世纪四十年代产生的。在法西斯侵略战争爆发成为威胁全世界人民的战争的时候,实际上帮助法西斯侵略者反对英美苏团结的反动势力,从许多主要国家(不是一切国家)的政治舞台上大批地被推落下去,赞成英美苏团结的反法西斯势力占了上风,这个条件就产生了。自从世界上出现了这个条件,世界的面目就改观了。整个法西斯势力及其在各国的游魂,必须被消灭。国际间的重大问题,必须以三大国或五大国为首的协议来解决。各国内部问题,无例外地必须按照民主原则来解决。世界将引向进步,决不是引向反动。这些就是我们这个世界的新面目。当然应该提起充分的警觉,了解到历史的若干暂时的甚至是严重的曲折,可能还会发生;许多国家中不愿看见本国人民与外国人民获得团结、进步与解放,不愿看见英美苏中法继续团结领导世界新秩序的世界分裂主义者的反动势力,还是强大的;谁要是忽视了这些,谁就将在政治上犯错误。但是,历史的总趋向已经确定,不能改变了,世界的新面目已经出现了。\\
  这个新面目,仅仅不利于法西斯和在实际上帮助法西斯的各国反动派(中国的也在内)。对于一切国家的人民及其有组织的民主势力,则都是福音。\\
  人民,只有人民,才是创造世界的动力。苏联人民创造了强大力量,充当了打倒法西斯的主力军。英、美、中、法四大国及其它反法西斯同盟国的人民的伟大努力,使打倒法西斯成为可能。法西斯被打倒以后,各国人民将建设一个巩固的与持久的和平世界。四月二十五日在旧金山举行的联合国会议,将是这种和平的起点。\\
  战争教育了人民,人民将赢得战争,赢得和平,又赢得进步,这就是目前世界新形势的规律。\\
  这一新形势,与第一次世界大战及在其后的所谓“和平”时代,是大不相同的。在那时,还没有现在这样的苏联,也没有现在这样的英、美、中、法及其它反法西斯同盟各国的人民的觉悟程度,自然也就不能有三大国或五大国为首的现在这样的世界团结。我们现在是处在完全新的局面之下。现在有的是:觉悟了与团结了并且正在更加觉悟与更加团结的世界人民以及人民的有组织的力量,这就规定了世界历史车轮所要走向的目标和到此目标所要选取的道路。\\
  法西斯侵略国家被打败,总的和平局面出现了以后,并不是说就没有了斗争。广泛地散布着的法西斯残余势力,一定还要捣乱。反法西斯侵略战争的阵营中,存在着反民主势力,他们仍然要压迫人民。所以,国际和平实现以后,反法西斯的人民大众与法西斯残余势力之争,民主与反民主之争,仍将充满世界的大部分地方。只有经过长期的努力,克服了法西斯残余势力及反民主势力,才能有最广泛的人民的胜利。到达这一天,决不是很快与很容易的,但是必然要到达这一天。反法西斯战争——正义的第二次世界大战的胜利,给这个战后人民斗争的胜利开辟了道路。也只有这后一种斗争胜利了,巩固的与持久的和平才得了保障。这就是世界人民的光明前途。\\
  由于英美苏三大民主国及其它欧洲国家的团结,最后地打败法西斯德国的战争很快就可结束,红军已攻击柏林,这个希特勒的神经中枢,不日可下。英美法盟军正在最后地打击希特勒残军。意大利人民发动了起义。这一切,将最后地打败希特勒。希特勒被打败以后,将在世界上出现这样的局面:解放欧洲,并立即增强着解放亚洲的可能性,从而使亚洲获得解放。\\
  英美中三大国团结在反对日本侵略者的事业上。由于中国人民在八年战争中的长期艰苦的奋斗,英国在东方的协同作战,特别是美国在太平洋上的胜利,使战争迫近了日本的大门。日本侵略者已处于极端的不利地位,它的军心民心已发生了更大的动摇。但是它还有力量,它正在准备持久的挣扎,并希望通过中国及同盟国内部的动摇分子谋取妥协的和平。但是一切太平洋国家全体人民的利益,均要求完全消灭日本侵略者。苏联已经废除了苏日中立条约,这件事给了中国人民及太平洋各国人民以极大的兴奋。在这种种情况下,我们应该这样说:到达最后地完全地消灭日本侵略者,还有一段艰难的路程,我们决不可轻敌;但是胜利的把握是更大了,我们一定能胜利。\\
  中国人民从来也没有遇到过象现在这样有利的国际条件,这个条件鼓励中国人民支持了长期的战争。\\
  目前中国的国内形势是怎样的呢?\\
  中国的长期战争,使中国人民付出了并且还将再付出重大的牺牲。但是同时,正是这个战争,锻炼了并且还将再锻炼英勇斗争的中国人民。这个战争促进了中国人民的觉悟与团结到了这样的程度:不但在中国古代没有过,就是近百年来中国人民的一切伟大斗争,也没有一次比得上的。在中国人民面前,不仅存在着强大的民族敌人,而且存在着强大的实际上帮助民族敌人的国内反动势力,这是一方面。但是另一方面,中国人民不但已经有了较之过去任何时候要高的觉悟程度,而且有了强大的中国解放区与日益高涨着的全国性的民主运动,所有这些,就是国内的有利条件。如果说,中国近百年来一切人民斗争都遭到了失败或挫折,而这是因为缺乏国际国内的必要的条件,那么,这一次,就不同了,比较以往历次,一切必要的条件更具备了,避免失败与取得胜利的可能性充分地存在。如果我们能够团结全国人民,努力奋斗,并给以适当的指导,我们就会有胜利。\\
  中国人民团结起来打败侵略者与建设新中国的信心,现在是极大地增强了,中国人民克服一切困难,实现其具有伟大历史意义的基本要求的时机,已经到来了。这一点还有疑义吗? 我以为没有疑义了。\\
  这些,就是目前国际与国内的一般形势。\\
\subsection*{\myformat{三 抗日战争中的两条路线}}
\subsubsection*{\myformat{中国问题的关键}}
谈到国内形势,我们还应对中国抗日战争加以具体的分析。\\
  中国是全世界反法西斯战争中五个最大国家之一,是在亚洲大陆上反对日本侵略者的主要国家。中国人民不但在抗日战争中起了与还将起极大的作用,而且在保障战后世界和平上将起极大的作用,在保障东方和平上则将起决定的作用。中国人民在其八年抗日战争中,为了自己的解放,为了帮助各同盟国的胜利,曾经作了伟大的努力。这种努力,主要地是属于中国人民方面的。中国军队的广大官兵,在前线流血战斗;中国的工人、农民、知识界、产业界,在后方努力工作;海外华侨,输财助战;一切抗日政党,除了那些反人民分子外,均对战争有所尽力。总之,中国人民以自己的血与汗和日本侵略者英勇地奋战了八年之久。但是多年以来,中国反动分子造作谣言,蒙蔽舆论,不使中国人民在抗日战争中所起作用的真相暴露于世。同时,对于中国八年抗日战争的各项经验,也还没有人作出全面的总结来。因此,我们的大会,应当对此作出适当的总结,借以教育人民并为我党决定政策的根据。\\
  提到总结经验,那么,大家可以很清楚地看到,中国存在着两条不同的指导路线,一条是能够打败日本侵略者的,一条是不但不能打败日本侵略者,而且在某些方面说来,它是在实际上帮助日本侵略者危害抗日战争的。\\
  由于国民党政府所采取的对日作战的消极政策与对内积极摧残人民的反动政策,招致了战争的失败,国土的大部沦陷,财政经济的危机,人民的被压迫,人民生活的痛苦,民族团结的被破坏,妨碍了动员与统一一切中国人民的抗日力量进行有效的战争,妨碍了中国人民的觉醒与团结。但是,中国人民的觉醒与团结的运动并没有停止,它是在日本侵略者与国民党政府的双重压迫之下曲折地发展着。两条路线:国民党政府压迫中国人民实行消极抗战的路线与中国人民觉醒与团结起来实行人民战争的路线, 很久以来,就明显地在中国存在着,这就是一切中国问题的关键所在。\\
\subsubsection*{\myformat{走着曲折道路的历史}}
为了使大家明了何以这个两条路线问题是一切中国问题的关键所在,必须回溯一下我们抗日战争的历史。\\
  中国人民的抗日战争,是在曲折的道路上发展起来的。这个战争,还是在一九三一年就开始了。一九三一年九月十八日,日本侵略者占领沈阳,几个月内,就把东三省占领了,国民党政府采取了不抵抗政策。但是东三省的人民,东三省的一部分爱国军队,在中国共产党领导或协助之下,违反国民党政府的意志,组织了东三省的抗日义勇军,从事英勇的游击战争。这个英勇的游击战争,曾经发展到很大的规模,中经许多困难挫折,但是始终没有被敌人消灭。一九三二年,日本侵略者进攻上海,国民党内的一派爱国分子,又一次违反国民党政府的意志,率领十九路军,抵抗了日本侵略者的进攻。一九三三年,日本侵略者进攻热河、察哈尔,国民党内的又一派爱国分子,第三次违反国民党政府的意志,组织了抗日同盟军,从事抵抗。但是一切这些抗日战争,除了中国人民、中国共产党、其他民主派别及海外华侨予以援助之外,国民党政府根据其不抵抗政策,是不给任何援助的。相反地,上海,察哈尔两次抗日行动,均被国民党政府一手破坏了。一九三三年,十九路军在福建成立了人民政府,也被国民党政府破坏了。\\
  那时的国民党政府为什么采取不抵抗政策呢? 主要的原因,在于国民党政府在一九二七年破坏了国共两党的合作与中国人民的团结。\\
  一九二四年,孙中山先生接受了中国共产党的建议,召集了有共产党人参加的第一次国民党全国代表大会,订出了联俄、联共、扶助农工的三大政策,建立了黄埔军校,实现了国共两党及各界人民的民族统一战线,因而在一九二五年,扫荡了广东的反动势力,在一九二六至一九二七年,举行了胜利的北伐战争,占领了长江流域及黄河流域,打败了北洋军阀政府,发动了广大的在中国历史上是空前的人民解放斗争。但是到了一九二七年春夏之交,正当北伐战争向前发展的紧要关头,这个代表中国人民解放事业的国共两党及各界人民的民族统一战线及其一切革命政策,就被国民党当局的叛卖性的反人民的“清党”政策与屠杀政策所破坏了。昨天的同盟者——中国共产党与中国人民,被看成了仇敌,昨天的敌人——帝国主义者与封建主义者,被看成了同盟者。就是这样,背信弃义地向着中国共产党与中国人民来一个突然的袭击,生气蓬勃的中国大革命就被葬送了。从此以后,内战代替了团结,独裁代替了民主,黑暗的中国代替了光明的中国。\\
  但是,中国共产党与中国人民并没有被吓倒、被征服、被杀绝,他们从地下爬起来,揩干净身上的血迹,掩埋好同伴的尸首,他们又继续战斗了。他们高举起革命的大旗,举行了武装的抵抗,在中国广大的区域内,组织了人民的政府,实行了土地的改革,创造了人民的军队——中国红军,保存了与发展了中国人民的革命力量。被国民党反动分子所抛弃的孙中山先生的革命的三民主义,由中国人民、中国共产党及其他民主分子继承下来了。\\
  到了日本侵略者打入东三省以后,中国共产党就在一九三三年,向一切进攻苏区与红军的国民党军队提议:在一、停止进攻;二、给予人民以自由权利;三、武装人民这样三个条件之下,订立停战协定,以便一致抗日。但是国民党当局拒绝了这个提议。\\
  从此以后,一方面,是国民党政府的内战政策越发猖狂。另一方面,是中国人民要求停止内战一致抗日的呼声越发高涨,各种人民爱国组织,在上海及其它许多地方建立起来。一九三五至一九三六年,长江南北各地的红军主力,在我党中央领导之下,经历了千辛万苦,移到了西北,并和西北红军汇合在一起。就在这两年,中国共产党适应新的情况,决定并执行了抗日民族统一战线的新的完整的政治路线,以团结抗日与建立新民主主义共和国为奋斗目标。一九三五年十二月九日,北平学生群众,在我党领导之下,发动了英勇的爱国运动,成立了中华民族解放先锋队,并使此种爱国运动推广到了全国各大城市。一九三六年十二月十二日,国民党内部主张抗日的两派爱国分子——东北军与十七路军,联合起来,勇敢地反对国民党当局的对日妥协与对内屠杀的反动政策,举行了有名的西安事变。同时,国民党内的其他爱国分子,也不满意国民党当局的当时政策。在此种形势下,国民党当局被迫放弃了内战政策,承认了人民的要求。以西安事变和平解决为时局转换的枢纽,形成了在新形势下的国内团结,发动了全国的抗日战争。在芦沟桥事变的前夜,即一九三七年五月,我党召集了一个具有历史意义的全国代表会议。在这个会议上,批准了我党自一九三五年以来的新的政治路线。\\
  从一九三七年七月七日芦沟桥事变到一九三八年十月武汉失守这一时期内,国民党政府的对日作战是比较努力的。在这个时期内,由于日本侵略者的大举进攻与全国人民民族义愤的高涨,使得国民党政府将其政策的重点放在反对日本侵略者身上,形成了全国军民抗日战争的高潮,一时出现了生气蓬勃的新气象。当时全国人民,我们共产党人,其他民主党派,都对国民党政府寄与极大的希望,就是说,希望它乘此民族艰危、人心振奋的时机,厉行民主改革,将孙中山先生的革命三民主义付诸实施。可是这个希望是失败了。就在这两年,一方面,有比较积极的抗战;另一方面,政府当局仍旧反对发动广大民众参加的人民战争,仍旧限制人民自动团结起来进行抗日与民主的活动。一方面,政府对待中国共产党及其他抗日党派的态度比较过去有了一个改变;另一方面,仍旧不给各党派以平等地位,并多方限制他们的活动,许多爱国政治犯并没有释放。主要的是国民党政府仍旧保持其自一九二七年发动内战以来的寡头专制形态,未能建立举国一致的民主的联合政府。\\
  这一时期内,我们共产党人就指出中国抗日战争的两条路线:或者是人民的全面的战争,这样就会胜利;或者是压迫人民的片面的战争,这样就会失败、我们又指出:战争将是长期的,必然要遇到许多艰难困苦;但是由于中国人民的努力,最后胜利必归于中国人民。\\
\subsubsection*{\myformat{人民战争}}
这一时期内,中国共产党领导的移到了西北的中国红军主力,改编为中国国民革命军第八路军,留在长江南北各地的中国红军游击部队,则改编为中国国民革命军新编第四军,相继开赴华北华中作战。内战时期的中国红军,保存了并发展了北伐时期黄埔军校及国民革命军的民主传统,曾经扩大到几十万人。由于国民党政府在南方各根据地内的残酷的摧毁、万里长征的消耗及其它原因,数量减少到几万人。于是有些人看不起这枝军队,以为抗日主要依靠国民党。但是人民是最好的鉴定人,他们知道八路军新四军此时数量虽小,质量很高,只有它才能执行真正的人民战争,它一旦开到抗日的前线,和那里的广大人民相结合,其前途是无限的。人民是正确的,当我在这里做报告的时候,我们的正式军队已发展到了九十一万人,民兵发展到了二百二十万以上。不管现在我们的正式军队比起国民党现存的军队来(包括中央系与地方系)在数量上还要少几十万,但是按其所抗击的敌伪军的数量与其所担负的战场的广大说来,按其战斗力说来,按其有广大的人民、民兵与自卫军配合作战说来,按其政治质量与其内部统一团结步调一致说来,它已经成了中国抗日战争的主力军。\\
  这个军队之所以有力量,是因为参加这个军队的一切人们,具有自觉的纪律,他们不是为着少数人的或狭隘集团的私利,而是为着正义的人民战争,为着广大人民群众的利益,为着全民族的利益,而结合,而战斗的。紧紧地和中国人民站在一起,全心全意地为中国人民服务,这就是这个军队的唯一宗旨。\\
  在这个宗旨下面,这个军队具有一往无前的精神,它要压倒一切敌人,而决不被敌人所屈服。不论在任何艰难困苦的场合,只要还存在一个人,这个人就要继续战斗下去。\\
  在这个宗旨下面,这个军队有一个很好的内部团结与外部团结。在内部:官兵之间,上下级之间,军事工作、政治工作与后勤工作之间,在外部:军民之间,军政之间,我友之间,均必须是团结的;一切妨害这些团结的现象,均在必须克服之列。\\
  在这个宗旨下面,这个军队有一个正确的争取敌军官兵与处理俘虏的政策。凡属投诚的,反正的,或在放下武器后愿意参加反对共同敌人的敌伪军人,一概表示欢迎,并给予适当的教育。一切俘虏,不许杀害、虐待与侮辱。\\
  在这个宗旨下面,这个军队形成了为人民战争所必需的一系列的战略战术,它善于按照变化着的具体条件从事机动灵活的游击战争,也善于作运动战。\\
  在这个宗旨下面,这个军队形成了为人民战争所必需的一系列的政治工作,为团结我军,团结友军,团结人民,瓦解敌军与保证战斗胜利而斗争。\\
  在这个宗旨下面,在游击战争的条件下,全军都可以并且已经是这样做了:利用战斗与训练的间隙,从事粮食与日用必需品的生产,达到军队自给,半自给,或部分自给之目的。借以克服经济困难,改善军队生活及减轻人民负担。在各个军事根据地上,也利用了一切可能性建立了许多小规模的军事工业。\\
  这个军队之所以有力量,还由于有人民自卫军与民兵这样广大的群众武装组织,和它一道配合作战。在中国解放区,一切青年,壮年,甚至老年的男人与女人,都在自愿的民主的与不脱离生产的原则下,组织在抗日人民自卫军之中。自卫军中的精干分子,除加入军队及游击队者外,则组织在民兵之中。没有这些群众武装力量的配合,要战胜日本侵略者是不可能的。\\
  这个军队之所以有力量,还由于它将自己划分为主力兵团与地方兵团两部分,前者可以随时执行超地方的作战任务,后者则固定在协同民兵、自卫军保卫地方与进攻当地敌人的任务上,这种划分,取得了人民的真心拥护。如果没有这种正确的划分,例如说,如果只注意主力兵团的作用,忽视地方兵团的作用,那么,在中国解放区的条件下,要战胜日本侵略者也是不可能的。在地方兵团方面,组织了许多经过良好训练,在军事、政治、民运各项工作上说来都是比较更健全的武装工作队,深人敌后之敌后,打击敌人,发动民众的抗日斗争,借以配合各个解放区正面战线的作战,收到了很大的成效。\\
  在中国解放区,在民主政府领导之下,号召一切抗日人民组织在工人的、农民的、青年的、妇女的、文化的及其他职业与工作的团体之中,热烈地从事援助军队的各项工作,例如动员人民参加军队,替军队运输粮食,优待抗日军人家属,帮助军队解决物质困难。在这方面更重要的,是动员游击队、民兵与自卫军,展开袭击运动与爆炸运动,侦察敌情,清除奸细,运输伤兵与保护伤兵,直接帮助了军队的作战。同时,全解放区人民又热烈地从事政治、经济、文化、卫生各项建设工作。在这方面最重要的,是动员全体人民从事粮食与日用品的生产,并使一切机关学校,除少数特殊情形者外,一律于工作或学习之暇,从事生产自给,以配合人民与军队的生产自给,造成了伟大的生产热潮,借以支持长期的抗日战争,尤为中国解放区的特色。在中国解放区,敌人的摧残是异常严重的。水、旱、虫灾,亦时常发生。但是,解放区民主政府领导全体人民,有组织地克服了与正在克服着各种困难,灭蝗、治水、救灾的伟大群众运动,收到了史无前例的成绩,使抗日战争能够长期地坚持下去。总之,一切为着前线,一切为着打倒日本侵略者与解放中国人民,这就是中国解放区全体军民的总口号,总方针。\\
  这就是真正的人民战争。只有这种人民战争,才能战胜民族敌人。国民党之所以失败,就是因为它拼命地反对人民战争。\\
  中国解放区的军队一旦得到新式武器的装备,它就会更加无敌,能够最后地打败日本侵略者了。\\
\subsubsection*{\myformat{两个战场}}
中国战场,一开始就分为两个战场:国民党战场与解放区战场。\\
  一九三八年十月武汉失守后,日本侵略者停止了向国民党战场的战略性的进攻,逐渐将其主要军事力量移到了解放区战场;同时,针对着国民党政府的失败情绪,声言愿意和它谋取妥协的和平,并将卖国贼汪精卫诱出重庆,在南京成立伪政府,实施民族的欺骗政策。从此时起,国民党政府开始了政策上的变化,将其重点由对外逐渐转移到对内。这首先表现在它的军事政策上,采取了对日作战的消极政策,保存军事实力,而把作战的重担放在解放区战场上,让日本人大举进攻解放区,自己则坐山观虎斗。\\
  一九三九年,国民党政府采取了反动的所谓“限制异党活动办法”,将抗战初期给予人民及抗日党派的某些权利,一概收回。从此时起,在国民党统治区内国民党政府将一切民主党派,首先与主要的是将中国共产党,打人地下。在国民党统治区各个省份的监狱与集中营内,充满了共产党人、爱国青年及其他民主战士。从一九三九年起的五年之内,直至一九四三年秋季为止,国民党政府举行了三次大规模的“反共高潮”,分裂国内的团结,造成严重的内战危险。有名的“解散”新四军及歼灭皖南新四军部队九千人的事变,就是发生在这个时期内。直到现时为止,国民党军队向解放区军队进攻的事件还未停止,并且看不出任何准备停止的征象。在此种情况下,一切诬蔑与谩骂,都从国民党反动分子的嘴里喷了出来。什么“奸党”、“奸军”、“奸区”,什么“破坏抗战、危害国家”,都是这些反动分子为着反对中国人民的得意的制造品。一九三九年七月七日,中国共产党中央委员会发表宣言,针对着当时的危机,提出了这样的口号:“坚持抗战,反对投降;坚持团结,反对分裂;坚持进步,反对倒退。”在这些适合时宜的口号之下,在五年之内,有力地打退了三次反动的反人民的“反共高潮”,克服了当时的危机。\\
  在这几年内,国民党战场实际上没有严重的战争。日本侵略者刀锋,主要地向着解放区战场。到一九四三年,侵华日军的百分之六十四及伪军的百分之九十五,为解放区战场担负着。国民党战场所担负的,不过日军的百分三十六及伪军的百分之五而已。\\
  一九四四年,日本侵略者举行打通大陆交通线的作战了,国民党战场表现出手足无措,毫无抵抗能力,几个月内,就将河南、湖南、广西、广东等省广大区域沦于敌手。仅在此时,两个战场分担抗敌的比例,才起了一些变化。然而就在我做这个报告的时候,在侵华日军(满洲的尚不在内)四十个师团,五十八万人中,解放区战场抗击的是二十二个半师团,三十二万人,占了百分之五十六;国民党战场抗击的,不过十七个半师团,二十六万人,仅占百分之四十四。伪军的分担则完全无变化。\\
  还应指出,数达八十万以上的伪军(包括伪正规军与伪地方武装在内),大部分是国民党将领率部投敌,或由国民党投敌军官所组成的。国民党反动分子事先即供给这些伪军以所谓“曲线救国”的叛国谬论,事后又在精神上与组织上支持他们,使他们配合日本侵略者反对中国解放区。此外,则动员大批军队封锁与进攻陕甘宁边区及敌后各解放区,其数量达到了七十九万七千人之多。这种严重情形,在国民党政府的新闻封锁政策下,很多的中国人外国人都无法知道。很多人只知道南斯拉夫有一个米海洛维奇,而不知道中国有几十个米海洛维奇。\\
\subsubsection*{\myformat{中国解放区}}
中国解放区,现在领有九千五百五十万人口。其地域,北起内蒙,南至海南岛,大部分敌人所到之处,都有八路军、新四军及其它人民军队的活动。在这个广大的中国解放区内,包括了十九个大的解放区,其地域则包括了辽宁、热河、察哈尔、绥远、陕西、甘肃、宁夏、山西、河北、河南、山东、江苏、浙江、安徽、江西、湖北、湖南、广东、福建等省,在这些省份中有些是大部分,有些是小部分,而以延安为各个解放区的指导中心。在这个广大的解放区内,黄河以西的陕甘宁边区,其人口一百五十万,不过是十九个解放区中的一个,而且除了浙东、琼崖两区之外,按其人口说来,它是一个最小的。有些人不明此种情形,以为所谓中国解放区,主要就是陕甘宁边区,这是一个误会,是由于国民党政府的封锁政策造成的。在这个广大的解放区内,实行了抗日民族统一战线的全部必要的政策,建立了或正在建立民选的共产党人和各党各派及无党无派代表人物合作的政府,亦即地方性的联合政府,全体人民的力量都动员起来了。所有这一切,使得中国解放区在强敌压迫之下,在国民党军队的封锁与进攻之下,在毫无外援之下,能够屹然树立,一天一天发展,缩小敌占区,扩大解放区,成为民主中国的模型,成为配合同盟国驱逐日本侵略者解放中国人民的重心。中国解放区的军队——八路军、新四军及其它人民军队,不但在对日战争的作战上,起了英勇的模范的作用,在执行抗日民族统一战线的各项民主政策上,也是起了模范作用的。\\
  一九三七年九月二十二日,中国共产党中央委员会发表宣言:承认孙中山先生的三民主义为中国今日之必需,本党愿为其彻底实现而奋斗。这一宣言,在中国解放区是完全实践了。\\
\subsubsection*{\myformat{国民党统治区}}
国民党内的主要统治集团,坚持独裁统治,实行了消极的抗日政策与反人民的国内政策。这样,就使得它的军队缩小了一半以上,并且大部分几乎丧失了战斗力;使得它自己和广大人民之间造成了深刻的裂痕,造成了民生凋敝,民怨沸腾,民变蜂起的严重危机;使得它在抗日战争中的作用,不但是极大地减少了,并且变成了动员与统一中国人民一切抗日力量的障碍物。\\
  为什么在国民党主要统治集团领导下会产生这种严重情况呢? 因为这个集团所代表的利益是中国的大地主、大银行家、大买办阶层的利益。这个极端少数的反动阶层,垄断着国民党政府管辖之下的军事、政治、经济、文化的一切重要的机构。他们将保全自己少数人的利益放在第一位,而把抗日放在第二位。他们也说“民族至上”,但是他们的行为却不符合于民族中大多数人民的要求。他们也说“国家至上”,但是他们所指的国家,就是大地主、大银行家、大买办阶层的封建法西斯独裁国家,并不是人民大众的民主国家。因此,他们惧怕人民起来,惧怕民主运动,惧怕认真的动员全民的抗日战争。这就是他们之所以采取对日作战的消极政策,对内的反人民、反民主、反共的反动政策之总根源。要问他们为什么采取这样的两面政策,例如,一面虽在抗日,一面又采取消极的作战政策,并且还被日本人经常选择为诱降的对象:一面在口头上宣称要发展中国经济,一面又在实际上扶助官僚资本,亦即大地主、大银行家、大买办的资本,垄断着中国的主要经济命脉,而残酷地压迫农民,压迫工人,压迫小资产阶级与自由资产阶级;一面在口头上宣称实行“民主”,“还政于民”,一面又在实际上残酷地压迫人民的民主运动,不愿实行丝毫的民主改革;一面在口头上宣称:“共党问题为一政治问题,应用政治方法解决”,一面又在军事上、政治上、经济上残酷地压迫中国共产党,把共产党看成他们的所谓“第一个敌人”,而把日本侵略者看成“第二个敌人”,并且每天都在积极地准备内战,处心积虑地要消灭共产党;一面在国民党内部,他们口头上宣称应有“精诚团结”,一面又在实际上鼓励中央系军队欺压地方系军队(所谓“杂牌军”),鼓励专制派欺压民主派,鼓励各派之间互相对立,以利其独裁统治;一面在口头上宣称要建立一个“近代国家”,一面又在实际上拼死命保持那个大地主、大银行家、大买办的封建法西斯独裁国家;一面和苏联在形式上保持外交关系,一面又在实际上采取仇视苏联的态度;一面依赖英美的援助,一面又反对英美的自由主义;一面同美国孤立派合唱“先亚后欧论”,藉以延长法西斯德国也就是延长一切法西斯的寿命,延长自己对于中国人民的法西斯统治的寿命,一面又在国际大家庭里投机取巧,把自己打扮成为反法西斯的英雄;要问如此种种的自相矛盾的两面政策从何而来,就是来自大地主、大银行家、大买办社会阶层这个总根源。\\
  但是国民党是一个复杂的政党。它是被这个代表大地主、大银行家、大买办阶层的反动集团所统治,所领导,但是整个国民党并不等于这个反动集团。它有许多领袖人物不属于这个集团,而且被这个集团所打击、排斥或轻视。它有更多的干部与党员群众及三青团团员群众并不满意这个集团的领导,而且有些甚至是反对它的领导的。这种情形,在被这个反动集团所统制的国民党的军队,国民党的政府机关,国民党的经济机关与国民党的文化机关中,都是存在着。在这些军队及机关里,包藏着广大的进步的民主分子。这个反动集团,其中又分为几派,互相斗争,并不是一个严密的统一体。把国民党看成清一色的反动派,无疑是很不适当的。\\
\subsubsection*{\myformat{比较}}
中国人民从中国解放区与国民党统治区,获得了明显的比较。\\
  难道还不明显吗? 两条路线,人民战争的路线与反对人民战争的消极抗日的路线,其结果:一条是胜利的,即使处在中国解放区这种环境恶劣与毫无外援的地位。另一条是失败的,即使处在国民党统治区这种极端有利与取得外国接济的地位。\\
  国民党政府把自己的失败归咎于缺乏武器。但是试问:缺乏武器的是国民党的军队呢? 还是解放区的军队?中国解放区的军队是中国军队中武器最缺乏的军队,他们只能从敌人手里夺取及在最恶劣条件下自己制造其武器。\\
  国民党中央系军队的武器,不是比起地方系军队来要好得多吗?但是比起战斗力来,中央系却多数劣于地方系。\\
  国民党拥有广大的人力资源,但是在它的错误的兵役政策下,人力补充却极端困难。中国解放区处在被敌人分割与战斗频繁的情况之下,却因适合人民需要的民兵与自卫军制度之普遍实施,又防止了对于人力资源的滥用与浪费,使人力动员源源不竭。\\
  国民党拥有粮食丰富的广大地区,人民每年供给七千万至一万万市担的粮食,但是大部分被经手人员中饱了,致使国民党的军队经常缺乏粮食,士兵饿得面黄肌瘦。中国解放区的主要部分隔在敌后,遭受敌人烧杀抢三光政策的摧残,其中有些是象陕北这样贫瘠的区域,但是却能用自己动手、发展农业生产的方法,很好地解决了粮食问题。\\
  国民党区域经济危机极端严重,工业大部分破产了,连布匹这样的日用品也要从美国运来。中国解放区却能用发展工业的方法,自己解决布匹及其它日用品。\\
  国民党区域工人,农民、店员、公务人员、知识分子及文化人,生活痛苦,达于极点。中国解放区的全体人民都有饭吃,有衣穿,有事做,有书读,有些地方做到了丰衣足食。\\
  利用抗战发国难财,官吏即商人,贪污成风,廉耻扫地,这是国民党区域的特色之一。艰苦奋斗,以身作则,工作之外,还要生产,奖励廉洁,禁绝贪污,这是中国解放区的特色之一。\\
  国民党区域剥夺人民的一切自由。中国解放区则给予人民以充分的自由。\\
  怪别人,还是怪自己? 怪外国缺乏援助,还是怪国民党政府的独裁统治与腐败无能? 这难道还不明白吗?\\
\subsubsection*{\myformat{“破坏抗战、危害国家”的是谁?}}
真凭实据地破坏了中国人民的抗战与危害了中国人民的国家的,难道不正是国民党政府吗? 这个政府一心一意地打了整十年的内战,将刀锋向着同胞,置一切国防事业于不顾,又用不抵抗政策送掉了东四省。日本人打进本部来了,仓皇应战,从芦沟桥退到了贵州省。但是国民党人却说:“共产党破坏抗战,危害国家”(见一九四三年九月国民党十一中全会的决议案)。唯一的证据,就是共产党联合了各党各派各界人民创造了英勇抗日的中国解放区。这些国民党人的逻辑,与中国人民的逻辑是这样的不相同,无怪乎很多问题都讲不通了。\\
  两个问题:\\
  第一个:究竟什么原因使得国民党政府抛弃了从黑龙江到芦沟桥,又从芦沟桥到贵州省这样广大的国土与这样众多的人民? 难道不是由于国民党政府所采取的不抵抗政策、消极的抗日政策与反人民战争的国内政策吗?\\
  第二个,究竟什么原因使得中国解放区战胜了敌伪军长期的残酷的进攻,从民族敌人手里恢复了这样广大的国土,解放了这样众多的人民?难道不是由于人民战争的正确路线吗?\\
\subsubsection*{\myformat{所谓“不服从政令、军令”}}
国民党政府还经常以“不服从政令、军令”责备中国共产党,但是我们只能这样说:幸喜中国共产党人还保存了中国人民的普通常识,没有服从那些实际上是把中国人民艰难困苦地从日本人手里夺回来的中国解放区再送交日本人的所谓“政令、军令”,例如一九三九年的所谓“限制异党活动办法”,一九四一年的所谓“解散新四军”与“退至旧黄河以北”,一九四三年的所谓“解散中国共产党”,一九四四年的所谓“限期取消十个师以外的全部军队”,以及在最近谈判中提出来的所谓将军队及地方政府移交给国民党,其交换条件是不许成立联合政府,只许收容几个共产党员到国民党独裁政府里去做官,并将此种办法称之为国民党政府的“让步”等等。幸喜我们没有服从这些东西,替中国人民保存了一片干净土,保存了一枝英勇抗日的军队,难道中国人民不应该庆贺这一个“不服从”吗? 难道国民党政府自己用自己的法西斯主义的政令与失败主义的军令,将黑龙江至贵州省的广大土地、人民送交日本人,还觉得不够吗?除了日本人及反动派欢迎这些“政令、军令”之外,难道还有什么爱国的有良心的中国人欢迎这些东西吗? 没有一个不是形式的而是实际的、不是独裁的而是民主的联合政府,能够设想中国人民会允许中国共产党人,擅自将这个获得了解放的中国解放区与抗日有功的人民军队,交给失败主义与法西斯主义的国民党独裁政府吗? 假如没有中国解放区及其军队,中国人民的抗日事业还有今日吗? 我们民族的前途还能设想吗?\\
\subsubsection*{\myformat{内战危险}}
迄今为止,国民党内的主要统治集团,坚持着独裁与内战的反动方针。有很多迹象表明,他们早已准备,龙其现在正在准备这样的行动:等候某一个同盟国的军队在中国大陆上驱除日本侵略者到了某一程度时,他们就要发动内战。他们并且希望某些同盟国将领们在中国境内执行斯科比将军在希腊所执行的职务,他们对于斯科比及希腊反动政府的屠杀事业,表示欢呼。他们企图把中国抛回到一九二七至一九三六年的国内战争的大海里去。国民党主要统治集团现在正在所谓“召开国民大会”及“政治解决”的烟幕之下,偷偷摸摸地进行其内战的准备工作。如果国人不加注意,不去揭露它的阴谋,阻止它的准备,那么,会有一个早上,要听到内战的炮声的。\\
\subsubsection*{\myformat{谈判}}
为着打败日本侵略者与建设新中国,为着防止内战,中国共产党在取得了其它民主派别的同意之后,于一九四四年九月间开会的国民参政会上,提出了立即废止国民党专政,成立民主的联合政府一项要求。无疑地,这项要求是适合时宜的,几个月内,获得了广大人民的响应。\\
  关于如何废止一党专政、成立联合政府以及实行必要的民主改革等项问题,我们和国民党政府之间曾经有了多次谈判,但是我们的一切建议,都遭到了国民党政府的拒绝。不但一党专政不愿废止,联合政府不愿成立,任何迫切需要的民主改革,例如取消特务机关,取消镇压人民自由的反动法令,释放政治犯,承认各党派合法地位,承认解放区,撤退封锁与进攻解放区的军队等,一项也不愿实行。就是这样,使得中国的政治关系处在非常严重的局面之下。\\
\subsubsection*{\myformat{两个前途}}
从整个形势看来,从上述一切国际国内实际情况的分析看来,我请大家注意,不要以为我们的事业,一切都将是顺利的,美妙的。不,不是这样,事实是好坏两个可能性,好坏两个前途都存在着。继续独裁统治,不许民主改革;不是将重点放在反对日本侵略者上面,而是放在反对人民上面;即使日本侵略者被打败,中国仍可能发动内战,将中国拖回到痛苦重重的不独立,不自由,不民主,不统一,不富强的老状态里去,这是一个可能性,这是一个前途。这个可能性,这个前途,依然存在,并不因为国际形势之好,国内人民觉悟程度之增长及有组织的人民力量之发展,它就似乎没有了,或自然地消失了。希望中国实现这个可能性,实现这个前途的,在中国是国民党内的反人民集团,在外国是那些怀抱帝国主义思想的反动分子。这是一方面,这是必须注意与不可不注意的一方面。\\
  但是,另一方面,同样是从整个形势看来,从上述一切内外情况的分析看来,使我们更有信心地与更有勇气地去把握第二个可能性,第二个前途。这就是克服一切困难,团结全国人民,废止独裁统治,实行民主改革,巩固与扩大抗日力量,彻底打败日本侵略者,将中国建设成为一个独立、自由、民主、统一与富强的新国家。希望中国实现这个可能性,实现这个前途的,在中国是广大人民,中国共产党及其他民主分子与民主派别,在外国是一切以平等地位待我之民族,外国的进步分子,外国的人民大众。\\
  我们清楚地懂得,在我们和中国人民面前,还有很大的困难,还有很多的障碍物,还要走很多的迂回路程。但是我们同样地懂得,任何困难与障碍物,我们和中国人民一定能够克服,而使中国的历史任务获得完成。竭尽全力地去反对第一个可能性,争取第二个可能性,反对第一个前途,争取第二个前途,是我们和中国人民的伟大任务。国际国内的主要条件,主要方面,是有利于我们与中国人民的。所有这些,我在前面已经说得很清楚了。我们希望国民党当局,鉴于世界大势之所趋,中国人心之所向,毅然改变其错误的现行政策,使抗日战争获得胜利,使中国人民少受痛苦,使新中国早日诞生。须知不论怎样迂回曲折,中国人民独立解放的任务总是要完成的,而且这种时机已经到来了。一百多年来无数先烈所怀抱的宏大志愿,一定要由我们这一代人去实现,谁要阻止,到底是阻止不了的。\\
\subsection*{\myformat{四 中国共产党的政策}}
上面,我已将中国抗日战争中的两条路线,给了一个分析。这样的一个分析,在我看来是完全必要的。因为在广大的中国人中间,至今还有很多人不明白中国抗日战争中的具体情况,在国民党统治区及在国外,由于国民党政府的封锁政策,很多人被蒙住了眼睛。在一九四四年中外新闻记者考察团及美军观察组来到中国解放区以前,那里的许多人对于解放区几乎是什么也不知道的。一九四五年一月二十八日“纽约时报”说“解决中国共产党问题的最好办法,莫如任令人们往来于两个区域之间,许多误会就会消失。”可是国民党政府非常害怕,一九四四年的一次新闻记者团回去之后,立即将大门堵上,不许一个新闻记者再来解放区。对于国民党区域的真相,国民党政府也是同样的加以封锁。因此,我感到我们有责任将“两个区域”的真相尽可 能使人们弄清楚。只有在弄清中国的全部情况之后,才有可能了解中国的两个最大政党——中国共产党与中国国民党的政策何以有这样的不同,何以有这样的两条路线之争。只有这样,才会使人们了解,两党的争论,不是如有些人们所说不过是一些不必要的,不重要的,或甚至是意气用事的争论,而是关系着几万万人民生死问题的原则的争论。\\
  在目前中国时局的严重形势下,中国人民,中国一切民主党派及民主份子,一切关心中国时局的外国人民以及许多同盟国的政府,都希望中国的分裂局面重趋于团结,都希望中国能实行民主改革,都愿意知道中国共产党对于解决当前许多重大问题上所持的政策。我们的党员对于这些,当然更加关心。\\
  我们的抗日民族统一战线的政策历来是明确的,八年战争中考验了这些政策。我们的大会应该对此作出结论,作为今后奋斗的指针。\\
  下面,我就来说明我党在为解决中国问题而得出的关于重要政策方面的若干确定的结论。\\
\subsubsection*{\myformat{我们的一般纲领}}
为着动员与统一中国人民一切抗日力量,彻底消灭日本侵略者,并建立独立自由、民主、统一与富强的新中国,中国人民、中国共产党及一切抗日民主党派迫切地需要一个互相同意的共同纲领。\\
  这种共同纲领,可分为一般性的与具体性的两部分,我们先来说一般性的纲领,然后再说到具体性的纲领。\\
  在彻底消灭日本侵略者与建设新中国的大前提之下,在中国现阶段上,我们共产党人在这样一个基本点上是和中国人口中最广大的成分相一致的,就是说:第一,中国既不应该是一个由大地主大资产阶级专政的、封建的、法西斯的、反人民的国家制度,因为这种反人民的制度,已由国民党主要统治集团在其十八年的统治中表现出完全破产了。第二,中国也不可能、因此就不应该企图建立一个纯粹自由资产阶级旧式民主主义专政的国家。因为在中国,一方面,自由资产阶级在经济上与政治上至今还表现得很软弱;另一方面,中国早已产生了一个觉悟了的,在中国政治舞台上表现了强大能力的,领导了广大的农民阶级、小资产阶级、知识分子及其他民主分子的中国无产阶级及其领袖——中国共产党这样的新条件。第三,在中国的现阶段上,在中国人民的任务还是反对民族压迫与封建压迫,在中国社会经济的必要条件还不具备时,中国人民也不可能、因此就不应该企图实现社会主义的国家制度。\\
  那么,我们的主张是什么呢? 我们主张在彻底消灭日本侵略者之后,建立一个以全国绝对大多数人民为基础的统一战线的民主联盟的国家制度,我们把这样的国家制度称之为新民主主义的国家制度。\\
  这是一个真正适合中国人口中最广大成分的要求的国家制度。因为第一,它取得了与可能取得数百万产业工人,数千万手工工人与雇佣农民的同意;其次,也取得了与可能取得占中国人口百分之八十,即在四万万五千万人口中占了三万万六千万的农民阶级的同意;又其次,也取得了与可能取得广大的小资产阶级,自由资产阶级,开明绅士,及其他爱国分子的同意。\\
  自然,这些阶级之间也有其不同的要求,在这一点上,它们之间仍然是存在着矛盾的,例如劳资之间的矛盾。抹杀这种不同要求,抹杀这种矛盾,是虚伪的与错误的。但是,这种不同要求,这种矛盾,在整个新民主主义制度的阶段上,不会也不应该使之发展到超过共同要求之上。这种不同要求与这种矛盾,可以获得调节。在这种调节下,可以共同完成新民主主义国家的政治、经济与文化的各项建设。\\
  我们主张的新民主主义的政治,就是推翻外来的民族压迫,废止国内的封建主义的与法西斯主义的压迫,并且主张不在推翻与废止这些之后又建立一个旧民主主义的政治制度,而是建立一个联合一切民主阶级的统一战线的政治制度,我们的这种主张,是和孙中山先生的主张完全一致的。孙先生在其所著国民党第一次代表大会的宣言里说:“近世各国所谓民权制度,往往为资产阶级所专有,适成为压迫平民之工具。盖国民党之民权主义,则为一般平民所共有,非少数人所得而私也。”这是孙先生的伟大政治指示,中国人民,中国共产党及其他一切民主分子,必须服从这个指示而坚决实行之,并和一切违背与敌对这一指示的任何人们与任何集团作坚决的斗争,借以保护与发扬这个完全正确的新民主主义的政治原则。\\
  新民主主义的政权构成,应该采取民主集中制,由各级人民代表大会决定大政方针,选举政府。它是民主的,又是集中的,就是说,在民主基础上的集中,在集中指导下的民主。只有这个制度,既能表现广泛的民主,使各级人民代表大会有最高的权力,又能集中处理国事,使各级政府能集中地处理被各级人民代表大会所委托的一切事务,并保障人民的一切必要的民主活动。\\
  在新民主主义的国家问题与政权问题上,包含着联邦的问题。中国境内各民族,应根据自愿与民主的原则,组织中华民主共和国联邦,并在这个联邦基础上组织联邦的中央政府。\\
  军队及其他武装力量,是新民主主义的国家权力机关的重要部分,没有它们,就不能保卫国家。新民主主义的一切武装力量,如同其他权力机关一样,是属于人民与保护人民的,它们和一切属于少数人与压迫人民的旧式军队、旧式警察等等完全不同。\\
  我们主张的新民主主义的经济,也是符合于孙先生的原则的。在土地问题上,孙先生主张“耕者有其田”。在工商业问题上,孙先生在上述宣言里这样说:“凡本国人及外国人之企业,或有独占性质,或规模过大为私人之力所不能办者,如银行、铁路、航路之属,由国家经营管理之,使私有资本制度不能操纵国民之生计,此即节制资本之要旨也。”在现阶段上,对于经济问题,我们完全同意孙先生的这些主张。\\
  有些人们怀疑中国共产党人不赞成发展个性,不赞成发展私人资本主义,不赞成保护私有财产,其实都是过虑。民族压迫与封建压迫残酷地束缚着中国人民的个性发展,束缚着私人资本主义的发展与破坏着广大人民的财产。我们主张的新民主主义制度的任务,则正是解除这些束缚与停止这种破坏,保障广大人民能够自由发展其在共同生活中的个性,能够自由发展那些不是“操纵国民生计”,而是有益于国民生计的私人资本主义经济,保障一切正当的私有财产。\\
  按照孙先生的原则与中国革命的经验,在现阶段上,中国的经济,必须是由国家经营,私人经营与合作社经营三者组成的。而这个国家经营的所谓国家,一定不应该是“少数人所得而私”的国家,而一定要是“为一般平民所共有”的新民主主义的国家。\\
  新民主主义的文化,同样应该是“为一般平民所共有”的,即是说,民族的、科学的、大众的文化,决不应该是“少数人所得而私”的文化。\\
  上述一切,就是我们共产党人在现阶段上,在整个资产阶级民主革命的阶段上所主张的一般纲领,或基本纲领。对于我们的社会主义与共产主义制度的将来纲领,或最高纲领说来,这是我们的最低纲领。实行这个纲领,可以把中国从现在的国家性质与社会性质上向前推进一步,即是说,从殖民地半殖民地与半封建的国家性质与社会性质,推进到新式资产阶级民主主义的国家性质与社会性质,在彻底消灭日本侵略者以后,可以建立一个具有这种性质的独立、自由、民主、统一与富强的国家。\\
  实行这个纲领,还没有把中国推进到社会主义。这不是一个由于什么人在主观上想做或不想做这种推进的问题,而是一个由于在客观上中国的政治条件与社会条件不许可人们这样做的问题。\\
  我们共产党人从来不隐瞒自己的政治主张,我们的将来纲领,或最高纲领,是要将中国推进到社会主义与共产主义的,这是确定的与毫无疑义的。我们的党的名称和我们的马克思主义的宇宙观,明确地指明了这个将来的,无限光明的,无限美妙的与最高理想的方向。我们每个人入党之时,心目中就悬着为现在的新式资产阶级民主主义革命而奋斗与为将来的无产阶级社会主义革命而奋斗这样两个明确的目标,而不顾那些共产主义敌人们的无知的与卑劣的敌视、诬蔑、谩骂或讥笑,对于这些,我们必须给以坚决的排击,对于那些善意的怀疑者,则不是排击而是给以善意的与耐心的解释,所有这些,都是异常清楚、异常确定与毫不含糊的。\\
  但是,一切中国共产党人,一切中国共产主义的同情者,必须为着现时阶段的目标而奋斗,为着反对民族压迫与封建压迫,为着使中国人民脱离殖民地、半殖民地、半封建的悲惨命运与建立新式资产阶级民主主义性质的,新民主主义性质的,亦即孙中山先生革命三民主义性质的独立、自由、民主、统一与富强的中国而奋斗。我们果然是这样做了,我们共产党人,协同广大的中国人民,为此而英勇奋斗了二十四年。\\
  对于任何一个共产党人及其同情者,如果不为这个目标而奋斗,如果看不起这个资产阶级民主革命而对它稍许放松,稍许怠工,稍许表现其不忠诚,不热情,不准备付出自己的鲜血与生命,而空谈什么社会主义与共产主义,那就是无意地,或有意地,或多或少地背叛社会主义与共产主义,就不是一个自觉的与忠诚的共产主义者。只有经过民主主义,才能到达社会主义,这是马克思主义的天经地义。而在中国,为民主主义而奋斗的时间还是长期的,没有一个新民主主义的联合统一的国家,没有新民主主义的国家经济的发展,没有广大的私人资本主义经济与合作社经济的发展,没有民族的科学的大众的文化即新民主主义文化的发展,没有几万万人民的个性解放与个性发展,一句话,没有一个新式资产阶级性质的彻底的民主革命,要想在殖民地半殖民地半封建的废墟上建立起社会主义来,那只是完全的空想。\\
  有些人不了解共产党人为什么不但不怕资本主义,反而提倡它的发展。我们的回答是这样简单:拿发展资本主义去代替外国帝国主义与本国封建主义的压迫,不但是一个进步,而且是一个不可避免的过程,它不但有利于资产阶级,同时也有利于无产阶级。现在的中国是多了一个外国的帝国主义与一个本国的封建主义,而不是多了一个本国的资本主义,相反地,我们的资本主义是太少了。说也奇怪,有些中国资产阶级代言人不敢正面提出发展资本主义的主张,而要转湾抹角地来说这个问题。另外有些人,则甚至一口否认中国应该让资本主义有一个广大的发展,而说什么一下发展到社会主义,什么要将三民主义与社会主义“毕其功于一役”。很明显地,这类现象,有些是反映着中国自由资产阶级的软弱性,有些则是反映大地主大资产阶级对于民众的欺骗手段。我们共产党人根据自己对于马克思主义的社会发展规律的认识,明确地知道,在中国的条件下、在新民主主义的国家统治下,除了国家自己的经济与劳动人民的个体经济及合作社经济之外,定要让私人资本主义经济获得广大发展的便利,才能有益于国家与人民,有益于社会的向前发展。对于中国共产党人,任何的空谈与欺骗,是不会让它迷惑我们的清醒头脑的。\\
  有些人怀疑共产党人“承认三民主义为中国今日之必需,本党愿为其彻底实现而奋斗”,似乎不是忠诚的。这是由于不了解我们所承认的孙中山先生在一九二四年国民党第一次代表大会宣言里所解释的三民主义的基本原则,和我党在现阶段的纲领即最低纲领里的若干基本原则,是互相一致的。应当指出,孙先生的这种三民主义,和我党在现阶段上的纲领,只是在若干基本原则上是一致的东西,并不是完全一致的东西。我党的新民主主义的纲领,比之孙先生的,当然要完备得多,特别是孙先生死后这二十年中中国革命的发展,使我党新民主主义的理论、纲领及其实践,有了一个极大的发展,今后还将有更大的发展。但是,孙先生的这种三民主义,按其基本性质说来,是一个新民主主义的三民主义,是一个和在此以前的旧三民主义相区别的新三民主义,当然这是“中国今日之必需”,当然“本党愿为其彻底实现而奋斗”。对于中国共产党人,为本党的最低纲领而奋斗与为孙先生的革命三民主义即新三民主义而奋斗,在其基本上(不是在一切上)是一件事情,并不是两件事情。因此,不但在过去,在现在,已经证明,而且在将来还要证明:中国共产党人是实现三民主义的最忠诚与最彻底者。\\
  有些人怀疑共产党得势之后,是否会学俄国那样,来一个无产阶级专政及一党制度?我们的答复是:几个民主阶级联盟的新民主主义国家,和无产阶级专政的社会主义国家,是原则上不同的。中国在整个新民主主义制度期间,不可能因此就不应该是一个阶级专政与一党独占政府机构的制度,只要共产党以外的其它任何政党,任何社会集团或个人,对于共产党是采取合作的而不是采取敌对的态度,我们是没有理由不和他们合作的。俄国的历史形成了俄国的制度,在那里,废除了人剥削人的社会制度,实现了最新式民主主义即社会主义的政治、经济、文化制度,一切反对社会主义的政党都被人民抛弃了,人民仅仅拥护布尔塞维克,因此形成了俄国的局面,这在他们是完全必要与完全合理的。但是在俄国的政权机关中,即使是处在除了布尔塞维克以外没有其它政党的条件下,实行的还是工人、农民与知识分子联盟,或党与非党联盟的制度,也不是只有工人阶级或只有布尔塞维克党人才可以在政权机关中工作。中国的历史将形成中国的制度,在一个长时期中,将产生一个对于我们是完全必要与完全合理同时又区别于俄国制度的特殊形态,即几个民主阶级联盟的新民主主义的国家形态与政权形态。\\
  在这里,也就同时回答了另一个问题。就是说,你们共产党人现在主张建立联合政府,是因为现在还没有民主选举制度,为着团结抗日,一个联合政府是需要的;但是在将来,在有了民主选举制度之后,为什么不由国民大会中的多数党组织一党政府,还要组织联合政府呢?我们回答说:中国的历史条件规定了这一点。我在前面已经说过,在中国,因为早就出现了不但代表无产阶级,而且按其纲领和实际斗争,同时也代表了最广大的农民阶级、小资产阶级、知识分子及其他民主份子的中国共产党这样一个新条件,一切情况就改变了。任何政府,如果把共产党排斥到门外,那是一件好事也做不成的,这就是中国进到了新民主主义的历史阶段的基本特点。还是孙中山先生在世时,在一九二四年,他就在实际上实现了一个有共产党参加的联合政府,从而他就做成了一件伟大的革命事业。一九二七年以后的国民党政府排斥共产党,并且举行了残酷的反共战争,这个政府所做的就变成了反革命事业了。八年抗战,虽然国民党政府直到今天,因为有一个日本侵略者站在前面,还没有向共产党公开宣布全面战争,还只用局部战争、特务镇压、封锁、诬蔑、准备内战及不许组织联合政府等方法排斥共产党,但是它已经给自己造成了这样的形势:越排斥共产党,就越走下坡路。如果今后还要排斥下去,那就是准备将下坡路走到底。在中国的条件下,一个政府里不要共产党,这是什么意思呢? 这就是说,不要最广大的人民。没有人会怀疑到共产党人是为着争夺官位而要求到政府里去的,共产党人如果加入政府,这就意味着实行新民主主义的改革。中国将来有了民主选举制度以后,不论共产党是国民大会中的多数党或是少数党,政府都应该是在一个共同承认的新民主主义的纲领之下从事工作的联合政府,才有利于更好地完成新民主主义的建设事业。这一点,现在已经可以看得很清楚了。\\
\subsubsection*{\myformat{我们的具体纲领}}
根据上述一般纲领,在各个时期中应有其具体的纲领。在整个资产阶级民主革命阶段中,在几十年中,我们的新民主主义的一般纲领是不变的。但是在这个大阶段的各个小阶段中,情形是变化了与变化着的,我们的具体纲领便不能不有所改变,这是当然的事情。例如在北伐战争时期、在土地革命时期与在抗日战争时期,我们的新民主主义的一般纲领并没有变化,但其具体纲领,三个时期中是有了变化的,这是因为我们的敌军与友军在三个时期中发生了变化的原故。\\
  目前中国人民是处在这样的情况中:(一)日本侵略者还未被打败;(二)中国人民迫切地需要团结起来,实现一个民主的改革,以便造成民族团结,迅速地动员与统一一切抗日力量,配合同盟国打败日本侵略者;(三)国民党政府分裂民族团结,阻碍这种民主的改革。在这些情况下,我们的具体纲领即中国人民的现时要求是什么呢?\\
  我们认为下面这些是适当的,并且是最低限度的。\\
  中国人民要求动员一切力量,配合同盟国,彻底消灭日本侵略者,并建立国际和平;要求废止国民党一党专政,建立民主的联合政府与联合统帅部;要求惩办那些分裂民族团结与反对人民的亲日分子,法西斯主义分子与失败主义分子,造成民族团结;要求惩办那些制造内战危机的反动分子,保障国内和平;要求惩办汉奸,讨伐降敌军官,惩办日本间谍;要求取消一切镇压人民的反动的特务机关与特务活动,取消集中营;要求取消一切镇压人民的言论、出版、集会、结社思想、信仰及身体等项自由的反动法令,使人民获得充分的自由权利;要求承认一切民主党派的合法地位;要求释放一切爱国政治犯;要求撤退一切包围与进攻中国解放区的军队,并将这些军队使用于抗日前线上去;要求承认中国解放区的一切抗日军队与民选政府;要求巩固与扩大解放区及其军队,缩小沦陷区;要求帮助沦陷区人民组织地下军,准备武装起义;要求允许中国人民自动武装起来,保乡卫国;要求从政治上军事上改造那些由国民党统帅部直接领导的经常打败仗,经常压迫人民与经常排斥异己的军队,惩办那些应对溃败负责的将领;要求改善兵役制度与改善官兵生活;要求优待抗日军人家属,使前线官兵安心作战;要求优待殉国战士的遗族,优待残废军人,对于退伍军人的生活与就业,应予帮助;要求发展军事工业,以利作战;要求将同盟国的武器与财政援助公平地分配于抗战各军;要求惩办贪官污吏,实现廉洁政治;要求改善中下级公务员的待遇;要求给予中国人民以民主的自治权利;要求取消压迫人民的保甲制度;要求救济难民与救济灾荒;要求设立大量的救济基金,在国土收复后,广泛地救济沦陷区受难的人民;要求取消苛捐杂税,实行统一的累进税;要求实行农村改革,减租减息,适当地保证佃权,对贫苦农民给予低利贷款,并使农民组织起来,以利于发展农业生产;要求取缔官僚资本;要求废止现行的经济统制政策;要求制止无限制的通货膨胀与无限制的物价高涨;要求扶助民间工业,给予民间工业以借贷资本、购买原料与推销产品的便利;要求改善工人生活,救济失业工人,并使工人组织起来,以利于发展工业生产;要求取消党化教育,发展民族的科学的大众的文化教育;要求保障教职员生活及学术自由;要求保护青年、妇女、儿童的利益,救济失学青年,并使青年、妇女组织起来,以平等地位参加有益于抗日战争与社会事业的各项工作,实现婚姻自由,男女平等,实现对于青年与儿童是有益的学习;要求改善国内少数民族的待遇,允许各少数民族有民族自决权及在自愿原则下和汉族联合建立联邦国家的权利;要求保护华侨利益,扶助回国的华侨;要求保护因被日本侵略者压迫而逃来中国的外国人民,并扶助其反对日本侵略者的斗争;要求改善中苏邦交等等。而要做到这一切,最重要的是要求立即取消国民党一党专政,建立一个包括一切抗日党派及无党无派的代表人物在内的举国一致的民主的联合的临时的中央政府。没有这个前提条件,要想在全国范围内,就是说,在国民党统治区域进行稍为认真的改革,是不可能的。\\
  只有这些,才是中国广大人民的呼声,也是同盟各国广大舆论界的呼声。\\
  一个为各个抗日民主党派互相同意的最低限度的具体纲领,是完全必要的,我们准备拿上述纲领为基础和他们进行协商。各党可以有不同的要求,但是各党之间应该协议一个共同的要求。\\
  这样的纲领,对于国民党统治区,暂时还是一个要求的纲领;对于沦陷区,除组织地下军准备武装起义一项外,是一个等到收复后才能实施的纲领;对于解放区,则是一个早已实施并正在继续实施的纲领。\\
\subsubsection*{\myformat{彻底消灭日本侵略者 不许中途妥协}}
在上述中国人民的目前要求,或具体纲领中,包含着许多战时战后的重大问题,需要在下面加以说明。在说明这些问题时,我们将批评国民党主要统治集团的一些错误观点,同时也将回答其他人们的一些疑问。\\
  第一,彻底消灭日本侵略者,不许中途妥协。\\
  开罗会议决定应使日本侵略者无条件投降,这是正确的。但是,现在日本侵略者正在暗地里进行活动,企图获得妥协的和平,国民党政府中的亲日分子,经过南京傀儡政府,亦正在和日本密使勾勾搭搭,并未遇到制止,因此,中途妥协的危险并未完全过去。开罗会议又决定将东北四省、台湾、澎湖群岛归还中国,这是很好的。但是根据国民党政府的现行政策,要想依靠它去打到鸭绿江边,收复一切失地,是不可能的。在此种情形下,中国人民应该怎么办呢? 中国人民应该要求国民党政府彻底消灭日本侵略者,不许中途妥协。一切妥协的阴谋活动必须立刻制止。中国人民应该要求国民党政府改变现在的消极的抗日政策,将其一切军事力量用于积极作战。中国人民应该扩大自己的军队——八路军、新四军及其他人民军队,并在一切敌人所到之处,广泛地自动地发展抗日武装,准备直接配合同盟国作战,收复一切失地,决不要单纯依靠国民党。消灭日本侵略者是中国人民的神圣权利。任何反动分子,要想剥夺中国人民的这种神圣权利,要想压制中国人民的抗日活动,要想破坏中国人民的抗日力量,中国人民在其劝说无效之后,应该站在自卫立场上给以坚决的回击。因为中国反动分子的这种背叛民族利益的反动行为,完全是帮助日本侵略者的。\\
\subsubsection*{\myformat{废止一党专政,建立联合政府}}
第二,废止国民党一党专政,建立民主的联合政府。\\
  为着彻底消灭日本侵略者,必须在全国范围内实行民主改革。而要这样做,不废止国民党一党专政,建立民主的联合政府,是不可能的。\\
  所谓国民党一党专政,实际上是国民党内反人民集团的专政,它是中国民族团结的破坏者,是抗日失败的负责者,是动员与统一中国人民抗日力量的根本障碍物,八年抗日战争中的惨痛经验,中国人民已经深刻地认识了这一点,很自然地要求立即废止它。这个反人民的专政,又是内战的祸胎,如不立即废止,内战惨祸又将降临。\\
  中国人民要求废止这个反人民专政的呼声是如此普遍而响亮了,使得国民党当局自己也不能不公开承认“提早结束训政”,可见这个所谓“训政”,或“一党专政”之丧失人心,威信扫地,到了何种地步了。在中国,已经没有一个人还敢说“训政”或“一党专政”有什么好处,不应该废止或结束了,这是当前时局的一大变化。\\
  应该结束是确定了,毫无疑义了。但是如何结束呢?可就意见纷歧了。一个说:立即结束,成立民主的临时的联合政府。一个说:等一会再结束,召开“国民大会”,“还政于民”,却不能还政于联合政府。\\
  这是什么意思呢?\\
  这是两种做法的表现:真做与假做。\\
  第一种,真做。这就是立即宣布废止国民党一党专政,成立一个由国民党、共产党、民主同盟及无党无派分子的代表人物联合组成的临时的中央政府,发布一个民主的施政纲领,如同我们在前面提出的那些“中国人民的现时要求”以便恢复民族团结,打败日本侵略者。为着讨论这些事情,召集一个各党各派及无党无派的代表人物的圆桌会议,成立协议,动手去做。这是一个团结的方针,中国人民是坚决拥护这个方针的。\\
  第二种,假做。不顾广大人民及一切民主党派的要求,一意孤行地召开一个由国民党反人民集团一手包办的所谓“国民大会”,在这个会上通过一个实际上维持独裁反对民主的所谓“宪法”,使那个仅仅由几十个国民党人私自委任的完全没有民意基础的强安在人民头上的不合法的所谓“国民”政府,披上“合法”的外衣,装模作样地“还政于民”,实际上,依然是“还政”于国民党内的反人民集团。谁要不赞成,就说他是破坏“民主”,破坏“统一”,就有“理由”向他宣布讨伐令。这是一个分裂的方针,中国人民是坚决反对这个方针的。\\
  我们的反人民的英雄们根据这种分裂方针所准备采取的步骤,有把他们自己推到绝路上去的危险性。他们准备把一条绳索套在自己的脖子上,并且让它永远也解不开,这条绳索的名称就叫做“国民大会”。他们的原意是想把所谓国民大会当作法宝,祭起来。一则抵制联合政府,二则维持独裁统治,三则准备内战理由的,可是历史的逻辑将向他们所设想的反面走去:“搬起石头打自己的脚”。因为现在谁也明白,在国民党统治区域,人民没有自由,在日本人占领区域,人民不能参加选举,在有了自由的中国解放区,国民党政府又不承认它,在此种情况下,那里来的国民代表?那里来的国民大会? 现在叫着要开的,是那个还在内战时期,还在八年以前,由国民党独裁政府一手伪造的所谓国民大会,如果这个会开成了,势必闹到全国人民群起反对,请问我们的反人民英雄们如何下台?归根结底,伪造国民大会如果开成了,不过将他们自己推到绝路上,造成分崩离析的局面。\\
  我们共产党人是不想中国走上这个局面的,所以提出解救中国的两个步骤:第一个,目前时期,经过各党各派及无党无派代表人物的协议,成立临时的联合政府;第二个,将来时期,经过自由的无拘束的选举,召开国民大会,成立正式的联合政府。总之,都是联合政府,团结一切愿意参加的阶级与政党的代表在一起,在一个民主的共同纲领之下,为现在的抗日与将来的建国而奋斗。\\
  不管国民党人或任何其他党派、集团及个人如何设想,愿意或不愿意,自觉或不自觉,中国只能走这条路。这是一个必然的、不可避免的历史法则,任何力量,都是扭转不过来的。\\
  在这个问题及其他任何有关民主改革的问题上,我们共产党人声言:不管国民党当局现在还是怎样坚持其错误政策与怎样借谈判为拖延时间、搪塞舆论的手段,只要他们一旦愿意放弃其错误的现行政策,同意民主改革,我们是愿意和他们恢复谈判的。但是谈判的基础必须放在抗日、团结与民主的总方针上,一切离开这个总方针的所谓办法、方案,或其他空话,不管它怎样说得好听,我们是不能赞成的。\\
\subsubsection*{\myformat{人民的自由}}
第三,人民的自由。\\
  目前中国人民争自由的目标,首先地与主要地是向着日本侵略者。但是国民党政府剥夺人民的自由,捆起人民的手足,使他们不能反对日本侵略者,不解决这个问题,就不能在全国范围内动员与统一一切抗日力量。我们在纲领中提出了废止一党专政,成立联合政府,取消特务,取消镇压自由的法令,惩办汉奸、间谍、亲日分子、法西斯分子与贪官污吏,释放政治犯,承认各民主党派的合法地位,撤退包围与进攻解放区的军队,承认解放区,废止保甲制度,以及其他许多经济的文化的与民众运动的要求,就是为着解开套在人民身上的绳索,使人民获得抗日、团结与民主的自由。\\
  自由是人民争来的,不是什么人恩赐的。中国解放区的人民已经争得了自由。其他地方的人民也可能与应该争得这种自由。中国人民争得的自由越多,有组织的民主力量越大,一个统一的临时的联合政府便越有成立之可能。这种联合政府一经成立,它将转过来给予人民以充分的自由,巩固联合政府的基础。然后才有可能,在日本侵略者被消灭之后,在全部国土上进行自由的无拘束的选举,产生民主的国民大会,成立统一的正式的联合政府。没有人民的自由,就没有真正民选的国民大会,就没有真正民选的政府。难道还不清楚么?\\
  人民的言论、出版、集会、结社、思想、信仰与身体这几项自由,是最重要的自由。在中国境内,只有解放区是彻底地实现了。\\
  一九二五年,孙中山先生在其临终的遗嘱上说:“余致力国民革命凡四十年,其目的在求中国之自由平等。积四十年之经验,深知欲达到此目的,必须唤起民众及联合世界上以平等待我之民族,共同奋斗。”背叛孙先生的不肖子孙,不是唤起民众,而是压迫民众,将民众的言论、出版、集会、结社、思想、信仰与身体等项自由权利剥夺干净。对于认真唤起民众、认真保护民众自由权利的共产党、八路军、新四军及解放区,则称之为“奸党”、“奸军”、“奸区”。我们希望这种颠倒是非的时代快要过去了。再要延长这种颠倒,中国人民将不能忍耐了。\\
\subsubsection*{\myformat{人民的统一}}
第四,人民的统一。\\
  为着消灭日本侵略者,为着防止内战,为着建设新中国,必须将分裂的中国变为统一的中国,这是中国人民的历史任务。\\
  但是如何统一呢?独裁者的专制的统一,还是人民的民主的统一? 从袁世凯以来,北洋军阀强调专制统一。但是结果怎么样呢?和这些军阀的志愿相反,所得的不是统一而是分裂,最后是自己从台上滚下去。国民党反人民集团抄袭袁世凯老路,追求专制的统一,打了整十年内战,结果把一个日本侵略者打了进来,自己也缩上了峨嵋山。现在又在山上大叫其专制统一论,这是叫给谁听呢?难道还有什么爱国的有良心的中国人愿听么? 经过了十六年的北洋军阀统治,又经过了十八年的国民党独裁统治,人民有了充分的经验,有了明亮的眼睛,他们要一个人民大众的民主的统一,不要独裁者的专制的统一。我们共产党人还在一九三五年就提出抗日民族统一战线的方针,没有一天不为此而奋斗。一九三九年国民党推行其反动的“限制异党活动办法”,发生投降、分裂、倒退的危机,国民党人大叫其专制统一论时,我们又说:“非统一于投降而统一于抗战,非统一于分裂而统一于团结,非统一于倒退而统一于进步,只有这后一种统一才是真统一,其它一切都是假统一。”又过了六年了,问题还是一样。\\
  没有人民的自由,没有人民的民主政治,能够统一么? 有了这些,立刻就统一了。中国人民争自由,争民主,争联合政府的运动,就是争统一的运动。我们在具体纲领中提出了许多争自由争民主的要求,提出了联合政府的要求,就是为了这个目的。不废止国民党内反人民集团的专政,成立民主的联合政府,不但在国民党统治区不能实行任何民主的改革,不能动员那里的全体军民有力地配合同盟国彻底消灭日本侵略者,而且还将发展为内战的惨祸,这是很多人都明白的常识了。为什么如此众多的有党有派无党无派的民主分子,包括国民党内的许多民主分子在内,一致要求联合政府,就是看清楚了时局的危机,非如此不能克服这种危机,不能达到团结对敌与团结建国之目的。\\
\subsubsection*{\myformat{人民的军队}}
第五,人民的军队。\\
  中国人民要自由,要统一,要联合政府,要彻底消灭日本侵略者与建设新中国,没有一枝站在人民立场上的军队,那是不行的。彻底站在人民立场的军队,现在还只有一个八路军、新四军,还很不够,可是国民党内的反人民集团却处心积虑地要破坏与消灭这枝军队。一九四四年,国民党政府提出了一个所谓“提示案”,叫共产党“限期取消”解放区军队的五分之四。一九四五年,即最近的一次谈判,又叫共产党将解放区军队全部交给它,然后它给共产党以“合法地位”。\\
  这些人们向共产党人说:你交出军队,我给你自由。根据这个学说,没有军队的党派该有自由了。但是一九二四至一九二七年,中国共产党只有很少一点军队,国民党政府的“清党”政策与屠杀政策一来,自由也光了。现在的中国民主同盟,中国国民党民主派并没有军队,同时也没有自由。十八年中,在国民党政府统治下的工人、农民、学生及一切要求进步的文化界、教育界、产业界,他们也一概没有自由。难道是由于他们组织了一枝什么军队,实行了什么“封建割据”,成立了什么“奸区”,违反了什么“政令军令”,因此才不给自由的么?否,适得其反,正是因为他们没有这样做。\\
  “军队是国家的”,非常之正确,世界上没有一个军队不是属于国家的。但是什么国家呢? 大地主,大银行家、大买办的封建法西斯独裁的国家? 还是人民大众的新民主主义的国家?中国只应该建立新民主主义的国家,并在这个国家基础之上建立新民主主义的联合政府,中国的一切军队均应属于这个国家的这个政府,借以保障人民的自由,有效地反对外国侵略者。什么时候中国有一个新民主主义的联合政府与联合统帅部出现了,中国解放区的军队将立即交给它。但是一切国民党的军队也必须同时交给它。\\
  为着打败日本侵略者与建设新中国,为着中国的自由与统一,为着防止内战,保障国内和平,中国人民必须做一项责无旁贷的工作,这就是将国民党政府的那些在对日作战时经常打败仗的,以反对人民,排斥异己与准备内战为目的的军队,加以改造,变为人民的军队。国民党军队的士兵与广大数量的军官原来是好的,他们对抗日原来是积极的,曾经英勇作战过。他们也并不愿意反对共产党、八路军、新四军与解放区。仅因国民党统帅机关及腐败将领违反孙中山先生在世时所手创的民主传统,实行自己的失败主义与法西斯主义的领导,反动的政治工作与反动的特务网,迫使这个军队站在反对人民的立场上;并使这个军队处于严重情况之中,内而官兵关系,外而军民关系,都极端恶化,战斗力薄弱,生活痛苦,兵役制度百弊丛生,又不许农村人民武装起来保乡卫国。一切这些,均须改革,决不许可再这样长期继续下去。这种改革,不但是中国人民的要求,也是同盟国广大舆论界的要求,也是国民党军队内部全体士兵与广大军官们的要求。\\
  一九二四年,孙中山先生说:“今日以后,当划一国民革命之新时代。……第一步,使武力与国民相结合。第二步,使武力为国民之武力。”八路军、新四军遵循这个方针,成了“国民之武力”,就是说,成了人民的军队,所以能打胜仗。国民党军队在北伐战争前期,做到了孙先生所说的“第一步”,所以打了胜仗。从北伐后期直至现在,连“第一步”也丢了,站在反人民的立场上,所以一天一天腐败堕落,除了“内战内行”之外,对于“外战”,就不能不是一个“外行”。国民党军队中一切爱国的有良心的军官们,应该起来恢复孙先生的精神,改造自己的军队。\\
  在改造旧军队的工作中,对于一切可以教育的军官,应当给予适当的教育,帮助他们学得正确观点,清除陈旧观点,为人民的军队而继续服务。\\
  为创造中国人民的军队而奋斗,是全国人民及一切民主党派的责任。没有一个人民的军队,便没有人民的一切。对于这个问题,切不可只发空论。\\
  我们共产党人愿意赞助改革中国军队的事业。八路军、新四军对于一切愿意团结人民、反对日本侵略者、而不反对中国解放区的军队,均应看作自己的友军,给以适当的协助。\\
\subsubsection*{\myformat{土地}}
第六,土地问题。\\
  为着消灭日本侵略者与建设新中国,必须实行土地改革、解放农民。孙中山先生的“耕者有其田”的主张,是目前资产阶级民主主义性质的革命时代的正确主张。\\
  为什么把目前时代的革命叫做“资产阶级民主主义性质的革命”?这就是说,这个革命的对象不是一般的资产阶级,而是民族压迫与封建压迫;这个革命的一切设施,不是一般地废除私有财产,而是一般地保护私有财产;这个革命的结果,将为资本主义扫清道路而使之获得发展。“耕者有其田”,是把土地从封建剥削者手里转移到农民手里,变为农民的私有财产,使农民从封建的土地关系上获得解放,使农业从旧式的落后的水平进到近代化的水平,从而使工业获得市场,造成了将农业国转变为工业国的可能性。因此,“耕者有其田”的主张,是一种资产阶级民主主义性质的主张,并不是无产阶级社会主义性质的主张,是一切革命民主派的主张,并不单是我们共产党人的主张。所不同的,在中国条件下,只有我们共产党人把这项主张看得特别认真,不但口讲,而且实做。那些人们是革命民主派呢?除了无产阶级是最彻底的革命民主派之外,农民是最大的革命民主派。农民的绝对大多数,就是说,除开那些带上了封建尾巴的一部分富农之外,无不积极地要求“耕者有其田”。城市小资产阶级也是革命民主派,“耕者有其田”使农业生产力获得发展,对于他们是有利的。自由资产阶级是一个动摇的阶级,他们需要市场,他们也赞成“耕者有其田”,他们多半和土地联系着,他们中的许多人又惧怕“耕者有其田”。坚决反对“耕者有其田”的是国民党内的反人民集团,因为他们是代表大地主、大银行家、大买办阶层的。中国没有单独代表农民的政党。自由资产阶级的政党没有坚决的土地纲领。因此,只有具备了坚决的土地纲领,为农民利益而认真奋斗,因而获得最广大农民群众作为自己伟大同盟军的中国共产党人,成了农民与一切革命民主派的领导者。\\
  一九二七至一九三六年,中国共产党执行了孙先生“耕者有其田”的主张。出而张牙舞爪,进行了十年反人民战争,亦即反“耕者有其田”的战争的,就是那个集中了孙中山先生一切不肖子孙在内的团体——国民党内的反人民集团。\\
  抗战期间,中国共产党让了一大步,将“耕者有其田”的政策,改为减租减息政策。这个让步是正确的,推动了国民党参加抗日,又使解放区的地主与农民联合起来反对日本侵略者。这个政策,如果没有特殊阻碍,我们准备在战后继续下去,首先在全国实现减租减息,然后寻找适当方法,有步骤地达到“耕者有其田”。\\
  但是背叛孙先生的人们不但反对“耕者有其田”,连减租减息也反对。国民党政府自己颁布的“二五减租”一类法令,自己不实行,仅仅在中国解放区实行了,因此也就成立了罪状:名之日“奸区”。\\
  在抗战期间,出现了所谓民族阶段与民主民生阶段的两阶段论,这是错误的。\\
  大敌当前,民主民生不该提起,等日本人走了再提好了——这是国民党反人民集团的谬论,其目的是不愿抗日战争获得彻底胜利,有些人居然随声附和,和自觉地作了这种谬论的尾巴。\\
  大敌当前,不解决民主民生就不能赶走日本人——这是中国革命民主派的正论,为整个中国近代史,尤其是八年抗战史所证明,也为法兰西、意大利、波兰、南斯拉夫、保加利亚、罗马尼亚、匈牙利等国人民的反法西斯斗争所证明,在波兰等国所执行的,不是如我们的减租减息,而是“耕者有其田”。\\
  在抗日期间,一切服从抗日,减租减息及一切民主改革是为着抗日的,不是自己孤立起来的。为着团结一切社会阶层反对共同敌人,不取消地主的土地所有权,并适当地交租交息,奖励地主的资本向工业方面转移,同时,更使开明绅士和其他人民的代表一道参加社会工作与政府工作。对于富农,则奖励其发展生产。所有这些,是在坚决执行农村民主改革的路线里包含着的,是完全必要的。\\
  两条路线:或者坚决反对中国农民解决民主民生问题,而使自己腐败无能,无力抗日。或者坚决赞助中国农民解决民主民生问题,而使自己获得占全人口百分之八十的最伟大的同盟军,借以组织雄厚的战斗力量。前者就是国民党政府的路线,后者就是中国解放区的路线。\\
  动摇于两者之间,口称赞助农民,但不坚决实行减租减息,武装农民与建立农村民主政权,这是机会主义者的路线。\\
  国民党反人民集团动员一切力量,向着中国共产党放出了一切恶毒的箭:明的和暗的,军事的和政治的,流血的和不流血的。两党的争论,就其社会性质说来,实质上是在农村关系的问题上。我们究竟在那一点上触怒了国民党反人民集团呢?难道不正是在这个问题上面吗? 国民党反人民集团之所以受到日本侵略者的欢迎与鼓励,难道不正是在这个问题上面,给日本侵略者帮了大忙吗?所谓“共产党破坏抗战、危害国家”,所谓“奸党”、“奸军”、“奸区”,所谓“不服从政令、军令”,难道不正是因为中国共产党在这个问题上做了真正符合于民族利益的认真的事业吗?\\
  农民——这是中国工人的前身。将来还要有几千万农民进人城市,进入工厂。如果中国需要建设强大的民族工业,建设很多的近代式的大城市,就要有一个变农村人口为城市人口的长过程。\\
  农民——这是中国工业的市场。只有他们能够供给最丰富的粮食、原料与吸收最广大的工业品。\\
  农民——这是军队的来源。士兵就是穿起军服的农民,他们是日本侵略者的死敌。\\
  农民——这是现阶段中国民主政治的主要基础。中国的民主主义者如不依靠三万万六千万农民群众的援助,他们就将一事无成。\\
  农民——这是现阶段中国文化运动的主要基础。所谓扫除文盲,所谓普及教育,所谓大众文艺,所谓国民卫生,离开了三万万六千万农民,岂非大半成了空话?\\
  我说的“主要基础”,当然不是忽视其他约占人口九千万的人民在政治上经济上文化上的重要性,尤其不是忽视在中国人民中政治上最觉悟与具有领导一切民主运动资格的工人阶级,这是不应该发生误会的。\\
  认识这一切,不但中国共产党人,而且一切民主派,都是完全必要的。\\
  土地制度获得改革,甚至仅获得初步的改革,例如减租减息之后,农民的生产兴趣就增加了。然后帮助农民在自愿原则下,逐渐地组织在农业生产合作社及其它合作社之中,生产力就会发展起来。这种农业生产合作社,现时还只能是建立在农民个体经济基础上的(农民私有财产基础上的)集体的互助的劳动组织,例如变工队、互助组、换工班之类,但是生产力的发展与生产量的增加,已属惊人。这种制度,已成为中国解放区的普遍的制度,今后应当尽量推广。\\
  这里应当指出一点,就是说,变工队一类的合作组织,原来在农民中就有了的,但在那时,不过是农民悲惨生活的表现。现在中国解放区的变工队,形式与内容都起了变化,它是农民群众为着发展自己的生产力,争取富裕生活的表现。\\
  中国一切政党的政策及其实践在中国人民中所表现的作用的好、坏、大、小,归根到底,看其对于中国人民的生产力的发展是否有帮助及其帮助之大小。它是束缚生产力的? 还是解放生产力的? 彻底消灭日本侵略者,实行土地改革,解放农民,发展现代工业,建立独立、自由、民主、统一与富强的新中国,只有这一切,才能使中国社会生产力获得解放,才是中国人民所欢迎的。\\
  这里还要指出一点,就是说,从城市到农村工作的知识分子,不容易了解现时还是分散的落后的个体经济这种农村特点,在解放区,则还要加上暂时还是被敌人分割的与游击战争的这些特点。因为不了解这些特点,他们就往往不适当地带着他们在城市里生活或工作的观点去观察农村问题,去处理农村工作,因而脱离农村实际,不能和农民打成一片。这种现象,必须用教育方法加以克服。\\
  中国广大的革命知识分子应该觉悟到将自己和农民结合起来之必要。农民正需要他们,等待他们的援助。他们应该热情地跑到农村中去,脱下学生装,穿起粗布衣,不惜从任何小事情做起,在那里了解农民的要求,帮助农民觉悟起来,组织起来,为着完成中国民主革命中一项极重要工作即农村民主革命而奋斗。\\
  在日本侵略者被消灭之后,日本侵略者及重要汉奸分子的土地应该被没收,并分配给无地及少地的农民。\\
\subsubsection*{\myformat{工业}}
第七,工业问题。\\
  为着打败日本侵略者与建设新中国,必须发展工业。但是,在国民党政府领导之下,一切依赖外国,自己的财政经济政策是破坏一切经济生活的,国民党统治区内仅有的一点小型工业,也不能不处于大部分破产的状态中。政治不改革,一切生产力都遇着破坏的命运,农业如此,工业也是如此。\\
  就整个来说,没有一个独立、自由、民主与统一的中国,不能有工业的中国。消灭日本侵略者,这是谋独立。废止国民党一党专政,成立民主的联合政府,实现人民的自由,人民的统一,人民的军队,实现土地改革,解放农民,这是谋自由、民主与统一。没有独立、自由、民主与统一,不可能有真正大规模的全国性的工业。没有工业,便没有巩固的国防,没有人民的福利,没有国家的富强。一八四〇年鸦片战争以来一百零五年的历史,特别是国民党当政以来十八年的历史,清楚地把这个要点告诉了中国人民。一个不是贫弱的而是富强的中国,是和一个不是殖民地半殖民地的而是独立的,不是半封建的而是自由的,民主的,不是分裂的而是统一的中国,相联结的。在一个半殖民地的、半封建的、分裂的中国里,要想发展工业,建设国防,福利人民,招致国家的富强,多少年来多少人做过这种梦,但是一概幻灭了。许多好心的教育家、科学家、学生们不问政治,自以为可以所学为国家服务,结果也化成了梦,一概幻灭了。这是好消息,这种幼稚的梦的幻灭,正是中国富强的起点。中国人民在抗日战争中学得了许多东西,知道在日本侵略者被消灭之后,有建立一个新民主主义的独立、自由、民主、统一、富强的中国之必要,而这些条件是互相关联的,不可缺一的。果然如此,中国就有希望了。解放中国人民的生产力,使之获得充分发展的可能性,有待于新民主主义的政治条件在全中国境内的实现。这一点,懂得的人是一天一天的多起来了。\\
  在新民主主义的政治条件获得之后,中国人民及其政府必须采取切实的步骤,在若干年内逐步地建立轻重工业,使中国由农业国地位升到工业国地位上去。中国的新民主主义的独立、自由、民主与统一,如无巩固的经济做它们的基础,如无进步的比较现时发达得多倍的农业,如无大规模的在全国经济比重上占极大优势的工业以及与此相适应的交通、贸易、金融等事业做它们的基础,所谓新民主主义的独立、自由、民主与统一,是不能巩固的。\\
  我们共产党人愿意协同全国一切抗日民主党派,各部分产业界,为上述目标而奋斗。中国工人阶级在这个任务中将起伟大的作用。\\
  中国工人阶级,自第一次世界大战以来,就开始以自觉的姿态,为中国的独立、解放而斗争。一九二一年,产生了它的先锋队——中国共产党。从此以后,使中国的解放斗争进入了新阶段。在北伐战争、土地革命与抗日战争三个时期中,中国工人阶级及其先锋队,对于中国人民的解放事业,作了极大的努力与极有价值的贡献。在最后消灭日本侵略者的斗争中,特别是在收复大城市与交通要道的斗争中,中国工人阶级将起着极大的作用。在抗日结束以后,可以预断,中国工人阶级的努力与贡献将会是更大的。中国工人阶级的任务,不但是为着中国的独立、自由、民主与统一而斗争,而且是为着中国的工业化与农业近代化而斗争。\\
  在新民主主义的国家制度下,将采取调节劳资间利害关系的政策。一方面,保护工人利益,根据情况之不同,而实行八小时到十小时的工作制以及适当的失业救济,社会保险,工会的权利等;另一万面,保证国家企业、私人企业与合作社企业在合理经营下的正当赢利。总之,使劳资双方共同为发展工业生产而努力。\\
  为着发展工业,需要大批资本。从什么地方来呢? 不外两方面:主要地依靠中国人民自己积累资本,同时借助于外援。在服从中国法令,有益中国经济的条件之下,外国投资是我们所欢迎的。对于中国人民与外国人民都有利的事业,是中国在得到一个巩固的国内和平与国际和平,得到一个彻底的政治改革与土地改革之后,能够蓬蓬勃勃地发展大规模的轻重工业与近代化的农业。在这个基础上,外国投资的容纳量将是产常广大的。一个政治上倒退与经济上贫困的中国,则不但对于中国人民非常不利,对于外国人民也是不利的。\\
  日本侵略者被消灭之后,日本侵略者及重要汉奸分子的企业及财产,应当没收,归政府处理。\\
\subsubsection*{\myformat{文化、教育、知识分子}}
第八,文化、教育、知识分子问题。\\
  民族压迫与封建压迫所给予中国人民的灾难中,包括了民族文化的灾难。特别是具有进步意义的文化与教育事业,文化人与教育家,所受灾难,更为深重。\\
  为着扫除民族压迫与封建压迫,为着建立新民主主义的独立、自由、民主、统一与富强的中国,需要大批的人民的教育家,教师,人民的科学家,工程师,技师,医生,新闻工作者,著作家,文学家,艺术家与普通文化工作者,以“为人民服务” “和人民打成一片”的精神,从事艰巨的工作。一切这些知识分子,只要是在为人民服务中著有成绩的,应受到政府与社会的尊重,把他们看作国家与社会的宝贵财富。中国是一个被民族压迫与封建压迫所造成的文化落后的国家,中国的人民解放斗争迫切地需要知识分子,因而知识分子问题就特别显得重要。而在过去半世纪的人民解放斗争,特别是五四运动以来的斗争中,在八年抗日中,广大革命知识分子对于中国人民解放事业所起的作用,是很伟大的,在今后的斗争中,他们将起更大的作用。因此,今后政府应有计划地从广大人民中培养各类知识分子干部,并注意团结与教育现有一切有用的知识分子。\\
  从百分之八十的人口中扫除文盲,是建立新中国的必要条件。\\
  一切奴化的、封建主义的与法西斯主义的文化、教育,应当采取适当的但是坚决的步骤,加以扫除。\\
  对于由民族压迫与封建压迫所造成的摧残中国人民的精神与肉体的那种不知卫生的愚昧与疾病疫疠的严重情况,应当讲求积极的改革与救治办法,推广国民卫生事业。\\
  对于旧文化工作者,旧教育工作者及旧医生们的态度,是采取适当方法,教育他们,使他们获得新观点,新方法,为中国人民服务。\\
  中国国民文化与国民教育的宗旨,应当是新民主主义的,就是说,中国应当建立自己的民族的、科学的、人民大众的新文化的新教育。\\
  对于外国文化,排外主义的方针是错误的,应当尽量吸收进步的外国文化,以为中国文化运动的借镜。盲目服从的方针也是错误的,应当以中国人民的实际需要为基础,批判地吸收外国文化。对于中国古代文化,同样,既不是一概排斥,也不是盲目服从,而是批判地接收它,以利于推进中国的新民主主义的文化。\\
  在为着中国人民解放事业而斗争的总方针下,共产党人应当不分阶级、信仰与党派,和一切知识分子很好地团结起来。\\
\subsubsection*{\myformat{少数民族}}
第九,少数民族问题。\\
  国民党反人民集团否认中国有多民族存在,而把蒙、回、藏、彝(“夷”)、苗、瑶(“猺”)各少数民族称之为“宗族”。他们对于各少数民族,完全继承满清政府及北洋军阀政府的反动政策,压迫剥削,无所不至。一九四三年对于伊克昭盟蒙族人民的屠杀事件,一九四四年直至现在对于新疆少数民族的武力镇压事件以及近几年对于甘肃回民的屠杀事件,就是明证。这是法西斯主义的大汉族主义的错误的民族思想与错误的民族政策,完全背叛了孙中山先生。\\
  一九二四年,孙中山先生在其所著的中国国民党第一次代表大会宣言里说:\\
  “国民党之民族主义,有两方面之意义:一则中国民族自求解放;二则中国境内各民族一律平等。” “国民党敢郑重宣言:承认中国以内各民族之自决权;于反对帝国主义及军阀之战争获得胜利以后,当组织一自由统一的(各民族自由联合的)中华民国。”\\
  中国共产党完全同意上述孙先生的民族政策,共产党人应当积极地帮助各少数民族的广大人民群众为实现这个政策而奋斗。应当帮助各少数民族的广大人民群众,包括一切联系民众的领袖在内,争取他们在政治上、经济上、文化上的解放与发展,并成立拥护民众利益的少数民族自己的军队。他们的言语、文字、风俗、习惯及宗教信仰,应被尊重。\\
  多年以来,陕甘宁边区及华北解放区对待蒙回两族的态度是正确的,其工作是有成绩的。\\
\subsubsection*{\myformat{外交}}
第十,外交问题。\\
  中国共产党同意大西洋宪章及莫斯科、开罗、德黑兰、克里米亚各次国际会议的决议,因为这些国际会议的决议都是有利于打败法西斯侵略者与维持世界和平的。\\
  中国共产党尤其对于克里米亚会议的决议表示热烈的赞同,克里米亚会议决定:最后地打败法西斯德国,并消灭法西斯主义及其产生的原因;在欧洲解放区消灭法西期主义的最后残余,确立各国的国内和平,建立各国人民自己所选择的民主制度;而在建立民主制度的程序方面:“组织临时政权机关,这种政权机关广泛地包罗人口中一切民主成分的代表;并保证尽早经过自由选举,以建立执行人民意志的政府。”克里米亚会议并决定:英美苏三国团结一致,保持“巩固的与持久的”世界和平,迅速成立世界和平机构。\\
  我们认为克里米亚的路线,和中国共产党关于解决东方问题与中国问题的基本方针,是一致的。在消灭日本侵略者与解决中国问题时,下面各点是必须实现的。第一,日本侵略者必须最后地被打败,并彻底消灭日本法西斯主义、军国主义及其产生的原因,不许中途妥协。第二,中国法西斯主义的最后残余必须被消灭,不许保留丝毫。第三,中国必须建立国内和平,不许再打内战。第四,国民党独裁统治必须废止;废止之后,首先代之以广泛地包罗中国人口中一切民主成份的代表所组成的举国一致的临时的联合政府,这是第一个步骤;然后,在国土收复之后,经过自由的无拘束的选举,建立执行人民意志的正式的联合政府,这是第二个步骤;这两个步骤,不许可省掉任何一个。按照克里米亚路线,按照中国国情,我们必须这样做。\\
  中国共产党的外交政策的基本原则,是在彻底消灭日本侵略者,保持世界和平,互相尊重国家的独立与平等地位,互相增进国家与人民的利益及友谊这些基础之上,和各国建立和巩固邦交,解决一切战时战后的相互关系问题,例如配合作战、和平会议、通商、投资等等。\\
  中国共产党对于保障战后国际和平与安全的机构之建立,完全同意敦巴顿橡树林会议所作的建议及在克里米亚会议上对此问题所作的决定。中国共产党欢迎旧金山联合国代表大会。中国共产党已经派遣自己的代表加入中国代表团去出席旧金山会议,借以表达中国人民的意志。\\
  我们认为国民党政府必须停止对于苏联的仇视态度,迅速地改善中苏邦交。苏联是第一个废除不平等条约,并与中国订立平等新约的国家。在一九二四年孙中山先生亲自召集的第一次国民党代表大会开会时及在其后举行北伐战争时,苏联是当时唯一援助中国解放战争的国家。在一九三七年七七抗战以后,苏联又是第一个援助中国反对日本侵略者的国家。中国人民对于苏联政府与苏联人民的这些援助,表示感谢。我们认为太平洋问题的最后的彻底的解决,没有苏联参加是不可能的。\\
  我们认为英美两大国,特别是美国,在反对日本侵略者的共同事业上所作的伟大努力,以及两国政府与两国人民对于中国的同情与援助,是值得感谢的。\\
  但是,我们要求各同盟国政府,首先是英美两国政府,对于中国最广大人民的呼声,加以严重的注意,不要使他们自己的外交政策违反中国人民的意志,因而损害或失去中国人民的友谊。我们认为任何外国政府,如果援助中国反动分子进行反对中国人民的民主事业,那就将要犯下绝大的错误。\\
  中国人民欢迎许多外国政府宣布废除对于中国的不平等条约、并与中国订立平等新约这种以平等地位对待中国人民的措施。但是,我们认为平等条约的订立,并不就是中国在实际上已经取得真正的平等地位。这种实际上的真正的平等地位,决不能单靠外国政府与外国人民的好意给与,主要地应靠中国人民自己努力,把中国在政治上经济上文化上建设成为一个新民主主义的独立、自由、民主、统一与富强的国家,否则便只会有形式上的独立、平等,在实际上是不会有的。就是说,依据国民党政府的现行政策,决不会使中国获得真正的独立与平等。\\
  我们认为在日本侵略者被打败并无条件投降之后,为着彻底消灭日本法西斯主义、军国主义及其所由产生的政治、经济、社会的原因,必须帮助一切日本人民的民主力量,建立日本人民的民主制度。没有这种日本人民的民主制度,便不能彻底消灭日本法西斯主义与军国主义,便不能保证太平洋的和平。\\
  我们认为开罗会议给予朝鲜独立的决定是正确的,中国人民应当帮助朝鲜人民获得解放。\\
  美国已经允许菲律宾独立。我们希望英国也能允许印度独立。因为一个独立的民主的印度,不但是印度人民的需要,也是世界和平的需要。\\
  对于南洋各国——缅甸、马来亚、荷属东印度、法属安南,我们希望英、美、法、荷诸国于帮助当地人民打败日本侵略者以后,能够仿照克里米亚会议对于欧洲解放区所取的态度,给予当地人民以建立独立的民主的国家制度之权利。对于泰国,则仿照对待欧洲法西斯附属国的方法去处理。\\
  已故的美国总统罗斯福先生说:“世界已经缩小了。”的确是这样,对于中国人民曾经感觉是住在十分遥远地方的美国人民,现在感觉成了近邻了。中国人民将和美英苏法各大国的人民,以及全世界上一切国家的人民一道,共同建设一个“巩固的与持久的”世界和平。\\
  关于具体纲领的说明,主要的就是这样。\\
  再说一遍,一切这些具体纲领,如果没有一个举国一致的民主的联合政府,就不可能顺利地在全中国实现。\\
  中国共产党在其为中国人民的解放事业而奋斗的二十四年中,创造了这样的地位,就是说,不论什么政党或社会集团,也不论是中国人或外国人,如果采取抹杀或不尊重中国共产党的意见的态度,那是极不妥当的。过去和现在都有这样的人,企图孤行己见,敢于抹杀或不尊重我们的意见,但是所得的结果只有一个,就是行不通。这是什么原故呢?不是别的,就是因为我们的意见,我们的政策,我们在中国现阶段上所提出与所实施的新民主主义的一般纲领与具体纲领,符合于最广大的中国人民的利益。我们是中国人民的最忠实的代言人,谁要是敢于抹杀或不尊重我们,谁就是在实际上抹杀或不尊重最广大的中国人民,谁就难免要失败。\\
\subsubsection*{\myformat{中国国民党统治区的任务}}
关于我党的新民主主义的一般纲领与具体纲领,我已在上面作了充分的说明。无疑地,这些纲领,是要在全中国实行的,整个国际国内的形势,给中国人民展开了这种想望。但是,目前的情形,不能不使我们在实行时有所区别,这就是在国民党统治区、在沦陷区、在解放区三种地方互不相同。根据三种地方的不同的情形,产生了不同的任务。这些任务,有些我已经在前面说到了,有些还须在下面加以补充。\\
  在国民党统治区,人民没有爱国活动的自由,民主运动被认为非法,但是包括许多阶层、许多民主党派与民主分子的积极活动是在发展中。中国民主同盟,在今年一月发表了要求结束国民党一党专政与成立联合政权的宣言。社会各界发表同类性质的宣言的,还有许多。国民党内相当广大的党员群众及许多重要人物,对于他们自己党的领导机关的政策,日益表示怀疑和不满,日益感觉他们自己的党从广大中国人民孤立起来的危险性,而要求有一种适合时宜的民主的改革。以重庆为中心的工人、农民、公教人员、商人、产业界、文化界、学生界、教育界、妇女界,乃至一部分军人的民主运动,正在发展。所有这些,预示着一切受压迫阶层的民主运动正在逐渐地向着同一的目标而汇合起来。运动的弱点,在于社会的基层分子还没有广泛地参加,非常重要的痛苦不堪的农民、工人、士兵与下层公教人员,还没有组织起来。运动的另一弱点,是参加运动的民主分子中,尚有许多人对于根据民主原则来转变时局,还缺乏明确的与坚决的精神。但是国际国内的时局,都迫着一切受压迫的阶层、党派、社会集团与个人,逐渐地觉悟与团结起来,执行自己的抗日、救国的神圣权利。 不管国民党政府如何镇压,也不能阻止这一运动的发展。\\
  国民党统治区内被压迫的一切阶层、党派、集团与个人的民主运动,应当有一个广大的发展,并把分散的力量逐渐统一起来,为着实现民族团结,建立联合政府,打败日本侵略者与建设新中国而斗争。为此目的,中国共产党与中国解放区,应当给予它们以一切可能的援助。\\
  在国民党统治区,共产党人应当继续执行广泛的抗日民族统一战线政策,不管什么人,那怕昨天还是反对我们的,只要他今天不反对了,就应和他团结起来,为共同目标而奋斗。中国共产党人的一切工作,服从于动员与统一一切力量,彻底消灭日本侵略者与建设新中国这个总目标。\\
\subsubsection*{\myformat{中国沦陷区的任务}}
在沦陷区,共产党人应当号召一切抗日人民,学习法国与意大利的榜样,将自己组织于各色团体中,组织地下军,准备武装起义,一俟时机成熟,配合从外部进攻的军队,里应外合地消灭日本侵略者。日本侵略者及其忠实走狗,对于我沦陷区内的兄弟姊妹们的无微不至的摧残、掠夺、奸淫与侮辱,激起了一切中国人的火一样的愤怒,报仇雪耻的时机快要到来了。沦陷区的人民,在东西战场及八路军新四军的胜利战争的鼓舞之下,极大地增高了他们的抗日情绪,他们迫切地需要组织起来,以便尽可能迅速地获得解放。因此,我们必须将沦陷区的工作提到和解放区的工作同等重要的地位上。必须有大批工作人员到沦陷区去工作。必须就沦陷区人民中训练与提拔大批的积极分子,参加当地的工作。在沦陷区中,东北四省沦陷最久,又是日本侵略者的产业中心与屯兵要地,我们应当加紧那里的地下工作。对于流亡的东北人民,应当加紧团结他们,准备收复失地。\\
  在一切沦陷区,共产党人应当执行最广泛的抗日民族统一战线政策,不管什么人,只要是反对日本侵略者及其忠实走狗的,就要联合起来,为打倒共同敌人而斗争。\\
  向一切帮助敌人反对同胞的伪军伪警及其它人员提出警告,是必要的。他们应该赶快认识自己的罪恶行为,及时回头,帮助同胞反对敌人,借以赎回自己的罪恶。否则敌人崩溃之日,民族纪律是不会对他们宽容的。\\
  共产党人及抗日人民应当向一切有群众的伪组织进行争取说服工作,使他们站在反对民族敌人的战线上来。同时,对于那些罪大恶极不愿改悔的汉奸分子进行调查工作,以便在国土收复之日,能够迅速地依法治罪。\\
  对于国民党内的反动分子组织汉奸反对中国人民、中国共产党、八路军、新四军及其他人民军队的背叛民族的罪恶行为,必须向他们提出警告,叫他们早日悔罪。否则,在国土收复之日,必然要将汉奸与汉奸组织者一体治罪,决不宽饶。\\
\subsubsection*{\myformat{中国解放区的任务}}
中国人民在中国解放区的伟大斗争中,已经将我党的全部新民主主义的纲领变成了并正在变成被他们所热烈拥护与坚决执行的纲领,因而收获了显著的成绩,聚集了巨大的抗日力量;今后应当从各方面发展与巩固这种力量。\\
  在目前条件下,中国解放区的军队应向一切被敌伪占领而又可能攻克的地方,不论是占领已久,或是新近占领的,发动广泛的进攻,借以扩大解放区,缩小沦陷区。\\
  但是同时应当注意,敌人在目前还是强大的,它一定还要向解放区发动它的进攻,解放区军民必须随时准备粉碎敌人的进攻,并注意解放区的各项巩固工作。\\
  应当争取在一切可能条件下扩大解放区的军队、游击队、民兵与自卫军,并加紧进行整训,增强其战斗力,为配合同盟国实行战略进攻,准备充分的军事力量。\\
  在解放区,一方面,军队应实行拥政爱民的工作,另一方面,政府应领导人民实行拥军优抗的工作,使军民关系获得更大的改善。\\
  共产党人在各个解放区的民选的三三制政府,即地方性联合政府的工作中,在社会工作中,应当继续过去的方针,和一切抗日民主分子,在新民主主义的共同纲领上,不分阶级、党派与信仰,进行很好的合作。\\
  同样,在军事工作中,共产党人应当和一切愿意和我们合作的抗日民主分子,在解放区军队的内部和外部,很好地合作,为着消灭日本侵略者与保卫民主中国之伟大目的,而共同建设一个强大的人民军队。\\
  为了提高工农劳动群众在抗日与生产力的积极性,适当的但是坚决的减租减息及改善工人与职员待遇的政策,应当继续执行。解放区的工作人员,必须努力学会做经济工作。必须动员一切可能的力量,大规模地发展解放区的农业、工业及贸易,改善军民生活,为坚持长期战争与消灭日本侵略者准备必要的物质条件。为此目的,必须实行劳动竞赛,奖励劳动英雄与模范工作者。在城市消灭日本侵略者以后,我们的工作人员,必须迅速学会做城市的经济工作。\\
  为着提高解放区人民大众首先是广大的工人、农民、士兵群众的觉悟程度与培养大批工作干部,应当发展解放区的文化、教育事业。解放区的文化工作者与教育工作者在推进他们的工作时,应当根据目前的农村特点,根据农村人民的需要与自愿原则,采用适宜的内容与形式。\\
  在推进解放区的各项工作时,必须十分爱惜当地的人力物力,任何地方均要作长期打算,避免滥用与浪费。这不但是为着打败日本人,而且是为着建设新中国。\\
  在推进解放区的各项工作时,必须十分注意扶助本地人管理本地的事业,必须十分注意从本地人民优秀分子中大批地培养本地的工作干部。一切从外地去的人,如果不和本地人打成一片,如果不是满腔热情地勤勤恳恳地并适合情况地去帮助本地干部,爱惜他们,如同爱惜自己的兄弟姊妹一样,那就不能完成农村民主革命这个伟大的任务。\\
  八路军、新四军及其他人民军队,每到一地,就应立即帮助本地人民,不但要组织以本地人民的干部为领导的民兵与自卫军,而且要组织以本地人民的干部为领导的地方部队与地方兵团。然后,就可以产生有本地人领导的主力部队与主力兵团。这是一项非常重要的任务。如果不能完成此项任务,就不能建立巩固的抗日根据地,也不能发展人民的军队。\\
  当然,一切本地人,应当热烈地欢迎与帮助从外地去的人员与军队,借以完成共同任务。\\
  关于对待暗藏的民族破坏分子问题,应当提起大家的注意,因为公开的敌人,公开的民族破坏分子,容易识别,也容易处置。暗藏的敌人,暗藏的民族破坏分子,则不容易识别,也就不容易处置。因此,对于这后一种人,既要采取严肃态度,又要采取谨慎态度。\\
  根据信教自由的原则,中国解放区容许各派宗教存在。不论是基督教、天主教、回教、佛教及其它宗教,只要教徒们遵守政府法律,政府就给以保护。信教的和不信教的各有他们的自由,不许加以强迫或歧视。\\
  最后,我们的大会应向各个解放区人民提议,尽可能迅速地在延安召开中国解放区人民代表会议,以便讨论统一各解放区的行动,加强各解放区的抗日工作,援助国民党统治区人民的抗日民主运动,援助沦陷区人民的地下军运动,促进全国人民的团结与联合政府的成立。中国解放区实际上现在已经成了全国广大人民所赖以抗日救国的重心,全国广大人民的希望寄托在我们身上,我们有责任不使他们失望。中国解放区人民代表会议的召集,将给中国人民的民族解放事业起一个巨大的推进作用。\\
\subsection*{\myformat{五 全党团结起来,为实现党的任务而斗争}}
同志们,我们的任务是这样的巨大,我们的政策是这样的具体与明确,我们应该用怎样的工作态度去执行这些政策与实现这些任务呢?\\
  目前国际国内的时局,在我们和中国人民面前,显示了光明的前途,具备了前所未有的有利条件,这是显然的,毫无疑义的。但是同时,依然存在着严重的困难条件。谁要是只看见光明一面,不看见困难一面,谁就会不能很好地为实现党的任务而斗争。\\
  我们的党和中国人民一道,不论在整个党的二十四年历史中,在八年抗日战争中,为中国人民创造了巨大的力量,我们的工作成绩是很显然的,毫无疑义的。但是同时,我们的工作中依然存在着缺点。谁要是只看见成绩一面,不看见缺点一面,谁也就不会很好地为实现党的任务而斗争。\\
  中国共产党自从它在一九二一年诞生以来,在其二十四年的历史中,经历了三次的伟大斗争,这就是北伐战争、土地革命和现在尚未完结的抗日战争。我们的党从它一开始,就是一个以马克思主义的理论为基础的党,这是因为这个主义是全世界无产阶级的最正确最革命的科学思想的结晶。马克思主义的普遍真理一经和中国革命的具体实践相结合,就使中国革命的面目为之一新,产生了新民主主义的整个历史阶段。在马克思主义的理论思想武装之下的中国共产党,在中国人民中产生了新的工作作风,这主要地就是理论与实践相结合的作风,和人民群众紧密地联系在一起的作风与自我批评的作风。\\
  反映了全世界无产阶级实践斗争的马克思主义的普遍真理,只有在它和中国无产阶级及广大人民群众的革命斗争的具体实践相结合,才变为中国人民的有用武器。中国共产党正是这样做了。我们党的发展与进步,是从和一切违反这个真理的教条主义与经验主义作坚决斗争的过程中发展与进步起来的。教条主义脱离具体的实践,经验主义以局部经验误认为普遍真理,这两种机会主义的思想,都是违背马克思主义的。我们党在自己的二十四年奋斗中,克服了和正在克服着这些错误思想,使得我们的党在思想上极大地巩固了。我们党现在已有了一百二十一万党员,绝大多数是在抗日时期入党的,他们中存在着各种不纯正的思想,在抗日以前入党的党员中也有这种情形。几年来的整风工作收到了巨大的成效,使这些不纯正的思想受到了很大的纠正。今后应当继续这种工作,以“惩前毖后,治病救人”的精神,更大地展开党内的思想教育。要使各地各级党的领导骨干都懂得,理论与实践这样密切地相结合,是我们共产党人区别于其他任何政党的显著标志之一。因此,掌握思想教育,是团结全党进行伟大政治斗争的中心环节。如果这个任务不解决,党的一切政治任务是不能完成的。\\
  我们共产党人区别于其它任何政党的又一个显著的标志,就是和最广大的人民群众取得最密切的联系。全心全意地为中国人民服务,一刻也不脱离群众;一切从人民的利益出发,而不是从自己小集团或自己个人的利益出发;向人民负责与向自己领导机关负责的一致性;这些就是我们的出发点。共产党人必须随时准备坚持真理,因为任何真理都是适合人民利益的。共产党人必须随时准备修正错误,因为任何错误都是不适合人民利益的。二十四年的经验告诉我们,凡属正确的任务、政策及工作作风,都是和当时当地的群众要求相适合,都是联系群众的。凡属错误了的任务、政策及工作作风,都是和当时当地的群众要求不相适合,都是脱离群众的。教条主义,经验主义,命令主义,尾巴主义,宗派主义,官僚主义,军阀主义,骄傲自大的工作态度等项弊病之所以一定不好,一定要不得,如果什么人有了这类弊病就一定要改正,就是因为它们脱离群众。我们的大会应该号召全党提起警觉,注意每一个工作环节上的每一个同志,不要让他脱离群众。教育每一个同志热爱人民群众,细心地倾听群众的呼声,每到一地就和那里的群众打成一片,不是高踞于群众之上,而是结合于群众之中,根据群众的觉悟程度,去启发与提高群众的觉悟,在群众出于内心自愿的原则之下,帮助群众逐步地组织起来,逐步地展开为当时当地内外环境所许可的一切必要的斗争。在一切工作中,命令主义是错误的,因为它超过群众的觉悟程度,违反了群众的自愿原则,害了急性病。我们的同志不要以为自己了解了的东西,广大群众也和自己一样一概都了解了。群众是否已经了解并且是否愿意行动起来,要到群众中去考察才会知道。如果我们这样做,就可以避免命令主义。在一切工作中,尾巴主义也是错误的,因为它落后于群众的觉悟程度,违反了领导群众前进一步的原则,害了慢性病。我们的同志不要以为自己尚不了解的东西,群众也一概不了解。许多时候,广大群众跑到我们的前头去了,迫切地需要前进一步了,我们的同志不能做广大群众的领导者,却反映了一部分落后分子的意见,并将此种落后分子的意见误认为广大群众的意见,做了落后分子的尾巴。总之,应使每个同志明了,共产党人的一切言论、行动,以是否合乎最广大人民群众的最大利益,是否为最广大人民群众所拥护为最高标准。应使每一个同志懂得,只要我们依靠人民,坚决地相信人民群众的创造力是无穷无尽的,因而信任人民,和人民打成一片,那就任何困难也能克服,任何敌人也不能压倒我们,而只会被我们所压倒。\\
  有无认真的自我批评,也是我们和其他政党互相区别的显著标志之一。我们曾经说过,房子是应该经常打扫的,不打扫就会积满了灰尘。脸是应该经常洗的,不洗也就会灰尘满面。我们同志的思想,我们党的工作,也会发生灰尘的,也应该打扫与洗涤。“流水不腐,户枢不蟗”,是说它们在不停的运动中抵抗了微生物或其它生物的侵蚀。对于我们,经常地检讨工作,在检讨中推广民主作风,不惧怕批评与自我批评,实行“知无不言,言无不尽”,“言者无罪,闻者足戒”,“有则改之,无则加勉”,这些中国人民的有益的格言,正是抵抗错误、缺点这类政治微生物侵蚀我们同志的思想与我们党的机体的唯一有效的方法。以“惩前毖后、治病救人”为宗旨的整风运动之所以发生了很大的效力,就是因为我们在这个运动中展开了正确的而不是歪曲的、认真的而不是敷衍的批评与自我批评。以中国最广大人民的最大利益为出发点的中国共产党人,相信自己的事业是完全合乎正义的,不惜牺牲自己个人的一切,随时准备拿出自己的生命去殉我们的事业,难道我们还有什么错误的不适合人民需要的思想,观点,意见,办法,舍不得丢掉的吗?难道我们还欢迎任何政治的灰尘,政治的微生物来点污我们的清洁的面貌与侵蚀我们的健全的机体吗?无数革命先烈为了人民的利益牺牲了他们的生命,使我们每个活着的人想起他们就心里难过,难道我们还有什么个人利益或错误、缺点,不能牺牲吗?\\
  同志们,我们的大会闭幕之后,我们就要上战场去,根据大会的决议,为着打倒日本侵略者与建设新中国而奋斗。为达此目的,我们要同全国人民团结起来。我重说一遍,不管什么阶级,什么政党,什么社会集团或个人,只要他是赞成打倒日本侵略者与建设新中国的,我们就要和他团结起来。为达此目的,我们要把我们的党在民主集中制的组织与纪律之下,比较过去还要好还要坚强地团结起来,不论什么同志,只要他是愿意服从党纲、党章和党的决议的,我们就要和他团结起来。我们的党,在北伐战争时期,不超过五万党员,并且后来大部分被当时的敌人打散了。在土地革命时期,不超过三十万党员,后来大部分也被当时的敌人打散了。现在我们有了一百二十余万党员,这一回无论如何不要被敌人打散。只要我们能吸取三个时期的经验,采取谦虚态度,防止骄傲态度,在党内,和全体同志更好地团结起来,在党外,和全国人民更好地团结起来,就可以保证,不但不会被敌人打散,相反地,一定要把日本侵略者及其忠实走狗坚决、彻底、干净、全部消灭之,并且在消灭他们之后,把一个独立、自由、民主、统一与富强的中国建设起来。\\
  三次革命的经验,尤其是抗日战争的经验,给了我们及中国人民这样一种信心:没有中国共产党的努力,没有中国共产党人做中国人民的中流砥柱,中国的独立、自由、民主、统一是不可能的,中国的工业化和农业近代化也是不可能的。\\
  同志们,有了三次革命经验的中国共产党,我坚决相信,我们是能够完成我们的伟大政治任务的。\\
  成千成万的人民的与党的先烈,为着人民的利益,在我们的前头英勇地牺牲了,让我们高举起他们的旗帜,踏着他们的血迹前进吧!\\
  一个独立、自由、民主、统一与富强的中国不久就要诞生了,让我们迎接这个伟大的日子吧!\\
  打倒日本侵略者!\\
  中国人民解放万岁!
\newpage\section*{\myformat{愚公移山}\\\myformat{(一九四五年六月十一日)}}\addcontentsline{toc}{section}{愚公移山}
\begin{introduction}\item  这是毛泽东在中国共产党第七次全国代表大会上的闭幕词。\end{introduction}
我们开了一个很好的大会。我们做了三件事:第一,决定了党的路线,这就是放手发动群众,壮大人民力量,在我党的领导下,打败日本侵略者,解放全国人民,建立一个新民主主义的中国。第二,通过了新的党章。第三,选举了党的领导机关——中央委员会。今后的任务就是领导全党实现党的路线。我们开了一个胜利的大会,一个团结的大会。代表们对三个报告\footnote[1]{ 指在中国共产党第七次全国代表大会上,毛泽东所作的政治报告、朱德所作的军事报告和刘少奇所作的关于修改党章的报告。}发表了很好的意见。许多同志作了自我批评,从团结的目标出发,经过自我批评,达到了团结。这次大会是团结的模范,是自我批评的模范,又是党内民主的模范。\\
  大会闭幕以后,很多同志将要回到自己的工作岗位上去,将要分赴各个战场。同志们到各地去,要宣传大会的路线,并经过全党同志向人民作广泛的解释。\\
  我们宣传大会的路线,就是要使全党和全国人民建立起一个信心,即革命一定要胜利。首先要使先锋队觉悟,下定决心,不怕牺牲,排除万难,去争取胜利。但这还不够,还必须使全国广大人民群众觉悟,甘心情愿和我们一起奋斗,去争取胜利。要使全国人民有这样的信心:中国是中国人民的,不是反动派的。中国古代有个寓言,叫做“愚公移山”。说的是古代有一位老人,住在华北,名叫北山愚公。他的家门南面有两座大山挡住他家的出路,一座叫做太行山,一座叫做王屋山。愚公下决心率领他的儿子们要用锄头挖去这两座大山。有个老头子名叫智叟的看了发笑,说是你们这样干未免太愚蠢了,你们父子数人要挖掉这样两座大山是完全不可能的。愚公回答说:我死了以后有我的儿子,儿子死了,又有孙子,子子孙孙是没有穷尽的。这两座山虽然很高,却是不会再增高了,挖一点就会少一点,为什么挖不平呢?愚公批驳了智叟的错误思想,毫不动摇,每天挖山不止。这件事感动了上帝,他就派了两个神仙下凡,把两座山背走了\footnote[2]{ 愚公移山的故事,见《列子•汤问》。原文是:“太行、王屋二山,方七百里,高万仞。本在冀州之南,河阳之北。北山愚公者,年且九十,面山而居。惩山北之塞,出入之迂也,聚室而谋曰:吾与汝毕力平险,指通豫南,达于汉阴,可乎?杂然相许。其妻献疑曰:以君之力,曾不能损魁父之丘,如太行王屋何?且焉置土石?杂曰:投诸渤海之尾,隐土之北。遂率子孙荷担者三夫,叩石垦壤,箕畚运于渤海之尾。邻人京城氏之孀妻,有遗男,始龀,跳往助之。寒暑易节,始一反焉。河曲智叟,笑而止之,曰:甚矣,汝之不惠。以残年余力,曾不能毁山之一毛,其如土石何?北山愚公长息曰:汝心之固,固不可彻,曾不若孀妻弱子。虽我之死,有子存焉;子又生孙,孙又生子;子又有子,子又有孙。子子孙孙,无穷匮也,而山不加增,何苦而不平?河曲智叟亡以应。操蛇之神闻之,惧其不已也,告之于帝。帝感其诚,命夸蛾氏二子负二山,一厝朔东,一厝雍南。自此,冀之南,汉之阴,无陇断焉。”}。现在也有两座压在中国人民头上的大山,一座叫做帝国主义,一座叫做封建主义。中国共产党早就下了决心,要挖掉这两座山。我们一定要坚持下去,一定要不断地工作,我们也会感动上帝的。这个上帝不是别人,就是全中国的人民大众。全国人民大众一齐起来和我们一道挖这两座山,有什么挖不平呢?\\
  昨天有两个美国人要回美国去,我对他们讲了,美国政府要破坏我们,这是不允许的。我们反对美国政府扶蒋反共的政策。但是我们第一要把美国人民和他们的政府相区别,第二要把美国政府中决定政策的人们和下面的普通工作人员相区别。我对这两个美国人说:告诉你们美国政府中决定政策的人们,我们解放区禁止你们到那里去,因为你们的政策是扶蒋反共,我们不放心。假如你们是为了打日本,要到解放区是可以去的,但要订一个条约。倘若你们偷偷摸摸到处乱跑,那是不许可的。赫尔利已经公开宣言不同中国共产党合作\footnote[3]{ 赫尔利(一八八三——一九六三),美国共和党人。他在一九四四年十一月底被任命为美国驻中国大使,因支持蒋介石的反共政策而受到中国人民的坚决反对,于一九四五年十一月被迫宣布离职。一九四五年四月二日他在华盛顿国务院记者招待会上的谈话中,公开宣言不同中国共产党合作。参见本卷《赫尔利和蒋介石的双簧已经破产》和《评赫尔利政策的危险》。},既然如此,为什么还要到我们解放区去乱跑呢?\\
  美国政府的扶蒋反共政策,说明了美国反动派的猖狂。但是一切中外反动派的阻止中国人民胜利的企图,都是注定要失败的。现在的世界潮流,民主是主流,反民主的反动只是一股逆流。目前反动的逆流企图压倒民族独立和人民民主的主流,但反动的逆流终究不会变为主流。现在依然如斯大林很早就说过的一样,旧世界有三个大矛盾:第一个是帝国主义国家中的无产阶级和资产阶级的矛盾,第二个是帝国主义国家之间的矛盾,第三个是殖民地半殖民地国家和帝国主义宗主国之间的矛盾\footnote[4]{ 参见斯大林《论列宁主义基础》第一部分《列宁主义的历史根源》(《斯大林选集》上卷,人民出版社1979年版,第186—187页)。}。这三种矛盾不但依然存在,而且发展得更尖锐了,更扩大了。由于这些矛盾的存在和发展,所以虽有反苏反共反民主的逆流存在,但是这种反动逆流总有一天会要被克服下去。\\
  现在中国正在开着两个大会,一个是国民党的第六次代表大会,一个是共产党的第七次代表大会。两个大会有完全不同的目的:一个要消灭共产党和中国民主势力,把中国引向黑暗;一个要打倒日本帝国主义和它的走狗中国封建势力,建设一个新民主主义的中国,把中国引向光明。这两条路线在互相斗争着。我们坚决相信,中国人民将要在中国共产党领导之下,在中国共产党第七次大会的路线的领导之下,得到完全的胜利,而国民党的反革命路线必然要失败。\\
\newpage\section*{\myformat{论军队生产自给,兼论整风和生产两大运动的重要性}\\\myformat{(一九四五年四月二十七日)}}\addcontentsline{toc}{section}{论军队生产自给,兼论整风和生产两大运动的重要性}
\begin{introduction}\item  这是毛泽东为延安《解放日报》写的社论。\end{introduction}
我们的军队在遭受极端物质困难的目前状况之下,在分散作战的目前状况之下,切不可将一切物质供给责任都由上面领导机关负起来,这样既束缚了下面广大人员的手足,而又不可能满足下面的要求。应该说:同志们,大家动手,克服困难吧。只要上面善于提出任务,放手让下面自力更生,问题就解决了,而且能够更加完善地解决它。如果上面不去这样作,而把一切事实上担负不起来的担子老是由自己担起来,不敢放手让下面去做,不去发动广大群众自力更生的积极性,虽然上面费尽了气力,结果将是上下交困,在目前条件下永远也不能解决问题。几年来的经验,已经充分证明了这一点。“统一领导,分散经营”的原则,已被证明是我们解放区在目前条件下组织一切经济生活的正确的原则。\\
  解放区的军队,已经达到了九十多万。为着打败日本侵略者,还需要扩大军队到几个九十万。但是我们还没有外援。就是假定将来有了外援,生活资料也只能由我们自己来供给,这是一点主观主义也来不得的。在不久的将来,我们需要集中必要的兵团,离开现在分散作战的地区,到一定的攻击目标上去作战。这种集中行动的大兵团,不但不能生产自给了,而且需要后方的大量的物质供给;只有被留下来的地方部队和地方兵团(其数目将还是广大的),还能照旧一面作战,一面生产。照此看来,我们全军应趁目前的时机,在不妨碍作战和训练的条件之下,一律学会完成部分的生产自给的任务,难道还有疑问吗?\\
  军队的生产自给,在我们的条件下,形式上是落后的、倒退的,实质上是进步的,具有重大历史意义的。在形式上,我们违背了分工的原则。但是,在我们的条件下——国家贫困、国家分裂(这些都是国民党主要统治集团所造成的罪恶结果)以及分散的长期的人民游击战争,我们这样做,就是进步的了。大家看,国民党的军队面黄肌瘦,解放区的军队身强力壮。大家看,我们自己,在没有生产自给的时候,何等困难,一经生产自给,何等舒服。现在,让站在我们面前的两个部队,例如说两个连,去选择两种办法中的一种:或者由上面全部供给生活资料;或者不给它或少给它,让它全部、大部、半部或小部地生产自给。哪一种结果要好些?哪一种它们愿意接受些呢?在认真地试行一年生产自给之后,一定会认为后一种办法结果要好些,愿意接受它;一定会认为前一种办法结果要差些,不愿意接受它。这是因为后者能使我们部队的一切成员改善生活;而前者,在目前的物质困难条件下,无论怎样由上面供给,也不能满足他们的要求。至于因为我们采用了这种表面上“落后的”、“倒退的”办法,而使我们的军队克服了生活资料的困难,改善了生活,个个身强力壮,足以减轻同在困难中的人民的赋税负担,因而取得人民的拥护,足以支持长期战争,并足以扩大军队,因而也就能够扩大解放区,缩小沦陷区,达到最后地消灭侵略者、解放全中国的目的。这种历史意义,难道还不伟大吗?\\
  军队生产自给,不但改善了生活,减轻了人民负担,并因而能够扩大军队,而且立即带来了许多副产物。这些副产物就是:(一)改善官兵关系。官兵一道生产劳动,亲如兄弟了。(二)增强劳动观念。我们现行的,既不是旧式的募兵制,也不是征兵制,而是第三种兵役制——动员制。它比募兵制要好些,它不会造成那样多的二流子;但比征兵制要差些。我们目前的条件,还只许可我们采取动员制,还不能采取征兵制。动员来的兵要过长期的军队生活,将减弱他们的劳动观念,因而也会产生二流子和沾染军阀军队中的若干坏习气。生产自给以来,劳动观念加强了,二流子的习气被改造了。(三)增强纪律性。在生产中执行劳动纪律,不但不会减弱战斗纪律和军人生活纪律,反而会增强它们。(四)改善军民关系。部队有了家务,侵害老百姓财物的事就少了,或者完全没有了。在生产中,军民变工互助,更增强他们之间的友好关系。(五)军队埋怨政府的事也会少了,军政关系也好了。(六)促进人民的大生产运动。军队生产了,机关生产更显得必要,更有劲了;全体人民的普遍增产运动,当然也更显得必要,更有劲了。\\
  一九四二和一九四三两年先后开始的带普遍性的整风运动和生产运动,曾经分别地在精神生活方面和物质生活方面起了和正在起着决定性的作用。这两个环子,如果不在适当的时机抓住它们,我们就无法抓住整个的革命链条,而我们的斗争也就不能继续前进。\\
  大家明白,我们在一九三七年以前入党的党员,剩下的不过数万人,而我们现在的党员是一百二十多万,其中大多数是农民及其它小资产阶级出身的,他们有很可爱的革命积极性,并愿接受马克思主义的训练;但是,他们是带了他们原来的不符合或不大符合于马克思主义的思想入党的。这种情形,就是在一九三七年以前入党的党员中也是存在着的。这是一个极其严重的矛盾,一个绝大的困难。在这种情形下,如果不进行一个普遍的马克思主义的教育运动,即整风运动,我们还能顺利地前进吗?显然是不能的。但是,我们在大量干部中解决了和正在解决着这个矛盾——党内无产阶级思想和非无产阶级思想(其中有小资产阶级、资产阶级甚至地主阶级的思想,而主要是小资产阶级的思想)之间的矛盾,即马克思主义思想和非马克思主义思想之间的矛盾,我们的党就能够在思想上、政治上、组织上空前统一地(不是完全统一地)大踏步地但又是稳步地前进了。在今后,我们党还会、也还应该有更大的发展,而我们是能够在马克思主义的思想原则下更好地掌握将来的发展了。\\
  另一个环子是生产运动。抗战八年了,我们开头还有饭吃,有衣穿。随后逐步困难起来,以至于大困难:粮食不足,油盐不足,被服不足,经费不足。这是伴随着一九四〇年至一九四三年敌人大举进攻和国民党政府发动三次大规模反人民斗争(所谓“反共高潮”)\footnote[1]{ 参见本卷《评国民党十一中全会和三届二次国民参政会》一文中关于国民党发动三次反共高潮的叙述。}而来的绝大的困难,绝大的矛盾。如果不解决这个困难,不解决这个矛盾,不抓住这个环节,我们的抗日斗争还能前进吗?显然是不能的。但是我们学会了并且正在学会着生产,这样一来,我们又活跃了,我们又生气勃勃了。再有几年,我们将不怕任何敌人,我们将要压倒一切敌人了。\\
  这样看来,整风和生产两大运动;具有何种历史重要性,是明白无疑的了。\\
  让我们进一步地、普遍地去推广这两大运动,以为其它各项战斗任务的基础。果能如此,那末,中国人民的彻底解放,就有把握了。\\
  目前正当春耕时节,希望一切解放区的领导同志、工作人员、人民群众,不失时机地掌握生产环节,取得比去年更大的成绩。特别是那些还没有学会生产的地区,今年应当更大地努一把力。\\
\newpage\section*{\myformat{赫尔利和蒋介石的双簧已经破产}\\\myformat{(一九四五年七月十日)}}\addcontentsline{toc}{section}{赫尔利和蒋介石的双簧已经破产}
\begin{introduction}\item  这是毛泽东为新华社写的评论。\end{introduction}
以粉饰蒋介石独裁统治为目的而召集的四届国民参政会,七月七日在重庆开会。第一次会议到会者之少,为历届参政会所未有。不但中共方面无人出席,其它方面也有很多人未出席。定数二百九十名的参政员中,出席的仅有一百八十名。蒋介石在开幕时说了一通话。蒋介石说:“政府对于国民大会召集有关的问题,拟不提出任何具体的方案,可使诸君得以充分的讨论。政府准备以最诚恳坦白的态度,聆取诸位对于这些问题的意见。”所谓今年十一月十二日召集国民大会一件公案,大概就此收场了。这件公案,也和帝国主义者赫尔利\footnote[1]{ 见本卷《愚公移山》注〔3〕。}有关系。原来这位帝国主义者是极力怂恿蒋介石干这一手的,蒋介石的腰这才敢于在今年元旦的演说\footnote[2]{ 这是指一九四五年一月一日蒋介石的广播演说。他在这个演说里,对过去一年国民党军队在日本侵略军进攻面前的溃败一字不提,反而大肆诬蔑人民,反对全国人民和各抗日党派所拥护的关于取消国民党一党专政及成立联合政府和联合统帅部的主张,坚持国民党一党专政,并且以准备召开为全国人民所唾弃的国民党御用的“国民大会”,作为反对人民的挡箭牌。}里稍稍硬了起来,至三月一日的演说\footnote[3]{ 这是指一九四五年三月一日蒋介石在重庆宪政实施协进会上的演说。蒋介石除坚持“元旦演说”的反动主张之外,又提出组织有美国代表参加的三人委员会来“整编”八路军新四军,公开地要求美帝国主义者来干涉中国的内政。}而大硬,说是一定要在十一月十二日“还政于民”。在蒋介石的三月一日的演说里,对于中国共产党代表中国人民的公意而提出的召开党派会议和成立联合政府一项主张,则拒之于千里之外。对于组织一个所谓有美国人参加的三人委员会来“整编”中共军队,则吹得得意忘形。蒋介石竟敢说:中共必须先将军队交给他,然后他才赏赐中共以“合法地位”。所有这一切,赫尔利老爷的撑腰起了决定的作用。四月二日,赫尔利在华盛顿发表声明,除了抹杀中共的地位,污蔑中共的活动,宣称不和中共合作等一派帝国主义的滥调而外,还极力替蒋介石的“国民大会”等项臭物捧场。如此,美国的赫尔利,中国的蒋介石,在以中国人民为牺牲品的共同目标下,一唱一和,达到了热闹的顶点。从此以后,似乎就走上了泄气的命运。反对者无论在中国人和外国人中,在国民党内和国民党外,在有党派人士和无党派人士中,到处皆是,不计其数。其原因只有一个,就是:赫尔利蒋介石这一套,不管他们怎样吹得像煞有介事,总之是要牺牲中国人民的利益,进一步破坏中国人民的团结,安放下中国大规模内战的地雷,从而也破坏美国人民及其它同盟国人民的反法西斯战争和战后和平共处的共同利益。到了今天,赫尔利不知在忙些什么,总之是似乎暂时地藏起来了,却累得蒋介石在参政会上说些不三不四的话。三月一日蒋介石说:“我国情形与他国不同,在国民大会召开以前,我们便无一个可以代表人民、使政府可以咨询民意之负责团体。”既然如此,不知道我们的委员长为什么又向参政会“聆取”起“意见”来了。按照委员长的说法,中国境内是并无任何“可以咨询民意的负责团体”的,参政会不过是一个吃饭的“团体”而已,今天的“聆取”,于法无据。可是不管怎样,只要参政会说一声停开那个伪造的“国民”大会,就说违反了三月一日的圣旨,犯了王法,也算做了一回好事,积了一件功德。当然,今天来评论参政会,为时尚早,因为参政会究竟拿什么东西让委员长“聆取”,还要等几天才能看到。不过有一点是确实的:自从中国人民群起反对之后,就是热心“君主立宪”的人们也替我们的君主担忧,劝他不要套上被称为猪仔国会的那条绞索,谨防袁世凯\footnote[4]{ 见本书第一卷《论反对日本帝国主义的策略》注〔1〕。}来找替死鬼。因此,我们的君主就此缩手,也未可知。然而我们的君主及其左右,是决不让人民轻易获得丝毫权力而使他们自己损失一根毫毛的。眼前的证据,就是这位君主将人民的合理批评,称之为“肆意攻击”。据说,“在战争状况之下,沦陷区域势必无法举行任何普遍的选举。因此,在两年以前,国民党中央全会乃有于战事结束一年以内召开国民大会、实行宪政的决定。若干方面,当时曾肆意攻击”,以为迟了。及至他“鉴于战事的完全结束为时容或延长,即使战事结束后各地秩序亦未必能于短时期内恢复,所以主张在战局稳定之时即行召集国民大会”,不料那些人们又“肆意攻击”。这样一来,闹得我们的君主很不好办。但是中国人民必须教训蒋介石及其一群:对于违反人民意志的任何欺骗,不管你们怎样说和怎样做,是断乎不许可的。中国人民所要的是立即实行民主改革,例如释放政治犯,取消特务,给人民以自由,给各党派以合法地位等项。对于这些,你们一件也不做,却在所谓召开“国民大会”的时间问题上耍花样,这是连三岁小孩子也欺骗不了的。没有认真的起码的民主改革,任何什么大会小会也只能被抛到毛屎坑里去。就叫做“肆意攻击”也罢,任何这类的欺骗,必须坚决、彻底、干净、全部地攻击掉,决不容许保留其一丝一毫。这原因不是别的,就是因为它是欺骗。有无国民大会是一件事,有无起码的民主改革又是一件事。可以暂时没有前者,不可以不立即实施后者。蒋介石及其一群,既然愿意“提早”“还政于民”,为什么不愿意“提早”实施若干起码的民主改革?国民党的先生们,当我写这最后几行时,你们得承认,中国共产党人总算不是向你们“肆意攻击”,仅仅提出一个问题,难道也不应该吗?难道你们也可以置之不答吗?你们得答复这个问题:为什么你们愿意“还政于民”,却不愿意实行民主改革呢?\\
\newpage\section*{\myformat{评赫尔利政策的危险}\\\myformat{(一九四五年七月十二日)}}\addcontentsline{toc}{section}{评赫尔利政策的危险}
\begin{introduction}\item  这是毛泽东为新华社写的评论。\end{introduction}
以美国驻华大使赫尔利\footnote[1]{ 见本卷《愚公移山》注〔3〕。}为代表的美国对华政策,越来越明显地造成了中国内战的危机。坚持反动政策的国民党政府,从它在十八年前成立之日起,就是以内战为生活的;仅在一九三六年西安事变\footnote[2]{ 参见本书第一卷《关于蒋介石声明的声明》注〔1〕。}和一九三七年日本侵入中国本部这样的时机,才被迫暂时地放弃全国规模的内战。但从一九三九年起,局部的内战又在发动,并且没有停止过。国民党政府在其内部的动员口号是“反共第一”,抗日被放在次要的地位。目前国民党政府一切军事布置的重心,并不是放在反对日本侵略者方面,而是放在向着中国解放区“收复失地”和消灭中国共产党方面。不论是为着抗日战争的胜利,或是战后的和平建设,这种情况均须严重地估计到。罗斯福总统在世时,他是估计到了这一点的,为了美国的利益,他没有采取帮助国民党以武力进攻中国共产党的政策。一九四四年十一月,赫尔利以罗斯福私人代表的资格来到延安的时候,他曾经赞同中共方面提出的废止国民党一党专政、成立民主的联合政府的计划。但是他后来变卦了,赫尔利背叛了他在延安所说的话。这样一种变卦,露骨地表现于四月二日赫尔利在华盛顿所发表的声明。这时候,在同一个赫尔利的嘴里,以蒋介石为代表的国民党政府变成了美人,而中共则变成了魔怪;并且他率直地宣称:美国只同蒋介石合作,不同中共合作。当然这不只是赫尔利个人的意见,而是美国政府中的一群人的意见,但这是错误的而且危险的意见。就在这个时候,罗斯福去世了,赫尔利得意忘形地回到重庆的美国大使馆。这个以赫尔利为代表的美国对华政策的危险性,就在于它助长了国民党政府的反动,增大了中国内战的危机。假如赫尔利政策继续下去,美国政府便将陷在中国反动派的又臭又深的粪坑里拔不出脚来,把它自己放在已经觉醒和正在继续觉醒的几万万中国人民的敌对方面,在目前,妨碍抗日战争,在将来,妨碍世界和平。这一种必然的趋势,难道还看不清楚吗?在中国的前途这个问题上,看清楚了中国人民要求独立、自由、统一的不可阻止的势力必然要代替民族压迫和封建压迫而勃兴的美国一部分舆论界,对于赫尔利式的危险的对华政策,是感到焦急的,他们要求改变这个政策。但是,美国的政策究竟是否改变和哪一天才改变,今天我们还不能说什么。可以确定地说的,就是赞助中国反人民势力和以如此广大的中国人民为敌的这个赫尔利式的政策,如果继续不变的话,那就将给美国政府和美国人民以千钧重负和无穷祸害,这一点,必须使美国人民认识清楚。\\
\newpage\section*{\myformat{给福斯特同志的电报}\\\myformat{(一九四五年七月二十九日)}}\addcontentsline{toc}{section}{给福斯特同志的电报}\\~~\\
福斯特同志和美国共产党中央委员会:\\
  欣悉美国共产主义政治协会特别会议决定抛弃白劳德的修正主义的即投降主义的路线\footnote[1]{ 白劳德(一八九一——一九七三),曾任美国共产党总书记。在第二次世界大战期间,美国共产党内以白劳德为代表的右倾思想,曾经形成反马克思主义的修正主义——投降主义路线,并于一九四四年四月出版了作为他的纲领性的著作《德黑兰:我们在战争与和平中的道路》一书。白劳德“修正”了列宁主义关于帝国主义是垄断的、腐朽的和垂死的资本主义的基本理论,否认美国资本主义的帝国主义性质,认为它还“保持着青年的资本主义制度的一些特点”,认为美国无产阶级和大资产阶级之间有“共同利害”,主张保护托拉斯制度,经过“阶级调和”来避免美国资本主义所不可避免的危机。白劳德于一九四四年五月,主持解散了美国无产阶级的政党——美国共产党,而另行组织非党的美国共产主义政治协会。白劳德的这一错误路线一开始就遭到以福斯特为首的许多美国共产党员的反对。一九四五年六月,在福斯特领导下,美国共产主义政治协会通过了批判白劳德路线的决议。同年七月,又举行美国共产主义政治协会特别代表大会,决定重建美国共产党。白劳德后来仍然坚持其错误主张,公开拥护杜鲁门政府的帝国主义政策,并进行反党的派别活动,因此在一九四六年二月被开除出党。},重新确立马克思主义的领导,并已恢复了美国共产党。我们对于美国工人阶级和马克思主义运动的这个伟大的胜利,谨致热烈的祝贺。白劳德的整个修正主义——投降主义路线(这条路线充分表现于白劳德所著《德黑兰》一书中),本质上是反映了美国反动资本集团在美国工人运动中的影响。这个反动资本集团现在也正在力图扩大其影响于中国,赞助中国国民党内反动集团的反民族反人民的错误政策,使中国人民面临着严重的内战危机,危害中美两大国人民的利益。美国工人阶级及其先锋队美国共产党反对白劳德修正主义——投降主义的胜利,对于中美两国人民目前所进行的反日战争和战后建设和平民主世界的伟大事业,无疑地将有重大的贡献。\\
\newpage\section*{\myformat{对日寇的最后一战}\\\myformat{(一九四五年八月九日)}}\addcontentsline{toc}{section}{对日寇的最后一战}
\begin{introduction}\item  这是毛泽东就苏联对日宣战发表的声明。\end{introduction}
八月八日,苏联政府宣布对日作战,中国人民表示热烈的欢迎。由于苏联这一行动,对日战争的时间将大大缩短。对日战争已处在最后阶段,最后地战胜日本侵略者及其一切走狗的时间已经到来了。在这种情况下,中国人民的一切抗日力量应举行全国规模的反攻,密切而有效力地配合苏联及其它同盟国作战。八路军、新四军及其它人民军队,应在一切可能条件下,对于一切不愿投降的侵略者及其走狗实行广泛的进攻,歼灭这些敌人的力量,夺取其武器和资财,猛烈地扩大解放区,缩小沦陷区。必须放手组织武装工作队,成百队成千队地深入敌后之敌后,组织人民,破击敌人的交通线,配合正规军作战。必须放手发动沦陷区的千百万群众,立即组织地下军,准备武装起义,配合从外部进攻的军队,消灭敌人。解放区的巩固工作仍应注意。今冬明春,应在现有一万万人民和一切新解放区的人民中,普遍地实行减租减息,发展生产,组织人民政权和人民武装,加强民兵工作,加强军队的纪律,坚持各界人民的统一战线,防止浪费人力物力。凡此一切,都是为着加强我军对敌人的进攻。全国人民必须注意制止内战危险,努力促成民主联合政府的建立。中国民族解放战争的新阶段已经到来了,全国人民应该加强团结,为夺取最后胜利而斗争。\\
\newpage

















\end{document}

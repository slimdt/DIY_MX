\documentclass[cn,11pt,chinese]{elegantbook}

\def\myformat#1{\hfil\hfil #1}


\title{毛泽东选集}
\subtitle{第四卷}

%\author{Ethan Deng \& Liam Huang}
%\institute{Elegant\LaTeX{} Program}
%\date{February 10, 2020}
%\version{3.10}
%\bioinfo{自定义}{信息}

\extrainfo{我们能够学会我们原来不懂的东西。我们不但善于破坏一个旧世界,我们还将善于建设一个新世界。}

%\logo{logo-blue.png}
\logo{mzd.jpg}
%\cover{cover.jpg}
\cover{jt.png}


% 本文档命令
\usepackage{array}
\newcommand{\ccr}[1]{\makecell{{\color{#1}\rule{1cm}{1cm}}}}
% 修改目录深度
\setcounter{tocdepth}{2}

\begin{document}

\maketitle
\frontmatter

\iffalse
\chapter*{特别声明}
\markboth{Introduction}{前言}

在过去的 2019 年,\href{https://elegantlatex.org/}{Elegant\LaTeX{}} 系列模板均逐步上线 \href{https://github.com/ElegantLaTeX}{GitHub}、\href{https://ctan.org/pkg/elegantbook}{CTAN}、\href{https://www.overleaf.com/latex/templates/elegantbook-template/zpsrbmdsxrgy}{Overleaf} 以及 \href{https://gitee.com/ElegantLaTeX/ElegantBook}{Gitee} 上。截止到 2019 年底,ElegantNote、ElegantBook、ElegantPaper 三个模板在 GitHub 上的收藏数达到了 194、333 和 220,从 2019 年 5 月开启捐赠之后收到了用户 33 笔合计超过 1500 元的捐赠,用户群人数也超过了 400 人。这些数字的背后,反映出 Elegant\LaTeX{} 越来越受用户的喜爱,在此非常感谢大家。

但是,我想声明的是:

\begin{center}
  由于某些原因,Elegant\LaTeX{} 项目 \underline{不再接受}\textbf{任何}非我本人预约的提交。
\end{center}

我是一个理想主义者,关于这个模板,我有自己的想法。我所关心的是,我周围的人能方便使用 \LaTeX{} 以及此模板,我自己会为自己的东西感到开心。如果维护模板让我不开心,那我就不会再维护了。诚然这个模板并不是完美的,但是相比 2.x 好很多了,这些改进离不开大家的反馈、China\TeX{} 和逐鹿人的鼓励以及支援人员的帮助!

\underline{如果你无法认同我的想法,建议直接删除本模板。}

\vskip 1.5cm

\begin{flushright}
Ethan Deng\\
February 10, 2020
\end{flushright}
\fi


\tableofcontents
%\listofchanges

\mainmatter


\chapter*{第三次国内革命战争时期}\addcontentsline{toc}{chapter}{第三次国内革命战争时期}\newpage\section*{\myformat{抗日战争胜利后的时局和我们的方针}\\\myformat{(一九四五年八月十三日)}}\addcontentsline{toc}{section}{抗日战争胜利后的时局和我们的方针}
\begin{introduction}\item  这是毛泽东在延安干部会议上的讲演。这篇讲演根据马克思列宁主义的阶级分析的方法,深刻地分析了抗日战争胜利后的中国政治的基本形势,并且提出了无产阶级的革命策略。正如毛泽东一九四五年四月在中国共产党第七次全国代表大会的开幕词中所指出的,在打败了日本帝国主义以后,中国仍然有成为一个新中国和还是一个老中国的两种命运,两个前途。以蒋介石为代表的中国大地主大资产阶级,要从人民手中夺取抗日战争胜利的果实,要使中国仍旧成为大地主大资产阶级专政的半殖民地半封建的国家。代表无产阶级和人民大众利益的中国共产党,一方面要尽力争取和平,反对内战,另一方面必须对于蒋介石发动全国规模内战的反革命计划有充分的准备,采取正确的方针,这就是说,对于帝国主义和反动派不抱幻想,不怕威吓,坚决保卫人民的斗争果实,努力建立无产阶级领导的、人民大众的、新民主主义的新中国。中国的两种命运、两个前途的决定胜败的斗争,就是从抗日战争结束直到中华人民共和国成立的这个历史时期的内容,这个历史时期就是中国人民解放战争或第三次国内革命战争时期。蒋介石在美国帝国主义的援助下,在抗日战争结束以后一再撕毁和平的协议,发动了空前的反革命的大内战,企图消灭人民力量。由于中国共产党的正确领导,中国人民只经历了四年的斗争,就在全国范围内取得了战胜蒋介石、建立新中国的伟大胜利。\end{introduction}
最近几天是远东时局发生极大变动的时候。日本帝国主义投降的大势已经定了。日本投降的决定因素是苏联参战。百万红军进入中国的东北,这个力量是不可抗拒的。日本帝国主义已经不能继续打下去了。中国人民的艰苦抗战,已经取得了胜利。抗日战争当作一个历史阶段来说,已经过去了。\\
  在这种形势下面,中国国内的阶级关系,国共两党的关系,现在怎么样,将来可能怎么样?我党的方针怎么样?这是全国人民很关心的问题,是全党同志很关心的问题。\\
  国民党怎么样?看它的过去,就可以知道它的现在;看它的过去和现在,就可以知道它的将来。这个党过去打过整整十年的反革命内战。在抗日战争中间,在一九四〇年、一九四一年和一九四三年,它发动过三次大规模的反共高潮\footnote[1]{ 关于国民党反动派发动三次反共高潮的经过,见本书第三卷《评国民党十一中全会和三届二次国民参政会》。},每一次都准备发展成为全国范围的内战,仅仅由于我党的正确政策和全国人民的反对,才没有实现。中国大地主大资产阶级的政治代表蒋介石,大家知道,是一个极端残忍和极端阴险的家伙。他的政策是袖手旁观,等待胜利,保存实力,准备内战。果然胜利被等来了,这位“委员长”现在要“下山”\footnote[2]{ 这里所说的“山”,即峨眉山,实际上是泛指中国西南、西北部的山区。自一九三八年武汉被日军侵占以后,蒋介石自己和他所指挥的很大一部分部队就躲在这些山区里,坐观解放区军民在敌后同日本侵略者作艰苦的斗争。}了。八年来我们和蒋介石调了一个位置:以前我们在山上,他在水边\footnote[3]{ 抗日战争以前,中国共产党领导的革命根据地,大多数建立在山区。当时,蒋介石的统治中心是在沿江、沿海的大城市。所以毛泽东说一在“山上”,一在“水边”。};抗日时期,我们在敌后,他上了山。现在他要下山了,要下山来抢夺抗战胜利的果实了。\\
  我们解放区的人民和军队,八年来在毫无外援的情况之下,完全靠着自己的努力,解放了广大的国土,抗击了大部的侵华日军和几乎全部的伪军。由于我们的坚决抗战,英勇奋斗,大后方\footnote[4]{ 见本书第二卷《和中央社、扫荡报、新民报三记者的谈话》注〔3〕。}的二万万人民才没有受到日本侵略者摧残,二万万人民所在的地方才没有被日本侵略者占领。蒋介石躲在峨眉山上,前面有给他守卫的,这就是解放区,就是解放区的人民和军队。我们保卫了大后方的二万万人民,同时也就保卫了这位“委员长”,给了他袖手旁观、坐待胜利的时间和地方。时间——八年零一个月,地方——二万万人民所在的地方,这些条件是我们给他的。没有我们,他是旁观不成的。那末,“委员长”是不是感谢我们呢?他不!此人历来是不知感恩的。蒋介石是怎样上台的?是靠北伐战争,靠第一次国共合作,靠那时候人民还没有摸清他的底细,还拥护他。他上了台,非但不感谢人民,还把人民一个巴掌打了下去,把人民推入了十年内战的血海。这段历史同志们都是知道的。这一次抗日战争,中国人民又保卫了他。现在抗日战争胜利了,日本要投降了,他绝不感谢人民,相反地,翻一翻一九二七年的老账,还想照样来干。蒋介石说中国过去没有过“内战”,只有过“剿匪”;不管叫做什么吧,总之是要发动反人民的内战,要屠杀人民。\\
  当全国规模的内战还没有爆发的时候,人民中间和我们党内的许多同志中间,对于这个问题还不是都认识得清楚的。因为大规模的内战还没有到来,内战还不普遍、不公开、不大量,就有许多人认为:“不一定吧!”还有许多人怕打内战。怕,是有理由的,因为过去打了十年,抗战又打八年,再打,怎么得了。产生怕的情绪是很自然的。对于蒋介石发动内战的阴谋,我党所采取的方针是明确的和一贯的,这就是坚决反对内战,不赞成内战,要阻止内战。今后我们还要以极大的努力和耐心领导着人民来制止内战。但是,必须清醒地看到,内战危险是十分严重的,因为蒋介石的方针已经定了。按照蒋介石的方针,是要打内战的。按照我们的方针,人民的方针,是不要打内战的。不要打内战的只是中国共产党和中国人民,可惜不包括蒋介石和国民党。一个不要打,一个要打。如果两方面都不要打,就打不起来。现在不要打的只是一个方面,并且这一方面的力量又还不足以制止那一方面,所以内战危险就十分严重。\\
  蒋介石要坚持独裁和内战的反动方针,我党曾经及时地指明了这一点。在党的七次代表大会以前、七次代表大会中间和七次代表大会以后,我们曾经进行了相当充分的工作,唤起人民对于内战危险的注意,使全国人民、我们的党员和军队,早有精神准备。这一点很重要,有这一点和没有这一点是大不相同的。一九二七年的时候,我党还是幼年的党,对于蒋介石的反革命的突然袭击毫无精神准备,以致人民已经取得的胜利果实跟着就失掉了,人民遭受了长期的灾难,光明的中国变成了黑暗的中国。这一次不同了,我党已经有了三次革命的丰富经验,党的政治成熟程度已经大大提高了。党中央再三再四地讲明内战危险,使全国人民、全党同志和党所领导的军队,都处于有准备的状态中。\\
  蒋介石对于人民是寸权必夺,寸利必得。我们呢?我们的方针是针锋相对,寸土必争。我们是按照蒋介石的办法办事。蒋介石总是要强迫人民接受战争,他左手拿着刀,右手也拿着刀。我们就按照他的办法,也拿起刀来。这是经过调查研究以后才找到的办法。这个调查研究很重要。看到人家手里拿着东西了,我们就要调查一下。他手里拿的是什么?是刀。刀有什么用处?可以杀人。他要拿刀杀谁?要杀人民。调查了这几件事,再调查一下:中国人民也有手,也可以拿刀,没有刀可以打一把。中国人民经过长期的调查研究,发现了这个真理。军阀、地主、土豪劣绅、帝国主义,手里都拿着刀,要杀人。人民懂得了,就照样办理。我们有些人,对于这个调查研究常不注意。例如陈独秀\footnote[5]{ 见本书第一卷《中国革命战争的战略问题》注〔4〕。},他就不知道拿着刀可以杀人。有人说,这是普遍的日常真理,共产党的领导人还会不知道?这很难说。他没有调查研究就不懂得这件事,所以我们给他起个名字,叫做机会主义者。没有调查研究就没有发言权,我们取消了他的发言权。我们采取了和陈独秀不同的办法,使被压迫、被屠杀的人民拿起刀来,谁如果再要杀我们,我们就照样办理。不久以前,国民党调了六个师来打我们关中分区,有三个师打进来了,占领了宽一百里、长二十里的地方。我们也照他的办法,把在这宽一百里、长二十里地面上的国民党军队,干净、彻底、全部消灭之\footnote[6]{ 一九四五年七月二十一日,胡宗南所部国民党军暂编第五十九师突向陕甘宁边区关中分区淳化县的爷台山发起攻击。随后又以预备第三师和暂编第十五师加入进攻。边区部队于七月二十七日主动撤出爷台山及其以西四十一个村庄。国民党军占领上述地区后,继续向边区腹地进犯。边区部队于八月八日对进犯的国民党军队发起自卫反击,收复了爷台山地区。}。我们是针锋相对,寸土必争,绝不让国民党轻轻易易地占我们的地方,杀我们的人。当然,寸土必争,并不是说要像过去“左”倾路线那样“不放弃根据地的一寸土地”。这一回我们就放弃了宽一百里、长二十里的地方。七月底放弃,八月初收回。在皖南事变以后,有一次,国民党的联络参谋问我们的动向如何。我说,你天天在延安还不清楚?“何反我亦反,何停我亦停”\footnote[7]{ 国民党的联络参谋,是抗日战争时期国民党政府派到延安做联络工作的人员。“何”,指何应钦。一九四〇年十月十九日和十二月八日,蒋介石曾经用国民政府军事委员会参谋总长何应钦和副参谋总长白崇禧的名义先后发出“皓”“齐”两电,对坚持敌后抗战的八路军、新四军大肆诬蔑,强迫命令黄河以南的八路军、新四军限期撤至黄河以北。接着,国民党反动派即制造了袭击新四军北移部队的皖南事变。中国共产党对此进行了针锋相对的斗争。毛泽东在这里指何应钦为发动反共高潮的国民党反动派的代表,实际上就是指蒋介石。}。那时候还没有提出蒋介石的名字,只提何应钦。现在是:“蒋反我亦反,蒋停我亦停。”照他的办法办理。现在蒋介石已经在磨刀了,因此,我们也要磨刀。\\
  人民得到的权利,绝不允许轻易丧失,必须用战斗来保卫。我们是不要内战的。如果蒋介石一定要强迫中国人民接受内战,为了自卫,为了保卫解放区人民的生命、财产、权利和幸福,我们就只好拿起武器和他作战。这个内战是他强迫我们打的。如果我们打不赢,不怪天也不怪地,只怪自己没有打赢。但是谁要想轻轻易易地把人民已经得到的权利抢去或者骗去,那是办不到的。去年有个美国记者问我:“你们办事,是谁给的权力?”我说:“人民给的。”如果不是人民给的,还有谁给呢?当权的国民党没有给。国民党是不承认我们的。我们参加国民参政会,按照参政会条例的规定,是以“文化团体”的资格\footnote[8]{ 国民参政会是一九三八年国民党政府成立的一个仅属咨询性质的机关,对国民党政府的政策措施没有任何约束权力。参政员都是由国民党政府指定的,虽也包含了一些各抗日党派的代表,但是国民党员占大多数。国民党政府不承认各抗日党派的平等合法地位,也不让它们的代表以党派代表的身份参加国民参政会。国民党政府一九三八年四月颁布的《国民参政会组织条例》第三条中规定:“曾在各重要文化团体或经济团体服务三年以上,着有信望,或努力国事,信望久着之人员”,得为国民参政会参政员。当时国民党就是按照这项规定,指定了中国共产党的参政员。}。我们说,我们不是“文化团体”,我们有军队,是“武化团体”。今年三月一日蒋介石说过:共产党交出军队,才有合法地位。蒋介石的这句话,现在还适用。我们没有交出军队,所以没有合法地位,我们是“无法无天”。我们的责任,是向人民负责。每句话,每个行动,每项政策,都要适合人民的利益,如果有了错误,定要改正,这就叫向人民负责。同志们,人民要解放,就把权力委托给能够代表他们的、能够忠实为他们办事的人,这就是我们共产党人。我们当了人民的代表,必须代表得好,不要像陈独秀。陈独秀对于反革命向人民的进攻,不是采取针锋相对、寸土必争的方针,结果在一九二七年的几个月内,把人民已经取得的权利统统丧失干净。这一次我们就要注意。我们和陈独秀的方针绝不相同,任何骗人的东西都骗不了我们。我们要有清醒的头脑和正确的方针,要不犯错误。\\
  抗战胜利的果实应该属谁?这是很明白的。比如一棵桃树,树上结了桃子,这桃子就是胜利果实。桃子该由谁摘?这要问桃树是谁栽的,谁挑水浇的。蒋介石蹲在山上一担水也不挑,现在他却把手伸得老长老长地要摘桃子。他说,此桃子的所有权属于我蒋介石,我是地主,你们是农奴,我不准你们摘。我们在报上驳了他\footnote[9]{ 见本卷《蒋介石在挑动内战》。}。我们说,你没有挑过水,所以没有摘桃子的权利。我们解放区的人民天天浇水,最有权利摘的应该是我们。同志们,抗战胜利是人民流血牺牲得来的,抗战的胜利应当是人民的胜利,抗战的果实应当归给人民。至于蒋介石呢,他消极抗战,积极反共,是人民抗战的绊脚石。现在这块绊脚石却要出来垄断胜利果实,要使抗战胜利后的中国仍然回到抗战前的老样子,不许有丝毫的改变。这样就发生了斗争。同志们,这是一场很严重的斗争。\\
  抗战胜利的果实应该属于人民,这是一个问题;但是,胜利果实究竟落到谁手,能不能归于人民,这是另一个问题。不要以为胜利的果实都靠得住落在人民的手里。一批大桃子,例如上海、南京、杭州等大城市,那是要被蒋介石抢去的。蒋介石勾结着美国帝国主义,在那些地方他们的力量占优势,革命的人民还基本上只能占领乡村。另一批桃子是双方要争夺的。太原以北的同蒲,平绥中段,北宁,郑州以北的平汉,正太,白晋\footnote[10]{ 白晋,指当时山西省东南部由祁县的白圭到晋城的一条未完成的铁路。},德石,津浦,胶济,郑州以东的陇海,这些地方的中小城市是必争的,这一批中小桃子都是解放区人民流血流汗灌溉起来的。究竟这些地方能不能落到人民的手里,现在还不能说。现在只能讲两个字:力争。靠得住落在人民手里的有没有呢?有的,河北、察哈尔、热河\footnote[11]{ 察哈尔,原来是一个省,一九五二年撤销,原辖地区划归河北、山西两省。热河,原来也是一个省,一九五五年撤销,原辖地区划归河北、辽宁两省和内蒙古自治区。}、山西的大部、山东、江苏的北部,这些地方的大块乡村和大批城市,乡村和乡村打成一片,上百的城市一块,七八十个城市一块,四五十个城市一块,大小三、四、五、六块。什么城市?中等城市和小城市。这是靠得住的,我们的力量能够取得这批胜利果实。得到了这批果实,在中国革命的历史上还是头一次。历史上,我们只在一九三一年下半年打破了敌人的第三次“围剿”以后,江西中央区联合起来有过二十一个县城\footnote[12]{ 这里所说的二十一个县城,是指江西省的瑞金、会昌、寻乌、安远、信丰、于都、兴国、宁都、广昌、石城、黎川和福建省的建宁、泰宁、宁化、清流、归化(今明溪)、龙岩、长汀、连城、上杭、永定。},但是还没有中等城市。二十一个小城市联在一起,最多的时候有过二百五十万人口。依靠着这些,中国人民就能奋斗那样久的时间,取得那样大的胜利,粉碎那样大的“围剿”。后来我们打输了,这不能怪蒋介石,要怪我们自己没有打好。如果这一次,大小城市几十个联成一块,有了三四五六块的话,中国人民就有了三四五六个大于江西中央区的革命根据地,中国革命的形势就很可观了。\\
  从整个形势看来,抗日战争的阶段过去了,新的情况和任务是国内斗争。蒋介石说要“建国”,今后就是建什么国的斗争。是建立一个无产阶级领导的人民大众的新民主主义的国家呢,还是建立一个大地主大资产阶级专政的半殖民地半封建的国家?这将是一场很复杂的斗争。目前这个斗争表现为蒋介石要篡夺抗战胜利果实和我们反对他的篡夺的斗争。这个时期如果有机会主义的话,那就是不力争,自愿地把人民应得的果实送给蒋介石。\\
  公开的全面的内战会不会爆发?这决定于国内的因素和国际的因素。国内的因素主要是我们的力量和觉悟程度。会不会因为国际国内的大势所趋和人心所向,经过我们的奋斗,使内战限制在局部的范围,或者使全面内战拖延时间爆发呢?这种可能性是有的。\\
  蒋介石要放手发动内战也有许多困难。第一,解放区有一万万人民、一百万军队、二百多万民兵。第二,国民党统治地区的觉悟的人民是反对内战的,这对蒋介石是一种牵制。第三,国民党内部也有一部分人不赞成内战。目前的形势和一九二七年的时候是大不相同了。特别是我党目前的情况和一九二七年时候的情况大不相同。那时候的党是幼年的党,没有清醒的头脑,没有武装斗争的经验,没有针锋相对的方针。现在党的觉悟程度已经大大地提高了。\\
  除了我们的觉悟,无产阶级先锋队的觉悟问题以外,还有一个人民群众的觉悟问题。当着人民还不觉悟的时候,把革命果实送给人家是完全可能的。这种事在历史上曾经有过。今天中国人民的觉悟程度也已经是大大地提高了。我党在人民中的威信从来没有过现在这样高。但是,在人民中间,主要是在日本占领区和国民党统治区的人民中间,还有相当多的人相信蒋介石,存在着对于国民党和美国的幻想,蒋介石也在努力散布这种幻想。中国人民中有这样一部分人还不觉悟,就是说明我们的宣传工作和组织工作还做得很不够。人民的觉悟不是容易的,要去掉人民脑子中的错误思想,需要我们做很多切切实实的工作。对于中国人民脑子中的落后的东西,我们要去扫除,就像用扫帚打扫房子一样。从来没有不经过打扫而自动去掉的灰尘。我们要在人民群众中间,广泛地进行宣传教育工作,使人民认识到中国的真实情况和动向,对于自己的力量具备信心。\\
  人民靠我们去组织。中国的反动分子,靠我们组织起人民去把他打倒。凡是反动的东西,你不打,他就不倒。这也和扫地一样,扫帚不到,灰尘照例不会自己跑掉。陕甘宁边区南面有条介子河。介子河南是洛川,河北是富县。河南河北两个世界。河南是国民党的,因为我们没有去,人民没有组织起来,龌龊的东西多得很。我们有些同志就是相信政治影响,以为靠着影响就可以解决问题。那是迷信。一九三六年,我们住在保安\footnote[13]{ 保安是当时陕西省西北部的一个县,即现在的志丹县。中共中央从一九三六年七月初至一九三七年一月上旬驻在保安。以后迁往延安。}。离保安四五十里的地方有个地主豪绅的土围子。那时候党中央的所在地就在保安,政治影响可谓大矣,可是那个土围子里的反革命就是死不投降。我们在南面扫、北面扫,都不行,后来把扫帚搞到里面去扫,他才说:“啊哟!我不干了。”\footnote[14]{ 这里说的土围子,是指保安县西南的旦八寨。该寨有二百余户人家,地形极为险要。当地地主豪绅兼民团团总曹俊章率反动武装百余人,长期盘据该寨。红军多次围攻未能打下。一九三六年八月,红军一面用地方武装围困,一面争取寨内基本群众,瓦解寨内敌军。同年十二月,曹俊章率少数人员逃跑,旦八寨获得解放。}世界上的事情,都是这样。钟不敲是不响的。桌子不搬是不走的。苏联红军不进入东北,日本就不投降。我们的军队不去打,敌伪就不缴枪。扫帚到了,政治影响才能充分发生效力。我们的扫帚就是共产党、八路军和新四军。手里拿着扫帚就要研究扫的办法,不要躺在床上,以为会来一阵什么大风,把灰尘统统刮掉。我们马克思主义者是革命的现实主义者,绝不作空想。中国有句古话说:“黎明即起,洒扫庭除。”\footnote[15]{ 见明末清初人朱柏庐所著《治家格言》。}黎明者,天刚亮也。古人告诉我们,在天刚亮的时候,就要起来打扫。这是告诉了我们一项任务。只有这样想,这样做,才有益处,也才有工作做。中国的地面很大,要靠我们一寸一寸地去扫。\\
  我们的方针要放在什么基点上?放在自己力量的基点上,叫做自力更生。我们并不孤立,全世界一切反对帝国主义的国家和人民都是我们的朋友。但是我们强调自力更生,我们能够依靠自己组织的力量,打败一切中外反动派。蒋介石同我们相反,他完全是依靠美国帝国主义的帮助,把美国帝国主义作为靠山。独裁、内战和卖国三位一体,这一贯是蒋介石方针的基本点。美国帝国主义要帮助蒋介石打内战,要把中国变成美国的附庸,它的这个方针也是老早定了的。但是,美国帝国主义是外强中干的。我们要有清醒的头脑,这里包括不相信帝国主义的“好话”和不害怕帝国主义的恐吓。曾经有个美国人向我说:“你们要听一听赫尔利的话,派几个人到国民党政府里去做官。”\footnote[16]{ 这里所说的“美国人”,是指美军在延安的观察组组长包瑞德上校。这个观察组是当时参加对日作战的美国军队在一九四四年取得中国共产党同意后派往延安的。赫尔利,参见本书第三卷《愚公移山》注〔3〕。}我说:“捆住手脚的官不好做,我们不做。要做,就得放开手放开脚,自由自在地做,这就是在民主的基础上成立联合政府。”他说:“不做不好。”我问:“为什么不好?”他说:“第一,美国人会骂你们;第二,美国人要给蒋介石撑腰。”我说:“你们吃饱了面包,睡足了觉,要骂人,要撑蒋介石的腰,这是你们美国人的事,我不干涉。现在我们有的是小米加步枪,你们有的是面包加大炮。你们爱撑蒋介石的腰就撑,愿撑多久就撑多久。不过要记住一条,中国是什么人的中国?中国绝不是蒋介石的,中国是中国人民的。总有一天你们会撑不下去!”同志们,这个美国人的话是吓人的。帝国主义者就会吓人的那一套,殖民地有许多人也就是怕吓。他们以为所有殖民地的人都怕吓,但是不知道中国有这么一些人是不怕那一套的。我们过去对于美国的扶蒋反共政策作了公开的批评和揭露,这是必要的,今后还要继续揭穿它。\\
  苏联出兵了,红军来援助中国人民驱逐侵略者,这是中国历史上从来没有过的事。这件事情所发生的影响,是不可估计的。美国和蒋介石的宣传机关,想拿两颗原子弹把红军的政治影响扫掉。但是扫不掉,没有那样容易。原子弹能不能解决战争?不能。原子弹不能使日本投降。只有原子弹而没有人民的斗争,原子弹是空的。假如原子弹能够解决战争,为什么还要请苏联出兵?为什么投了两颗原子弹日本还不投降,而苏联一出兵日本就投降了呢?我们有些同志也相信原子弹了不起,这是很错误的。这些同志看问题,还不如一个英国贵族。英国有个勋爵,叫蒙巴顿。他说,认为原子弹能解决战争是最大的错误\footnote[17]{ 蒙巴顿(一九〇〇——一九七九),当时担任东南亚盟军最高指挥官。一九四五年八月九日,他发表谈话,欢迎苏联参加对日作战,并说:“认为原子弹会停止远东战争是一个最大的错误。”}。我们这些同志比蒙巴顿还落后。这些同志把原子弹看得神乎其神,是受了什么影响呢?是资产阶级的影响。这种影响是从哪里来的呢?是从资产阶级的学校教育中来的,是从资产阶级的报纸、通讯社来的。有两种世界观、方法论:无产阶级的世界观、方法论和资产阶级的世界观、方法论。这些同志把资产阶级的世界观、方法论,经常拿在手里;无产阶级的世界观、方法论,却经常丢在脑后。我们队伍中的唯武器论,单纯军事观点,官僚主义、脱离群众的作风,个人主义思想,等等,都是资产阶级的影响。对于我们队伍中的这些资产阶级的东西,也要像打扫灰尘一样,常常扫除。\\
  苏联的参战,决定了日本的投降,中国的时局发展到了一个新的时期。新时期和抗日战争时期之间有一个过渡阶段。过渡阶段的斗争,就是反对蒋介石篡夺抗战胜利果实的斗争。蒋介石要发动全国规模的内战,他的方针已经定了,我们对此要有准备。全国性的内战不论哪一天爆发,我们都要准备好。早一点,明天早上就打吧,我们也在准备着。这是第一条。现在的国际国内形势,有可能把内战暂时限制在局部范围,内战可能暂时是若干地方性的战争。这是第二条。第一条我们准备着,第二条是早已如此。总而言之,我们要有准备。有了准备,就能恰当地应付各种复杂的局面。\\
\newpage\section*{\myformat{蒋介石在挑动内战}\\\myformat{(一九四五年八月十三日)}}\addcontentsline{toc}{section}{蒋介石在挑动内战}
\begin{introduction}\item  这是毛泽东为新华社写的评论。\end{introduction}
国民党中央宣传部发言人发表谈话说,第十八集团军朱德总司令于八月十日在延安总部所发表的限令敌伪投降的命令\footnote[1]{ 一九四五年八月十日,延安总部朱德总司令为日本投降事向各解放区所有武装部队发布命令,全文如下:“日本已宣布无条件投降,同盟国在波茨坦宣言基础上将会商受降办法。因此,我特向各解放区所有武装部队发布下列命令:一、各解放区任何抗日武装部队均得依据波茨坦宣言规定,向其附近各城镇交通要道之敌人军队及其指挥机关送出通牒,限其于一定时间向我作战部队缴出全部武装,在缴械后,我军当依优待俘虏条例给以生命安全之保护。二、各解放区任何抗日武装部队均得向其附近之一切伪军伪政权送出通牒,限其于敌寇投降签字前,率队反正,听候编遣,过期即须全部缴出武装。三、各解放区所有抗日武装部队,如遇敌伪武装部队拒绝投降缴械,即应予以坚决消灭。四、我军对任何敌伪所占城镇交通要道,都有全权派兵接收,进入占领,实行军事管制,维持秩序,并委任专员负责管理该地区之一切行政事宜,如有任何破坏或反抗事件发生,均须以汉奸论罪。”接着,在八月十一日,延安总部又连续发布了六道命令,命令晋绥解放区贺龙领导的武装部队、晋察冀解放区聂荣臻领导的武装部队、冀热辽解放区的武装部队向内蒙和东北进军;命令山西解放区的武装部队肃清同蒲路沿线和汾河流域的日伪军;命令各解放区的武装部队,向一切敌占交通要道展开积极进攻,迫使日伪军投降。各解放区的解放军坚决地执行了这些命令,并取得了重大的胜利。},是一种“唐突和非法之行动”。这种评论,荒谬绝伦。根据这种意见,可以逻辑地解释为朱德总司令根据波茨坦公告\footnote[2]{ 指一九四五年七月二十六日中、美、英三国在波茨坦会议过程中发表的促令日本投降的公告。其主要内容为:盟军对日作战,直至其停止抵抗为止;日本政府应立即宣布无条件投降;日本军国主义必须永远肃清;日本军队必须完全解除武装;日本的军事工业必须拆除;日本的战犯必须交付审判;开罗宣言必须实施,即日本必须放弃前所掠取的土地,如朝鲜,中国的满洲、台湾、澎湖列岛等地,日本的领土限于本州岛、北海道、九州岛、四国和其它小岛之内;同盟国军队占领日本直到日本民主政府建立以后为止。八月八日,苏联对日宣战后,亦签字于该公告。}和敌人投降的意向,下令给所属部队促使敌伪投降,倒反错了,应该劝使敌伪拒绝投降,才是对的,才算合法。无怪中国法西斯头子独夫民贼蒋介石,在敌人尚未真正接受投降之前,敢于“命令”解放区抗日军队“应就原地驻防待命”,束手让敌人来打。无怪这同一个法西斯头子,又敢于“命令”所谓地下军(实际上就是实行“曲线救国”\footnote[3]{ 见本书第三卷《论联合政府》注〔9〕。}的伪军和与敌伪合流的戴笠系特务\footnote[4]{ 戴笠系特务,是以国民政府军事委员会调查统计局(简称“军统”)局长戴笠为首的特务系统。“军统”成立于一九三八年八月,其前身是一九三二年四月成立的复兴社的核心组织力行社的特务处。“军统”是国民党庞大的特务组织之一,它残酷地迫害共产党人和革命人士,破坏人民革命,同时也对国民党军队、警察及行政机关等进行监视和控制。})和伪军,“负责维持地方治安”,而不许解放区抗日军队向敌伪“擅自行动”。这样的敌我倒置,真是由蒋介石自己招供,活画出他一贯勾结敌伪、消除异己的全部心理了。可是中国解放区的人民抗日军队,绝不会中此毒计。他们知道:朱德总司令的命令,正是坚决地执行波茨坦公告第二项的规定:“对日作战,直至其停止抵抗为止。”而蒋介石的所谓“命令”,正是违反了他自己签字的波茨坦公告。只要拿这一比,就知道谁是不“恪守盟邦共同协议之规定”了。\\
  国民党中央宣传部发言人的评论和蒋介石的“命令”,从头到尾都是在挑拨内战,其目的是在当着国内外集中注意力于日本无条件投降之际,找一个借口,好在抗战结束时,马上转入内战。其实,国民党反动派是蠢得可怜的。他们找了朱德总司令命令敌伪投降缴械当作借口。这难道也算得一个聪明的借口吗?不,这样来找借口,只足以证明国民党反动派把敌伪看得比同胞还可亲些,把同胞看得比敌伪还可恨些。淳化事件\footnote[5]{ 淳化事件,即一九四五年七月国民党军队进犯陕甘宁边区关中分区淳化县爷台山等地的事件。见本卷《抗日战争胜利后的时局和我们的方针》注〔6〕。},明明是胡宗南军队攻入陕甘宁边区,挑拨内战,国民党反动派却说是中共的“谣言攻势”。淳化事件这个借口,好容易被国民党反动派找着了,却被中外舆论界一下子识破,于是又说八路军、新四军不该要敌伪缴枪了。八年抗战,八路军、新四军受尽了蒋介石和日本人夹击围攻的苦楚,现在抗战瞬将结束,蒋介石又在暗示日本人(加上他亲爱的伪军),叫他们不要向八路军、新四军缴枪,说是只能缴给我蒋介石。蒋介石剩下一句话没有说,这一句就是:好使我拿了这些枪杀共产党,并破坏中国和世界的和平。不是吗?叫日本人缴枪给蒋介石,叫伪军“负责维持地方治安”,这会有什么结果呢?只有一个结果,就是以宁渝合流\footnote[6]{ “宁”指当时南京汉奸集团的伪政权,“渝”指当时重庆的蒋介石政权。“宁渝合流”,是当时日本帝国主义与南京方面以及重庆蒋介石方面共同策动的一个政治阴谋。}、蒋伪合作,去代替“中日提携”、日伪合作;以蒋介石的反共建国,去代替日本人、汪精卫\footnote[7]{ 见本书第一卷《论反对日本帝国主义的策略》注〔31〕。}的反共建国。这难道还不是违背波茨坦公告吗?抗战一旦结束,内战危险立即严重威胁全国人民,这一点难道还有疑义吗?现在我们向全国同胞和世界盟邦呼吁,一致起来,同解放区人民一道,坚决制止这个危及世界和平的中国内战。\\
  究竟谁有权接受日伪的投降呢?中国解放区的抗日军队,在国民党政府毫无接济又不承认的条件下,完全靠自己的努力和人民的拥护,得以独力解放了广大的国土和一万万以上的人民,抗击着侵华敌军百分之五十六和伪军的百分之九十五。要是没有这一个军队,中国绝无今天的局面!实在说,在中国境内,只有解放区抗日军队才有接受敌伪军投降的权利。至于蒋介石,他的政策是袖手旁观,坐待胜利,实在没有丝毫权利接受敌伪投降。\\
  我们要向全国同胞和全世界人民宣布:重庆统帅部,不能代表中国人民和中国真正抗日的军队;中国人民要求,中国解放区抗日军队有在朱德总司令指挥之下,直接派遣他的代表参加四大盟国接受日本投降和军事管制日本的权利,并且有参加将来和会的权利。要不是这样做,中国人民将认为是很不恰当的。\\
\newpage\section*{\myformat{第十八集团军总司令给蒋介石的两个电报}\\\myformat{(一九四五年八月)}}\addcontentsline{toc}{section}{第十八集团军总司令给蒋介石的两个电报}
\begin{introduction}\item  这两个电报是毛泽东为第十八集团军总司令写的。当时蒋介石政府,在日本侵略者宣布投降但尚未实行投降之际,在美国帝国主义的武力援助下,垄断接受日本投降的权利,并且借口受降调运大军向解放区进逼,积极准备反革命内战。毛泽东写第一个电报的目的,就在于揭露蒋介石的反革命面目,教育全国人民警惕蒋介石的内战阴谋。在第二个电报里,进一步揭穿了蒋介石集团准备内战的阴谋,并提出了中国共产党关于制止内战的六项主张。为着同样的目的,毛泽东还为新华社写了两篇评论,即本卷《蒋介石在挑动内战》和《评蒋介石发言人谈话》。由于中国共产党采取了这种决不被蒋介石的反动气焰所吓倒的坚定的果断的立场,就使解放区和解放军得到了迅速的扩大,并且使蒋介石在国内外反对中国内战的强大政治压力之下,不得不改变策略,装出和平姿态,邀请毛泽东到重庆举行和平谈判。\end{introduction}
\subsection*{\myformat{一 八月十三日的电报}}
我们从重庆广播电台收到中央社两个消息,一个是你给我们的命令,一个是你给各战区将士的命令。在你给我们的命令上说:“所有该集团军所属部队,应就原地驻防待命。”此外,还有不许向敌人收缴枪械一类的话。你给各战区将士的命令,据中央社重庆十一日电是这样说的:“最高统帅部今日电令各战区将士加紧作战努力,一切依照既定军事计划与命令积极推进,勿稍松懈。”我们认为这两个命令是互相矛盾的。照前一个命令,“驻防待命”,不进攻了,不打仗了。现在日本侵略者尚未实行投降,而且每时每刻都在杀中国人,都在同中国军队作战,都在同苏联、美国、英国的军队作战,苏美英的军队也在每时每刻同日本侵略者作战,为什么你叫我们不要打了呢?照后一个命令,我们认为是很好的。“加紧作战,积极推进,勿稍松懈”,这才像个样子。可惜你只把这个命令发给你的嫡系军队,不是发给我们,而发给我们的另是一套。朱德在八月十日下了一个命令给中国各解放区的一切抗日军队\footnote[1]{ 见本卷《蒋介石在挑动内战》注〔1〕。},正是“加紧作战”的意思。再有一点,叫他们在“加紧作战”时,必须命令日本侵略者投降过来,将敌、伪军的武装等件收缴过来。难道这样不是很好的吗?无疑这是很好的,无疑这是符合于中华民族的利益的。可是“驻防待命”一说,确与民族利益不符合。我们认为这个命令你是下错了,并且错得很厉害,使我们不得不向你表示:坚决地拒绝这个命令。因为你给我们的这个命令,不但不公道,而且违背中华民族的民族利益,仅仅有利于日本侵略者和背叛祖国的汉奸们。\\
\subsection*{\myformat{二 八月十六日的电报}}
在我们共同敌人——日本政府已接受波茨坦公告\footnote[2]{ 见本卷《蒋介石在挑动内战》注〔2〕。}宣布投降,但尚未实行投降之际,我代表中国解放区、中国沦陷区一切抗日武装力量和二亿六千万人民,特向你提出下列的声明和要求。\\
  在抗日战争将要胜利结束的时候,我提起你注意目前中国战场上的这样的事实,即在敌伪侵占而为你所放弃的广大沦陷地区中,违背你的意志,经过我们八年的苦战,夺回了近百万平方公里的土地,解放了过一万万的人民,组织了过一百万的正规部队和二百二十多万的民兵,在辽宁、热河、察哈尔、绥远\footnote[3]{ 热河、察哈尔,见本卷《抗日战争胜利后的时局和我们的方针》注〔11〕。绥远,原来也是一个省,一九五四年撤销,原辖地区划归内蒙古自治区。}、河北、山西、陕西、甘肃、宁夏、河南、山东、江苏、安徽、湖北、湖南、江西、浙江、福建、广东十九个省区内建立了十九个大块的解放区\footnote[4]{ 十九个大块的解放区,指陕甘宁、晋绥、晋察冀、冀热辽、晋冀豫、冀鲁豫、山东、苏北、苏中、苏南、淮北、淮南、皖中、浙东、广东、琼崖、湘鄂赣、鄂豫皖、河南。},除少数地区外,大部包围了自一九三七年七七事变以来敌伪所侵占的中国城镇、交通要道和沿海海岸。此外,我们还在中国沦陷区(在这里,有一亿六千万人口)中组织了广大的地下军,打击敌伪。在作战中,我们至今还抗击和包围着侵华(东北不在内)日军的百分之六十九和伪军的百分之九十五。而你的政府和军队,却一向采取袖手旁观、坐待胜利、保存实力、准备内战的方针,对于我们解放区及其军队,不仅不予承认,不予接济,而且以九十四万大军包围和进攻它们。中国解放区全体军民虽受尽了敌伪和你的军队两方面夹击之苦,但丝毫未减弱他们坚持抗战、团结和民主的意志。中国解放区人民和中国共产党,曾经多次向你和你的政府提议召开各党派会议,成立民主的举国一致的联合政府,以便停止内部纷争,动员和统一全中国人民的抗日力量,领导抗日战争取得胜利,保证战后的和平,但都被你和你的政府所拒绝。凡此一切,我们是非常之不满意的。\\
  现在敌国投降将要签字了,而你和你的政府仍然漠视我们的意见,并且于八月十一日下了一个非常无理的命令给我,又命令你的军队以收缴敌人枪械为借口大举向解放区压迫,内战危机空前严重。凡此种种,使得我们不得不向你和你的政府提出下列的要求:\\
  一、你和你的政府及其统帅部,在接受日伪投降、缔结受降后的一切协定和条约的时候,我要求你事先和我们商量,取得一致意见。因为你和你的政府为人民所不满,不能代表中国解放区、中国沦陷区的广大人民和一切抗日的人民武装力量。如果协定和条约中,有涉及中国解放区、中国沦陷区一切抗日的人民武装力量之处,而未事先取得我们同意的时候,我们将保留自己的发言权。\\
  二、中国解放区、中国沦陷区的一切抗日的人民武装力量,有权根据波茨坦公告和同盟国规定的受降办法\footnote[5]{ 一九四五年八月十日,日本政府向苏、中、美、英四国请降。十一日,四国政府复文规定,“日本一切陆海空军当局以及彼等控制下之一切部队(不论其在何处)”,必须“停止积极活动,缴出武器”。},接受我们所包围的日伪军队的投降,收缴其武器资材,并负责实施同盟国在受降后的一切规定。我在八月十日下了一道命令给中国解放区军队,叫他们努力进击敌军,并准备接受敌人投降。八月十五日,我已下令给敌军统帅冈村宁次,叫他率部投降\footnote[6]{ 冈村宁次当时是日本的中国派遣军总司令官。朱德总司令给冈村宁次的命令如下:“(一)日本政府已正式接受波茨坦宣言条款宣布投降。(二)你应下令你所指挥下的一切部队,停止一切军事行动,听候中国解放区八路军、新四军及华南抗日纵队的命令,向我方投降,除被国民党政府的军队所包围的部分外。(三)关于投降事宜,在华北的日军,应由你命令下村定将军派出代表至八路军阜平地区,接受聂荣臻将军的命令;在华东的日军,应由你直接派出代表至新四军军部所在地天长地区,接受陈毅将军的命令;在鄂豫两省的日军,应由你命令在武汉的代表,至新四军第五师大悟山地区,接受李先念将军的命令;在广东的日军,应由你指定在广州的代表,至华南抗日纵队东莞地区,接受曾生将军的命令。(四)所有在华北、华东、华中及华南的日军(被国民党军队包围的日军在外),应暂时保存一切武器、资材,静候我军受降,不得接受八路军、新四军及华南抗日纵队以外之命令。(五)所有华北、华东之飞机、舰船,应即停留原地;但沿黄海、渤海之中国海岸的舰船,应分别集中于连云港、青岛、威海卫、天津。(六)一切物资设备,不得破坏。(七)你及你所指挥的在华北、华东、华中及华南的日军指挥官,对执行上述命令应负绝对的责任。”},但这只限于解放区军队作战的范围内,并不干涉其它区域。我的这些命令,我认为是非常合理、非常符合中国和同盟国的共同利益的。\\
  三、中国解放区、中国沦陷区的广大人民和一切抗日武装力量,应有权派遣自己的代表参加同盟国接受敌人的投降,和处理敌国投降后的工作。\\
  四、中国解放区和一切抗日武装力量,应有权选出自己的代表团,参加将来关于处理日本的和平会议和联合国会议。\\
  五、请你制止内战。其办法就是:凡被解放区军队所包围的敌伪军由解放区军队接受其投降,你的军队则接受被你的军队所包围的敌伪军的投降。这不但是一切战争的通例,尤其是为了避免内战,必须如此。如果你不这样做,势将引起不良后果。关于这一点,我现在向你提出严重的警告,请你不要等闲视之。\\
  六、请你立即废止一党专政,召开各党派会议,成立民主的联合政府,罢免贪官污吏和一切反动分子,惩办汉奸,废止特务机关,承认各党派的合法地位(中国共产党和一切民主党派至今被你和你的政府认为是非法的),取消一切镇压人民自由的反动法令,承认中国解放区的民选政府和抗日军队,撤退包围解放区的军队,释放政治犯,实行经济改革和其它各项民主改革。\\
  此外,我在八月十三日发了一个电报给你,回答你在八月十一日给我的命令,谅你已经收到了。我这里重复声言,你那个命令是完全错误的。你在八月十一日叫我的军队“就原地驻防待命”,不打敌人了。但是不但在八月十一日,就是在今天(八月十六日),日本政府还只在口头上宣布投降,并没有在事实上投降,投降协定尚未签字,投降事实尚未发生。我的这个意见,和英美苏各同盟国的意见是完全一致的。就在你下命令给我的那一天(八月十一日),缅甸前线英军当局宣布:“对日战争仍在进行中。”美军统帅尼米兹\footnote[7]{ 尼米兹(一八八五——一九六六),美国海军上将。当时是美国太平洋舰队总司令兼太平洋战区总司令。}宣布:“不仅战争状态是存在的,而且具有一切毁灭结果的战争,必须继续进行。”苏联远东红军宣布:“敌人必须粉碎,不要留情。”八月十五日,红军总参谋长安东诺夫大将还作了下列声明:“八月十四日日皇所发表的日本投降声明,仅仅是无条件投降的一般宣言,给武装部队关于停止敌对行动的命令尚未发布,而且日本军队还在继续进行抵抗。因此,日本实际投降尚未发生。我们只有在日皇命令其军队停止敌对行为和放下武器,而且这个命令被实际执行的时候,才承认日本军队投降了。鉴于上述各点,远东苏军将继续进行对日攻势作战。”由此看来,一切同盟国的统帅中,只有你一个人下了一个绝对错误的命令。我认为你的这个错误,是由于你的私心而产生的,带着非常严重的性质,这就是说,你的命令有利于敌人。因此,我站在中国和同盟国的共同利益的立场上,坚决地彻底地反对你的命令,直至你公开承认错误,并公开收回这个错误命令之时为止。我现在继续命令我所统帅的军队,配合苏联、美国、英国的军队,坚决向敌人进攻,直至敌人在实际上停止敌对行为、缴出武器,一切祖国的国土完全收复之时为止。我向你声明:我是一个爱国军人,我不能不这样做。\\
  以上各项,我请你早日回答。\\
\newpage\section*{\myformat{评蒋介石发言人谈话}\\\myformat{(一九四五年八月十六日)}}\addcontentsline{toc}{section}{评蒋介石发言人谈话}
\begin{introduction}\item  这是毛泽东为新华社写的一篇评论。\end{introduction}
蒋介石的发言人,于十五日下午在重庆记者招待会上讲关于所谓共产党违反蒋介石委员长对朱德总司令的命令时说:“委员长之命令,必须服从。”“违反者即为人民之公敌。”新华社记者说:这是蒋介石公开发出的全面内战的信号。蒋介石于十一日发出一个背叛民族的命令,在最后消灭日寇的关头,禁止八路军新四军和一切人民军队打日本打伪军。这个命令,当然是绝对不能接受和绝对不应接受的。随后,蒋介石经过他的发言人,就把中国人民的军队宣布为“人民公敌”。这样就表示:蒋介石向中国人民宣布了内战。蒋介石的内战阴谋,当然不是从十一日的命令开始的,这是他在抗战八年中的一贯计划。在这八年中,蒋介石曾于一九四〇年、一九四一年、一九四三年发动过三次大规模的反共高潮\footnote[1]{ 蒋介石发动三次大规模反共高潮的经过,见本书第三卷《评国民党十一中全会和三届二次国民参政会》。},每一次都准备将其发展为全国范围的内战,仅由于中国人民和盟邦人士的反对,才未实现,使蒋介石引为恨事。因此,蒋介石不得不把全国内战改期到抗战结束的时候,这样就来了本月十一日的命令和十五日的谈话。蒋介石为了发动内战,已经发明了种种词令,如所谓“异党”、“奸党”、“奸军”、“叛军”、“奸区”、“匪区”、“不服从军令政令”、“封建割据”、“破坏抗战”、“危害国家”;以及所谓中国过去只有过“剿共”,没有过“内战”,因此也不会有“内战”等等。这一次稍为特别的,是增加了一个新词令,叫做“人民公敌”。但是人们会感觉到,这个发明是愚蠢的。因为在中国,只要提起“人民公敌”,谁都知道这是指着谁。在中国,有这样一个人,他叛变了孙中山的三民主义\footnote[2]{ 见本书第一卷《湖南农民运动考察报告》注〔8〕。}和一九二七年的大革命。他将中国人民推入了十年内战的血海,因而引来了日本帝国主义的侵略。然后,他失魂落魄地拔步便跑,率领一群人,从黑龙江一直退到贵州省。他袖手旁观,坐待胜利。果然,胜利到来了,他叫人民军队“驻防待命”,他叫敌人汉奸“维持治安”,以便他摇摇摆摆地回南京。只要提到这些,中国人民就知道是蒋介石。蒋介石干了这一切,他是不是人民公敌的问题,是否还有争论呢?争论是有的。人民说:是。人民公敌说:不是。只有这个争论。至于人民群里,这样的争论是越来越少了。现在成为问题的,是这个人民公敌,要打内战了。人民怎么办呢?新华社记者说:中国共产党对于蒋介石发动内战一事所取的方针,是明确的和一贯的,这就是反对内战。中国共产党早在日本帝国主义开始侵入中国的时候,就要求停止内战,一致对外。并于一九三六年至一九三七年,以惊人的努力,迫使蒋介石接受了自己的主张,因而实现了抗日战争。在抗日的八年中,中国共产党从没有一次放松了提醒人民,制止内战的危险。去年以来,共产党更以蒋介石所准备好了的在抗战结束时发动全国内战的大阴谋,再三再四地唤起人们的注意。共产党同中国人民和全世界关心中国和平的人士一样,认为新的内战将是一个灾难。但是共产党认为,内战仍然是可以制止和必须制止的。共产党主张成立联合政府,就为制止内战。现在蒋介石拒绝了这个主张,致使内战有一触即发之势。然而,制止蒋介石这一手,是完全有办法的。坚决迅速努力壮大人民的民主力量,由人民解放敌占大城市和解除敌伪武装,如有独夫民贼敢于进犯人民,则取自卫立场,给以坚决的反击,使内战挑拨者无所逞其伎。这就是办法,也只有这个唯一的办法。新华社记者唤起全中国和全世界来反对这样一种最虚伪和最无耻的谎言。这些谎言是说:蒋介石禁止中国人民去解放敌占大城市,禁止他们去解除敌伪武装和建立民主政治,而由他自己到这些大城市去“世袭”(而不是破坏)敌伪的统治,中国的内战反而可以避免。新华社记者说,这是谎言,这种谎言不但显然违反中国人民的民族利益和民主利益,而且直接违反中国近代历史的全部事实。必须永远记得:蒋介石所进行的自一九二七年至一九三七年的十年内战,并不是因为大城市在共产党手中而不在蒋介石手中,恰恰相反,从一九二七年到现在,大城市都不在共产党手中,而是在蒋介石或蒋介石所让与的日本和汉奸手中,正是这样,内战就在全国范围内进行了十年,并局部地继续到现在。必须永远记得:十年内战之所以被停止,抗战中三次反共高潮以及其它无数次挑战(直到最近陕甘宁边区南部蒋介石入犯事件\footnote[3]{ 指一九四五年七月国民党军队进犯陕甘宁边区关中分区淳化县爷台山等地的事件。见本卷《抗日战争胜利后的时局和我们的方针》注〔6〕。})之所以被制止,并不是由于蒋介石的力量强大,恰恰相反,都是由于蒋介石力量相对地不够强大,由于共产党和人民力量相对地强大。十年内战,不是因为全国一切愿望和平害怕战争人士的呼吁(例如过去的“废止内战大同盟”\footnote[4]{ “废止内战大同盟”,一九三二年八月成立于上海,主要是由一些资产阶级人物组成的。他们发表了主张“消弭内战、共御外侮”的宣言。}之类的呼吁)而停止,而是因为中国共产党的武装要求和张学良杨虎城所领导的东北军西北军的武装要求而停止的。三次反共高潮以及其它无数次挑战,不是因为共产党的无限制的让步和服从而打退的,而是因为共产党坚持“人不犯我,我不犯人;人若犯我,我必犯人”的严正自卫态度而打退的。如果共产党毫无力量,毫无骨气,不为民族和人民的利益而奋斗到底,十年内战何能结束?抗日战争何能开始?即令开始,又何能坚持到今天的胜利?又何能让蒋介石辈直到今天还安然活着,在离前线那么远的山坳里发表什么命令谈话呢?中国共产党是坚决反对内战的。“确立内部和平状态”,“成立临时政府,使民众中一切民主分子的代表广泛参加,并确保尽可能从速经由自由选举以建立对于人民意志负责的政府”,这是苏美英三国在克里米亚说的话\footnote[5]{ 这些话引自一九四五年二月十一日苏、美、英三国克里米亚(雅尔塔)会议公报。}。中国共产党正是坚持这个主张,这就是“联合政府”的主张。实现这个主张,就可制止内战。一个条件:要力量。全体人民团结起来,壮大自己的力量,内战就可以制止。\\
\newpage\section*{\myformat{中共中央关于同国民党进行和平谈判的通知}\\\myformat{(一九四五年八月二十六日)}}\addcontentsline{toc}{section}{中共中央关于同国民党进行和平谈判的通知}
\begin{introduction}\item  这是毛泽东在去重庆同蒋介石进行和平谈判的前两天为中共中央起草的对党内的通知。由于中国共产党和中国广大人民坚定地反对蒋介石的内战阴谋,也由于美国帝国主义当时还顾忌世界民主舆论一致反对蒋介石的内战政策和独裁政策,蒋介石在一九四五年八月十四日、二十日和二十三日三次电邀毛泽东到重庆进行和平谈判,当时美国驻中国大使赫尔利还在八月二十七日为此来到延安。中国共产党为了尽一切可能争取和平,也为了在争取和平的过程中揭露美国帝国主义和蒋介石的真面目,以利于团结和教育广大人民,决定派遣毛泽东、周恩来、王若飞到重庆去同国民党进行和平谈判。毛泽东所起草的这个通知,分析了日本宣布投降以后两个星期内中国形势的发展,说明了中共中央关于和平谈判的方针,在谈判中准备作出的某些让步以及对谈判结果的两种可能情况的对策,分别对华北、华东解放区和华中、华南解放区的斗争作了原则的指示,告诉全党绝对不要因为谈判而放松对蒋介石的警惕和斗争。毛泽东等在八月二十八日到重庆,同国民党进行了四十三天的谈判。这次谈判的结果,虽然只发表了一个国共双方代表会谈纪要(即《双十协定》),但是在政治上却使中国共产党获得了极大的主动,而使国民党陷入被动,因而是成功的。十月十一日,毛泽东回到延安。周恩来、王若飞仍在重庆继续谈判。关于这次谈判的结果,见本卷《关于重庆谈判》一文。\end{introduction}
日寇迅速投降,改变了整个形势。蒋介石垄断了受降权利,大城要道暂时(一个阶段内)不能属于我们。但是华北方面,我们还要力争,凡能争得者应用全力争之。两星期来,我军收复大小五十九个城市和广大乡村,连以前所有,共有城市一百七十五个,获得了伟大的胜利。华北方面,收复了威海卫、烟台、龙口、益都、淄川、杨柳青、毕克齐、博爱、张家口、集宁、丰镇等处,我军威震华北,配合苏军和蒙古军进抵长城之声势,造成了我党的有利地位。今后一时期内仍应继续攻势,以期尽可能夺取平绥线、同蒲北段、正太路、德石路、白晋路\footnote[1]{ 白晋路,指当时山西省东南部由祁县的白圭到晋城的一条未完成的铁路。}、道清路,切断北宁、平汉、津浦、胶济、陇海、沪宁各路,凡能控制者均控制之,哪怕暂时也好。同时以必要力量,尽量广占乡村和府城县城小市镇。例如新四军占领了南京、太湖、天目山之间许多县城和江淮间许多县城,山东占领了整个胶东半岛,晋绥占领了平绥路南北许多城市,就造成了极好的形势。再有一时期攻势,我党可能控制江北、淮北、山东、河北、山西、绥远\footnote[2]{ 绥远,原来是一个省,一九五四年撤销,原辖地区划归内蒙古自治区。}的绝对大部分,热察\footnote[3]{ 指热河、察哈尔,见本卷《抗日战争胜利后的时局和我们的方针》注〔11〕。}两个全省和辽宁一部。\\
  现在苏美英三国均不赞成中国内战\footnote[4]{ 在日本投降前后的一个时期内,苏、美、英三国都表示不赞成中国发生内战。但是不久以后的事实证明,美国所谓不赞成中国内战的声明,只不过是它用来作为积极帮助国民党反动政府准备反革命内战的掩护而已。},我党又提出和平、民主、团结三大口号\footnote[5]{ 和平、民主、团结三大口号,是一九四五年八月二十五日中国共产党中央委员会《对于目前时局的宣言》中提出的。宣言指出:在日本帝国主义投降以后,“我全民族面前的重大任务是:巩固国内团结,保证国内和平,实现民主,改善民生,以便在和平民主团结的基础上,实现全国的统一,建设独立自由与富强的新中国”。},并派毛泽东、周恩来、王若飞三同志赴渝和蒋介石商量团结建国大计,中国反动派的内战阴谋,可能被挫折下去。国民党在取得沪宁等地、接通海洋和收缴敌械、收编伪军之后,较之过去加强了它的地位,但是仍然百孔千疮,内部矛盾甚多,困难甚大。在内外压力下,可能在谈判后,有条件地承认我党地位,我党亦有条件地承认国民党的地位,造成两党合作(加上民主同盟\footnote[6]{ 见本书第三卷《论联合政府》注〔16〕。}等)、和平发展的新阶段。假如此种局面出现之后,我党应当努力学会合法斗争的一切方法,加紧国民党区域城市、农村、军队三大工作(均是我之弱点)。在谈判中,国民党必定要求我方大大缩小解放区的土地和解放军的数量,并不许发纸币,我方亦准备给以必要的不伤害人民根本利益的让步。无此让步,不能击破国民党的内战阴谋,不能取得政治上的主动地位,不能取得国际舆论和国内中间派的同情,不能换得我党的合法地位和和平局面。但是让步是有限度的,以不伤害人民根本利益为原则。\\
  在我党采取上述步骤后,如果国民党还要发动内战,它就在全国全世界面前输了理,我党就有理由采取自卫战争,击破其进攻。同时我党力量强大,有来犯者,只要好打,我党必定站在自卫立场上坚决彻底干净全部消灭之(不要轻易打,打则必胜),绝对不要被反动派的其势汹汹所吓倒。但是不论何时,又团结,又斗争,以斗争之手段,达团结之目的;有理有利有节;利用矛盾,争取多数,反对少数,各个击破等项原则\footnote[7]{ 参见本书第二卷《目前抗日统一战线中的策略问题》和《论政策》。},必须坚持,不可忘记。\\
  在广东、湖南、湖北、河南等省的我党力量比华北、江淮所处地位较为困难,中央对于这些地方的同志们深为关怀。但是国民党空隙甚多,地区甚广,只要同志们对于军事政策(行动和作战)和团结人民的政策,不犯大错误,谦虚谨慎,不骄不躁,是完全有办法的。除中央给予必要的指示外,这些地方的同志必须独立地分析环境,解决问题,冲破困难,获得生存和发展。待到国民党对于你们无可奈何的时候,可能在两党谈判中被迫承认你们的力量,而允许作有利于双方的处置。但是你们绝对不要依靠谈判,绝对不要希望国民党发善心,它是不会发善心的。必须依靠自己手里的力量,行动指导上的正确,党内兄弟一样的团结和对人民有良好的关系。坚决依靠人民,就是你们的出路。\\
  总之,我党面前困难甚多,不可忽视,全党同志必须作充分的精神准备。但是整个国际国内大势有利于我党和人民,只要全党能团结一致,是能逐步地战胜各种困难的。\\
\newpage\section*{\myformat{关于重庆谈判}\\\myformat{(一九四五年十月十七日)}}\addcontentsline{toc}{section}{关于重庆谈判}
\begin{introduction}\item  这是毛泽东从重庆回到延安以后,在延安干部会上的报告。\end{introduction}
讲一讲目前的时局问题。这是同志们所关心的问题。这一次,国共两党在重庆谈判,谈了四十三天。谈判的结果,已经在报上公布了\footnote[1]{ 这里是指一九四五年十月十日国共双方代表签订的会谈纪要,即《双十协定》。在这个纪要中,国民党表面上不得不同意中国共产党提出的和平建国的基本方针,承认“以和平、民主、团结、统一为基础,……长期合作,坚决避免内战,建设独立、自由和富强的新中国”,“政治民主化、军队国家化及党派平等合法,为达到和平建国必由之途径”;也不得不同意迅速结束国民党的训政,召开政治协商会议,“保证人民享受一切民主国家人民在平时应享受的身体、信仰、言论、出版、集会、结社之自由,现行法令当依此原则,分别予以废止或修正”,取消特务机关,“严禁司法和警察以外机关有拘捕、审讯和处罚人民之权”,“释放政治犯”,“积极推行地方自治,实行由下而上的普选”等。同时,国民党却顽固地拒绝承认人民军队和解放区民主政权的合法地位,并妄图在“统一军令”和“统一政令”的借口下,根本取消中国共产党领导的人民军队和解放区,以致无法就这个问题达成协议。下面是《会谈纪要》上关于解放区的军队和政权问题谈判经过的记载,里面的所谓“政府方面”是说国民党政府。}。现在两党的代表,还在继续谈判。这次谈判是有收获的。国民党承认了和平团结的方针和人民的某些民主权利,承认了避免内战,两党和平合作建设新中国。这是达成了协议的。还有没有达成协议的。解放区的问题没有解决,军队的问题实际上也没有解决。已经达成的协议,还只是纸上的东西。纸上的东西并不等于现实的东西。事实证明,要把它变成现实的东西,还要经过很大的努力。\\
  国民党一方面同我们谈判,另一方面又在积极进攻解放区。包围陕甘宁边区的军队不算,直接进攻解放区的国民党军队已经有八十万人。现在一切有解放区的地方,都在打仗,或者在准备打仗。《双十协定》上第一条就是“和平建国”,写在纸上的话和事实岂不矛盾?是的,是矛盾的。所以说,要把纸上的东西变成实际,还要靠我们的努力。为什么国民党要动员那么多的军队向我们进攻呢?因为它的主意老早定了,就是要消灭人民的力量,消灭我们。最好是很快消灭;纵然不能很快消灭,也要使我们的形势更不利,它的形势更有利一些。和平这一条写在协定上面,但是事实上并没有实现。现在有些地方的仗打得相当大,例如在山西的上党区。太行山、太岳山、中条山的中间,有一个脚盆,就是上党区。在那个脚盆里,有鱼有肉,阎锡山派了十三个师去抢。我们的方针也是老早定了的,就是针锋相对,寸土必争。这一回,我们“对”了,“争”了,而且“对”得很好,“争”得很好。就是说,把他们的十三个师全部消灭。他们进攻的军队共计三万八千人,我们出动三万一千人。他们的三万八千被消灭了三万五千,逃掉两千,散掉一千\footnote[2]{ 上党区,指山西省东南部以长治为中心的地区,古属上党郡。这一带的山区在抗日战争时期是八路军一二九师的根据地,属于晋冀鲁豫解放区。一九四五年八月中旬,国民党军阎锡山部集中十三个师的兵力,在日伪军的配合下,先后自临汾、浮山、翼城和太原、榆次出发,侵入晋东南解放区的襄垣、屯留、长治、潞城等地。九月十日至十月十二日,解放区军民展开自卫反击,举行了上党战役。这次战役,共歼灭国民党军十一个师及一个挺进纵队三万五千余人,生俘军长史泽波和师长多名。}。这样的仗,还要打下去。我们解放区的地方,他们要拚命来争。这个问题好像不可解释。他们为什么要这样地争呢?在我们手里,在人民手里,不是很好吗?这是我们的想法,人民的想法。要是他们也是这样想,那就统一了,都是“同志”了。可是,他们不会这样想,他们要坚决反对我们。不反对我们,他们想不开。他们来进攻,是很自然的。我们解放区的地方让他们抢了去,我们也想不开。我们反击,也是很自然的。两个想不开,合在一块,就要打仗。既然是两个想不开,为什么又谈判,又成立《双十协定》呢?世界上的事情是复杂的,是由各方面的因素决定的。看问题要从各方面去看,不能只从单方面看。在重庆,有些人认为,蒋介石是靠不住的,是骗人的,要同他谈判出什么结果是不可能的。我遇到许多人都给我这样说过,其中也有国民党员。我向他们说,你们说的是有理由的,有根据的,积十八年之经验\footnote[3]{ “积十八年之经验”,指自一九二七年国民党背叛革命起到一九四五年止中国共产党同它作斗争的经验。},深知是这么一回事。国共两党一定谈判不好,一定要打仗,一定要破裂,但是这只是事情的一个方面。事情还有另外一个方面,还有许多因素,使得蒋介石还不能不有很多顾忌。这里主要有三个因素:解放区的强大,大后方\footnote[4]{ 见本书第二卷《和中央社、扫荡报、新民报三记者的谈话》注〔3〕。}人民的反对内战和国际形势。我们解放区有一万万人民、一百万军队、两百万民兵,这个力量,任何人也不敢小视。我们党在国内政治生活中所处的地位,已经不是一九二七年时候的情况了,也不是一九三七年时候的情况了。国民党从来不肯承认共产党的平等地位,现在也只好承认了。我们解放区的工作,已经影响到全中国、全世界了。大后方的人民都希望和平,需要民主。我这次在重庆,就深深地感到广大的人民热烈地支持我们,他们不满意国民党政府,把希望寄托在我们方面。我又看到许多外国人,其中也有美国人,对我们很同情。广大的外国人民不满意中国的反动势力,同情中国人民的力量。他们也不赞成蒋介石的政策。我们在全国、全世界有很多朋友,我们不是孤立的。反对中国内战,主张和平、民主的,不只是我们解放区的人民,还有大后方的广大人民和全世界的广大人民。蒋介石的主观愿望是要坚持独裁和消灭共产党,但是要实现他的愿望,客观上有很多困难。这样,使他不能不讲讲现实主义。人家讲现实主义,我们也讲现实主义。人家讲现实主义来邀请,我们讲现实主义去谈判。我们八月二十八日到达重庆,二十九日晚上,我就向国民党的代表说:从九一八事变以后,就产生了和平团结的需要。我们要求了,但是没有实现。到西安事变以后、“七七”抗战以前,才实现了。抗战八年,大家一致打日本。但是内战是没有断的,不断的大大小小的磨擦。要说没有内战,是欺骗,是不符合实际的。八年中,我们一再表示愿意谈判。我们在党的七次代表大会上也这样声明:只要国民党当局“一旦愿意放弃其错误的现行政策,同意民主改革,我们是愿意和他们恢复谈判的”\footnote[5]{ 见《论联合政府》(本书第3卷第1069页)。}。在谈判中间,我们提出,第一条中国要和平,第二条中国要民主,蒋介石没有理由反对,只好赞成。《会谈纪要》上所发表的和平方针和若干民主协议,一方面是写在纸上的,还不是现实的东西;另一方面也是由各方面力量决定的。解放区人民的力量,大后方人民的力量,国际形势,大势所趋,使得国民党不得不承认这些东西。\\
  “针锋相对”,要看形势。有时候不去谈,是针锋相对;有时候去谈,也是针锋相对。从前不去是对的,这次去也是对的,都是针锋相对。这一次我们去得好,击破了国民党说共产党不要和平、不要团结的谣言。他们连发三封电报邀请我们,我们去了,可是他们毫无准备,一切提案都要由我们提出。谈判的结果,国民党承认了和平团结的方针。这样很好。国民党再发动内战,他们就在全国和全世界面前输了理,我们就更有理由采取自卫战争,粉碎他们的进攻。成立了《双十协定》以后,我们的任务就是坚持这个协定,要国民党兑现,继续争取和平。如果他们要打,就把他们彻底消灭。事情就是这样,他来进攻,我们把他消灭了,他就舒服了。消灭一点,舒服一点;消灭得多,舒服得多;彻底消灭,彻底舒服。中国的问题是复杂的,我们的脑子也要复杂一点。人家打来了,我们就打,打是为了争取和平。不给敢于进攻解放区的反动派很大的打击,和平是不会来的。\\
  有些同志问,为什么要让出八个解放区\footnote[6]{ 指分布在广东、浙江、苏南、皖南、皖中、湖南、湖北、河南(豫北不在内)等八个省区内的人民军队在抗日战争时期所建立的根据地。}?让出这八块地方非常可惜,但是以让出为好。为什么可惜?因为这是人民用血汗创造出来的、艰苦地建设起来的解放区。所以在让出的地方,必须和当地的人民解释清楚,要作妥善的处置。为什么要让出呢?因为国民党不安心。人家要回南京,南方的一些解放区,在他的床旁边,或者在他的过道上,我们在那里,人家就是不能安心睡觉,所以无论如何也要来争。在这一点上我们采取让步,就有利于击破国民党的内战阴谋,取得国内外广大中间分子的同情。现在全国所有的宣传机关,除了新华社,都控制在国民党手里。它们都是谣言制造厂。这一次谈判,它们造谣说:共产党就是要地盘,不肯让步。我们的方针是保护人民的基本利益。在不损害人民基本利益的原则下,容许作一些让步,用这些让步去换得全国人民需要的和平和民主。我们过去和蒋介石办交涉,也作过让步,并且比现在的还大。在一九三七年,为了实现全国抗战,我们自动取消了工农革命政府的名称,红军也改名为国民革命军,还把没收地主土地改为减租减息。这一次,我们在南方让出若干地区,就在全国人民和全世界人民面前,使国民党的谣言完全破产。军队的问题也是这样。国民党宣传说,共产党就是争枪杆子。我们说,准备让步。我们先提出把我们的军队由现在的数目缩编成四十八个师。国民党的军队是二百六十三个师,我们占六分之一。后来我们又提出缩编到四十三个师,占七分之一。国民党说,他们的军队要缩编到一百二十个师。我们说,照比例减下来,我们的军队可以缩编到二十四个师,还可以少到二十个师,还是占七分之一。国民党军队官多兵少,一个师不到六千人。照他们的编法,我们一百二十万人的军队,就可以编二百个师。但是我们不这样做。这样一来,他们无话可说,一切谣言都破产了。是不是要把我们的枪交给他们呢?那也不是。交给他们,他们岂不又多了!人民的武装,一枝枪、一粒子弹,都要保存,不能交出去。\\
  上面就是我向同志们讲的时局问题。目前时局的发展,有许多矛盾现象。为什么国共谈判中有些问题可以达成协议,有些问题又不能达成协议?为什么《会谈纪要》上说要和平团结,而实际上又在打仗?这种矛盾现象,有些同志想不开。我的讲话就是答复这些问题。有的同志不能了解,蒋介石历来反共反人民,为什么我们又愿意同他谈判呢?我党七次代表大会决定,只要国民党的政策有所转变,我们就愿意同他们谈判,这对不对呢?这是完全对的。中国的革命是长期的,胜利的取得是逐步的。中国的前途如何,靠我们大家的努力如何来决定。在半年左右的时间内,局势还会是动荡不定的。我们要加倍地努力,争取局势的发展有利于全国人民。\\
  还讲一点我们的工作。在座的有些同志要往前方去。许多同志满腔热忱,争着出去工作,这种积极性和热情,是很可贵的。但是也有个别的同志抱着错误的想法,不是想到那里有许多困难需要解决,而是认为那里的一切都很顺利,比延安舒服。有没有人这样想呢?我看是有的。我劝这些同志改正自己的想法。去,是为了工作去的。什么叫工作,工作就是斗争。那些地方有困难、有问题,需要我们去解决。我们是为着解决困难去工作、去斗争的。越是困难的地方越是要去,这才是好同志。那些地方的工作是很艰苦的。艰苦的工作就像担子,摆在我们的面前,看我们敢不敢承担。担子有轻有重。有的人拈轻怕重,把重担子推给人家,自己拣轻的挑。这就不是好的态度。有的同志不是这样,享受让给人家,担子拣重的挑,吃苦在别人前头,享受在别人后头。这样的同志就是好同志。这种共产主义者的精神,我们都要学习。\\
  有许多本地的干部,现在要离乡背井,到前方去。还有许多出生在南方的干部,从前从南方到了延安,现在也要到前方去。所有到前方去的同志,都应当做好精神准备,准备到了那里,就要生根、开花、结果。我们共产党人好比种子,人民好比土地。我们到了一个地方,就要同那里的人民结合起来,在人民中间生根、开花。我们的同志不论到什么地方,都要把和群众的关系搞好,要关心群众,帮助他们解决困难。团结广大人民,团结得越多越好。放手发动群众,壮大人民力量,在我们党的领导下,打败侵略者,建设新中国。这是党的七次代表大会的方针,我们要为这个方针奋斗。中国的事情,要靠共产党办,靠人民办。我们有决心、有办法实现和平,实现民主。只要我们同全体人民更好地团结起来了,中国的事情就好办了。\\
  第二次世界大战以后的世界,前途是光明的。这是总的趋势。伦敦五国外长会议失败了\footnote[7]{ 苏、中、美、英、法五国外长会议是根据一九四五年八月波茨坦协定而设立的,自一九四五年至一九四九年共举行六次会议。这里所说的伦敦五国外长会议是指一九四五年九月十一日至十月二日在伦敦举行的第一次五国外长会议。会议讨论了同曾参与法西斯德国侵略战争的意大利、罗马尼亚、保加利亚、匈牙利、芬兰签订和平条约以及处理意大利的殖民地等问题。在讨论对罗、保、匈三国和约草案时,美英无理要求罗、保、匈三国民主政府辞职或改组;美英还违背波茨坦协定,给予法国参加讨论和约的权利,遭到苏联拒绝;苏联提议讨论成立对日本的管制委员会问题这一合理主张,也被美国拒绝。由于上述分歧,这次会议没有达成协议。},是不是就要打第三次世界大战呢?不会的。试想第二次世界大战刚刚打完,怎么就可能打第三次世界大战呢?资本主义国家和社会主义国家在许多国际事务上,还是会妥协的,因为妥协有好处\footnote[8]{ 参见本卷《关于目前国际形势的几点估计》。}。反苏反共的战争,全世界的无产阶级和人民都坚决反对。在最近的三十年内,打过两次世界大战。在第一次大战和第二次大战之间,间隔了二十几年。人类历史五十万年,只有在这三十年内才打过世界战争。第一次大战以后,世界有很大进步。这一次大战以后,世界一定会进步得更快。第一次大战以后,产生了苏联,全世界产生了几十个共产党,这是从前没有过的。第二次世界大战以后,苏联更强盛了,欧洲的面貌改观了,全世界无产阶级和人民的政治觉悟更提高了,全世界的进步力量更团结了。我们中国也处在急剧的变动中间。中国发展的总趋势,也必定要变好,不能变坏。世界是在进步的,前途是光明的,这个历史的总趋势任何人也改变不了。我们应当把世界进步的情况和光明的前途,常常向人民宣传,使人民建立起胜利的信心。同时,我们还要告诉人民,告诉同志们,道路是曲折的。在革命的道路上还有许多障碍物,还有许多困难。我们党的七次代表大会设想过许多困难,我们宁肯把困难想得更多一些。有些同志不愿意多想困难。但是困难是事实,有多少就得承认多少,不能采取“不承认主义”。我们要承认困难,分析困难,向困难作斗争。世界上没有直路,要准备走曲折的路,不要贪便宜。不能设想,哪一天早上,一切反动派会统统自己跪在地下。总之,前途是光明的,道路是曲折的。我们面前困难还多,不可忽视。我们和全体人民团结起来,共同努力,一定能够排除万难,达到胜利的目的。\\
\newpage\section*{\myformat{国民党进攻的真相}\\\myformat{(一九四五年十一月五日)}}\addcontentsline{toc}{section}{国民党进攻的真相}
\begin{introduction}\item  这是毛泽东以中共发言人的名义发表的谈话。这时蒋介石已经撕毁《双十协定》,进攻解放区的内战规模已经日趋扩大。\end{introduction}
合众社重庆三日电报道,国民党中央宣传部长吴国桢宣称,“政府在此次战争中全居守势”,并提出所谓恢复交通的办法\footnote[1]{ 抗日战争结束时,中国的大部分铁路交通线都在解放区军民的控制或包围之中。国民党反动派在所谓“恢复交通”的借口下,企图利用这些交通线分割解放区,并且把它的几百万军队运往东北、华北、华东、华中,进攻解放区,抢占大城市。}。新华社记者为此询问中共方面发言人。\\
  中共发言人告记者称:吴氏所说“守势”云云,全系撒谎。除我军已撤退的浙东、苏南、皖中、皖南、湖南五个解放区全被国民党军队进占、大肆蹂躏人民外,其它大多数解放区,例如广东、湖北、河南、苏北、皖北、山东、河北等省,国民党正规军已有七十余师开到我解放区及其附近,压迫人民,进攻我军,或准备进攻。正在向我解放区开进者,尚有数十师。这难道是取守势吗?其中由彰德\footnote[2]{ 彰德,今河南省安阳市。}北进一路,攻至邯郸地区之八个师,两个师反对内战,主张和平,六个师(其中有三个美械师)在我解放区军民举行自卫的反击之后,始被迫放下武器。这一路国民党军的许多军官,其中有副长官、军长、副军长多人,现在都在解放区\footnote[3]{ 一九四五年九月,国民党军队自郑州、新乡一带沿平汉路进攻晋冀鲁豫解放区。十月下旬,其先头三个军,侵入磁县、邯郸地区。解放区军民奋起自卫,经一周激战,国民党第十一战区副司令长官兼新八军军长高树勋率新编第八军等部一万余人,在邯郸地区起义,其余的两个军,在溃退中被人民解放军围歼,放下武器。当时被迫放下武器的高级军官有:国民党第十一战区副司令长官兼第四十军军长马法五,第四十军副军长刘世荣、军参谋长李旭东、副师长刘树森等多人。},他们都可以证明他们是从何处开来、如何奉命进攻的全部真情。这难道也是取守势吗?我豫鄂两省解放区军队,现被国民党第一、第五、第六等三个战区的军队共二十几个师四面包围,刘峙任该区“剿共”总指挥。我豫西、豫中、鄂南、鄂东、鄂中等处解放区都被国民党军队侵占,大肆烧杀,迫得我李先念、王树声等部无处存身,不得不向豫鄂交界地区觅一驻地,以求生存,但又被国民党军队紧紧追击\footnote[4]{ 日本投降以后,国民党调集了三个战区的二十多个师的兵力大举进犯豫、鄂两省解放区。国民党第一战区司令长官胡宗南分兵一部,自西北方向,沿陇海路两侧东犯豫西解放区;第五战区司令长官刘峙所部,沿平汉路两侧,自北向南进犯豫中、鄂中、鄂东解放区;第六战区的部队自鄂南北犯加以配合。以上的国民党军队大都归刘峙指挥。豫鄂解放区的人民军队,对进犯军作了坚决的斗争,保存了实力,于一九四五年十月下旬,转移至豫、鄂交界之大洪山、桐柏山、枣阳地区,后因国民党军队继续追逼,又转移至平汉路东之宣化店地区。}。这难道也是取守势吗?在晋绥察三省,也是如此。十月上旬,阎锡山指挥十三个师,攻入我上党解放区襄垣、屯留区域,被当地军民在自卫战斗中全部缴械,被俘人员中亦有军长师长多人。他们现在我太行解放区,一个个活着,足以证明他们是从何处开来、如何奉命进攻的全部真情。最近阎锡山在重庆报道他如何被攻,而他则仅取“守势”,说了种种谎话。他大概忘记了他的十九军军长史泽波,暂四十六师师长郭溶,暂四十九师师长张宏,六十六师师长李佩膺,六十八师师长郭天兴,暂三十七师师长杨文彩等位将军\footnote[5]{ 这里所列举的一些军官,都是上党战役中被人民解放军俘虏的阎锡山部队的高级将领。关于上党战役,见本卷《关于重庆谈判》注〔2〕。},现正住在我解放区,足以驳斥吴国桢氏、阎锡山氏和一切挑拨内战的反动派的任何谎话。傅作义将军奉命进攻我绥远、察哈尔、热河三省解放区,已两个多月,曾打到张家口的门口,占领我整个绥远解放区和察哈尔西部。难道这也是取守势和未放“第一枪”吗?我察绥两省军民起而自卫,在反攻战斗中亦俘虏大批官兵,他们都可以证明他们是从何处开来、如何进攻等等\footnote[6]{ 绥远,一九五四年撤销,原辖地区划归内蒙古自治区。察哈尔,一九五二年撤销,原辖地区划归河北、山西两省。热河,一九五五年撤销,原辖地区划归河北、辽宁两省及内蒙古自治区。傅作义当时担任国民党第十二战区的司令长官。他的部队,抗日战争时期驻扎在绥远西部的五原、临河一带。日本投降后,他奉命进攻绥远、热河、察哈尔三省解放区。一九四五年八月攻占归绥(今呼和浩特市)、集宁、丰镇。九月初攻占兴和、尚义、武川、陶林、新堂、凉城,大举向察哈尔解放区进攻,迫近张家口。人民解放军起而自卫,将其击退,并俘虏其大批官兵。}。在各次自卫战斗中,我方缴获大批“剿匪”和反共文件,其中有国民党最高当局所颁发而被吴国桢氏称为不过是“笑话”的《剿匪手本》、“剿匪”命令\footnote[7]{ 《剿匪手本》,是一九三三年蒋介石编的专门讲述进攻中国人民军队和革命根据地的方法的反革命小册子。一九四五年抗日战争结束以后,蒋介石重印这个小册子发给国民党军官,并下达密令称:“此次剿匪为人民幸福之所系,务本以往抗战之精神,遵照中正所订《剿匪手本》,督励所属,努力进剿,迅速完成任务。其功于国家者必得膺赏,其迟滞贻误者当必执法以罪。希转饬所属剿匪部队官兵一体悉遵为要。”}和其它反共文件,正在向延安解送中。这些反共文件,都是国民党军队进攻解放区的铁证。\\
  记者又问中共发言人,吴国桢氏所提恢复交通办法,你的意见如何?该发言人答道:这不过是缓兵之计而已。国民党当局正在大举调兵,像洪水一样,想要淹没我整个解放区。他们在九、十两月几个进攻失败之后,正在布置新的更大规模的进攻。而阻碍这种进攻,亦即有效地制止内战的武器之一,就是不许他们在铁路上运兵。我们和旁人一样,主张交通线迅速恢复,但是必须在受降、处置伪军和实行解放区自治三项问题获得解决之后,才能恢复。先解决交通问题,后解决三项问题呢,还是先解决三项问题,后解决交通问题呢?解放区军队艰苦抗日八年,为什么没有受降资格,而劳其它军队从远远的地方开去受降呢?伪军人人得而诛之,为什么一律编为“国军”,并且指挥他们进攻解放区呢?地方自治,《双十协定》\footnote[8]{ 见本卷《关于重庆谈判》注〔1〕。}上已有明文规定,孙中山先生早主省长民选,为什么还要政府派遣官吏呢?交通问题应该迅速解决,这三大问题尤其应该迅速解决。三大问题不解决而言恢复交通,只是使内战扩大延长,达到内战发动者们淹没解放区的目的。为着迅速制止已经普及全国的反人民反民主的内战,我们主张:(一)已经进入华北、苏北、皖北、华中各解放区及其附近的政府受降军队和进攻军队,立即撤返原防,由解放区军队去接受敌人投降和驻防各城市与交通线,恢复被侵占的解放区;(二)全部伪军立即缴械遣散,在华北、苏北、皖北者,由解放区负责缴械遣散;(三)承认一切解放区的人民民主自治,中央政府不得委派官吏,实现《双十协定》的规定。发言人说:只有如此,才能制止内战;否则是完全没有保障的。绥远、上党、邯郸三次自卫战斗中所缴获的文件以及大举调兵和大举进攻的实际行动,已充分证明国民党当局所谓恢复交通是为着人民,不是为着内战,乃是毫不足信的。中国人民被欺骗得已经够了,现在再不能被欺骗。现在的中心问题,是全国人民动员起来,用一切方法制止内战。\\
\newpage\section*{\myformat{减租和生产是保卫解放区的两件大事}\\\myformat{(一九四五年十一月七日)}}\addcontentsline{toc}{section}{减租和生产是保卫解放区的两件大事}
\begin{introduction}\item  这是毛泽东为中共中央起草的对党内的指示。\end{introduction}
(一)国民党在美国援助下,动员一切力量进攻我解放区。全国规模的内战已经存在。我党当前任务,是动员一切力量,站在自卫立场上,粉碎国民党的进攻,保卫解放区,争取和平局面的出现。为达此目的,使解放区农民普遍取得减租利益,使工人和其它劳动人民取得酌量增加工资和改善待遇的利益;同时又使地主还能生活,使工商业资本家还有利可图;并于明年发展大规模的生产运动,增加粮食和日用必需品的生产,改善人民的生活,救济饥民、难民,供给军队的需要,成为非常迫切的任务。只有减租和生产两件大事办好了,才能克服困难,援助战争,取得胜利。\\
  (二)目前战争的规模很大,许多领导同志在前方指挥,不能分心照顾减租和生产。因此,必须实行分工。留在后方的领导同志,除了作直接援助前线的许多工作之外,一定要不失时机,布置减租和生产两件大工作。务使整个解放区,特别是广大的新解放区,在最近几个月内(冬春两季)发动一次大的减租运动,普遍地实行减租,借以发动大多数农民群众的革命热情。同时,在一九四六年内,全解放区的农业和工业的生产,务使有一个新的发展。不要因为新的大规模战争而疏忽减租和生产;恰好相反,正是为了战胜国民党的进攻,而要加紧减租和生产。\\
  (三)减租必须是群众斗争的结果,不能是政府恩赐的。这是减租成败的关键。减租斗争中发生过火现象是难免的,只要真正是广大群众的自觉斗争,可以在过火现象发生后,再去改正。只有在那时,才能说服群众,使他们懂得让地主能够活下去,不去帮助国民党,对于农民和全体人民是有利的。目前我党方针,仍然是减租而不是没收土地。在减租中和减租后,必须帮助大多数农民组织在农会中。\\
  (四)使大多数生产者组织在生产互助团体中,是生产运动胜利的关键。政府发放农贷、工贷,是必不可少的步骤。不违农时,减少误工,也十分重要。现在一面要为战争动员民力,一面又要尽可能地不违农时,应当研究调节的办法。在不妨碍战争、工作和学习的条件下,部队、机关、学校仍要适当地参加生产,才能改善生活,减轻人民的负担。\\
  (五)我们已得到了一些大城市和许多中等城市。掌握这些城市的经济,发展工业、商业和金融业,成了我党的重要任务。为此目的,利用一切可用的社会现成人材,说服党员同他们合作,向他们学习技术和管理的方法,非常必要。\\
  (六)告诉党员坚决同人民一道,关心人民的经济困难,而以实行减租和发展生产两件大事作为帮助人民解决困难的重要关键,我们就会获得人民的真心拥护,任何反动派的进攻是能够战胜的。一切仍要从长期支持着想,爱惜人力、物力,事事作长期打算,我们就一定能够胜利。\\
\newpage\section*{\myformat{一九四六年解放区工作的方针}\\\myformat{(一九四五年十二月十五日)}}\addcontentsline{toc}{section}{一九四六年解放区工作的方针}
\begin{introduction}\item  这是毛泽东为中共中央起草的对党内的指示。\end{introduction}
过去几个月内,我党领导人民在肃清敌伪和粉碎国民党向解放区进攻的激烈斗争中,得到了伟大的胜利。全党同志齐心协力,在各项工作中得到了显着的成绩。一九四五年即将过去,一九四六年各解放区的工作必须注意如下各点:\\
  (一)粉碎新的进攻。国民党自从在绥远\footnote[1]{ 绥远,原来是一个省,一九五四年撤销,原辖地区划归内蒙古自治区。}、山西、冀南三处向我解放区大举进攻被我军粉碎后,又在调动更大的兵力,准备新的进攻。假如没有新的情况足以使国民党迅速停止内战,则一九四六年春季的战斗,将是紧张的。因此,站在自卫立场上,尽一切努力粉碎国民党的进攻,仍是各解放区的中心任务。\\
  (二)开展高树勋运动\footnote[2]{ 一九四五年十月三十日,国民党第十一战区副司令长官兼新八军军长高树勋率新编第八军等部一万余人在邯郸内战前线起义,在全国影响很大。为了进一步加强分化、瓦解国民党军队和争取国民党军队起义的工作,中共中央决定对国民党军队开展宣传运动,号召国民党军队中的官兵学习高树勋部队的榜样,拒绝进攻解放区,在内战战场上实行怠工,和人民解放军联欢,举行起义,站到人民方面来。这个运动,被称为“高树勋运动”。}。为着粉碎国民党的进攻,我党必须对一切准备进攻和正在进攻的国民党军队进行分化的工作。一方面,由我军对国民党军队进行公开的广大的政治宣传和政治攻势,以瓦解国民党内战军的战斗意志。另一方面,须从国民党军队内部去准备和组织起义,开展高树勋运动,使大量国民党军队在战争紧急关头,仿照高树勋榜样,站到人民方面来,反对内战,主张和平。为使此项工作切实进行和迅速生效起见,各地必须依照中央指示,设置专门部门,调派大批干部,专心致志,从事此项工作。各地领导机关,则要给以密切指导。\\
  (三)练兵。各解放区野战军,一般业已组成,地方军亦不在少。目前扩兵一般应该停止,而应利用作战间隙着重练兵。不论野战军、地方军、民兵,都是如此。练兵项目,仍以提高射击、刺杀、投弹等项技术程度为主,提高战术程度为辅,特别着重于练习夜战。练兵方法,应开展官教兵、兵教官、兵教兵的群众练兵运动。一九四六年必须进一步实现改进军队政治工作的任务,克服军队中存在着的教条主义和形式主义作风,为团结官兵,团结军民,团结友军,瓦解敌军,保证练兵、供给和作战任务的完成而奋斗。各地民兵,须按目前条件,重新组织。军队的后方勤务工作,须重新调整。应尽一切可能建立和扩充各地的炮兵和工兵。军事学校应继续办理,着重技术人材的训练。\\
  (四)减租。按照中央一九四五年十一月七日指示\footnote[3]{ 即本卷《减租和生产是保卫解放区的两件大事》。},各地务必在一九四六年,在一切新解放区,发动大规模的、群众性的、但是有领导的减租减息运动。工人则酌量增加工资。使广大群众,在此运动中翻过身来,并组织起来,成为解放区自觉的主人翁。在新解放区,如无此项坚决措施,群众便不能区别国共两党的优劣,便会动摇于两党之间,而不能坚决地援助我党。在老解放区,则应复查减租减息的工作,进一步巩固老解放区。\\
  (五)生产。按照十一月七日指示,各地立即准备一切,务使一九四六年我全解放区的公私生产超过以前任何一年的规模和成绩。人民中发生的疲劳情绪,只有在认真实现减租和生产两项任务,并有了显着成绩之后,才能克服。减租和生产两大任务是否能够完成,将最后地决定解放区政治军事斗争的胜负,各地切不可疏忽。\\
  (六)财政。为着应付最近时期的紧张工作而增重了的财政负担,在一九四六年中,必须有计划有步骤地转到正常状态。人民负担太重者必须酌量减轻。各地脱离生产人员,必须不超过当地财力负担所许可的限度,以利持久。兵贵精不贵多,仍是今后建军原则之一。发展生产,保障供给,集中领导,分散经营,军民兼顾,公私兼顾\footnote[4]{ 这里所说的“公私兼顾”的“公私”,是指公家和个人两方面,而不是指公营企业和私营企业两方面。},生产和节约并重等项原则,仍是解决财经问题的适当的方针。\\
  (七)拥政爱民和拥军优抗\footnote[5]{ 拥政爱民,是人民解放军“拥护政府、爱护人民”的口号的简称。拥军优抗,是解放区的党政机关、群众团体和人民群众“拥护军队、优待抗日军人家属”的口号的简称。“拥军优抗”的口号,后来改为“拥军优属”,即拥护人民解放军、优待革命军人家属。}。一九四六年,这两项工作,必须比过去几年做得更好些。这对于粉碎国民党进攻和巩固解放区,将有重大意义。军队中应当从每个指战员的思想上解决问题,使他们彻底认识拥政爱民的重要性。只要军队方面做好了,地方对军队的关系必会跟着改善。\\
  (八)救济。各解放区有许多灾民、难民、失业者和半失业者,亟待救济。此问题解决的好坏,对各方面影响甚大。救济之法,除政府所设各项办法外,主要应依靠群众互助去解决。此种互助救济,应由党政鼓励群众组织之。\\
  (九)爱护本地干部。现在每个解放区,都有大批外来干部做各级领导工作。东北各省,此种情形尤为显着。对于这些外来干部,各地领导机关,务须谆谆告诫他们,以充分的热情和善意,爱护本地干部。将识别、培养和提拔本地干部,当作自己的重要任务。只有这样,我党在解放区才能生根。外来人轻视本地人的作风,应当受到批评。\\
  (十)一切作持久打算。不论时局发展的情况如何,我党均须作持久打算,才能立于不败之地。目前我党一方面坚持解放区自治自卫立场,坚决反对国民党的进攻,巩固解放区人民已得的果实;一方面,援助国民党区域正在发展的民主运动(以昆明罢课\footnote[6]{ 一九四五年十一月二十五日晚,云南省昆明市大中学生六千余人,在西南联合大学举行反内战时事晚会,国民党反动派派遣军队包围会场,发射小钢炮、机关枪、步枪,并在学校附近戒严,禁阻师生通行返家。各校学生于第二天起联合罢课。十二月一日,国民党反动派派大批军警和特务在西南联合大学校舍、师范学院两处投掷手榴弹,并骚扰联大工学院、联大附中和南英中学等处。当日,师生死四人,伤数十人。一般称这个血案为“一二一惨案”。}为标志),使反动派陷于孤立,使我党获得广大的同盟者,扩大在我党影响下的民族民主统一战线。同时,我党代表团即将出席各党派和无党派人物的政治协商会议,并和国民党重新谈判,为全国的和平民主而奋斗。但事情可能还有曲折。我们面前还有许多困难,例如新区域、新部队还不巩固和财政困难等。我们必须正视这些困难,克服这些困难,在一切工作布置中作持久打算,十分注意人力物力的节省使用,力戒侥幸成功的心理。\\
  以上十项,为一九四六年尤其是上半年工作应加特别注意之点。望各地同志根据当地情况,灵活地实现上述方针。至于各地政权建设工作,统一战线工作,从党内外推广时事教育的工作,解放区附近城市的工作等项,都是重要的,这里不来多说。\\
\newpage\section*{\myformat{建立巩固的东北根据地}\\\myformat{(一九四五年十二月二十八日)}}\addcontentsline{toc}{section}{建立巩固的东北根据地}
\begin{introduction}\item  这是毛泽东为中共中央起草的给中共中央东北局的指示。在苏联宣布对日作战,苏联军队进入东北以后,中共中央和中共中央军事委员会派遣大批干部和部队进入东北,与东北抗日联军会合,领导东北人民,消灭日军和伪满的残余,肃清汉奸,剿除土匪,建立各级地方民主政府。但是这时坚持要独占全东北的国民党反动派,在美帝国主义援助下,经过海陆空三路向东北大举运兵,并攻占了已被人民解放军解放的山海关、锦州等要地。东北的严重斗争已经不可避免,而这一斗争对于全国局势显然具有特别重大的意义。毛泽东在为中共中央起草的这个指示中,预见到东北斗争的艰苦性,确定了中国共产党在东北的任务是在距离国民党占领中心较远的城市和广大乡村,建立巩固的根据地,发动群众,逐步积蓄力量,准备在将来转入反攻。中共中央和毛泽东的这个正确的方针,由中共中央东北局有效地实现了,因而能在三年以后的一九四八年十一月,取得解放全东北的伟大胜利。\end{introduction}
(一)我党现时在东北的任务,是建立根据地,是在东满、北满、西满建立巩固的军事政治的根据地\footnote[1]{ 当时,东满根据地是指中长路沈阳至长春段以东的吉林、西安(今东辽)、安图、延吉、敦化等地区;北满根据地是指哈尔滨、牡丹江、北安、佳木斯、齐齐哈尔等地区;西满根据地是指中长路沈阳至哈尔滨段以西的洮安、开鲁、阜新、郑家屯(今双辽)、扶余等地区。此外,中国共产党还在南满建立了根据地,南满根据地是指中长路沈阳至大连段以东的安东(今丹东市)、庄河、通化、临江清原和沈阳西南的辽中等地区。坚持南满的对敌斗争,对东北根据地的建设也起了重要作用。}。建立这种根据地,不是轻而易举的事,必须经过艰苦奋斗。建立这种根据地的时间,需要三四年。但是在一九四六年一年内,必须完成初步的可靠的创建工作。否则,我们就有可能站不住脚。\\
  (二)建立这种根据地的地区,现在应当确定不是在国民党已占或将占的大城市和交通干线,这是在现时条件下所作不到的。也不是在国民党占领的大城市和交通干线的附近地区内。这是因为国民党既然得了大城市和交通干线,就不会容许我们在其靠得很近的地区内建立巩固的根据地。这种地区,我党应当作充分的工作,在军事上建立第一道防线,决不可轻易放弃。但是,这种地区将是两党的游击区,而不是我们的巩固根据地。因此,建立巩固根据地的地区,是距离国民党占领中心较远的城市和广大乡村。目前,应当确定这种地区,以便部署力量,引导全党向此目标前进。\\
  (三)在确定建立巩固根据地的地区和部署力量之后,又在我军数量上已有广大发展之后,我党在东北的工作重心是群众工作。必须使一切干部明白,国民党在东北一个时期内将强过我党,如果我们不从发动群众斗争、替群众解决问题、一切依靠群众这一点出发,并动员一切力量从事细心的群众工作,在一年之内,特别是在最近几个月的紧急时机内,打下初步的可靠的基础,那末,我们在东北就将陷于孤立,不能建立巩固根据地,不能战胜国民党的进攻,而有遭遇极大困难甚至失败的可能;反之,如果我们紧紧依靠群众,我们就将战胜一切困难,一步一步地达到自己的目的。群众工作的内容,是发动人民进行清算汉奸的斗争,是减租和增加工资运动,是生产运动。应当在这些斗争中,组织各种群众团体,建立党的核心,建立群众的武装和人民的政权,把群众斗争从经济斗争迅速提高到政治斗争,参加根据地的建设。最近热河省委的发动群众斗争的指示\footnote[2]{ 指一九四五年十二月中共热河省委颁发的《发动群众的指示》。这个指示指出:发动群众对汉奸、特务的控诉复仇的清算运动是目前发动群众的中心环节,应当通过这一运动,发扬群众的积极性,提高群众在社会上、政治上和经济上的地位,组织工会、农会和其它群众团体,并准备在这个运动告一段落后转入减租减息的群众运动。在城市中发动群众,必须首先发动工人,使工人在清算汉奸、特务的运动中起先锋作用和领导作用。这个指示还提出要学会管理城市的一套办法,爱护民力,一切作长期打算。},可以应用于东北。我党必须给东北人民以看得见的物质利益,群众才会拥护我们,反对国民党的进攻。否则,群众分不清国民党和共产党的优劣,可能一时接受国民党的欺骗宣传,甚至反对我党,造成我们在东北非常不利的形势。\\
  (四)我党现时在东北有一项主观上的困难。这就是大批干部和军队初到东北,地理民情不熟。干部对于不能占领大城市表示不满,对于发动群众建立根据地的艰苦工作表示不耐心。这些情况,都是同当前形势和党的任务相矛盾的。必须反复教育一切外来干部,注重调查研究,熟悉地理民情,并下决心和东北人民打成一片,从人民群众中培养出大批积极分子和干部。应向干部说明,即使大城市和交通线归于国民党,东北形势对于我们仍然是有利的。只要我们能够将发动群众、建立根据地的思想普及到一切干部和战士中去,动员一切力量,迅速从事建立根据地的伟大斗争,我们就能在东北和热河\footnote[3]{ 热河,原来是一个省,一九五五年撤销,原辖地区划归河北、辽宁两省和内蒙古自治区。}立住脚跟,并取得确定的胜利。必须告诉干部,对于国民党势力切不可估计太低,也不可以为国民党将向东满和北满进攻,因而产生不耐心作艰苦工作的情绪。这样说明时,当然不要使干部觉得国民党势力大得了不得,国民党的进攻是不能粉碎的。应当指出,国民党在东北没有深厚的有组织的基础,它的进攻是可以粉碎的,这就给我党以建立根据地的可能性。但是,国民党军队现在正向热辽边境进攻,如果没有受到打击,他们不久即将向东满和北满进攻。因此,我党必须人人下决心,从事最艰苦的工作,迅速发动群众,建立根据地,在西满和热河,坚决地有计划地粉碎国民党的进攻。在东满和北满,则是迅速准备粉碎国民党进攻的条件。干部中一切不经过自己艰苦奋斗、流血流汗,而依靠意外便利、侥幸取胜的心理,必须扫除干净。\\
  (五)迅速在西满、东满、北满划分军区和军分区,将军队划分为野战军和地方军。将正规军队的相当部分,分散到各军分区去,从事发动群众,消灭土匪,建立政权,组织游击队、民兵和自卫军,以便稳固地方,配合野战军,粉碎国民党的进攻。一切军队,均须有确定的地区和任务,才能迅速和人民结合起来,建立巩固的根据地。\\
  (六)此次我军十余万人进入东北和热河,新扩大者又达二十余万人,还有继续扩大的趋势。加上党政工作人员,估计在一年内,将达四十万人以上。如此大量的脱离生产人员,专靠东北人民供给,是决不能持久的,是很危险的。因此,除集中行动负有重大作战任务的野战兵团外,一切部队和机关,必须在战斗和工作之暇从事生产。一九四六年决不可空过,全东北必须立即计划此事。\\
  (七)在东北,工人和知识分子的动向,对于我们建立根据地,同争取将来的胜利关系极大。因此,我党对于大城市和交通干线的工作,特别是争取工人和知识分子,应当充分注意。鉴于抗战初期我党争取工人和知识分子进入根据地注意不够,此次东北党组织除注意国民党占领区的地下工作外,还应尽可能吸引工人和知识分子参加军队和根据地的各项建设工作。\\
\newpage\section*{\myformat{关于目前国际形势的几点估计}\\\myformat{(一九四六年四月)}}\addcontentsline{toc}{section}{关于目前国际形势的几点估计}
\begin{introduction}\item  这个文件是针对当时对于国际形势的一种悲观估计而写的。一九四六年春季,以美国为首的帝国主义和各国反动派,日益加紧反苏、反共、反人民的活动,鼓吹所谓“美苏必战”,所谓“第三次世界大战必然爆发”。在这种情况下,当时有一些同志,由于过高地估计帝国主义力量,过低地估计人民力量,惧怕美帝国主义,惧怕爆发新的世界战争,因而在美蒋反动派武装进攻的面前,表示软弱,不敢坚决地用革命战争反对反革命战争。毛泽东在这个文件里,反对了这种错误思想。毛泽东指出,只要世界人民力量向世界反动力量进行坚决的和有效的斗争,就可以克服新的世界战争的危险;同时,又指出,帝国主义国家和社会主义国家有可能取得某些妥协,但是这种妥协,“并不要求资本主义世界各国人民随之实行国内的妥协”,“各国人民仍将按照不同情况进行不同斗争”。这个文件,当时没有发表,只在中共中央一部分领导同志中间传阅过。一九四七年十二月的中共中央会议,印发了这个文件。由于到会同志一致同意这个文件的内容,后来将全文收入了中共中央一九四八年一月发出的《关于一九四七年十二月中央会议决议事项的通知》中。\end{introduction}
(一)世界反动力量确在准备第三次世界大战,战争危险是存在着的。但是,世界人民的民主力量超过世界反动力量,并且正在向前发展,必须和必能克服战争危险。因此,美、英、法同苏联的关系,不是或者妥协或者破裂的问题,而是或者较早妥协或者较迟妥协的问题。所谓妥协,是指经过和平协商达成协议。所谓较早较迟,是指在几年或者十几年之内,或者更长时间。\\
  (二)上述这种妥协,不是说在一切国际问题上。这在美、英、法继续由反动派统治的条件下是不可能的。这种妥协,是说在若干问题上,包括在某些重大问题上。但是,这一类的妥协在目前短时期内还不会很多。美、英、法同苏联的通商贸易关系则有扩大的可能。\\
  (三)美、英、法同苏联之间的这种妥协,只能是全世界一切民主力量向美、英、法反动力量作了坚决的和有效的斗争的结果。这种妥协,并不要求资本主义世界各国人民随之实行国内的妥协。各国人民仍将按照不同情况进行不同斗争。反动势力对于人民的民主势力的原则,是能够消灭者一定消灭之,暂时不能消灭者准备将来消灭之。针对这种情况,人民的民主势力对于反动势力,亦应采取同样的原则。\\
\newpage\section*{\myformat{以自卫战争粉碎蒋介石的进攻}\\\myformat{(一九四六年七月二十日)}}\addcontentsline{toc}{section}{以自卫战争粉碎蒋介石的进攻}
\begin{introduction}\item  这是毛泽东为中共中央起草的对党内的指示。一九四五年《双十协定》签订后蒋介石就不断地破坏这个协定,但是当时他对全面内战还没有准备好,主要是大批国民党军队还没有运到内战前线。因此,在一九四六年一月,国民党政府在全国人民要求和平民主的压力下,仍然不能不召集有中国共产党和其它民主党派参加的政治协商会议,在这个会议上通过一系列有利于和平民主的决议,并且在一月十日发布停战令。蒋介石不愿意遵守政治协商会议的决议和停战令。在一九四六年上半年国民党军队继续在许多地方向解放区进攻,在东北进攻的规模更大,形成关内小打、关外大打的局面。同时美国用极大力量运输和装备国民党军队。到了一九四六年六月底,蒋介石和他的美国主子认为已经有了充分准备,认为可以在三个月至六个月的时间内消灭全部人民解放军,就以六月二十六日大举围攻中原解放区为起点,发动了对解放区的全面进攻。从七月起到九月止,国民党军队先后向苏皖解放区、山东解放区、晋冀鲁豫解放区、晋察冀解放区、晋绥解放区大举进攻。十月,对东北解放区再次发动了大规模的进攻。同时,继续以大军包围陕甘宁解放区。在全国规模内战爆发的时候,国民党用于进攻解放区的兵力,正规军共达一百九十三个旅(师),约一百六十万人,占其总兵力正规军二百四十八个旅(师)二百万人的百分之八十。各解放区军民,在中共中央和各中央局、分局的领导下,英勇地抗击了蒋介石军队的进攻。当时,解放区共有六个大的作战区域。这六个作战区域和在这些区域作战的人民解放军是:晋冀鲁豫解放区,在那里作战的是由刘伯承、邓小平等领导的人民解放军;华东解放区(包括山东解放区和苏皖解放区),在那里作战的是由陈毅、粟裕、谭震林等领导的人民解放军;东北解放区,在那里作战的是由林彪、罗荣桓等领导的人民解放军;晋察冀解放区,在那里作战的是由聂荣臻等领导的人民解放军;晋绥解放区,在那里作战的是由贺龙等领导的人民解放军;中原解放区,在那里作战的是由李先念、郑位三等领导的人民解放军。当时人民解放军的总兵力约为一百二十万人,较之敌人的兵力,在数量上居于劣势。人民解放军正确地执行了毛泽东所制定的作战方针,不断地给进犯的敌人以有力的打击。经过约八个月的时间,在消灭了敌人正规军六十六个旅,加上非正规军,共七十一万多人以后,便停止了敌人的全面进攻。接着,人民解放军又粉碎了敌人的重点进攻,逐步地展开了战略性的反攻。\end{introduction}
(一)蒋介石破坏停战协定\footnote[1]{《停战协定》,即一九四六年一月十日公布的中共代表和蒋介石国民党政府代表签订的关于停止军事冲突的协定。这个协定规定双方军队应在一月十三日午夜就各自位置上停止军事行动。但是蒋介石实际上是利用这个停战协定作为布置大战的幌子,在停战令下达的同时,即密令国民党军队“抢占战略要点”,接着又不断地调动军队,向解放区进攻。到六月下旬蒋介石便公开撕毁了停战协定,向解放区发动了全面的进攻。},破坏政协决议\footnote[2]{“政协”,即国民党、共产党、其它党派和无党派人士的代表,在一九四六年一月十日至三十一日在重庆举行的政治协商会议。这个会议通过了五项议案。(一)关于政府组织问题的协议。这个协议确定“修改国民政府组织法,以充实国民政府委员会”。增加国民政府委员的名额;“国民政府委员由国民政府主席就中国国民党内外人士选任之”,“国民政府主席提请选任各党派人士为国府委员时,由各党派自行提名,但主席不同意时,由各该党派另提人选”;“国民政府主席提请选任无党派人士为国府委员时,如所提人选有为各被选人三分之一所反对者,则主席须重新考虑,另行选任之”。“国府委员名额之半由国民党人员充任,其余半数由其它各党派及社会贤达充任”。国民政府委员会抽象地规定为“政府之最高国务机关”,其权力为讨论和决定立法原则、施政方针、军政大计、财政计划和预算以及国民政府主席交议的事项等;同时,国民政府主席却有很大的权力,有指定权、议案的相对否决权和紧急处置权。又规定国民政府“行政院现有部会及拟设之不管部会政务委员总额中,将以七席或八席,约请国民党以外人士充任”。(二)和平建国纲领。这个纲领包括《总则》、《人民权利》、《政治》、《军事》、《外交》、《经济及财政》、《教育及文化》、《善后救济》、《侨务》等九章。在《总则》一章中,规定全国各党派“团结一致,建设统一自由民主之新中国”;实行“政治民主化、军队国家化及党派平等合法”;“用政治方法解决政治纠纷,以保持国家之和平发展”。在《人民权利》一章中,规定“确保人民享有身体、思想、宗教、信仰、言论、出版、集会、结社、居住、迁徙、通讯之自由”,“严禁司法及警察以外任何机关或个人,有拘捕、审讯及处罚人民之行为,犯者应予惩处”。在《政治》一章中规定“整饬各级行政机构,统一并划清权责,取消一切骈枝机关,简化行政手续,实行分层负责”;“保障称职人员,用人不分派别,以能力、资历为标准,禁止兼职及私人援引”;“厉行监察制度,严惩贪污,便利人民自由告发”;“积极推行地方自治,实行由下而上之普选”;“中央与地方之权限,采均权主义,各地得采取因地制宜之措施,但省、县所颁之法规,不得与中央法令相抵触”。在《军事》一章中,规定“军队建制应适合国防需要,依民主政制与国情改革军制,实行军党分立,军民分治,改进军事教育,充实装备,健全人事、经理制度,以建设现代化之国军”;“全国军队应按照整军计划切实缩编”。在《经济及财政》一章中,规定“防止官僚资本之发展,并严禁官吏利用其权势地位,从事于投机、垄断、逃税、走私、挪用公款与非法使用交通工具”;“实行减租减息,保护佃权,保证交租,扩大农贷,严禁高利盘剥,以改善农民生活,并实行土地法,以期达到‘耕者有其田’之目的”;“实行劳动法,改善劳动条件”;“财政公开。厉行预算决算制度,紧缩支出,平衡收支,划分中央与地方财政,收缩通货,稳定币制,并公布内外债之募集及用途,由民意机关监督之”;“改革税制,根绝苛杂与非法摊派”。在《教育及文化》一章中,规定“保障学术自由,不以宗教信仰政治思想干涉学校行政”;“在国家预算中,增加教育及文化事业经费之比率”;“废止战时实施之新闻、出版、电影、戏剧、邮电检查办法”。(三)关于国民大会问题的协议。这个协议规定,国民大会的代表“增加党派及社会贤达代表七百名”,“第一届国民大会之职权为制定宪法”。(四)关于宪法草案问题的协议。这个协议规定,组织宪草审议委员会,修改国民党的宪法草案;并且规定了宪草修改的原则。除对国民大会、政府机构的职权作了原则规定外,特别对于地方制度和人民之权利义务作了规定。关于地方制度,规定“确定省为地方自治之最高单位”;“省与中央权限之划分依照均权主义规定”;“省长民选”;“省得制定省宪,但不得与国宪抵触”。关于人民之权利义务,规定“凡民主国家人民应享之自由及权利,均应受宪法之保障,不受非法之侵犯”;“关于人民自由,如用法律规定,须出之于保障自由之精神,非以限制为目的”;“工役应规定于自治法内,不在宪法内规定”;“聚居于一定地方之少数民族,应保障其自治权”。(五)关于军事问题的协议。这个协议规定“军队制度应依我国民主政制与国情实行改革”;“改善征兵制度”;“军队教育应依建军原则办理,永远超出于党派系统及个人关系以外”;“实行军党分立”,“任何党派及个人不得利用军队为政争之工具”;“实行军民分治”,“凡在军队中任职之现役军人,不得兼任行政官吏”。关于整编国民党军队和解放区军队,规定“军事三人小组应照原定计划尽速商定中共军队整编办法,整编完竣”;国民党军队“依军政部原定计划,尽速于六个月内完成其九十师之整编”;“上两项整编完竣,应再将全国所有军队统一整编为五十师或六十师”。政治协商会议的这些协议,在各种不同程度上有利于人民而不利于蒋介石的反动统治。蒋介石一方面表示承认这些协议,企图利用这些协议来进行和平欺骗,另一方面则积极备战,准备发动全国规模的内战。政治协商会议的这些协议,在后来不久都被蒋介石一一撕毁。},在东北占我四平、长春等地后,现在又在华东、华北向我大举进攻,将来亦有可能再向东北进攻。只有在自卫战争中彻底粉碎蒋介石的进攻之后,中国人民才能恢复和平。\\
  (二)我党我军正准备一切,粉碎蒋介石的进攻,借此以争取和平。蒋介石虽有美国援助,但是人心不顺,士气不高,经济困难。我们虽无外国援助,但是人心归向,士气高涨,经济亦有办法。因此,我们是能够战胜蒋介石的。全党对此应当有充分的信心。\\
  (三)战胜蒋介石的作战方法,一般地是运动战。因此,若干地方,若干城市的暂时放弃,不但是不可避免的,而且是必要的。暂时放弃若干地方若干城市,是为了取得最后胜利,否则就不能取得最后胜利。此点,应使全党和全解放区人民都能明白,都有精神准备。\\
  (四)为着粉碎蒋介石的进攻,必须和人民群众亲密合作,必须争取一切可能争取的人。在农村中,一方面应坚定地解决土地问题,紧紧依靠雇农、贫农,团结中农;另方面在进行解以自卫战争粉碎蒋介石的进攻决土地问题时,应将一般富农、中小地主分子和汉奸、豪绅、恶霸分子,加以区别。对待汉奸、豪绅、恶霸要放严些,对待富农、中小地主要放宽些。在一切土地问题已经解决的地方,除少数反动分子外,应对整个地主阶级改取缓和态度。对一切生活困难的地主给以帮助,对逃亡地主招引其回来,给以生活出路,借以减少敌对分子,使解放区得到巩固。在城市中,除团结工人阶级、小资产阶级和一切进步分子外,应注意团结一切中间分子,孤立反动派。在国民党军队中,应争取一切可能反对内战的人,孤立好战分子。\\
  (五)为着粉碎蒋介石的进攻,必须作持久打算。必须十分节省地使用我们的人力资源和物质资源,力戒浪费。必须检查和纠正各地已经发生的贪污现象。必须努力生产,使一切必需品,首先是粮食和布匹,完全自给。必须提倡普遍植棉,家家纺纱,村村织布。即在东北亦应开始提倡。在财政供给上,必须使自卫战争的物质需要得到满足,同时又必须使人民负担较前减轻,使我解放区人民虽然处在战争环境,而其生活仍能有所改善。总之,我们是一切依靠自力更生,立于不败之地,和蒋介石的一切依靠外国,完全相反。我们是艰苦奋斗,军民兼顾,和蒋介石统治区的上面贪污腐化,下面民不聊生,完全相反。在这种情形下,我们是一定要胜利的。\\
  (六)我们面前存在着困难,但是这些困难能够克服和必须克服。全党同志和全解放区军民,必须团结一致,彻底粉碎蒋介石的进攻,建立独立、和平、民主的新中国。\\
\newpage\section*{\myformat{和美国记者安娜·刘易斯·斯特朗\footnote[1]{ 安娜·刘易斯·斯特朗(一八八五——一九七〇),美国进步作家和记者。从一九二五年起多次访华。一九四六年六月第五次到中国访问,八月毛泽东在延安杨家岭会见了她,发表了这篇谈话。一九五八年她来中国定居。一九六二年后,定期编写《中国通讯》,向国外读者宣传新中国建设的成就。着有《千千万万中国人民》、《人类的五分之一》和《中国人征服中国》等书。}的谈话}\\\myformat{(一九四六年八月六日)}}\addcontentsline{toc}{section}{和美国记者安娜·刘易斯·斯特朗\footnote[1]{ 安娜·刘易斯·斯特朗(一八八五——一九七〇),美国进步作家和记者。从一九二五年起多次访华。一九四六年六月第五次到中国访问,八月毛泽东在延安杨家岭会见了她,发表了这篇谈话。一九五八年她来中国定居。一九六二年后,定期编写《中国通讯》,向国外读者宣传新中国建设的成就。着有《千千万万中国人民》、《人类的五分之一》和《中国人征服中国》等书。}的谈话}
\begin{introduction}\item  这是毛泽东在第二次世界大战结束不久,关于国际形势和国内形势的一篇很重要的谈话。在这篇谈话里,毛泽东提出了“一切反动派都是纸老虎”的著名论点。这个论点,武装了中国人民的思想,加强了中国人民的胜利信心,在人民解放战争中,起了极其伟大的作用。同列宁把帝国主义看做“泥足巨人”一样,毛泽东把帝国主义和一切反动派看做纸老虎,都是从它们的本质说的。这是革命人民的一个根本战略思想。从第二次国内革命战争时期以来,毛泽东曾经多次地指出,革命者必须在战略上,在全体上,藐视敌人,敢于同他们斗争,敢于夺取胜利;同时,又要在战术上,在策略上,在每一个局部上,在每一个具体斗争问题上,重视敌人,采取谨慎态度,讲究斗争艺术,根据不同的时间、地点和条件,采取适当的斗争形式,以便一步一步地孤立敌人和消灭敌人。毛泽东一九五八年十二月一日在中共八届六中全会期间写的《关于帝国主义和一切反动派是不是真老虎的问题》一文中指出:“同世界上一切事物无不具有两重性(即对立统一规律)一样,帝国主义和一切反动派也有两重性,它们是真老虎又是纸老虎。历史上奴隶主阶级、封建地主阶级和资产阶级,在它们取得统治权力以前和取得统治权力以后的一段时间内,它们是生气勃勃的,是革命者,是先进者,是真老虎。在随后的一段时间,由于它们的对立面,奴隶阶级、农民阶级和无产阶级,逐步壮大,并同它们进行斗争,越来越厉害,它们就逐步向反面转化,化为反动派,化为落后的人们,化为纸老虎,终究被或者将被人民所推翻。反动的、落后的、腐朽的阶级,在面临人民的决死斗争的时候,也还有这样的两重性。一面,真老虎,吃人,成百万人成千万人地吃。人民斗争事业处在艰难困苦的时代,出现许多弯弯曲曲的道路。中国人民为了消灭帝国主义、封建主义和官僚资本主义在中国的统治,花了一百多年时间,死了大概几千万人之多,才取得一九四九年的胜利。你看,这不是活老虎,铁老虎,真老虎吗?但是,它们终究转化成了纸老虎,死老虎,豆腐老虎。这是历史的事实。人们难道没有看见听见过这些吗?真是成千成万!成千成万!所以,从本质上看,从长期上看,从战略上看,必须如实地把帝国主义和一切反动派,都看成纸老虎。从这点上,建立我们的战略思想。另一方面,它们又是活的铁的真的老虎,它们会吃人的。从这点上,建立我们的策略思想和战术思想。”关于必须在战略上藐视敌人战术上重视敌人的问题,参看本书第一卷《中国革命战争的战略问题》的第五章第六节和本卷《关于目前党的政策中的几个重要问题》的第一部分。\end{introduction}
\textbf{斯特朗问}:你觉得中国的问题,在不久的将来,有政治解决、和平解决的希望没有?\\
  \textbf{毛答}:这要看美国政府的态度。如果美国人民拖住了帮助蒋介石打内战的美国反动派的手的话,和平是有希望的。\\
  \textbf{问}:如果美国除了它所已经给的以外不再帮助了\footnote[2]{ 美帝国主义为了支持蒋介石发动反人民内战,给了国民党政府以巨大援助。到一九四六年六月为止,美国共装备国民党军四十五个师。为国民党训练陆军、海军、空军、特务、交通警察、参谋、军医、军需等军事人员十五万人。美国用军舰、飞机运送了国民党军队十四个军共四十一个师,另外八个交通警察总队,共计五十四万余人,到进攻解放区的前线。美国政府以九万人的海军陆战队在中国登陆,占驻上海、青岛、天津、北平、秦皇岛等重要城市,在华北并为国民党保护交通线。美国给国民党政府的各种援助,据一九四九年八月五日美国国务院发表的《美国与中国的关系(白皮书)》披露的材料,从一九三七年抗日战争开始到一九四八年,总数共达四十五亿多美元(美国在抗日战争中给予国民党政府的援助,绝大部分也被国民党留作以后进行反人民内战之用)。但实际上,美国援蒋数字远不止这个数目。美国白皮书承认,美国给国民党政府的援助,等于国民党政府“货币支出的百分之五十以上,与该国政府预算相较,在比例上超过战后美国对任何西欧国家的援助数量”。},那末蒋介石还可以打多久?\\
  \textbf{答}:一年以上。\\
  \textbf{问}:蒋介石在经济上可能支持那样久吗?\\
  \textbf{答}:可以的。\\
  \textbf{问}:如果美国说明此后不再给蒋介石以什么帮助了呢?\\
  \textbf{答}:在现时还没有什么征象,表示美国政府和蒋介石有任何在短时期内停止战争的愿望。\\
  \textbf{问}:共产党能支持多久?\\
  \textbf{答}:就我们自己的愿望说,我们连一天也不愿意打。但是如果形势迫使我们不得不打的话,我们是能够一直打到底的。\\
  \textbf{问}:如果美国人民问到共产党为什么作战,我该怎样回答呢?\\
  \textbf{答}:因为蒋介石要屠杀中国人民,人民要生存就必须自卫。这是美国人民所能够理解的。\\
  \textbf{问}:你对于美国是否可能举行反苏战争如何看法?\\
  \textbf{答}:关于反苏战争的宣传,包括两个方面。在一方面,美国帝国主义确是在准备着反苏战争的,目前的反苏战争宣传和其它的反苏宣传,就是对于反苏战争的政治准备。在另一方面,这种宣传,是美国反动派用以掩盖当前美国帝国主义所直接面对着的许多实际矛盾,所放的烟幕。这些矛盾,就是美国反动派同美国人民之间的矛盾,以及美国帝国主义同其它资本主义国家和殖民地、半殖民地国家之间的矛盾。美国反苏战争的口号,在目前的实际意义,是压迫美国人民和向资本主义世界扩张它的侵略势力。你知道,希特勒和他的伙伴日本军阀,在一个长时期中,都曾经把反苏的口号作为奴役本国人民和侵略其它国家的托辞。现在美国反动派的做法,也正是这样。\\
  美国反动派要掀动战争,首先必须进攻美国人民。他们已经在进攻美国人民了,他们从政治上、经济上压迫美国的工人和民主分子,准备在美国实行法西斯主义。美国人民应当起来抵抗美国反动派的进攻。我相信他们是会这样做的。\\
  美国和苏联中间隔着极其辽阔的地带,这里有欧、亚、非三洲的许多资本主义国家和殖民地、半殖民地国家。美国反动派在没有压服这些国家之前,是谈不到进攻苏联的。现在美国在太平洋控制了比英国过去的全部势力范围还要多的地方,它控制着日本、国民党统治的中国、半个朝鲜和南太平洋;它早已控制着中南美;它还想控制整个大英帝国和西欧。美国在各种借口之下,在许多国家进行大规模的军事布置,建立军事基地。美国反动派说,他们在世界各地已经建立和准备建立的一切军事基地,都是为着反对苏联的。不错,这些军事基地是指向苏联。但是,在现时,首先受到美国侵略的不是苏联,而是这些被建立军事基地的国家。我相信,不要很久,这些国家将会认识到真正压迫它们的是谁,是苏联还是美国。美国反动派终有一天将会发现他们自己是处在全世界人民的反对中。\\
  当然,我不是说,美国反动派不想进攻苏联。苏联是世界和平的保卫者,是阻碍美国反动派建立世界霸权的强大的因素,有了苏联,美国和世界反动派的野心就根本不能实现。因此,美国反动派非常痛恨苏联,确实梦想消灭这个社会主义国家。但是在目前,在第二次世界大战结束不久的时候,美国反动派如此大吹大擂地强调美苏战争,闹得乌烟瘴气,就使人不能不来看看他们的实际目的。原来他们是在反苏的口号下面,疯狂地进攻美国的工人和民主分子,和把美国向外扩张的一切对象国都变成美国的附属物。我以为,美国人民和一切受到美国侵略威胁的国家的人民,应当团结起来,反对美国反动派及其在各国的走狗的进攻。只有这个斗争胜利了,第三次世界大战才可以避免,否则是不能避免的。\\
  \textbf{问}:这是一个很好的说明。但是如果美国使用原子炸弹呢?如果美国从冰岛、冲绳岛以及中国的基地轰炸苏联呢?\\
  \textbf{答}:原子弹是美国反动派用来吓人的一只纸老虎,看样子可怕,实际上并不可怕。当然,原子弹是一种大规模屠杀的武器,但是决定战争胜败的是人民,而不是一两件新式武器。\\
  一切反动派都是纸老虎。看起来,反动派的样子是可怕的,但是实际上并没有什么了不起的力量。从长远的观点看问题,真正强大的力量不是属于反动派,而是属于人民。在一九一七年俄国二月革命以前,俄国国内究竟哪一方面拥有真正的力量呢?从表面上看,当时的沙皇是有力量的;但是二月革命的一阵风,就把沙皇吹走了。归根结蒂,俄国的力量是在工农兵苏维埃这方面。沙皇不过是一只纸老虎。希特勒不是曾经被人们看作很有力量的吗?但是历史证明了他是一只纸老虎。墨索里尼也是如此,日本帝国主义也是如此。相反的,苏联以及各国爱好民主自由的人民的力量,却是比人们所预料的强大得多。\\
  蒋介石和他的支持者美国反动派也都是纸老虎。提起美国帝国主义,人们似乎觉得它是强大得不得了的,中国的反动派正在拿美国的“强大”来吓唬中国人民。但是美国反动派也将要同一切历史上的反动派一样,被证明为并没有什么力量。在美国,另有一类人是真正有力量的,这就是美国人民。\\
  拿中国的情形来说,我们所依靠的不过是小米加步枪,但是历史最后将证明,这小米加步枪比蒋介石的飞机加坦克还要强些。虽然在中国人民面前还存在着许多困难,中国人民在美国帝国主义和中国反动派的联合进攻之下,将要受到长时间的苦难,但是这些反动派总有一天要失败,我们总有一天要胜利。这原因不是别的,就在于反动派代表反动,而我们代表进步。
\newpage\section*{\myformat{集中优势兵力,各个歼灭敌人}\\\myformat{(一九四六年九月十六日)}}\addcontentsline{toc}{section}{集中优势兵力,各个歼灭敌人}
\begin{introduction}\item  这是毛泽东为中共中央军事委员会起草的对党内的指示。\end{introduction}
(一)集中优势兵力、各个歼灭敌人\footnote[1]{ 在本书各篇中所说的“歼灭敌人”或者“消灭敌人”,都是包括把敌人击毙、击伤或俘虏。}的作战方法,不但必须应用于战役的部署方面,而且必须应用于战术的部署方面。\\
  (二)在战役的部署方面,当着敌人使用许多个旅\footnote[2]{ 国民党正规军的编制,原来是一个军下辖三个师或两个师,每师下辖三个团。一九四六年三月起,国民党分期整编了当时在黄河以南地区的正规军,将原来的军改为整编师,师改编为旅,旅下辖三个团或两个团。在黄河以北地区的军队没有整编,仍按原来编制。一九四八年九月以后,国民党又将整编师恢复为军,旅改称师。}(或团)分几路向我军前进的时候,我军必须集中绝对优势的兵力,即集中六倍、或五倍、或四倍于敌的兵力、至少也要有三倍于敌的兵力,于适当时机,首先包围歼击敌军的一个旅(或团)。这个旅(或团),应当是敌军诸旅中较弱的,或者是较少援助的,或者是其驻地的地形和民情对我最为有利而对敌不利的。我军以少数兵力牵制敌军的其余各旅(或团),使其不能向被我军围击的旅(或团)迅速增援,以利我军首先歼灭这个旅(或团)。得手后,依情况,或者再歼敌军一个旅至几个旅(例如我粟谭军在如皋附近,八月二十一、二十二日歼敌交通警察部队五千,八月二十六日又歼敌一个旅,八月二十七日又歼敌一个半旅\footnote[3]{ 一九四六年七月,国民党军队大举进犯苏皖解放区,人民解放军奋起自卫。进攻苏中解放区的敌军,是汤恩伯指挥的十五个旅,约十二万人。华中野战军司令员粟裕、政治委员谭震林指挥十九个团(后增至二十多个团)的兵力,自七月十三日至八月三十一日,在苏中的泰兴、如皋、海安、邵伯一带,集中优势兵力,连续作战七次,歼敌军六个旅、五个交通警察大队,共五万三千余人。本文所提到的是其中三次作战的战果。}。又如我刘邓军在定陶附近,九月三日至九月六日歼敌一个旅,九月六日下午又歼敌一个旅,九月七日至九月八日又歼敌两个旅\footnote[4]{ 一九四六年八月,国民党军队自徐州、郑州一带,分两路进犯晋冀鲁豫解放区。晋冀鲁豫野战军主力在司令员刘伯承、政治委员邓小平等指挥下,集中优势兵力迎击自郑州出发的一路敌军,自九月三日至九月八日,先后在山东的菏泽、定陶、曹县一带歼灭敌军四个旅共一万七千余人。});或者收兵休整,准备再战。在战役部署上,必须反对那种轻视敌人、因而平分兵力对付诸路之敌、以致一路也不能歼灭、使自己陷于被动地位的错误的作战方法。\\
  (三)在战术的部署方面,当着我军已经集中绝对优势兵力包围敌军诸路中的一路(一个旅或一个团)的时候,我军担任攻击的各兵团(或各部队),不应企图一下子同时全部地歼灭这个被我包围之敌,因而平分兵力,处处攻击,处处不得力,拖延时间,难于奏效。而应集中绝对优势兵力,即集中六倍、五倍、四倍于敌,至少也是三倍于敌的兵力,并集中全部或大部的炮兵,从敌军诸阵地中,选择较弱的一点(不是两点),猛烈地攻击之,务期必克。得手后,迅速扩张战果,各个歼灭该敌。\\
  (四)这种战法的效果是:一能全歼;二能速决。全歼,方能最有效地打击敌军,使敌军被歼一团少一团,被歼一旅少一旅。对于缺乏第二线兵力的敌人,这种战法最为有用。全歼,方能最充分地补充自己。这不但是我军目前武器弹药的主要来源,而且是兵员的重要来源。全歼,在敌则士气沮丧,人心不振;在我则士气高涨,人心振奋。速决,则使我军有可能各个歼灭敌军的增援队,也使我军有可能避开敌军的增援队。在战术和战役上的速决,是战略上持久的必要条件。\\
  (五)现在我军干部中,还有许多人,在平时,他们赞成集中兵力各个歼敌的原则;但到临战,则往往不能应用这一原则。这是轻敌的结果,也是没有加强教育和着重研究的结果。必须详举战例,反复说明这种作战方法的好处,指出这是战胜蒋介石进攻的主要方法。实行这种方法,就会胜利。违背这种方法,就会失败。\\
  (六)集中兵力各个歼敌的原则,是我军从开始建军起十余年以来的优良传统,并不是现在才提出的。但是在抗日时期,我军以分散兵力打游击战为主,以集中兵力打运动战为辅。在现在的内战时期,情况改变了,作战方法也应改变,我军应以集中兵力打运动战为主,以分散兵力打游击战为辅。而在蒋军武器加强的条件下,我军必须特别强调集中优势兵力、各个歼灭敌人的作战方法。\\
  (七)在敌处进攻地位、我处防御地位的时候,必须应用这一方法。在敌处防御地位、我处进攻地位的时候,则应分为两种情况,采取不同的方法。如果我军兵力多,当地敌军较弱,或者我军出敌不意举行袭击的时候,可以同时攻击若干部分的敌军。例如,六月五日至六月十日,山东我军同时攻击胶济、津浦两路十几个城镇而占领之\footnote[5]{ 一九四六年六月上旬,山东野战军和山东军区所属一部,为了打击国民党收编的在停战令生效后不断向解放区挑衅的伪军,决定在胶济线和津浦线上发动讨逆战役。先后解放胶县、张店、周村、德州、泰安、枣庄、高密、即墨等十余座城镇。}。又如,八月十日至八月二十一日,我刘邓军攻击陇海路汴徐线十几个城镇而占领之\footnote[6]{ 晋冀鲁豫野战军主力和冀鲁豫军区部队一部,为了配合中原、华东的人民解放军作战,于一九四六年八月十日至二十一日,分数路向驻在陇海路上开封至徐州一线的国民党军出击,先后占领砀山、兰封(今并入兰考)、李庄、杨集等十余座城镇。}。如果我军兵力不足,则应对敌军所占诸城一个一个地夺取之,而不要同时攻击几个城镇的敌人。例如山西我军夺取同蒲路上诸城\footnote[7]{ 一九四六年七月,国民党军胡宗南部和阎锡山部联合进犯晋南解放区。晋冀鲁豫野战军第四纵队、太岳军区部队和晋绥军区所属一部举行反击,击退敌军的进犯;八月,又向驻在同蒲路上临汾至灵石一线的敌军发起攻势,至九月一日战役结束,先后解放洪洞、赵城、霍县、灵石、汾西诸城。},就是这样打的。\\
  (八)我军主力集中歼敌的时候,必须同地方兵团、地方游击队和民兵的积极活动,互相配合。地方兵团(或部队)在打敌一团一营一连的时候,也适用集中兵力各个歼敌的原则。\\
  (九)集中兵力各个歼敌的原则,以歼灭敌军有生力量为主要目标,不以保守或夺取地方为主要目标。有些时机,为着集中兵力歼击敌军的目的,或使我军主力避免遭受敌军的严重打击以利休整再战的目的,可以允许放弃某些地方。只要我军能够将敌军有生力量大量地歼灭了,就有可能恢复失地,并夺取新的地方。因此,凡能歼灭敌军有生力量者,均应奖励之。不但歼灭敌军的正规部队应当受到奖励;就是歼灭敌军的保安队、还乡队\footnote[8]{ 人民解放战争时期,有一些解放区的地主、恶霸逃到国民党统治区。国民党把他们组织成为“还乡队”、“还乡团”等反动武装,随国民党军队进攻解放区。这些地主反动武装,到处抢掠屠杀,无恶不作。}等反动的地方武装,也应当受到奖励。但是,凡在敌我力量对比上能够保守或夺取的地方和在战役上战术上有意义的地方,则必须保守或夺取之,否则就是犯错误。因此,凡能保守或夺取这些地方者,也应受到奖励。\\
\newpage\section*{\myformat{美国“调解”真相和中国内战前途}\\\myformat{——和美国记者斯蒂尔的谈话}\\\myformat{(一九四六年九月二十九日)}}\addcontentsline{toc}{section}{美国“调解”真相和中国内战前途}
\textbf{斯蒂尔问}:阁下是否认为美国调解中国内战之举已告失败?如美国政策按目前形式继续实行,则结局将如何?\\
  \textbf{毛答}:我很怀疑美国政府的政策是所谓调解\footnote[1]{ 一九四五年十二月,美国政府派马歇尔为总统特使来华,以“调解国共军事冲突”为名,从各方面加强美国侵略势力和国民党反动派的统治地位。蒋介石为着争取时间布置内战,在表面上接受了中国共产党和中国人民关于停止内战的要求。一九四六年一月,国民党政府的代表和中国共产党的代表签订了停战协定,发布了停战令,并组成了有美国代表参加的“三人小组”和“北平军事调处执行部”。在进行所谓“调处”的时期内,美方首先在东北,后来又在华北、华东、华中,协助国民党军队进犯解放区,并积极训练和装备国民党军队,供给蒋介石以大量的军火和其它作战物资。至一九四六年六月,蒋介石已将国民党正规军总兵力(大约二百万人)的百分之八十调集到进攻解放区的前线,其中有五十四万多人是美国武装部队直接用军舰、飞机帮助运送的。蒋介石在布置就绪之后,即于六月下旬发动了全国规模的反革命战争。接着,马歇尔就在八月十日和美国驻中国大使司徒雷登发表联合声明,宣布“调处”失败,以便让蒋介石放手打内战。}。根据美国大量援助蒋介石使得他能够举行空前大规模内战的事实看来,美国政府的政策是在借所谓调解作掩护,以便从各方面加强蒋介石,并经过蒋介石的屠杀政策,压迫中国民主力量,使中国在实际上变为美国的殖民地。这一政策继续实行下去,必将激起全中国一切爱国人民起来作坚决的反抗。\\
  \textbf{问}:中国内战将延长多久?其结果将如何?\\
  \textbf{答}:如果美国政府放弃现行援蒋政策,撤退驻华美军,实行莫斯科苏美英三国外长会议的协定\footnote[2]{ 指一九四五年十二月苏、美、英三国外长莫斯科会议关于中国问题的协议。在莫斯科会议公报中,三国外长重申坚持不干涉中国内政的政策;苏美外长并一致同意苏美两国军队尽早撤离中国。苏联忠实地履行了这个协议。只是由于国民党政府的一再请求,苏联军队才推迟了撤军的时间。一九四六年五月三日,苏军即全部从中国东北境内撤出。但美国政府却完全违背了自己的诺言,拒不撤走自己的军队,并且变本加厉地干涉中国内政。},则中国内战必能早日结束。如果不是这样,就有变为长期战争的可能。其结果,一方面,当然是中国人民受痛苦;但是,另一方面,中国人民必将团结起来,保卫自己的生存,决定自己的命运。不管怎样艰难困苦,中国人民的独立、和平、民主的任务是一定要实现的。任何本国和外国的压迫力量,不可能阻止这一任务的实现。\\
  \textbf{问}:阁下是否认为蒋介石是中国人民的“当然领袖”?共产党是否将在任何情况之下都不接受蒋介石的五项要求\footnote[3]{ 一九四六年六月中旬,蒋介石曾向中国共产党提出了一些无理要求。同年八月,蒋介石又提出五项要求,要中国人民解放军退出下列各地:一、陇海路以南的一切地区;二、胶济线全线;三、承德和承德以南的地区;四、东北的大部分;五、一九四六年六月七日以后解放区人民武装在山东、山西两省从伪军手里解放出来的一切地区。蒋介石还声称只有中共方面承认了这些要求,才能考虑停止内战。中国共产党坚决地拒绝了这些无理的要求。}?如果国民党企图召集一个无共产党参加的国民大会\footnote[4]{ 按照一九四六年一月政治协商会议的决议和会议中协商的精神,国民大会应该是各党派参加的民主的团结的大会,必须在政协各项协议付诸实施之后,在改组后的政府领导之下才能召开。但是国民党反动派违反政协决议,在同年六月下旬发动全国性内战,十月十一日国民党军队占领张家口,蒋介石被这一“胜利”冲昏了头脑,即于当日下令召开国民大会。十一月十五日至十二月二十五日这个由国民党一手包办的“国民大会”在南京召开。除青年党、民社党及极少数“社会贤达”外,中国共产党和各民主党派、各人民团体均拒绝参加并严正声明不承认这个国民大会,使蒋介石国民党在政治上陷于孤立。},则共产党将采取何种行动?\\
  \textbf{答}:世界上无所谓“当然领袖”。蒋介石如能按照今年一月间的停战协定\footnote[5]{ 见本卷《以自卫战争粉碎蒋介石的进攻》注〔1〕。}和政治协商会议的共同决议\footnote[6]{ 见本卷《以自卫战争粉碎蒋介石的进攻》注〔2〕。}处理中国政治军事经济等项问题,而不是按照所谓“五项”或十项违反上述协定和决议的片面要求,那末,我们是仍然愿意和他共事的。国民大会只应当按照政治协商会议的决议由各党派共同负责去召集,否则我们将采取坚决反对的态度。\\
\newpage\section*{\myformat{三个月总结}\\\myformat{(一九四六年十月一日)}}\addcontentsline{toc}{section}{三个月总结}
\begin{introduction}\item  这是毛泽东为中共中央起草的对党内的指示。这个指示,详细地总结了一九四六年七月全国规模内战爆发以来的三个月战争的一系列经验,提出了人民解放军在今后的作战方针和作战任务,指出了人民解放战争在克服一个时期的困难以后必然能够取得胜利。对于支持和配合人民解放战争所需要解决的解放区内部的土地改革问题,发展解放区的生产问题,加强国民党统治区的群众斗争的领导问题,以及其它有关问题,这个文件也作了原则的说明。\end{introduction}
(一)七月二十日中央对时局的指示\footnote[1]{ 即本卷《以自卫战争粉碎蒋介石的进攻》。}上说:“我们是能够战胜蒋介石的。全党对此应当有充分的信心。”七、八、九三个月的作战,业已证明此项断语是正确的。\\
  (二)除了政治上经济上的基本矛盾,蒋介石无法克服,为我必胜蒋必败的基本原因之外,在军事上,蒋军战线太广与其兵力不足之间,业已发生了尖锐的矛盾。此种矛盾,必然要成为我胜蒋败的直接原因。\\
  (三)向解放区进攻的全部正规蒋军,除伪军、保安队、交通警察部队等不计外,共计一百九十几个旅。此数以外,至多再从南方抽调一部分兵力向北增援,此后即难再调。而此一百九十几个旅中,过去三个月内,已被我军歼灭二十五个旅。今年二月至六月被我军在东北所歼灭者,尚未计算在内。\\
  (四)蒋军一百九十几个旅中,须以差不多半数任守备,能任野战者不过半数多一点。而这些任野战的兵力进到一定地区,又必不可免地要以一部至大部改任守备。敌人的野战军,一方面,不断地被我歼灭,另方面,大量地担任守备,因此,它就必定越打越少。\\
  (五)三个月被我歼灭的二十五个旅中,计汤恩伯(原为李默庵)七个旅,薛岳两个旅,顾祝同(原为刘峙)七个旅,胡宗南两个旅,阎锡山四个旅,王耀武两个旅,杜聿明一个旅。除李宗仁、傅作义、马鸿逵、程潜四部,尚未受到我军歼灭性的打击之外,其余七部,或者受到我军相当严重的打击,或者受了初步的打击。受我严重打击者,有杜聿明(包括今年二月至六月在东北的作战)、汤恩伯、顾祝同、阎锡山。受我初步打击者,有薛岳、胡宗南、王耀武。所有这些,都证明我军能够战胜蒋介石。\\
  (六)今后一个时期内的任务,是再歼灭敌军约二十五个旅。这个任务完成了,即可能停止蒋军的进攻,并可能部分地收复失地。可以预计,在歼灭第二个二十五个旅这一任务完成的时候,我军必能夺取战略上的主动,由防御转入进攻。那时的任务,是歼灭敌军第三个二十五个旅。果能如此,就可以收复大部至全部失地,并可以扩大解放区。那时国共军力对比,必起重大变化。欲达此目的,必须在今后三个月内外,继续过去三个月歼敌二十五个旅的伟大成绩,再歼敌二十五个旅左右。这是改变敌我形势的关键\footnote[2]{ 后来情况证明,开始转变敌我形势是在一九四七年六月底晋冀鲁豫野战军主力强渡黄河向大别山进军的时候。这时人民解放军作战已十二个月,平均每月歼灭国民党军八个旅,共已歼灭敌军约一百个旅,比较这里估计的数目要多些。这是蒋介石在美帝国主义的支持下使用其一切可能的力量来进攻的缘故。}。\\
  (七)过去三个月内,我们损失淮阴、菏泽、承德、集宁等几十个中小城市。其中多数是不可避免地要放弃和应当主动地暂时放弃的;一部分是仗打得不好被迫放弃的。不管怎样,只要今后仗打得好,失地即可收复。今后还可能有一部分地方,在不得已时被敌占去,但是将来均可收复。各地应检讨过去作战经验,吸取教训,避免重犯错误。\\
  (八)过去三个月内,我中原解放军以无比毅力克服艰难困苦,除一部已转入老解放区外,主力在陕南、鄂西两区,创造了两个游击根据地\footnote[3]{ 一九四六年六月底,李先念、郑位三等领导的中原军区部队在国民党军队三十万人的包围进攻下开始主动地作战略转移,胜利地突出了敌人的包围。毛泽东这里所说的转入老解放区的一部,是指突围后转入陕甘宁边区的由王震等所领导的部队。另有一部,由皮定钧等领导转入苏皖解放区。在陕南创造的游击根据地,是指由李先念、郑位三率领一部主力突围到达陕南后,与当地游击队会合,在卢氏、洛南、灵宝、潼关、华阴等地创建的豫鄂陕游击根据地。在鄂西创造的游击根据地,是指由王树声率领的另一部主力突围后进入武当山区,创建了以武当山为中心的鄂西北游击根据地。还有一部分散在大别山地区坚持游击战争。}。此外,在鄂东和鄂中均有部队坚持游击战争。这些都极大地援助了和正在继续援助着老解放区的作战,并将对今后长期战争起更大的作用。\\
  (九)过去三个月战争,吸引了蒋介石原拟调赴东北的几支有力部队于关内,使我们在东北得到休整军队、发动群众的时间,这对将来斗争也有重大的意义。\\
  (十)集中优势兵力,各个歼灭敌人,是过去三个月歼敌二十五个旅时所采用的唯一正确的作战方法。我们集中的兵力必须六倍、五倍、四倍、至少三倍于敌,方能有效地歼敌。不论在战役上,战术上,都须如此。不论是高级指挥员,或中下级干部,都须学会此种作战方法。\\
  (十一)过去三个月内,我军不但歼灭敌正规军二十五个旅,又歼灭伪军、保安队、交通警察部队等反动军队为数不少,这也是一个大成绩。今后仍应大量地歼灭此类军队。\\
  (十二)过去三个月经验证明:歼灭敌军一万人,自己须付出二千至三千人的伤亡作代价。这是不可避免的。为应付长期战争(各地应处处从长期战争着眼),今后必须有计划地扩兵,保证主力军经常满员,并大量训练军事干部。必须有计划地发展生产和整理财政,遵照发展经济,保障供给,统一领导,分散经营,军民兼顾,公私兼顾\footnote[4]{ 见本卷《一九四六年解放区工作的方针》注〔4〕。}等项原则,坚决地实施之。\\
  (十三)三个月经验证明:在一月至六月休战期间,凡依照中央指示加紧进行了军事训练的军队(中央曾反复指示各地应以练兵、生产和土地改革等三项工作为中心任务),其作战效能就高些,否则效能就低得很多。今后各区必须利用作战间隙,加强军事训练。一切军队必须加强政治工作。\\
  (十四)三个月经验证明:凡坚决和迅速地执行了中央五月四日的指示\footnote[5]{ 即一九四六年五月四日中共中央《关于土地问题的指示》,通称《五四指示》。日本投降以后,中共中央根据农民对土地的迫切要求,决定改变党在抗日战争时期的土地政策,由减租减息改为没收地主土地分配给农民。《五四指示》的制定就是表现这种改变。},深入和彻底地解决了土地问题的地方,农民即和我党我军站在一道反对蒋军进攻。凡对《五四指示》执行得不坚决,或布置太晚,或机械地分为几个阶段,或借口战争忙而忽视土地改革的地方,农民即站在观望地位。各地必须在今后几个月内,不论战争如何忙,坚决地领导农民群众解决土地问题,并在土地改革基础上布置明年的大规模的生产工作。\\
  (十五)三个月经验证明:凡民兵、游击队、武装工作队等地方武装组织得好的地方,虽然被敌人暂时占领许多的点线,我们仍能控制广大的乡村。凡地方武装薄弱和领导不好的地方,就给敌人以很大的便利。今后必须加强党的领导,在暂时被敌占领地区,发展地方武装,坚持游击战争,保护群众利益,打击反动派活动。\\
  (十六)三个月战争,使国民党的后备力量快要用完了,国民党统治区的军事力量大为减弱了。同时,国民党恢复征兵征实\footnote[6]{ 征实,指田赋征收实物(粮食)。},引起人民不满,利于群众斗争的发展。全党必须加强国民党统治区内的群众斗争的领导,加强瓦解国民党军队的工作。\\
  (十七)国民党反动派,在美国指使下,破坏今年一月的停战协定\footnote[7]{ 见本卷《以自卫战争粉碎蒋介石的进攻》注〔1〕。}和政协决议\footnote[8]{ 见本卷《以自卫战争粉碎蒋介石的进攻》注〔2〕。},决心发动内战,企图消灭人民民主力量。他们的一切花言巧语都是骗人的,我们必须揭穿美蒋的一切阴谋。\\
  (十八)三个月以来,国民党区最广大阶层的人民,包括民族资产阶级在内,对于国民党和美国政府互相勾结,发动内战,压迫人民这一种情况的认识,很快地提高了。关于马歇尔调解\footnote[9]{ 参见本卷《美国“调解”真相和中国内战前途》注〔1〕。}是骗局、国民党是内战祸首这些真理,明白的人已日益增多。广大群众在对美国和国民党失望之余,转而寄希望于我党的胜利。这是极有利的国内政治形势。美国帝国主义的反动政策,已引起各国广大人民日益不满。各国人民的觉悟程度日益提高。一切资本主义国家的人民民主斗争日益高涨,各国共产党力量有很大的发展,反动派想要压服它们是不可能的。苏联的国力及其在各国人民中的威信日益高涨。美国反动派和被美国反动派所扶助的各国反动派,必然日益陷于孤立。这些就是极有利的国际政治形势。凡此国内国际形势,都比第一次世界大战以后时期,大不相同。第二次世界大战以后的革命力量是极大地发展了。不管中外反动派如何猖獗(这种猖獗是历史必然性,毫不足奇),我们是能够战胜他们的。各地领导同志,应当向党内一部分同志,即对于国内国际有利形势认识不足、因而对于斗争前途怀抱悲观情绪的人们,作充分的解释。必须明白,敌人还有力量,我们自己也还有弱点,斗争的性质依然是长期的,残酷的。但是我们一定能够胜利。此项认识和信心,必须在全党巩固地建立起来。\\
  (十九)今后数月是一个重要而困难的时期,必须实行全党紧张的动员和精心计划的作战,从根本上转变军事形势。各地必须依照上述各项方针坚决实施,力争军事形势的根本转变。\\
\newpage\section*{\myformat{迎接中国革命的新高潮}\\\myformat{(一九四七年二月一日)}}\addcontentsline{toc}{section}{迎接中国革命的新高潮}
\begin{introduction}\item  这是毛泽东为中共中央起草的对党内的指示。\end{introduction}
(一)目前各方面情况显示,中国时局将要发展到一个新的阶段。这个新的阶段,即是全国范围的反帝反封建斗争发展到新的人民大革命的阶段。现在是它的前夜。我党的任务是为争取这一高潮的到来及其胜利而斗争。\\
  (二)目前军事形势,已向有利于人民的方向发展。去年七月至今年一月的七个月作战,已歼灭蒋介石进犯解放区的正规军五十六个旅,平均每月歼敌八个旅;被歼灭的大量伪军和保安部队,被击溃的正规军,都未计算在内。蒋介石的攻势,在鲁南、鲁西、陕甘宁边区、平汉北段和南满等地虽然还在继续,但是比较去年秋季已经衰弱得多了。蒋军兵力不敷分配,征兵不足规定数额,这同他的战线之广和兵员消耗之多,发生了严重的矛盾。蒋军士气日益下降。最近在苏北、鲁南、鲁西、晋西等地几次作战中,许多蒋军部队士气的下降已到了很大的程度。我军已在几个战场上开始夺取了主动,蒋军则开始失去了主动。预料今后数月内可能达到歼灭蒋军连前共计一百个旅的目的。蒋介石共有正规军步骑九十三个师(军),二百四十八个旅(师),一百九十一万六千人,伪军、警察、地方保安部队、交通警察部队、后勤部队和技术兵种等,都未计算在内。进攻解放区的为七十八个师(军),二百一十八个旅(师),一百七十一万三千人,约占蒋军正规军兵力百分之九十。留在蒋管区后方的仅十五个师,三十个旅,二十万三千人,约占百分之十。因此,蒋介石不可能再从他的后方调动很多有战斗力的军队向解放区进攻。进攻解放区的二百一十八个旅中,被我歼灭者已超过四分之一。虽然有些部队在被歼灭后又以原番号补充恢复,但其战斗力很弱。有些补充后又被歼灭,有些则根本没有补充。我军如能于今后数月内,再歼其四十至五十个旅,连前共达一百个旅左右,则军事形势必将发生重大的变化。\\
  (三)同时,蒋介石区域的伟大的人民运动发展起来了。去年十一月三十日因国民党压迫摊贩而引起的上海市民骚动\footnote[1]{ 从一九四六年八月起,上海国民党当局禁止黄浦、老闸两区的摊贩营业,至十一月下旬,拘捕和拘押继续营业的摊贩近千人。摊贩于十一月三十日举行三千余人的请愿游行,并包围黄浦区警察局。国民党当局下令开枪镇压,摊贩死七人,受伤被捕者甚多。十二月一日,摊贩请愿游行队伍增至五千余人,继续进行斗争,当日又被杀十人,受伤百余人。上海全市商店曾经停业表示同情。这样就形成了全市性的反蒋群众运动。}和去年十二月三十日因美军强奸中国女学生而引起的北平学生运动\footnote[2]{ 一九四六年十二月二十四日,北平发生了美军强奸北京大学一名女生的事件。十二月三十日至一九四七年一月,国民党统治区几十个大中城市的学生,为此相继罢课,举行反美反蒋的示威游行,要求美军退出中国。参加这一运动的学生人数在五十万以上。},标志着蒋管区人民斗争的新高涨。由北平开始的学生运动,已向全国各大城市发展,参加人数达数十万,超过“一二九”抗日学生运动\footnote[3]{ 指一九三五年十二月九日在北平爆发的学生爱国运动。见本书第一卷《论反对日本帝国主义的策略》注〔8〕。}的规模。\\
  (四)解放区人民解放军的胜利和蒋管区人民运动的发展,预示着中国新的反帝反封建斗争的人民大革命毫无疑义地将要到来,并可能取得胜利。\\
  (五)这一形势,是在美国帝国主义及其走狗蒋介石代替日本帝国主义及其走狗汪精卫的地位,采取了变中国为美国殖民地的政策、发动内战的政策和加强法西斯独裁统治的政策的情况之下形成的。在美蒋这些反动政策下,全国人民除了斗争,再无出路。为独立、为和平、为民主而斗争,仍然是现时期中国人民的基本要求。还在前年四月,我党第七次全国代表大会即曾预见美蒋实施这些反动政策的可能性,并为战胜这些反动政策而制定了完整的和完全正确的政治路线。\\
  (六)美蒋的上述反动政策,迫使中国各阶层人民处于团结自救的地位。这里包括工人、农民、城市小资产阶级、民族资产阶级、开明绅士、其它爱国分子、少数民族和海外华侨在内。这是一个极其广泛的全民族的统一战线。它和抗日时期的统一战线相比较,不但规模同样广大,而且有更加深刻的基础。全党同志必须为这个统一战线的巩固和发展而奋斗。解放区在坚决地毫不犹豫地实现耕者有其田的条件下,“三三制”\footnote[4]{ 见本书第二卷《论政策》注〔7〕。}政策仍然不变。在政权机关和社会事业中,除共产党人外,必须继续吸收广大的党外进步分子、中间分子(开明绅士等)参加工作。解放区内,除汉奸分子和反对人民利益而为人民所痛恨的反动分子外,一切公民不分阶级、男女、信仰,都有选举权和被选举权。在彻底实现耕者有其田的制度以后,解放区人民的私有财产权仍将受到保障。\\
  (七)由于蒋介石政府长期施行反动的财政经济政策,由于蒋介石的官僚买办资本在著名的卖国条约——中美商约\footnote[5]{ “中美商约”即《中美友好通商航海条约》,一九四六年十一月四日国民党政府和美国政府在南京签订。这个大量出卖中国主权的条约,共有三十条,其主要内容是:第一,美国人有在中国“领土全境内”居住,旅行,从事商务、制造、加工、科学、教育、宗教、慈善事业,采勘和开发矿产资源,租赁和保有土地,以及从事各种职业的权利。美国人在中国,在经济权利上得与中国人享受同样待遇。第二,美国商品在中国的征税、销售、分配或使用,享有不低于任何第三国和中国商品的待遇。中国对美国任何种植物、出产物或制造品的输入,以及由中国运往美国的任何物品,“不得加以任何禁止或限制”。第三,美国船舶可以在中国开放的任何口岸、地方或领水内自由航行,其人员和物品有经由“最便捷之途径”通过中国领土的自由。美国船舶,包括军舰在内,可以在遇到“任何危难”的借口下,开入中国“对外国商务或航业不开放之任何口岸、地方或领水”。}中同美国的帝国主义资本相结合,使恶性通货膨胀迅速发展,中国民族工商业日趋于破产,劳动群众和公教人员的生活日趋于恶化,为数众多的中等阶级分子日益丧失了他们的积蓄而变为毫无财产的人,罢工、罢课等项斗争因之不断发生。中国空前严重的经济危机,已经威胁着各阶层人民。蒋介石为了继续内战,恢复了抗日时期极端恶劣的征兵、征粮制度,这将迫使广大的乡村人民首先是贫苦农民不能生活,因而民变运动已经起来,并将继续发展。这样,蒋介石反动统治集团就将在广大人民面前日益丧失自己的威信,遭到严重的政治危机和军事危机。这个形势,一方面推动蒋管区反帝反封建的人民运动日益向前发展,另方面影响蒋军士气更加下降,增加人民解放军的胜利的可能性。\\
  (八)以孤立我党和其它民主势力为目标而召开的蒋介石的非法的分裂的国民大会\footnote[6]{ 见本卷《美国“调解”真相和中国内战前途》注〔4〕。}及其所制造的伪宪法,在人民面前没有任何威信。它们没有使我党和其它民主势力陷于孤立,反而使蒋介石反动统治集团自己孤立起来。我党和其它民主势力采取了拒绝参加伪国大的方针,这是完全正确的。蒋介石反动统治集团已将青年党、民社党\footnote[7]{ 青年党,即中国青年党,一九二三年成立。参见本书第一卷《中国社会各阶级的分析》注〔1〕。民社党,即中国民主社会党,一九四六年八月由中国国家社会党和民主宪政党合并组成。青年党和民社党的主要成员都是大地主、大资产阶级的代表人物。抗日战争时期,它们曾参加中国民主同盟,后又依附国民党。一九四六年十一月参加了国民党包办的“国民大会”,拥护这次国民大会通过的所谓“中华民国宪法”,支持蒋介石发动反共反人民的内战。}两个历来在社会上毫无威信的小党派和某些所谓“社会贤达”拉拢到自己方面,并且中间派队伍中预计今后还可能有一部分人投到反动派方面去,这是因为中国民主势力日益壮大和反动势力日益孤立,所以敌我两条阵线不得不划分得这样清楚。一切隐藏在民主阵线中欺骗人民的分子,最后都将露出自己的原形而为人民所唾弃,而人民的反帝反封建的队伍则将因为同隐藏的反动分子分清了界限,而更加壮大起来。\\
  (九)国际形势的发展,对于中国人民的斗争,极为有利。苏联力量的增长及其外交政策的胜利,世界各国人民的日益左倾及其反对本国和外国反动势力的斗争的日益发展,这两大因素,已经迫使并将继续迫使美帝国主义及其在各国的走狗日益陷于孤立。如果再加上无可避免的美国的经济危机这一因素,必将迫使美帝国主义及其在各国的走狗更加处于困难地位。美帝国主义及其走狗蒋介石的强大仅仅是暂时的,他们的进攻是可以粉碎的。所谓反动派进攻不能粉碎的神话,在我们队伍中不应有它的位置。中央曾经多次指出这一点,国际国内形势的发展日益证明这一判断的正确性。\\
  (十)为着取得休息时间补充军队,重新进攻,为着向美国取得新的借款和军火,为着缓和人民的愤怒,蒋介石又在施行新的骗术,要求和我党恢复所谓和谈\footnote[8]{ 一九四七年一月十六日,国民党政府由于它的军事进攻的不断失败和军事形势的日益恶化,为了取得喘息时间,准备再攻,经过美国驻中国大使司徒雷登向中国共产党要求允许派遣代表到延安进行“和平谈判”。美蒋这一新的骗局,立即为中国共产党彻底揭穿。中国共产党指出,恢复谈判,必须实现两个最低条件,即(一)废除国民党违背政治协商会议协议制定的伪宪法,(二)国民党军队退出一九四六年一月十三日停战协定生效以后侵占的解放区的一切土地;否则无法保证以后谈判中所获协议不再被国民党撕毁。国民党政府看到“和平”骗术无法实施,即于二月二十七日、二十八日,先后通知中国共产党驻在南京、上海、重庆等地担任谈判联络工作的代表全部撤退,宣布国共谈判完全破裂。}。我党方针是不拒绝谈判,借以揭露其欺骗。\\
  (十一)为着彻底粉碎蒋军的进攻,必须在今后几个月内再歼蒋军四十至五十个旅,这是决定一切的关键。为达此目的,必须充分地实行去年十月一日中央关于三个月总结的指示和去年九月十六日军委关于集中兵力各个歼敌的指示。这里特再着重指出几点,引起各地同志注意:\\
  甲、军事问题。我军在过去七个月艰苦奋战中,已经证明自己有一切把握粉碎蒋介石的进攻,取得最后的胜利。我军的装备和战术,均有进步。今后军事建设方面的中心任务,是用一切努力加强炮兵和工兵的建设。各大小军区,各野战兵团,必须具体地解决为了加强炮兵和工兵而发生的各项问题,主要是训练干部和制造弹药两项问题。\\
  乙、土地问题。各区都有约三分之二的地方执行了中央一九四六年五月四日的指示\footnote[9]{ 指一九四六年五月四日中共中央《关于土地问题的指示》。见本卷《三个月总结》注〔5〕。},解决了土地问题,实现了耕者有其田,这是一个伟大的胜利。但是还有约三分之一的地方,必须于今后继续努力,放手发动群众,实现耕者有其田。在已实现耕者有其田的地方,还有解决不彻底的缺点存在,主要是因为没有放手发动群众,以致没收和分配土地都不彻底,引起群众不满意。在这种地方,必须认真检查,实行填平补齐\footnote[10]{ 填平补齐,是在土地改革比较彻底的老区,为了解决某些贫雇农土地和其它生产资料不足以及土地改革中遗留下的其它问题,在较小的范围内,采用抽肥补瘦、抽多补少的办法,合理地调剂土地和其它生产资料。},务使无地和少地的农民都能获得土地,而豪绅恶霸分子则必须受到惩罚。在实现耕者有其田的全部过程中,必须坚决联合中农,绝对不许侵犯中农利益(包括富裕中农在内),如有侵犯中农利益的事,必须赔偿道歉。此外,对于一般的富农和中小地主,在土地改革中和土地改革后,应有适当的出于群众愿意的照顾之处,都照《五四指示》办理。总之,在农村土地改革运动中,务须团结赞成土地改革的百分之九十以上的群众,孤立反对土地改革的少数封建反动分子,以期迅速完成实现耕者有其田的任务。\\
  丙、生产问题。各地必须作长期打算,努力生产,厉行节约,并在生产和节约的基础上,正确地解决财政问题。这里第一个原则是发展生产,保障供给。因此,必须反对片面地着重财政和商业、忽视农业生产和工业生产的错误观点。第二个原则是军民兼顾,公私兼顾\footnote[11]{ 见本卷《一九四六年解放区工作的方针》注〔4〕。}。因此,必须反对只顾一方面、忽视另一方面的错误观点。第三个原则是统一领导,分散经营。因此,除依情况应当集中经营者外,必须反对不顾情况,一切集中,不敢放手分散经营的错误观点。\\
  (十二)我党和中国人民有一切把握取得最后胜利,这是毫无疑义的。但这并不是说我们面前已没有困难。中国反帝反封建斗争的长期性,中外反动派将继续用全力反对中国人民,蒋管区的法西斯统治将更加紧,解放区的某些部分将暂时变为沦陷区或游击区,部分的革命力量可能暂时遭受损失,在长期战争中人力物力将受到消耗,凡此种种,全党同志都必须充分地估计到,并准备用百折不回的毅力,有计划地克服所有的困难。反动势力面前和我们面前都有困难。但是反动势力的困难是不可能克服的,因为他们是接近于死亡的没有前途的势力。我们的困难是能够克服的,因为我们是新兴的有光明前途的势力。\\
\newpage\section*{\myformat{中共中央关于暂时放弃延安和保卫陕甘宁边区的两个文件}\\\myformat{(一九四六年十一月、一九四七年四月)}}\addcontentsline{toc}{section}{中共中央关于暂时放弃延安和保卫陕甘宁边区的两个文件}
\begin{introduction}\item  毛泽东起草的这两个文件,一个是在一九四六年冬季国民党军队准备进攻延安的时候在延安写的,一个是在国民党军队已经在一九四七年三月十九日占领延安以后的二十天在陕北靖边县青阳岔写的。蒋介石在全面进攻解放区的计划破产以后,为了挽救自己的垂死统治,采取了召开伪国民大会,驱逐中国共产党驻在南京、上海、重庆的代表,进攻中共中央所在地延安等项疯狂的步骤。蒋介石采取这些步骤的结果,如本文所说,在政治上完全是自取灭亡。在军事上,他企图集中兵力于解放区的东西两翼,即山东解放区和陕甘宁解放区,实行所谓重点进攻,结果也完全失败。进攻陕甘宁边区的国民党军兵力达到二十五万人,西北战场的人民解放军在陕甘宁边区的部队只有两万多人,因而国民党军曾经先后占领过人民解放军主动放弃的延安和陕甘宁边区的所有县城。但是国民党军不但没有达到消灭中共中央首脑机关和陕甘宁边区的人民解放军或者把它们赶到黄河以东的目的,而且受到人民解放军多次的沉重打击,损失兵力约达十万人,最后不得不逃出边区,而人民解放军胜利地转入解放大西北的进攻。同时,西北战场人民解放军以很少兵力吸引和歼灭了敌军的大量主力部队,也有力地支持了其它战场首先是晋冀鲁豫战场人民解放军的作战,帮助他们较快地转入进攻。毛泽东和中共中央、人民解放军总部,从一九四七年三月人民解放军撤出延安起,到一年以后人民解放军在西北战场转入进攻止,一直留在陕甘宁边区,这个事实具有重大的政治意义。它极大地鼓舞了和增强了陕甘宁边区以及全国解放区军民的战斗意志和胜利信心。毛泽东在留驻陕甘宁边区期间,不但继续指导了全国各个战场的人民解放战争,而且直接指挥了西北战场的人民解放战争,胜利地达到了本文所提出的“用坚决战斗精神保卫和发展陕甘宁边区和西北解放区”的目的。关于西北战场的作战,参看本卷《关于西北战场的作战方针》和《评西北大捷兼论解放军的新式整军运动》两文。\end{introduction}
\subsection*{\myformat{一 一九四六年十一月十八日的指示}}
蒋介石日暮途穷,欲以开“国大”\footnote[1]{ “国大”,即“国民大会”。见本卷《美国“调解”真相和中国内战前途》注〔4〕。}、打延安两项办法,打击我党,加强自己。其实,将适得其反。中国人民坚决反对蒋介石一手包办的分裂的“国民大会”,此会开幕之日,即蒋介石集团开始自取灭亡之时。蒋介石军队在被我歼灭了三十五个旅\footnote[2]{ 这是一九四六年七月初到十一月十三日的统计。}之后,在其进攻能力快要枯竭之时,即使用突袭方法,占领延安,亦无损于人民解放战争胜利的大局,挽救不了蒋介石灭亡的前途。总之,蒋介石自走绝路,开“国大”、打延安两着一做,他的一切欺骗全被揭破,这是有利于人民解放战争的发展的。各地对于蒋介石开“国大”、打延安两点,应向党内外作充分说明,团结全党全军和全体人民,为粉碎蒋介石进攻、建立民主的中国而奋斗。\\
\subsection*{\myformat{二 一九四七年四月九日的通知}}
国民党为着挽救其垂死统治,除了采取召开伪国大,制定伪宪法,驱逐我党驻南京、上海、重庆等地代表机关,宣布国共破裂\footnote[3]{ 一九四七年二月二十七日、二十八日,国民党政府强迫中国共产党驻在南京、上海、重庆等地担任谈判联络工作的全部代表和工作人员,限于三月五日前撤退。一九四七年三月十五日,国民党召开六届三中全会,蒋介石在会上宣称国共破裂,决心作战到底。}等项步骤之外,又采取进攻我党中央和人民解放军总部所在地之延安和陕甘宁边区一项步骤。\\
  国民党之所以采取这些步骤,丝毫不是表示国民党统治的强有力,而是表示国民党统治的危机业已异常深刻化。其进攻延安和陕甘宁边区,还为着妄图首先解决西北问题,割断我党右臂,并且驱逐我党中央和人民解放军总部出西北,然后调动兵力进攻华北,达到其各个击破之目的。\\
  在上述情况下,中央决定:\\
  一、必须用坚决战斗精神保卫和发展陕甘宁边区和西北解放区,而此项目的是完全能够实现的。\\
  二、我党中央和人民解放军总部必须继续留在陕甘宁边区。此区地形险要,群众条件好,回旋地区大,安全方面完全有保障。\\
  三、同时,为着工作上的便利,以刘少奇同志为书记,组织中央工作委员会,前往晋西北或其它适当地点进行中央委托之工作\footnote[4]{ 人民解放军于一九四七年三月十九日撤出延安后,中共中央书记处的三位书记,即毛泽东、周恩来和任弼时,继续留在陕甘宁边区领导全国的解放战争;另两位书记刘少奇、朱德和其它一部分中央委员组成以刘少奇为首的中共中央工作委员会,经晋绥解放区进入晋察冀解放区,到河北省平山县西柏坡村进行中央委托的工作。一九四八年五月,中共中央和毛泽东到达西柏坡村以后,中共中央工作委员会即行结束。}。\\
  以上三项,为上月所决定,业已分别实行。特此通知。\\
\newpage\section*{\myformat{关于西北战场的作战方针}\\\myformat{(一九四七年四月十五日)}}\addcontentsline{toc}{section}{关于西北战场的作战方针}
\begin{introduction}\item  这是毛泽东给西北野战兵团(一九四七年七月改称西北野战军)的电报。西北战场的人民解放军,是由彭德怀、贺龙、习仲勋等领导的陕甘宁解放区和晋绥解放区的人民解放军所组成的。\end{introduction}
(一)敌现已相当疲劳,尚未十分疲劳;敌粮已相当困难,尚未极端困难。我军自歼敌第三十一旅\footnote[1]{ 西北人民解放军主动撤出延安后,以小部兵力诱敌主力进至安塞(在延安西北),主力埋伏于延安东北的青化砭地区,待机歼敌。一九四七年三月二十五日,国民党军胡宗南部整编二十七师之三十一旅旅部率一个团共二千九百余人进入人民解放军伏击圈内,经过一个多小时的战斗,人民解放军将其全部歼灭。这是中共中央撤出延安后陕北战场的第一个胜仗。}后,虽未大量歼敌,但在二十天中已经达到使敌相当疲劳和相当缺粮之目的,给今后使敌十分疲劳、断绝粮食和最后被歼造成有利条件。\\
  (二)目前敌之方针是不顾疲劳粮缺,将我军主力赶到黄河以东,然后封锁绥德、米脂,分兵“清剿”。敌三月三十一日到清涧不即北进,目的是让一条路给我走;敌西进瓦窑堡,是赶我向绥、米。现在因发现我军,故又折向瓦市以南以西,再向瓦市赶我北上。\\
  (三)我之方针是继续过去办法,同敌在现地区再周旋一时期(一个月左右),目的在使敌达到十分疲劳和十分缺粮之程度,然后寻机歼击之。我军主力不急于北上打榆林,也不急于南下打敌后路。应向指战员和人民群众说明,我军此种办法是最后战胜敌人必经之路。如不使敌十分疲劳和完全饿饭,是不能最后获胜的。这种办法叫“蘑菇”战术,将敌磨得精疲力竭,然后消灭之。\\
  (四)你们现在位于瓦市以东和以北地区,引敌向瓦市以北最为有利;然后可向敌廖昂\footnote[2]{ 廖昂是国民党军胡宗南部整编七十六师师长,当时率部驻守于延川、清涧一带,后来在一九四七年十月十一日清涧战斗中被人民解放军活捉。}薄弱部分攻击,引敌向东;再后你们可折向安塞方面,引敌再向西。\\
  (五)但你们在数日内即应令三五九旅(全部)完成向南袭击之准备,以便在一星期以后派其向南担任袭击延长延安之线以南、宜川洛川之线以北地区,断敌粮运。\\
  (六)以上意见妥否望复。\\
\newpage\section*{\myformat{蒋介石政府已处在全民的包围中}\\\myformat{(一九四七年五月三十日)}}\addcontentsline{toc}{section}{蒋介石政府已处在全民的包围中}
\begin{introduction}\item  这是毛泽东为新华社写的一篇评论。这篇评论指出中国事变的发展,比人们预料的要快些,号召人民为中国革命在全国的胜利迅速地准备一切必要的条件。这个预言,不久以后就得到了证实。本篇和《关于西北战场的作战方针》,都是毛泽东在陕北靖边县王家湾写的。\end{introduction}
和全民为敌的蒋介石政府,现在已经发现它自己处在全民的包围中。无论是在军事战线上,或者是在政治战线上,蒋介石政府都打了败仗,都已被它所宣布为敌人的力量所包围,并且想不出逃脱的方法。\\
  蒋介石卖国集团及其主人美国帝国主义者,错误地估计了形势。他们曾经过高地估计了自己的力量,过低地估计了人民的力量。他们把第二次世界大战以后的中国和世界,看成和过去一样,不许改变任何事物的样式,不许任何人违背他们的意志。在日本投降以后,他们决定要使中国回复到过去的旧秩序。经过政治协商和军事调处等项欺骗办法赢得时间之后,蒋介石卖国政府就调动了二百万军队实行了全面的进攻。\\
  中国境内已有了两条战线。蒋介石进犯军和人民解放军的战争,这是第一条战线。现在又出现了第二条战线,这就是伟大的正义的学生运动和蒋介石反动政府之间的尖锐斗争\footnote[1]{ 从一九四六年十二月起,随着人民解放战争的发展,国民党统治区广大学生的爱国民主运动,有了新的高涨,逐步形成为反对蒋介石反动统治斗争的第二条战线。一九四六年十二月底到一九四七年一月初,北平、天津、上海、南京等几十个大中城市,五十多万学生,相继举行罢课和游行示威,抗议美国士兵强奸北京大学一名女生的暴行,要求美军撤出中国。这一斗争,迅速获得了工人、教员和其它人民群众的支持。一九四七年五月四日,上海各校学生举行游行示威,反对内战。同时,发生了上海八千工人、学生包围国民党警察局的事件。这一爱国运动,立即扩大到南京、北平、杭州、沈阳、青岛、开封等许多城市。国民党反动派对学生的爱国民主运动采取了极端野蛮的镇压办法。五月二十日,同时在南京和天津殴伤和逮捕学生共百余人,造成有名的“五二〇血案”。但是学生的爱国运动,在广大人民支持之下,并没有被镇压下去。以“反饥饿、反内战、反迫害”为口号的学生罢课示威运动,和工人罢工、教员罢教等各界人民的反美反蒋斗争,当时遍及六十多个大中城市。一九四八年五月间,上海学生又同文化界、新闻界和其它各界一起,展开了反对美国扶植日本侵略势力复活的爱国运动,这个运动也迅速地扩展到其它许多城市。直到全国胜利为止,学生的爱国斗争从未停止过,给了国民党以严重的打击。}。学生运动的口号是要饭吃,要和平,要自由,亦即反饥饿,反内战,反迫害。蒋介石颁布了《维持社会秩序临时办法》\footnote[2]{ 国民党政府于一九四七年五月十八日颁布所谓《维持社会秩序临时办法》,严禁人民十人以上的请愿和一切罢工、罢课、游行示威,并授权各地方政府,对于人民的爱国民主运动,采取“紧急措施”,进行镇压。}。蒋介石的军警宪特同学生群众之间,到处发生冲突。蒋介石用逮捕、监禁、殴打、屠杀等项暴力行为对付赤手空拳的学生,学生运动因而日益扩大。一切社会同情都在学生方面,蒋介石及其走狗完全陷于孤立,蒋介石的狰狞面貌暴露无遗。学生运动是整个人民运动的一部分。学生运动的高涨,不可避免地要促进整个人民运动的高涨。过去五四运动\footnote[3]{ 见本书第一卷《实践论》注〔6〕。}时期和一二九运动\footnote[4]{ 见本书第一卷《论反对日本帝国主义的策略》注〔8〕。}时期的历史经验,已经表明了这一点。\\
  由于美国帝国主义及其走狗蒋介石代替了日本帝国主义及其走狗汪精卫的地位,采取了变中国为美国殖民地的政策、发动内战的政策和加强法西斯独裁统治的政策,他们就宣布他们自己和全国人民为敌,他们就将全国各阶层人民放在饥饿和死亡的界线上,因而就迫使全国各阶层人民团结起来,同蒋介石反动政府作你死我活的斗争,并使这个斗争迅速发展下去。全国人民除此以外,再无出路。被蒋介石政府各项反动政策所压迫、处于团结自救地位的中国各阶层人民,包括了工人、农民、城市小资产阶级、民族资产阶级、开明绅士、其它爱国分子、少数民族和海外华侨在内。这是一个极其广泛的全民族的统一战线。\\
  蒋介石政府所长期施行的极端反动的财政经济政策,现在被空前的卖国条约即中美商约\footnote[5]{ 见本卷《迎接中国革命的新高潮》注〔5〕。}所加强了。在中美商约的基础上,美国的独占资本和蒋介石的官僚买办资本紧紧地结合在一起,控制着全国的经济生活。其结果,就是极端的通货膨胀,空前的物价高涨,民族工商业日益破产,劳动群众和公教人员的生活日益恶化。这种情形,迫使各阶层人民不得不团结起来为救死而斗争。\\
  军事镇压和政治欺骗,是蒋介石维持自己反动统治的两个主要工具,现在人们已经看到这些工具的迅速破产。\\
  蒋介石的军队,无论在哪个战场,都打了败仗。从去年七月到现在共计十一个月中,仅就其正规军来说,即已被歼灭约九十个旅。不但去年占长春、占承德、占张家口、占菏泽、占淮阴、占安东\footnote[6]{ 安东,今辽宁省丹东市。}时候的那种神气,现在没有了,就是今年占临沂、占延安时候的那种神气,现在也没有了。蒋介石、陈诚曾经错误地估计了人民解放军的力量和人民解放军的作战方法,以为退却就是胆怯,放弃若干城市就是失败,妄想在三个月或六个月内解决关内问题,然后再解决东北问题。但在十个月之后,蒋介石全部进犯军已经深入绝境,被解放区人民和人民解放军所重重包围,想要逃脱,已很困难。\\
  蒋介石军队在前线打败仗的消息传到后方的日益增多,被蒋介石反动政府压迫得喘不过气来的广大人民群众,就日益感觉自己的出头翻身有了希望。恰在这时,蒋介石的一切政治欺骗,由于蒋介石的迅速扮演而迅速破产。一切出于反动派意料之外。什么召开国民大会制定宪法呀,什么改组一党政府为多党政府呀,其目的原是为着孤立中共和其它民主力量;结果却是相反,被孤立的不是中共,也不是任何民主力量,而是反动派自己。从此以后,中国人民从自己的经验中,知道什么是蒋介石的国民大会,什么是蒋介石的宪法,什么是蒋介石的多党政府。在这以前,中国人民中的许多人,主要地是中间阶层的分子,对于蒋介石的这些手法是多少存了幻想的。对于蒋介石的所谓和谈也是这样。在几次庄严的停战协定被蒋介石撕毁得干干净净之后,在用刺刀向着要和平反内战的学生群众之后,除了存心欺骗的人们或者政治上毫无经验的人们之外,什么人也不会相信蒋介石的所谓和谈了。\\
  一切事变都证明我们估计的正确。我们曾经不断地向人们指出,蒋介石政府不是别的,仅仅是一个卖国内战独裁的政府。这个政府欲以内战的手段,削平中共和一切民主力量,达到变中国为美国殖民地和维持自己独裁统治的目的。这个政府因为采取了这些反动政策,它就在政治上变得毫无威信,毫无力量。蒋介石政府的强大只是暂时的,表面的,它实际上是一个外强中干的政府。它的进攻是能够打败的,不论是在什么地方和在什么战线上。它的前途必然是众叛亲离,全军覆灭。一切事变,都已经证明并且将继续证明这些估计的正确性。\\
  中国事变的发展,比人们预料的要快些。一方面是人民解放军的胜利,一方面是蒋管区人民斗争的前进,其速度都是很快的。为了建立一个和平的、民主的、独立的新中国,中国人民应当迅速地准备一切必要的条件。\\
\newpage\section*{\myformat{解放战争第二年的战略方针}\\\myformat{(一九四七年九月一日)}}\addcontentsline{toc}{section}{解放战争第二年的战略方针}
\begin{introduction}\item  这是毛泽东为中共中央起草的对党内的指示,当时他和中共中央住在陕北佳县朱官寨。这个指示规定,解放战争第二年的基本任务,是以主力打到国民党区域,由内线作战转入外线作战,也就是由战略防御阶段转入战略进攻阶段。人民解放军按照毛泽东所规定的战略计划,从一九四七年七月至九月,转入了全国规模的进攻。晋冀鲁豫野战军于六月三十日在鲁西南地区强渡黄河,八月上旬越过陇海线,挺进大别山。晋冀鲁豫野战军的太岳兵团,八月下旬由晋南强渡黄河,挺进豫西地区。华东野战军在打破敌人的重点进攻以后,其主力部队于八月初挺进鲁西南,九月下旬进入豫皖苏地区。华东野战军的内线兵团(一九四八年三月改称山东兵团),从十月初起向胶东地区之敌发起攻势作战。西北野战军八月下旬转入反攻。晋察冀野战军九月初对平汉线北段之敌发起攻势作战。东北野战军紧接着全东北范围的夏季攻势之后,从九月起,在长春、吉林、四平地区和北宁线锦西至义县地区发起大规模的秋季攻势。所有这些战场上的攻势,组成了人民解放军全面进攻的总形势。人民解放军的大举进攻,使解放战争达到了一个转折点,标志着战争形势的根本改变,参看本卷《目前形势和我们的任务》一文。\end{introduction}
(一)第一年作战(去年七月至今年六月),歼灭敌正规军九十七个半旅,七十八万人,伪军、保安队等杂部三十四万人,共计一百十二万人。这是一个伟大的胜利。这一胜利,给了敌人以严重打击,在整个敌人营垒中引起了极端深刻的失败情绪,兴奋了全国人民,奠定了我军歼灭全部敌军、争取最后胜利的基础。\\
  (二)第一年作战,敌人以二百四十八个正规旅中的二百十八个旅一百六十多万人,近百万的特种兵(海军、空军、炮兵、工兵、装甲兵),以及伪军、交通警察部队、保安部队等,向我解放区大举进攻。我军正确地采取战略上的内线作战方针,不惜付出三十余万人的伤亡,大块土地的被敌占领,使自己随时随地立于主动地位,因而争取了歼敌一百十二万人,分散了敌军,锻炼和壮大了我军,并且在东北、热河、冀东、晋南、豫北举行了战略性的反攻,收复和新解放了广大的土地\footnote[1]{ 本文所说我军在东北、热河(现分属河北、辽宁两省和内蒙古自治区)、冀东的战略反攻,是指东北民主联军一九四七年的夏季攻势。自五月十三日开始,东北民主联军在东北和热河、冀东各个战场上同时展开进攻,到七月一日止,歼灭国民党军八万二千余人,收复和一度攻克城镇四十二座,打通了南满、北满的联系,迫使敌军收缩于长春至大石桥、沈阳至山海关铁路沿线,转为防御,从而改变了东北整个局势。本文所说我军在晋南、豫北的战略性反攻,是指晋冀鲁豫野战军一九四七年三、四、五月间在豫北和晋南同蒲路两侧所展开的攻势。豫北攻势于三月二十三日发起,在接连攻克延津、阳武(今并入原阳)、濮阳、封丘后向北扩张战果,四月三日挥师北上,至五月二十八日止又攻克淇县、浚县、滑县、汤阴等城,歼敌四万余人。晋南攻势于四月四日发起,至五月十一日,连克曲沃、新绛、永济等二十二座县城和黄河渡口的禹门口、风陵渡要点,歼敌一万四千余人。}。\\
  (三)我军第二年作战的基本任务是:举行全国性的反攻,即以主力打到外线去,将战争引向国民党区域,在外线大量歼敌,彻底破坏国民党将战争继续引向解放区、进一步破坏和消耗解放区的人力物力、使我不能持久的反革命战略方针。我军第二年作战的部分任务是:以一部分主力和广大地方部队继续在内线作战,歼灭内线敌人,收复失地。\\
  (四)我军执行外线作战、将战争引向国民党区域的方针,当然要遇到许多困难。因为到国民党区域创立新根据地需要时间,需要在多次往返机动的作战中大量歼灭敌人、发动群众、分配土地、建立政权、建立人民武装之后,方能创立巩固的根据地。在这以前,困难将是不少的。但是,这种困难能够克服和必须克服。因为敌人将被迫更加分散,有广大地区作为我军机动作战的战场,可以求得运动战;那里的广大民众是痛恨国民党拥护我军的;虽然部分敌军仍然有较强的战斗力,但一般地敌军士气比一年前低落得多,其战斗力比一年前削弱得多了。\\
  (五)到国民党区域作战争取胜利的关键:第一是在善于捕捉战机,勇敢坚决,多打胜仗;第二是在坚决执行争取群众的政策,使广大群众获得利益,站在我军方面。只要这两点做到了,我们就胜利了。\\
  (六)敌军分布,到今年八月底止,连被歼灭和受歼灭性打击者都算在内,南线一百五十七个旅,北线七十个旅,国民党后方二十一个旅,全国总数仍是二百四十八个旅,实际人数约一百五十万人;特种部队、伪军、交通警察、保安部队等约一百二十万人;敌后方军事机关非战斗人员约一百万人。敌全军共约三百七十万人。南线各军中,顾祝同系统一百十七个旅,程潜系统和其它七个旅,胡宗南系统三十三个旅。顾军一百十七个旅中,被我歼灭和受歼灭性打击者有六十三个旅。其中一部尚未补充;一部虽已补充,但人数很少,战斗力很弱;另一部虽有较多人员武器补充,战斗力也恢复到某种程度,但仍然远不如前。尚未被歼和尚未受歼灭性打击者只有五十四个旅。全部顾军,使用于守备和仅能作地方性机动之用者占了八十二到八十五个旅,能用于战略性机动者只有三十二到三十五个旅。程潜系统和其它的七个旅大体均只能任守备,其中一个旅曾受歼灭性打击。胡宗南系统(包括兰州以东,宁夏榆林以南,临汾洛阳以西)之三十三个旅中,被歼灭和受歼灭性打击者有十二个旅,能用于战略性机动者只有七个旅,其余均任守备。北线敌军共有七十个旅。其中,东北系统二十六个旅,内有十六个旅被歼灭和受歼灭性打击;孙连仲系统十九个旅,内有八个旅被歼灭和受歼灭性打击;傅作义十个旅,内有二个旅受歼灭性打击;阎锡山十五个旅,内有九个旅被歼灭和受歼灭性打击。这些敌军现在大体均取守势,能机动作战的兵力只有一小部分。国民党后方任守备的兵力仅有二十一个旅。其中,新疆和甘西八个旅,川、康七个旅,云南两个旅,广东两个旅(即被歼灭的第六十九师),台湾两个旅,湖南、广西、贵州、福建、浙江、江西六省全无正规军。国民党在美国援助下,今年计划征兵一百万补充前线并训练若干新旅和若干补充团。但是,只要我军能如第一年作战平均每月歼敌八个旅,在第二年再歼敌九十六至一百个旅(七、八两月已歼敌十六个半旅),则敌军将进一步大受削弱,其战略性机动兵力将减少至极度,势将被迫在全国一切地方处于防御地位,到处受我攻击。国民党虽有征兵百万训练新旅和补充团之计划,也将无济于事。其征兵纯用捕捉和购买方法,必难达到百万,而且逃亡甚多。我军执行外线作战方针,又可缩小其人力资源和物质资源。\\
  (七)我军作战方针,仍如过去所确立者,先打分散孤立之敌(包括一次打几个旅的大规模歼灭性战役在内,例如今年二月莱芜战役\footnote[2]{ 莱芜战役,是华东野战军在山东莱芜(位于济南东南)地区所进行的运动战。一九四七年一月底,国民党军队分南北两线进攻山东解放区。南线国民党军以八个整编师,分三路沿沂河、沭河北犯临沂,北线国民党军李仙洲集团三个军由明水(今章丘)、淄川、博山等地南下莱芜、新泰策应,同时从冀南、豫北抽调一个军及三个整编师集结在鲁西南地区,阻止华东野战军西撤和晋冀鲁豫野战军东援,企图同华东野战军主力在临沂地区决战。华东野战军以一部阻击南线之敌,佯作决战模样,主力则隐蔽兼程北上莱芜歼击李仙洲集团。战斗自二月二十日开始至二十三日下午结束,歼灭该敌五万六千余人,生俘国民党第二绥靖区副司令官李仙洲。},七月鲁西南战役\footnote[3]{ 鲁西南战役,是晋冀鲁豫野战军在山东省西南部菏泽、郓城、巨野、定陶、金乡、曹县地区所进行的一次战役。这次战役于一九四七年六月三十日夜发起,至七月二十八日结束,先后歼灭国民党军四个整编师师部和九个半旅,共六万余人。}),后打集中强大之敌。先取中、小城市和广大乡村,后取大城市。以歼灭敌人有生力量为主要目标,不以保守和夺取地方为主要目标;保守或夺取地方是歼敌有生力量的结果,往往须反复多次才能最后地保守或夺取之。每战集中绝对优势兵力,四面包围敌人,力求全歼,不使漏网。在特殊情况下,则采用给敌以歼灭性打击之方法,即集中全力打敌正面及其一翼或两翼,以求达到歼灭其一部、击溃其另一部之目的,以便我军能够迅速转移兵力,歼击他部敌军。一方面,必须注意不打无准备之仗,不打无把握之仗,每战都应力求有准备,力求在敌我条件对比上有胜利之把握;另方面,必须发扬勇敢战斗、不惜牺牲、不怕疲劳和连续作战(即短期内接连打几仗)的优良作风。必须力求调动敌人打运动战,但同时必须极大地注重学习阵地攻击战术,加强炮兵、工兵建设,以便广泛地夺取敌人据点和城市。一切守备薄弱之据点和城市则坚决攻取之,一切有中等程度的守备而又环境许可之据点和城市则相机攻取之,一切守备强固之据点和城市则暂时弃置之。以俘获敌人的全部武器和大部兵员(十分之八九的士兵和少数下级官佐)补充自己。主要向敌军和国民党区域求补充,只有一部分向老解放区求补充,特别是南线各军应当如此。在一切新老解放区必须坚决实行土地改革(这是支持长期战争取得全国胜利的最基本条件),发展生产,厉行节约,加强军事工业的建设,一切为了前线的胜利。只有这样做,才能支持长期战争,取得全国胜利。果然这样做了,就一定可以支持长期战争,取得全国胜利。\\
  (八)以上是一年战争的总结和今后战争的方针。望各地领导同志传达给军队团级以上、地方地委和专署以上的各级干部,使大家明白自己的任务而坚决地毫不动摇地执行之。\\
\newpage\section*{\myformat{中国人民解放军宣言}\\\myformat{(一九四七年十月十日)}}\addcontentsline{toc}{section}{中国人民解放军宣言}
\begin{introduction}\item  这是毛泽东为中国人民解放军总部所起草的政治宣言。在这个宣言里,分析了当时的国内政治形势,提出了“打倒蒋介石,解放全中国”的口号,宣布了中国人民解放军的也就是中国共产党的八项基本政策。这个宣言在一九四七年十月十日公布,被称为《双十宣言》。宣言是在陕北佳县神泉堡起草的。\end{introduction}
中国人民解放军,在粉碎蒋介石的进攻之后,现已大举反攻。南线我军已向长江流域进击,北线我军已向中长、北宁两路进击。我军所到之处,敌人望风披靡,人民欢声雷动。整个敌我形势,和一年前比较,已经起了基本上的变化。\\
  本军作战目的,迭经宣告中外,是为了中国人民和中华民族的解放。而在今天,则是实现全国人民的迫切要求,打倒内战祸首蒋介石,组织民主联合政府,借以达到解放人民和民族的总目标。\\
  中国人民,为了自己的解放和民族的独立,同日本帝国主义英勇奋战了八年之久。日本投降后,人民渴望和平,蒋介石则破坏人民一切争取和平的努力,而以空前的内战灾难压在人民的头上。这样,就逼得全国各阶层人民,除了团结起来打倒蒋介石以外,再无出路。\\
  蒋介石现在的内战政策,不是偶然的,这是蒋介石及其反动集团一贯反人民政策的必然结果。早在民国十六年(一九二七年),蒋介石就忘恩负义地背叛了国共两党的革命联盟,背叛了孙中山的革命的三民主义和三大政策,从此建立独裁统治,投降帝国主义,打了十年内战,造成日寇侵略。民国二十五年(一九三六年)西安事变\footnote[1]{ 参见本书第一卷《关于蒋介石声明的声明》注〔1〕。}时期,中国共产党以德报怨,协同张学良、杨虎城两将军,释放蒋介石,希望蒋介石悔过自新,共同抗日。但是蒋介石又一次忘恩负义,对于日寇则消极应战,对于人民则积极镇压,对于共产党则极端仇视。前年(一九四五年)日本投降,中国人民又一次宽恕蒋介石,要求蒋介石停止已经发动的内战,实行民主政治,团结各党派和平建国。但是毫无信义的蒋介石,在签订停战协定\footnote[2]{ 见本卷《以自卫战争粉碎蒋介石的进攻》注〔1〕。}、通过政协决议\footnote[3]{ 见本卷《以自卫战争粉碎蒋介石的进攻》注〔2〕。}、宣布四项诺言\footnote[4]{ 四项诺言,指蒋介石一九四六年一月在政治协商会议开幕词中宣布的保障人民自由、保障各党派合法地位、实行普选和释放政治犯。}以后,随即将其全部推翻。人民方面,虽则再三忍让求全,但是蒋介石在美帝国主义援助之下,决心不顾国家民族的死活,向人民作空前的全面的进攻。从去年(一九四六年)一月停战协定宣布到现在,蒋介石先后动员了二百二十多个正规旅和近百万的杂色部队,向中国人民从日本帝国主义手里用血战夺取过来的解放区,实行大举进攻,先后侵占了沈阳、抚顺、本溪、四平、长春、永吉、承德、集宁、张家口、淮阴、菏泽、临沂、延安、烟台等城市和广大的乡村。蒋军所到之处,杀人放火,奸淫掳掠,实行三光政策,同日本强盗的行为完全一样。去年十一月,蒋介石召集了伪国大\footnote[5]{ 见本卷《美国“调解”真相和中国内战前途》注〔4〕。},宣布了伪宪法。今年三月,蒋介石驱逐了共产党的代表。今年七月,蒋介石下了反人民的总动员令\footnote[6]{ 一九四七年七月四日,国民党反动政府通过了蒋介石的“国家总动员提案”,随即下了所谓《戡平共匪叛乱总动员令》。其实,蒋介石的反革命内战的总动员早已实行了。这时,中国人民解放军已经在全国范围内开始转入进攻。蒋介石自己也供认他的统治已到了“严重危机”。这个“总动员令”,只是表示蒋介石的垂死挣扎。}。对于全国各地反对内战、反对饥饿、反对美帝国主义侵略的正义的人民运动,对于工人、农民、学生、市民和公教人员的争生存的斗争,蒋介石的方针就是镇压、逮捕和屠杀。对于国内各少数民族,蒋介石的方针就是实施大汉族主义,摧残镇压,无所不至。在一切蒋介石统治区域,贪污遍地,特务横行,捐税繁重,物价高涨,经济破产,百业萧条,征兵征粮,怨声载道,这样就使全国绝大多数人民,处于水深火热之中。而以蒋介石为首的金融寡头,贪官污吏,土豪劣绅,则集中了巨大的财富。这些财富,都是蒋介石等利用其独裁权力横征暴敛、假公济私而来的。蒋介石为着维持独裁,进行内战,不惜出卖国家权利于外国帝国主义,勾结美国军队留驻青岛等地,从美国招致顾问人员,参加内战的指挥和军队的训练,残杀自己的同胞。内战的飞机、坦克、枪炮、弹药,大批从美国运来。内战的经费,大批从美国借来。蒋介石则以出卖军事基地、出卖空海航权、签订奴役性商约\footnote[7]{ 奴役性商约,指一九四六年十一月四日国民党政府和美国政府在南京签订的出卖中国主权的所谓《中美友好通商航海条约》。见本卷《迎接中国革命的新高潮》注〔5〕。}等项比袁世凯卖国行为还要严重多倍的条件,作为酬谢美国帝国主义的礼物。总而言之,蒋介石二十年的统治,就是卖国独裁反人民的统治。到了今天,全国绝大多数人民,地无分南北,年无分老幼,都认识了蒋介石的滔天罪恶,盼望本军从速反攻,打倒蒋介石,解放全中国。\\
  本军是中国人民的军队,一切以中国人民的意志为意志。本军的政策,代表中国人民的迫切要求,主要的有如下各项:\\
  一、联合工农兵学商各被压迫阶级、各人民团体、各民主党派、各少数民族、各地华侨和其它爱国分子,组成民族统一战线,打倒蒋介石独裁政府,成立民主联合政府。\\
  二、逮捕、审判和惩办以蒋介石为首的内战罪犯。\\
  三、废除蒋介石统治的独裁制度,实行人民民主制度,保障人民言论、出版、集会、结社等项自由。\\
  四、废除蒋介石统治的腐败制度,肃清贪官污吏,建立廉洁政治。\\
  五、没收蒋介石、宋子文、孔祥熙、陈立夫兄弟等四大家族和其它首要战犯的财产,没收官僚资本,发展民族工商业,改善职工生活,救济灾民贫民。\\
  六、废除封建剥削制度,实行耕者有其田的制度。\\
  七、承认中国境内各少数民族有平等自治的权利。\\
  八、否认蒋介石独裁政府的一切卖国外交,废除一切卖国条约,否认内战期间蒋介石所借的一切外债。要求美国政府撤退其威胁中国独立的驻华军队,反对任何外国帮助蒋介石打内战和使日本侵略势力复兴。同外国订立平等互惠通商友好条约。联合世界上一切以平等待我之民族共同奋斗。\\
  上述各项,就是本军的基本政策。本军所到之处,立即实施这些政策。这些政策是适合全国百分之九十以上人民的要求的。\\
  本军对于蒋方人员,并不一概排斥,而是采取分别对待的方针。这就是首恶者必办,胁从者不问,立功者受奖。对于罪大恶极的内战祸首蒋介石和一切坚决助蒋为恶、残害人民、而为广大人民所公认的战争罪犯,本军必将追寻他们至天涯海角,务使归案法办。本军警告一切蒋军官兵,蒋政府官员,蒋党党员,凡是尚未沾染无辜人民鲜血的人们,切勿跟那些罪犯们同流合污。凡是已经做过坏事的人们,赶快停止作恶,悔过自新,脱离蒋介石,准其将功赎罪。本军对于放下武器的蒋军官兵,一律不杀不辱,愿留者收容,愿去者遣送。对于起义加入本军的蒋军部队和公开或秘密为本军工作的人们,则给予奖励。\\
  为了早日打倒蒋介石,建立民主联合政府,我们号召全国各界同胞,在本军到达之处,同我们积极合作,肃清反动势力,建立民主秩序。在本军未到之处,则自动拿起武器,实行抗丁抗粮,分田废债,利用敌人空隙,发展游击战争。\\
  为了早日打倒蒋介石,建立民主联合政府,我们号召解放区人民贯彻土地改革,巩固民主基础,发展生产,厉行节约,加强人民武装,肃清敌人残留据点,支援前线作战。\\
  本军全体指挥员、战斗员同志们!我们现在担负了我国革命历史上最重要最光荣的任务,我们应当积极努力,完成自己的任务。我伟大祖国哪一天能由黑暗转入光明,我亲爱同胞哪一天能过人的生活,能按自己的愿望选择自己的政府,依靠我们的努力来决定。我全军将士必须提高军事艺术,在必胜的战争中勇猛前进,坚决彻底干净全部地歼灭一切敌人。必须提高觉悟性,人人学会歼灭敌人、唤起民众两套本领,亲密团结群众,把新区迅速建设成为巩固区。必须提高纪律性,坚决执行命令,执行政策,执行三大纪律八项注意\footnote[8]{ 见本卷《中国人民解放军总部关于重行颁布三大纪律八项注意的训令》注〔1〕。},军民一致,军政一致,官兵一致,全军一致,不允许任何破坏纪律的现象存在。我全军将士必须时刻牢记,我们是伟大的人民解放军,是伟大的中国共产党领导的队伍。只要我们时刻遵守党的指示,我们就一定胜利。\\
  打倒蒋介石!\\
  新中国万岁!\\
\newpage\section*{\myformat{中国人民解放军总部关于重行颁布三大纪律八项注意的训令}\\\myformat{(一九四七年十月十日)}}\addcontentsline{toc}{section}{中国人民解放军总部关于重行颁布三大纪律八项注意的训令}
一、本军三大纪律八项注意,实行多年\footnote[1]{ 三大纪律、八项注意,是毛泽东等在第二次国内革命战争中为中国工农红军制订的纪律,其具体内容在不同时候和不同部队略有出入。在红军初创时期,就已提出部队对待群众要说话和气,买卖公平,不拉夫,不打人,不骂人。一九二八年春工农红军在井冈山的时候,规定了三项纪律:第一、行动听指挥;第二、不拿工人农民一点东西;第三、打土豪要归公。一九二八年夏提出了六项注意:一、上门板,二、捆铺草,三、说话和气,四、买卖公平,五、借东西要还,六、损坏东西要赔。一九二九年以后,毛泽东又将三大纪律中的“不拿工人农民一点东西”,改为“不拿群众一针一线”;“打土豪要归公”改为“筹款要归公”,后来又改为“一切缴获要归公”。对于六项注意,增加了“洗澡避女人”和“不搜俘虏腰包”两项内容,从而成为三大纪律、八项注意。这些纪律,曾经是红军以及后来的八路军、新四军、人民解放军政治工作的重要内容,对于人民军队的建设,对于正确处理军队内部关系、团结人民群众和确立人民军队对待俘虏的正确政策,都起了伟大的作用。},其内容各地各军略有出入。现在统一规定,重行颁布。望即以此为准,深入教育,严格执行。至于其它应当注意事项,各地各军最高首长,可根据具体情况,规定若干项目,以命令施行之。\\
  二、三大纪律如下:\\
  (一)一切行动听指挥;(二)不拿群众一针一线;(三)一切缴获要归公。\\
  三、八项注意如下:\\
  (一)说话和气;(二)买卖公平;(三)借东西要还;(四)损坏东西要赔;(五)不打人骂人;(六)不损坏庄稼;(七)不调戏妇女;(八)不虐待俘虏。\\
\newpage\section*{\myformat{目前形势和我们的任务}\\\myformat{(一九四七年十二月二十五日)}}\addcontentsline{toc}{section}{目前形势和我们的任务}
\begin{introduction}\item  这是毛泽东在中共中央一九四七年十二月二十五日至二十八日在陕北米脂县杨家沟召集的会议上的报告。这次会议除有当时能够到会的中央委员和候补中央委员以外,还有陕甘宁边区和晋绥边区负责同志参加。会议讨论了和通过了毛泽东的这个报告和他所写的《关于目前国际形势的几点估计》(见本卷第一一八四页)。关于毛泽东的报告,会议的决定指出:“这个报告是整个打倒蒋介石反动统治集团,建立新民主主义中国的时期内,在政治、军事、经济各方面带纲领性的文件。全党全军应将这个文件联系一九四七年双十节各项文件(按指一九四七年十月十日公布的《中国人民解放军宣言》、《中国人民解放军口号》、《中国人民解放军总部关于重行颁布三大纪律八项注意的训令》、《中国土地法大纲》和《中共中央关于公布中国土地法大纲的决议》),进行深入教育,并在实践中严格地遵照实施。各地实施政策中如果有和报告所指出的原则不相符合的地方,应即加以改正。”这次会议的其它重要决定有:(1)中国人民革命战争应该力争不间断地发展到完全胜利,应该不让敌人用缓兵之计(和谈)获得休整时间,然后再来打人民。(2)组织革命的中央政府的时机目前尚未成熟,须待我军取得更大胜利,然后考虑此项问题,颁布宪法更是将来的问题。会议还详细讨论了当时党内的倾向问题以及土地改革和群众运动中的几个具体政策问题。讨论的结果后来由毛泽东写在《关于目前党的政策中的几个重要问题》(见本卷第一二六七页)一文中。从本篇起,直到一九四八年三月二十日《关于情况的通报》止,都是在陕北米脂县杨家沟写的。\end{introduction}
\subsubsection*{\myformat{一}}
中国人民的革命战争,现在已经达到了一个转折点。这即是中国人民解放军已经打退了美国走狗蒋介石的数百万反动军队的进攻,并使自己转入了进攻。还在一九四六年七月至一九四七年六月此次战争的第一个年头内,人民解放军即已在几个战场上打退了蒋介石的进攻,迫使蒋介石转入防御地位。而从战争第二年的第一季,即一九四七年七月至九月间,人民解放军即已转入了全国规模的进攻,破坏了蒋介石将战争继续引向解放区、企图彻底破坏解放区的反革命计划。现在,战争主要地已经不是在解放区内进行,而是在国民党统治区内进行了,人民解放军的主力已经打到国民党统治区域里去了\footnote[1]{ 关于人民解放军在各个战场陆续转入进攻,打到国民党统治区的情况,见本卷《评西北大捷兼论解放军的新式整军运动》注〔2〕。}。中国人民解放军已经在中国这一块土地上扭转了美国帝国主义及其走狗蒋介石匪帮的反革命车轮,使之走向覆灭的道路,推进了自己的革命车轮,使之走向胜利的道路。这是一个历史的转折点。这是蒋介石的二十年反革命统治由发展到消灭的转折点。这是一百多年以来帝国主义在中国的统治由发展到消灭的转折点。这是一个伟大的事变。这个事变所以带着伟大性,是因为这个事变发生在一个拥有四亿七千五百万人口的国家内,这个事变一经发生,它就将必然地走向全国的胜利。这个事变所以带着伟大性,还因为这个事变发生在世界的东方,在这里,共有十万万以上人口(占人类的一半)遭受帝国主义的压迫。中国人民的解放战争由防御转到进攻,不能不引起这些被压迫民族的欢欣鼓舞。同时,对于正在斗争的欧洲和美洲各国的被压迫人民,也是一种援助。\\
\subsubsection*{\myformat{二}}
从蒋介石发动反革命战争的一天起,我们就说,我们不但必须打败蒋介石,而且能够打败他。我们必须打败蒋介石,是因为蒋介石发动的战争,是一个在美帝国主义指挥之下的反对中国民族独立和中国人民解放的反革命的战争。中国人民的任务,是要在第二次世界大战结束、日本帝国主义被打倒以后,在政治上、经济上、文化上完成新民主主义的改革,实现国家的统一和独立,由农业国变成工业国。然而恰在这时,在反法西斯的第二次世界大战胜利地结束以后,美国帝国主义及其在各国的走狗代替德国和日本帝国主义及其走狗的地位,组成反动阵营,反对苏联,反对欧洲各人民民主国家,反对各资本主义国家的工人运动,反对各殖民地半殖民地的民族运动,反对中国人民的解放。在这种时候,以蒋介石为首的中国反动派,和日本帝国主义的走狗汪精卫一模一样,充当美国帝国主义的走狗,将中国出卖给美国,发动战争,反对中国人民,阻止中国人民解放事业的前进。在这种时候,如果我们表示软弱,表示退让,不敢坚决地起来用革命战争反对反革命战争,中国就将变成黑暗世界,我们民族的前途就将被断送。中国共产党领导中国人民解放军坚决地进行了爱国的正义的革命的战争,反对蒋介石的进攻。中国共产党依据马克思列宁主义的科学,清醒地估计了国际和国内的形势,知道一切内外反动派的进攻,不但是必须打败的,而且是能够打败的。当着天空中出现乌云的时候,我们就指出:这不过是暂时的现象,黑暗即将过去,曙光即在前头。当着一九四六年七月,蒋介石匪帮发动全国规模的反革命战争的时候,蒋介石匪帮认为,只须三个月至六个月,就可以打败人民解放军。他们认为他们有正规军二百万,非正规军一百余万,后方军事机关和部队一百余万,共有军事力量四百余万人;他们已经利用时间完成了进攻的准备;他们重新控制了大城市;他们拥有三万万以上的人口;他们接收了日本侵华军队一百万人的全部装备;他们取得了美国政府在军事上和财政上的巨大援助。他们又认为,中国人民解放军在八年抗日战争中已经打得很疲倦,而且在数量上和装备上远不及国民党军队;中国解放区还只有一万万多一点的人口,其中大部分地区的反动封建势力还没有被肃清,土地改革还不普遍和不彻底,就是说,人民解放军的后方还不是巩固的。从这种估计出发,蒋介石匪帮就不顾中国人民的和平愿望,最后地撕毁在一九四六年一月间签订的国共两党的停战协定\footnote[2]{ 见本卷《以自卫战争粉碎蒋介石的进攻》注〔1〕。}和各党派政治协商会议的决议\footnote[3]{ 见本卷《以自卫战争粉碎蒋介石的进攻》注〔2〕。},发动了冒险的战争。那时,我们说,蒋介石军事力量的优势,只是暂时的现象,只是临时起作用的因素;美国帝国主义的援助,也只是临时起作用的因素;蒋介石战争的反人民的性质,人心的向背,则是经常起作用的因素;而在这方面,人民解放军则占着优势。人民解放军的战争所具有的爱国的正义的革命的性质,必然要获得全国人民的拥护。这就是战胜蒋介石的政治基础。十八个月战争的经验,充分地证明了我们的论断。\\
\subsubsection*{\myformat{三}}
十七个月(一九四六年七月至一九四七年十一月为止,十二月的尚未计入)作战,共打死、打伤、俘虏了蒋介石正规军和非正规军一百六十九万人,其中打死和打伤的有六十四万人,俘虏的有一百零五万人。这样,就使我军打退了蒋介石的进攻,保存了解放区的基本区域,并使自己转入了进攻。我们所以能够如此,在军事方面来说,是因为执行了正确的战略方针。我们的军事原则是:(1)先打分散和孤立之敌,后打集中和强大之敌。(2)先取小城市、中等城市和广大乡村,后取大城市。(3)以歼灭敌人有生力量为主要目标,不以保守或夺取城市和地方为主要目标。保守或夺取城市和地方,是歼灭敌人有生力量的结果,往往需要反复多次才能最后地保守或夺取之。(4)每战集中绝对优势兵力(两倍、三倍、四倍、有时甚至是五倍或六倍于敌之兵力),四面包围敌人,力求全歼,不使漏网。在特殊情况下,则采用给敌以歼灭性打击的方法,即集中全力打敌正面及其一翼或两翼,求达歼灭其一部、击溃其另一部的目的,以便我军能够迅速转移兵力歼击他部敌军。力求避免打那种得不偿失的、或得失相当的消耗战。这样,在全体上,我们是劣势(就数量来说),但在每一个局部上,在每一个具体战役上,我们是绝对的优势,这就保证了战役的胜利。随着时间的推移,我们就将在全体上转变为优势,直到歼灭一切敌人。(5)不打无准备之仗,不打无把握之仗,每战都应力求有准备,力求在敌我条件对比下有胜利的把握。(6)发扬勇敢战斗、不怕牺牲、不怕疲劳和连续作战(即在短期内不休息地接连打几仗)的作风。(7)力求在运动中歼灭敌人。同时,注重阵地攻击战术,夺取敌人的据点和城市。(8)在攻城问题上,一切敌人守备薄弱的据点和城市,坚决夺取之。一切敌人有中等程度的守备、而环境又许可加以夺取的据点和城市,相机夺取之。一切敌人守备强固的据点和城市,则等候条件成熟时然后夺取之。(9)以俘获敌人的全部武器和大部人员,补充自己。我军人力物力的来源,主要在前线。(10)善于利用两个战役之间的间隙,休息和整训部队。休整的时间,一般地不要过长,尽可能不使敌人获得喘息的时间。以上这些,就是人民解放军打败蒋介石的主要的方法。这些方法,是人民解放军在和国内外敌人长期作战的锻炼中产生出来,并完全适合我们目前的情况的\footnote[4]{ 一九五八年六月二十三日,毛泽东在中央军委扩大会议小组长座谈会上的讲话中指出:十大军事原则,是根据十年内战、抗日战争、解放战争前期的经验,在解放战争进入反攻时期提出来的,是马克思列宁主义普遍真理同中国革命战争实践相结合的产物。运用十大原则,取得了解放战争、抗美援朝战争的胜利(当然还有其它原因)。十大原则目前还可以用,今后有许多地方还可以用。但马克思列宁主义不是停止的,是向前发展的,十大原则也要根据今后战争的实际情况,加以补充和发展,有的可能要修正。}。蒋介石匪帮和美国帝国主义的在华军事人员,熟知我们的这些军事方法。蒋介石曾多次集训他的将校,将我们的军事书籍和从战争中获得的文件发给他们研究,企图寻找对付的方法。美国军事人员曾向蒋介石建议这样那样的消灭人民解放军的战略战术;并替蒋介石训练军队,接济军事装备。但是所有这些努力,都不能挽救蒋介石匪帮的失败。这是因为我们的战略战术是建立在人民战争这个基础上的,任何反人民的军队都不能利用我们的战略战术。在人民战争的基础上,在军队和人民团结一致、指挥员和战斗员团结一致以及瓦解敌军等项原则的基础上,人民解放军建立了自己的强有力的革命的政治工作,这是我们战胜敌人的重大因素。当着我们避开优势敌人的致命打击,并转移军力求得在运动中歼灭敌人,而主动地放弃许多城市的时候,我们的敌人是兴高采烈了。他们认为这就是他们的胜利和我们的失败。他们被一时的所谓胜利冲昏了头脑。张家口被占领的当天下午,蒋介石即下令召集他的反动的国民大会\footnote[5]{ 见本卷《美国“调解”真相和中国内战前途》注〔4〕。},似乎他的反动统治从此可以安如泰山了。美国帝国主义分子也手舞足蹈,似乎他们将中国变为美国殖民地的狂妄计划,从此可以毫无阻碍地实现了。但是,随着时间的推移,蒋介石及其美国主子的腔调也发生了变化。现在是一切国内外敌人都被他们的悲观情绪所统治的时候。他们唉声叹气,大叫危机,一点欢乐的影子也看不见了。十八个月中,蒋介石的前线高级指挥官,大部分因为战败被撤换。这里有郑州的刘峙,徐州的薛岳,苏北的吴奇伟,鲁南的汤恩伯,豫北的王仲廉,沈阳的杜聿明、熊式辉,北平的孙连仲等人。负指挥全部作战责任的蒋介石的参谋总长陈诚,亦被取消此种指挥职权,降为东北一个战场的指挥官\footnote[6]{ 刘峙原任国民党郑州绥靖公署主任,因为一九四六年九月在定陶战役中失败,当月被撤职。薛岳原任国民党徐州绥靖公署主任,因为他所指挥的国民党军队在一九四六年十二月宿北战役、一九四七年一月鲁南战役、同年二月莱芜战役中,接连遭受严重失败,于三月被撤职。吴奇伟原任国民党徐州绥靖公署副主任,因为一九四六年十二月宿北战役失败,于一九四七年三月被撤职。汤恩伯原任国民党第一兵团司令官,因为一九四七年五月孟良崮战役中国民党整编第七十四师被歼,于六月被撤职。王仲廉原任国民党第四兵团司令官,因为一九四七年七月在鲁西南战役中失败,于八月被撤职。杜聿明原任国民党东北保安司令长官,熊式辉原任国民党东北行辕主任,均因在东北民主联军一九四七年夏季攻势中遭到失败而于八月被撤职。孙连仲原任国民党保定绥靖公署主任,因为一九四七年十月至十一月在清风店、石家庄战役中遭到失败而被撤职。陈诚原任国民党军参谋总长,因指挥山东历次战役屡遭失败,于一九四七年八月兼任东北行辕主任,被取消了参谋总长的实际权力。}。而在蒋介石自己代替陈诚担任全局指挥的期间,却发生了蒋军由进攻转入防御,人民解放军由防御转入进攻这样一个局面。蒋介石反动集团及其美国主子,现在应当感觉到他们自己的错误了。他们将日本投降以后一个长时间内,中国共产党代表中国人民的愿望,力争和平反对内战的一切努力,看作是胆怯和力量薄弱的表现。他们过高地估计了自己力量,过低地估计了革命力量,冒险地发动战争,因而落在他们自己布置的陷阱里。我们敌人的战略打算是彻底地输了。\\
\subsubsection*{\myformat{四}}
现在,比较十八个月以前,人民解放军的后方也巩固得多了。这是由于我党坚决地站在农民方面实行土地改革的结果。在抗日战争时期,为着同国民党建立抗日统一战线和团结当时尚能反对日本帝国主义的人们起见,我党主动地把抗日以前的没收地主土地分配给农民的政策,改变为减租减息的政策,这是完全必需的。日本投降以后,农民迫切地要求土地,我们就及时地作出决定,改变土地政策,由减租减息改为没收地主阶级的土地分配给农民。我党中央一九四六年五月四日发出的指示,就是表现这种改变。一九四七年九月,我党召集了全国土地会议,制定了中国土地法大纲\footnote[7]{ 中国共产党全国土地会议,一九四七年九月举行于河北省平山县西柏坡村。这个会议在九月十三日通过的《中国土地法大纲》,于同年十月十日由中共中央公布。土地法大纲规定:“废除封建性及半封建性剥削的土地制度,实行耕者有其田的土地制度”;“乡村中一切地主的土地及公地,由乡村农会接收,连同乡村中其它一切土地,按乡村全部人口,不分男女老幼,统一平均分配”;“乡村农会接收地主的牲畜、农具、房屋、粮食及其它财产,并征收富农的上述财产的多余部分,分给缺乏这些财产的农民及其它贫民,并分给地主同样的一份”。这样,土地法大纲就不但肯定了一九四六年《五四指示》所提出的“没收地主土地分配给农民”的原则,而且改正了《五四指示》中对某些地主照顾过多的不彻底性。},并立即在各地普遍实行。这个步骤,不但肯定了去年《五四指示》的方针,而且对于去年《五四指示》中的某些不彻底性作了明确的改正。中国土地法大纲规定,在消灭封建性和半封建性剥削的土地制度、实行耕者有其田的土地制度的原则下,按人口平均分配土地\footnote[8]{ 《中国土地法大纲》所规定的平分土地的办法,在以后的执行过程中有了一些改变。一九四八年二月,中共中央在《关于在老区半老区进行土地改革工作与整党工作的指示》中规定:在一切封建制度已被推翻的老区半老区,不再平分土地,而只在必要时采取抽多补少、抽肥补瘦的办法,调剂一部分土地和其它生产资料给尚未彻底翻身的贫雇农,并容许中农保有比较一般贫农所得土地的平均水平为高的土地数量。在封建制度还存在的地方,平分的重点,也限于地主的土地财产和旧式富农的多余的土地财产方面。无论在哪一种地方,对于中农和新式富农的多余土地,只有在确有调剂必要和本人确实同意的条件下,才允许抽出调剂。在新解放区的土地改革中,对一切中农的土地都不再抽动。}。这是最彻底地消灭封建制度的一种方法,这是完全适合于中国广大农民群众的要求的。为着坚决地彻底地进行土地改革,乡村中不但必须组织包括雇农贫农中农在内的最广泛群众性的农会及其选出的委员会,而且必须首先组织包括贫农雇农群众的贫农团及其选出的委员会,以为执行土地改革的合法机关,而贫农团则应当成为一切农村斗争的领导骨干。我们的方针是依靠贫农,巩固地联合中农,消灭地主阶级和旧式富农的封建的和半封建的剥削制度。地主富农应得的土地和财产,不能超过农民群众。但是,曾经在一九三一年至一九三四年期间实行过的所谓“地主不分田,富农分坏田”的过左的错误的政策,也不应重复。地主富农在乡村人口中所占的比例,虽然各地有多有少,但按一般情况来说,大约只占百分之八左右(以户为单位计算),而他们占有的土地,按照一般情况,则达全部土地的百分之七十至八十。因此,我们的土地改革所反对的对象,人数甚少,而乡村中能够参加和应当参加土地改革统一战线的人数(户数),则有大约百分之九十以上这样多。这里必须注意两条基本原则:第一,必须满足贫农和雇农的要求,这是土地改革的最基本的任务;第二,必须坚决地团结中农,不要损害中农的利益。只要我们掌握了这两条基本原则,我们的土地改革任务就一定能够胜利地完成。旧式富农按照平分原则所多余的土地及其一部分财产之所以应当拿出来分配,是因为中国的富农一般地带着很重的封建和半封建剥削的性质,富农大都兼出租土地和放高利贷,其雇佣劳动的条件亦是半封建的\footnote[9]{ 中国土地改革中的富农问题,是在中国具体的历史经济条件下形成的一个特殊问题。中国的富农,一般具有很重的封建和半封建剥削的性质,而这种富农经济在全国农业经济中又不占重要地位,这两点都不同于许多资本主义国家中的富农。在中国的反对地主阶级封建剥削的斗争中,广大的贫雇农要求同时废除富农的封建和半封建剥削。在解放战争时期,中国共产党采取了征收富农多余的土地和财产分给农民的政策,从而满足了广大贫雇农的要求,保证了人民解放战争的胜利。随着战争的胜利发展,在一九四八年二月,中共中央规定在新解放区实行土地改革的新政策,即将新区土地改革分为两个阶段,在第一阶段,中立富农,专门打击地主,首先是打击大地主;第二阶段,在平分地主土地的时候,也分配富农出租和多余的土地,但是对待富农和对待地主仍然有所区别(见本卷《新解放区土地改革要点》)。在中华人民共和国成立以后,中央人民政府在一九五〇年六月发布土地改革法,规定在土地改革中对富农只征收其出租的土地的一部或全部,对富农的其它土地和财产则予以保护。在以后的社会主义改造进程中,随着农业合作化运动的深入和农村经济的发展,富农经济就不再存在了。}。还因为他们所占的土地数量较多,质量较好\footnote[10]{ 这里所说的富农所占的土地数量较多,质量较好,是就每户富农同每户贫农所占的土地的比较而言。从全国说来,中国的富农所占的生产资料和所生产的农产品的数量,都是很小的。富农经济在全国农村经济中并不占有重要的地位。},如不平分则不能满足贫雇农的要求。但是按照土地法大纲的规定,对待富农和对待地主一般地应当有所区别。土地改革中,中农表现赞成平分,这是因为平分并不损害中农利益。在平分时,中农中一部分土地不变动,一部分增加了土地,只有一部分富裕中农有少数多余的土地,他们也愿意拿出来平分,这是因为在平分后他们的土地税的负担也减轻了。虽然如此,各地在平分土地时,仍须注意中农的意见,如果中农不同意,则应向中农让步。在没收分配封建阶级的土地财产时应当注意某些中农的需要。在划分阶级成分时,必须注意不要把本来是中农成分的人,错误地划到富农圈子里去。在农会委员会中,在政府中,必须吸收中农积极分子参加工作。在土地税和支援战争的负担上,必须采取公平合理的原则。这些,就是我党在执行巩固地联合中农这一战略任务时所必须采取的具体政策。全党必须明白,土地制度的彻底改革,是现阶段中国革命的一项基本任务。如果我们能够普遍地彻底地解决土地问题,我们就获得了足以战胜一切敌人的最基本的条件。\\
\subsubsection*{\myformat{五}}
为了坚决地彻底地实行土地改革,巩固人民解放军的后方,必须整编党的队伍。抗日战争时期我党内部的整风运动\footnote[11]{ 见本书第三卷《学习和时局》注〔11〕。},是一般地收到了成效的。这种成效,主要地是在于使我们的领导机关和广大的干部和党员,进一步地掌握了马克思列宁主义的普遍真理和中国革命的具体实践的统一这样一个基本的方向。在这点上我们党是比抗日以前的几个历史时期,大进一步了。但是,在党的地方组织方面,特别是在党的农村基层组织方面所存在的成分不纯和作风不纯的问题,则没有获得解决。一九三七年至一九四七年,十一年时间内,我们党的组织,由几万党员,发展到了二百七十万党员,这是一个极大的跃进。这使我们的党成了一个在中国历史上空前强大的党。这使我们有可能打败日本帝国主义,并打退蒋介石的进攻,领导一万万以上人口的解放区和二百万人民解放军。但是缺点也就跟着来了。这即是有许多地主分子、富农分子和流氓分子乘机混进了我们的党。他们在农村中把持许多党的、政府的和民众团体的组织,作威作福,欺压人民,歪曲党的政策,使这些组织脱离群众,使土地改革不能彻底。这种严重情况,就在我们面前提出了整编党的队伍的任务。这个任务如果不解决,我们在农村中就不能前进。党的全国土地会议彻底地讨论了这个问题,并规定了适当的步骤和方法。这些步骤和方法,现在正和平分土地的决定一道在各地坚决地实施。其中首先重要的,是在党内展开批评和自我批评,彻底地揭发各地组织内的离开党的路线的错误思想和严重现象。全党同志必须明白,解决这个党内不纯的问题,整编党的队伍,使党能够和最广大的劳动群众完全站在一个方向,并领导他们前进,是解决土地问题和支援长期战争的一个决定性的环节。\\
\subsubsection*{\myformat{六}}
没收封建阶级的土地归农民所有,没收蒋介石、宋子文、孔祥熙、陈立夫为首的垄断资本归新民主主义的国家所有,保护民族工商业。这就是新民主主义革命的三大经济纲领。蒋宋孔陈四大家族,在他们当权的二十年中,已经集中了价值达一百万万至二百万万美元的巨大财产,垄断了全国的经济命脉。这个垄断资本,和国家政权结合在一起,成为国家垄断资本主义。这个垄断资本主义,同外国帝国主义、本国地主阶级和旧式富农密切地结合着,成为买办的封建的国家垄断资本主义。这就是蒋介石反动政权的经济基础。这个国家垄断资本主义,不但压迫工人农民,而且压迫城市小资产阶级,损害中等资产阶级。这个国家垄断资本主义,在抗日战争期间和日本投降以后,达到了最高峰,它替新民主主义革命准备了充分的物质条件。这个资本,在中国的通俗名称,叫做官僚资本。这个资产阶级,叫做官僚资产阶级,即是中国的大资产阶级。新民主主义的革命任务,除了取消帝国主义在中国的特权以外,在国内,就是要消灭地主阶级和官僚资产阶级(大资产阶级)的剥削和压迫,改变买办的封建的生产关系,解放被束缚的生产力。被这些阶级及其国家政权所压迫和损害的上层小资产阶级和中等资产阶级,虽然也是资产阶级,却是可以参加新民主主义革命,或者保守中立的。他们和帝国主义没有联系,或者联系较少,他们是真正的民族资产阶级。在新民主主义的国家权力到达的地方,对于这些阶级,必须坚决地毫不犹豫地给以保护。蒋介石统治区域的上层小资产阶级和中等资产阶级,其中有为数不多的一部分人,即这些阶级的右翼分子,存在着反动的政治倾向,他们替美国帝国主义和蒋介石反动集团散布幻想,他们反对人民民主革命。当着他们的反动倾向尚能影响群众时,我们应当向着接受他们影响的群众进行揭露的工作,打击他们在群众中的政治影响,使群众从他们的影响之下解放出来。但是,政治上的打击和经济上的消灭是两件事,如果混同这两件事,我们就要犯错误。新民主主义革命所要消灭的对象,只是封建主义和垄断资本主义,只是地主阶级和官僚资产阶级(大资产阶级),而不是一般地消灭资本主义,不是消灭上层小资产阶级和中等资产阶级。由于中国经济的落后性,广大的上层小资产阶级和中等资产阶级所代表的资本主义经济,即使革命在全国胜利以后,在一个长时期内,还是必须允许它们存在;并且按照国民经济的分工,还需要它们中一切有益于国民经济的部分有一个发展;它们在整个国民经济中,还是不可缺少的一部分。这里所说的上层小资产阶级,是指雇佣工人或店员的小规模的工商业者。此外,还有不雇佣工人或店员的广大的独立的小工商业者,对于这些小工商业者,不待说,是应当坚决地保护的。革命在全国胜利以后,由于新民主主义国家手里有着从官僚资产阶级接收过来的控制全国经济命脉的巨大的国家企业,又有从封建制度解放出来、虽则在一个颇长时间内在基本上仍然是分散的个体的、但是在将来可以逐步地引向合作社方向发展的农业经济,在这些条件下,这种小的和中等的资本主义成分,其存在和发展,并没有什么危险。土地改革后,在农村中必然发生的新的富农经济,也是如此。对于上层小资产阶级和中等资产阶级经济成分采取过左的错误的政策,如像我们党在一九三一年至一九三四年期间所犯过的那样(过高的劳动条件,过高的所得税率,在土地改革中侵犯工商业者,不以发展生产、繁荣经济、公私兼顾、劳资两利为目标,而以近视的片面的所谓劳动者福利为目标),是绝对不许重复的。这些错误如果重犯,必然要损害劳动群众的利益和新民主主义国家的利益。中国土地法大纲上有一条规定:“保护工商业者的财产及其合法的营业,不受侵犯。”这里所说的工商业者,就是指的一切独立的小工商业者和一切小的和中等的资本主义成分。总起来说,新中国的经济构成是:(1)国营经济,这是领导的成分;(2)由个体逐步地向着集体方向发展的农业经济;(3)独立小工商业者的经济和小的、中等的私人资本经济。这些,就是新民主主义的全部国民经济。而新民主主义国民经济的指导方针,必须紧紧地追随着发展生产、繁荣经济、公私兼顾、劳资两利这个总目标。一切离开这个总目标的方针、政策、办法,都是错误的。\\
\subsubsection*{\myformat{七}}
一九四七年十月,人民解放军发表宣言,其中说:“联合工农兵学商各被压迫阶级、各人民团体、各民主党派、各少数民族、各地华侨和其它爱国分子,组成民族统一战线,打倒蒋介石独裁政府,成立民主联合政府。”这就是人民解放军的、也是中国共产党的最基本的政治纲领。从表面上看来,现在时期,比较抗日时期,我们的革命的民族统一战线,似乎是缩小了。但是在实际上,只是在现在时期,只是在蒋介石出卖民族利益给美国帝国主义,发动反人民的全国规模的国内战争之后,只是在美国帝国主义和蒋介石反动统治集团的罪恶已经在中国人民面前暴露无遗之后,我们的民族统一战线才是真正地扩大了。在抗日时期,蒋介石和国民党在中国人民中还没有完全丧失威信,他们还有许多的欺骗作用。现在不同了,他们的一切欺骗都已被他们自己的行为所揭穿,他们已经没有什么群众,他们已经完全孤立了。和国民党相反,中国共产党不但在解放区得到最广大人民群众的信任;在国民党统治区,在国民党控制的大城市,也得到了广大人民群众的拥护。如果说,在一九四六年,在蒋介石统治下的上层小资产阶级和中等资产阶级的知识分子中,还有一部分人怀着所谓第三条道路\footnote[12]{ 在人民解放战争初期,有一些民主人士幻想在国民党的大地主、大资产阶级专政和中国共产党领导的人民民主专政之外,另找所谓第三条道路。这条道路,实际上就是英美式的资产阶级专政的道路。}的想法,那末,在现在,这种想法已经破产了。由于我党采取了彻底的土地政策,使我党获得了比较抗日时期广大得多的农民群众的衷心拥护。由于美国帝国主义的侵略、蒋介石的压迫和我党坚决保护群众利益的正确方针,我党获得了蒋介石统治区域工人阶级、农民阶级、城市小资产阶级和中等资产阶级的广大群众的同情。这些群众,因为挨饿,因为政治上受压迫,因为蒋介石的反人民的内战夺去了人民的一切活路,他们就不断地掀起了反对美国帝国主义和蒋介石反动政府的斗争,他们的基本口号是反饥饿,反迫害,反内战和反对美国干涉中国内政。而在抗日以前,在抗日时期,乃至在日本投降后的一个时期,他们的觉悟都没有达到这样的程度。因此我们说,我们的新民主主义的革命的统一战线,现在比过去任何时期都要广大,也比过去任何时期都要巩固。这件事,不但同我们的土地政策和城市政策相联系,而且同人民解放军的胜利,同蒋介石由进攻转入防御,人民解放军由防御转入进攻,中国革命已经进入新的高潮时期,这一总的政治形势,密切地联系着。现在,人们看到了蒋介石统治的灭亡已经不可避免,因而将希望寄托在中国共产党和人民解放军身上,这是很自然的道理。中国新民主主义的革命要胜利,没有一个包括全民族绝大多数人口的最广泛的统一战线,是不可能的。不但如此,这个统一战线还必须是在中国共产党的坚强的领导之下。没有中国共产党的坚强的领导,任何革命统一战线也是不能胜利的。在一九二七年北伐战争达到高潮的时期,我党领导机关的投降主义分子,自愿地放弃对于农民群众、城市小资产阶级和中等资产阶级的领导权,尤其是放弃对于武装力量的领导权,使那次革命遭到了失败。抗日战争时期,我党反对了和这种投降主义思想相类似的思想,即是对于国民党的反人民政策让步,信任国民党超过信任人民群众,不敢放手发动群众斗争,不敢在日本占领地区扩大解放区和扩大人民的军队,将抗日战争的领导权送给国民党。我党对于这样一种软弱无能的腐朽的违背马克思列宁主义原则的思想,进行了坚决的斗争,坚决地执行了“发展进步势力,争取中间势力,孤立顽固势力”的政治路线,坚决地扩大了解放区和人民解放军。这样,就不但保证了我党在日本帝国主义侵略时期能够战胜日本帝国主义,而且保证了我党在日本投降以后蒋介石举行反革命战争时期,能够顺利地不受损失地转变到用人民革命战争反对蒋介石反革命战争的轨道上,并在短时期内取得了伟大的胜利。这些历史教训,全党同志都要牢记。\\
\subsubsection*{\myformat{八}}
蒋介石反动集团在一九四六年发动全国规模的反人民的国内战争的时候,他们之所以敢于冒险,不但依靠他们自己的优势的军事力量,而且主要地依靠他们认为是“异常强大”的、“举世无敌”的、手里拿着原子弹的美国帝国主义。一方面,以为它能够像流水一样地供给他们以军事上和财政上的需要;另一方面,狂妄地设想所谓“美苏必战”,所谓“第三次世界大战必然爆发”。这种对于美国帝国主义的依赖,是第二次世界大战结束以后全世界各国反动势力的共同特点。这件事,反映了第二次世界大战给予世界资本主义的打击的严重性,反映了各国反动派力量的薄弱及其心理的恐慌和丧失信心,反映了全世界革命力量的强大,使得各国反动派除了依靠美国帝国主义的援助,就感到毫无出路。但是,在实际上,在第二次世界大战以后的美国帝国主义,是否真如蒋介石和各国反动派所设想的那么强大呢?是否真能像流水一样地接济蒋介石和各国反动派呢?并不如此。美国帝国主义在第二次世界大战期间所增强起来的经济力量,遇着了不稳定的日趋缩小的国内市场和国际市场。这种市场的进一步缩小,就要引起经济危机的爆发。美国的战争景气,仅仅是一时的现象。它的强大,只是表面的和暂时的。国内国外的各种不可调和的矛盾,就像一座火山,每天都在威胁美国帝国主义,美国帝国主义就是坐在这座火山上。这种情况,迫使美国帝国主义分子建立了奴役世界的计划,像野兽一样,向欧亚两洲和其它地方乱窜,集合各国的反动势力,那些被人民唾弃的渣滓,组成帝国主义和反民主的阵营,反对以苏联为首的一切民主势力,准备战争,企图在将来,在遥远的时间内,有一天发动第三次世界大战打败民主力量。这是一个狂妄的计划。全世界民主势力必须打败这个计划,也完全能够打败它。全世界反帝国主义阵营的力量超过了帝国主义阵营的力量。优势是在我们方面,不是在敌人方面。以苏联为首的反帝国主义阵营,已经形成。没有危机的、向上发展的、受到全世界广大人民群众爱护的社会主义的苏联,它的力量,现在就已经超过了被危机严重威胁着的、向下衰落的、受到全世界广大人民群众反对的帝国主义的美国。欧洲各人民民主国家,正在巩固其内部,并互相团结起来。以法意为首的欧洲各资本主义国家内的人民的反帝国主义力量,正在发展。美国内部,存在着日趋强大的人民民主势力。拉丁美洲的人民,并不是顺从美国帝国主义的奴隶。整个亚洲,兴起了伟大的民族解放运动。反帝国主义阵营的一切力量,正在团结起来,并正在向前发展。欧洲九个国家的共产党和工人党,业已组成了情报局,发表了号召全世界人民起来反对帝国主义奴役计划的檄文\footnote[13]{ 共产党和工人党情报局,是一九四七年九月在波兰华沙举行的保加利亚、罗马尼亚、匈牙利、波兰、苏联、法国、捷克斯洛伐克、意大利、南斯拉夫等九国共产党和工人党代表会议上通过成立的。一九五六年四月宣布停止活动。毛泽东在这里所说情报局发表的号召全世界人民起来反对帝国主义奴役计划的檄文,是指情报局一九四七年九月的会议通过的《关于国际形势的宣言》。}。这篇檄文,振奋了全世界被压迫人民的精神,指示了他们的斗争方向,巩固了他们的胜利信心。全世界的反动派,在这篇檄文面前惊惶失措。东方各国一切反帝国主义的力量,也应当团结起来,反对帝国主义和各国内部反动派的压迫,以东方十万万以上被压迫人民获得解放为奋斗的目标。我们自己的命运完全应当由我们自己来掌握。我们应当在自己内部肃清一切软弱无能的思想。一切过高地估计敌人力量和过低地估计人民力量的观点,都是错误的。我们和全世界民主力量一道,只要大家努力,一定能够打败帝国主义的奴役计划,阻止第三次世界大战使之不能发生,推翻一切反动派的统治,争取人类永久和平的胜利。我们清醒地知道,在我们的前进道路上,还会有种种障碍,种种困难,我们应当准备对付一切内外敌人的最大限度的抵抗和挣扎。但是,只要我们能够掌握马克思列宁主义的科学,信任群众,紧紧地和群众一道,并领导他们前进,我们是完全能够超越任何障碍和战胜任何困难的,我们的力量是无敌的。现在是全世界资本主义和帝国主义走向灭亡,全世界社会主义和人民民主主义走向胜利的历史时代,曙光就在前面,我们应当努力。\\
\newpage\section*{\myformat{关于建立报告制度}\\\myformat{(一九四八年一月七日)}}\addcontentsline{toc}{section}{关于建立报告制度}
\begin{introduction}\item 这是毛泽东为中共中央起草的对党内的指示。这个指示中所规定的报告制度,是中共中央坚持民主集中制、反对无纪律无政府倾向的长期斗争在新条件下的一个发展。这个问题在这时之所以特别重要,是因为革命形势已经有了极大的进展,许多解放区已经连成一片,许多城市已经解放或者即将解放,人民解放军和人民解放战争的正规性程度大为提高,全国胜利已经在望。这种情况,要求党迅速克服存在于党内和军队内的任何无纪律无政府状态,把一切必须和可能集中的权力集中于中央。建立严格的报告制度,就是中共中央为此目的而采取的一个重要步骤。关于这个问题,参看本卷《一九四八年的土地改革工作和整党工作》的第六部分和《中共中央关于九月会议的通知》的第四点。\end{introduction}
为了及时反映情况,使中央有可能在事先或事后帮助各地不犯或少犯错误,争取革命战争更加伟大的胜利起见,从今年起,规定如下的报告制度。\\
  (一)各中央局和分局,由书记负责(自己动手,不要秘书代劳),每两个月,向中央和中央主席作一次综合报告。报告内容包括该区军事、政治、土地改革、整党、经济、宣传和文化等各项活动的动态,活动中发生的问题和倾向,对于这些问题和倾向的解决方法。报告文字每次一千字左右为限,除特殊情况外,至多不要超过两千字。一次不能写完全部问题时,分两次写。或一次着重写几个问题,对其余问题则不着重写,只略带几笔;另一次,则着重写其余问题,而对上次着重写过的只略带几笔。综合报告内容要扼要,文字要简练,要指出问题或争论之所在。写发综合报告的日期是单月的上旬,报告用电报发来。这是各中央局、分局书记个人负责向中央和中央主席作的经常性的报告和请示。书记在前线指挥作战时,除自己报告外,指定代理书记或副书记作后方活动的报告。此外,各中央局和分局向中央所作的临时性的报告和请示,照过去一样,不在此内。\\
  我们所以规定这项政策性的经常的综合的报告和请示的制度,是因为党的第七次全国代表大会以后,仍然有一些(不是一切)中央局和分局的同志,不认识事先或事后向中央作报告并请求指示的必要和重要性,或仅仅作了一些技术性的报告和请示,以致中央不明了或者不充分明了他们重要的(不是次要的或技术性的)活动和政策的内容,因而发生了某些不可挽救的、或难以挽救的、或能够挽救但已受了损失的事情。而那些事前请示、事后报告的中央局或分局,则避免了或减少了这样的损失。从今年起,全党各级领导机关,必须改正对上级事前不请示、事后不报告的不良习惯。各中央局和分局是受中央委任、代表中央执行其所委托的任务的机关,必须同中央发生最密切的联系。各省委或区党委,同各中央局和分局也须密切联系。当此革命已进入新的高潮时期,加强此种联系,极为必要。\\
  (二)各野战军首长和军区首长,除作战方针必须随时报告和请示,并且照过去规定,每月作一次战绩报告、损耗报告和实力报告外,从今年起,每两个月要作一次政策性的综合报告和请示。此项报告和请示的内容是:关于该军纪律,物质生活,指战员情绪,指战员中发生的偏向,克服偏向的方法,技术、战术进步或退步的情况,敌军的长处、短处和士气高低,我军政治工作的情况,我军对土地政策、城市政策、俘虏政策的执行情况和克服偏向的方法,军民关系和各阶层人民的动向等。此项报告的字数、写作方法以及发报时间,和各中央局、分局报告的办法相同。如规定的写报告时间(逢单月的上旬)恰在作战紧张的时候,则可提前或推迟若干天,但须申明原因。其中关于政治工作部分,由该军政治部主任起草,经司令员、政治委员审查修改,并且共同署名。报告用电报发给军委主席。我们规定此项政策性综合报告的理由,和上述中央局、分局应作综合报告的理由相同。\\
\newpage\section*{\myformat{关于目前党的政策中的几个重要问题}\\\myformat{(一九四八年一月十八日)}}\addcontentsline{toc}{section}{关于目前党的政策中的几个重要问题}
\begin{introduction}\item  这是毛泽东为中共中央起草的决定草案。参看本卷《目前形势和我们的任务》一文的题解。\end{introduction}
\subsection*{\myformat{一 党内反对错误倾向问题}}
反对对敌人的力量估计过高。例如,惧怕美帝国主义,惧怕到国民党区域作战,惧怕消灭买办封建制度、平分地主土地和没收官僚资本,惧怕长期战争等。这些都是不正确的。全世界帝国主义和中国蒋介石反动集团的统治,已经腐烂,没有前途。我们有理由轻视它们,我们有把握、有信心战胜中国人民的一切内外敌人。但是在每一个局部上,在每一个具体斗争问题上(不论是军事的、政治的、经济的或思想的斗争),却又决不可轻视敌人,相反,应当重视敌人,集中全力作战,方能取得胜利。当着我们正确地指出在全体上,在战略上,应当轻视敌人的时候,却决不可在每一个局部上,在每一个具体问题上,也轻视敌人。如果我们在全体上过高估计敌人力量,因而不敢推翻他们,不敢胜利,我们就要犯右倾机会主义错误。如果我们在每一个局部上,在每一个具体问题上,不采取谨慎态度,不讲究斗争艺术,不集中全力作战,不注意争取一切应当争取的同盟者(中农,独立工商业者,中产阶级,学生、教员、教授和一般知识分子,一般公务人员,自由职业者和开明绅士),我们就要犯“左”倾机会主义错误。\\
  反对党内“左”、右倾向,必须依据具体情况决定方针。例如:军队在打胜仗的时候,必须防止“左”倾;在打败仗或者未能多打胜仗的时候,必须防止右倾。土地改革在群众尚未认真发动和尚未展开斗争的地方,必须反对右倾;在群众已经认真发动和已经展开斗争的地方,必须防止“左”倾。\\
\subsection*{\myformat{二 土地改革和群众运动中的几个具体政策问题}}
一、必须将贫雇农的利益和贫农团的带头作用,放在第一位。我党必须经过贫雇农发动土地改革,必须使贫雇农在农会中在乡村政权中起带头作用,这种带头作用即是团结中农和自己一道行动,而不是抛弃中农由贫雇农包办一切。在老解放区中农占多数贫雇农占少数的地方,中农的地位尤为重要。“贫雇农打江山坐江山”的口号是错误的。在乡村,是雇农、贫农、中农和其它劳动人民联合一道,在共产党领导之下打江山坐江山,而不是单独贫雇农打江山坐江山。在全国,是工人,农民(包括新富农),独立工商业者,被反动势力所压迫和损害的中小资本家,学生、教员、教授、一般知识分子,自由职业者,开明绅士,一般公务人员,被压迫的少数民族和海外华侨,联合一道,在工人阶级(经过共产党)的领导之下,打江山坐江山,而不是少数人打江山坐江山。\\
  二、必须避免对中农采取任何冒险政策。对中农和其它阶层订错了成分的,应一律改正,分了的东西应尽可能退还。在农民代表中、农民委员会中排斥中农的倾向和在土地改革斗争中将贫雇农同中农对立起来的倾向,必须纠正。有剥削收入的农民,其剥削收入占总收入百分之二十五(四分之一)以下者,应订为中农,以上者为富农\footnote[1]{ 关于农村划分阶级的标准,参见本书第一卷《怎样分析农村阶级》一文和第二卷《中国革命和中国共产党》的第二章第四节。}。富裕中农的土地不得本人同意不能平分。\\
  三、必须避免对中小工商业者采取任何冒险政策。各解放区过去保护并奖励一切于国民经济有益的私人工商业发展的政策是正确的,今后仍应继续。减租减息时期鼓励地主富农转入工商业的政策也是正确的,认为“化形”而加以反对和没收分配是错误的。地主富农的工商业一般应当保护,只有官僚资本和真正恶霸反革命分子的工商业,才可以没收。这种应当没收的工商业,凡属有益于国民经济的,在国家和人民接收过来之后,必须使其继续营业,不得分散或停闭。对于一切有益于国民经济的工商业征收营业税,必须以不妨碍其发展为限度。在公营企业中,必须由行政方面和工会方面组织联合的管理委员会,以加强管理工作,达到降低成本、增加生产、公私两利的目的。私人资本主义企业也应当试行这种办法,以达到降低成本、增加生产、劳资两利的目的。工人生活必须酌量改善,但是必须避免过高的待遇。\\
  四、对于学生、教员、教授、科学工作者、艺术工作者和一般知识分子,必须避免采取任何冒险政策。中国学生运动和革命斗争的经验证明,学生、教员、教授、科学工作者、艺术工作者和一般知识分子的绝大多数,是可以参加革命或者保持中立的,坚决的反革命分子只占极少数。因此,我党对于学生、教员、教授、科学工作者、艺术工作者和一般知识分子,必须采取慎重态度。必须分别情况,加以团结、教育和任用,只对其中极少数坚决的反革命分子,才经过群众路线予以适当的处置。\\
  五、关于开明绅士问题。抗日时期,我党在各解放区政权机关(参议会和政府)中同开明绅士合作,是完全必需的,并且是成功的。对于那些同我党共过患难确有相当贡献的开明绅士,在不妨碍土地改革的条件下,必须分别情况,予以照顾。其中政治上较好又有工作能力者,应当继续留在高级政府中给以适当的工作。政治上较好但缺乏工作能力者,应当维持其生活。其为地主富农出身而人民对他们没有很大恶感者,按土地法平分其封建的土地财产,但应使其避免受斗争。对于过去混进我政权机关中来、实际上一贯是坏人、对人民并无好处而为广大群众所极端痛恨者,则照一般处理恶霸分子的办法交由人民法庭审处。\\
  六、必须将新富农和旧富农加以区别。在减租减息时期提出鼓励新富农和富裕中农,对于稳定中农、发展解放区农业生产是收了成效的。平分土地以后,必须号召农民发展生产,丰衣足食,并劝告农民组织变工队、互助组或换工班一类的农业互助合作组织。平分土地时,对于老解放区的新富农,照富裕中农待遇,不得本人同意,不能平分其土地。\\
  七、地主富农在老解放区减租减息时期改变生活方式,地主转入劳动满五年以上,富农降为中贫农满三年以上者,如果表现良好,即可依其现在状况改变成分。其中确仍保有大量多余财产(不是少量多余财产)者,则应依照农民要求拿出其多余部分。\\
  八、土地改革的中心是平分封建阶级的土地及其粮食、牲畜、农具等财产(富农只拿出其多余部分),不应过分强调斗地财\footnote[2]{ 地财,指地主埋藏在地下的财物。},尤其不应在斗地财上耗费很长时间,妨碍主要工作。\\
  九、对待地主和对待富农必须依照土地法大纲\footnote[3]{ 见本卷《目前形势和我们的任务》注〔7〕。}加以区别。\\
  十、对大、中、小地主,对地主富农中的恶霸和非恶霸,在平分土地的原则下,也应有所区别。\\
  十一、极少数真正罪大恶极分子经人民法庭认真审讯判决,并经一定政府机关(县级或分区一级所组织的委员会)批准枪决予以公布,这是完全必要的革命秩序。这是一方面。另一方面,必须坚持少杀,严禁乱杀。主张多杀乱杀的意见是完全错误的,它只会使我党丧失同情,脱离群众,陷于孤立。土地法大纲上规定的经过人民法庭审讯判决的这一斗争方式,必须认真实行,它是农民群众打击最坏的地主富农分子的有力武器,又可免犯乱打乱杀的错误。应在适当时机(在土地斗争达到高潮之后),教育群众懂得自己的远大利益,要把一切不是坚决破坏战争、坚决破坏土地改革,而在全国数以千万计(在全国约三亿六千万乡村人口中占有约三千六百万之多)的地主富农,看作是国家的劳动力,而加以保存和改造。我们的任务是消灭封建制度,消灭地主之为阶级,而不是消灭地主个人。必须按照土地法给以不高于农民所得的生产资料和生活资料。\\
  十二、对于某些犯有重大错误的干部和党员,以及工农群众中的某些坏分子,必须进行批评和斗争。在批评和斗争的时候,应当说服群众,采取正确的方法和方式,避免粗暴行动。这是一方面。另一方面,则应使这些干部、党员和坏分子提出保证,不对群众采取报复。应当宣布,群众不但有权对他们放手批评,而且有权在必要时将他们撤职,或建议撤职,或建议开除党籍,直至将其中最坏的分子送交人民法庭审处。\\
\subsection*{\myformat{三 关于政权问题}}
一、新民主主义的政权是工人阶级领导的人民大众的反帝反封建的政权。所谓人民大众,是包括工人阶级、农民阶级、城市小资产阶级、被帝国主义和国民党反动政权及其所代表的官僚资产阶级(大资产阶级)和地主阶级所压迫和损害的民族资产阶级,而以工人、农民(兵士主要是穿军服的农民)和其它劳动人民为主体。这个人民大众组成自己的国家(中华人民共和国)并建立代表国家的政府(中华人民共和国的中央政府),工人阶级经过自己的先锋队中国共产党实现对于人民大众的国家及其政府的领导。这个人民共和国及其政府所要反对的敌人,是外国帝国主义、本国国民党反动派及其所代表的官僚资产阶级和地主阶级。\\
  二、中华人民共和国的权力机关是各级人民代表大会及其选出的各级政府。\\
  三、现在时期,在乡村中可以而且应当依据农民的要求,召集乡村农民大会选举乡村政府,召集区农民代表大会选举区政府。县、市和县市以上的政府,因其不但代表乡村的农民,而且代表市镇、县城、省城和大工商业都市的各阶层各职业人民,就应召集县的、市的、省的或边区的人民代表大会,选举各级政府。在将来,革命在全国胜利之后,中央和地方各级政府,都应当由各级人民代表大会选举。\\
\subsection*{\myformat{四 在革命统一战线中领导者和被领导者的关系问题}}
领导的阶级和政党,要实现自己对于被领导的阶级、阶层、政党和人民团体的领导,必须具备两个条件:(甲)率领被领导者(同盟者)向着共同敌人作坚决的斗争,并取得胜利;(乙)对被领导者给以物质福利,至少不损害其利益,同时对被领导者给以政治教育。没有这两个条件或两个条件缺一,就不能实现领导。例如共产党要领导中农,必须率领中农和自己一道向封建阶级作坚决的斗争,并取得胜利(消灭地主武装,平分地主土地)。如果没有坚决的斗争,或虽有斗争而没有胜利,中农就会动摇。再则,必须以地主土地财产的一部分分配给中农中的较贫困者,对于富裕中农则不要损害其利益。在农会中和乡区政府中,必须吸收中农积极分子参加工作,并须在数量上做适当规定(例如占委员的三分之一)。不要订错中农的成分,对中农的土地税和战争勤务要公道。同时,还要给中农以政治教育。如果没有这些,我们就要丧失中农的拥护。城市中工人阶级和共产党要实现对于被反动势力所压迫和损害的中产阶级、民主党派、人民团体的领导,也是如此。\\
\newpage\section*{\myformat{军队内部的民主运动}\\\myformat{(一九四八年一月三十日)}}\addcontentsline{toc}{section}{军队内部的民主运动}
\begin{introduction}\item
这是毛泽东为中共中央军事委员会起草的对党内的指示。
\end{introduction}
部队内部政治工作方针,是放手发动士兵群众、指挥员和一切工作人员,通过集中领导下的民主运动,达到政治上高度团结、生活上获得改善、军事上提高技术和战术的三大目的。目前在我军部队中热烈进行的三查、三整\footnote[1]{ 三查、三整,是中国共产党在人民解放战争时期,结合土地改革所进行的整党整军的一个重要运动。三查,在地方上是指查阶级、查思想、查作风;在部队中是指查阶级、查工作、查斗志。三整,是指整顿组织、整顿思想、整顿作风。},就是用政治民主、经济民主的方法,达到前两项目的。\\
  关于经济民主,必须使士兵选出的代表有权协助(不是超过)连队首长管理连队的给养和伙食。\\
  关于军事民主,必须在练兵时实行官兵互教,兵兵互教;在作战时,实行在火线上连队开各种大、小会,在连队首长指导下,发动士兵群众讨论如何攻克敌阵,如何完成战斗任务。在连续几天的战斗中,此种会应开几次。此项军事民主,在陕北蟠龙战役\footnote[2]{ 蟠龙是位于延安东北的一个市镇。一九四七年五月,人民解放军西北野战部队在蟠龙全歼国民党军胡宗南部守军六千七百余人,并缴获大量武器、装备和给养。}和晋察冀石家庄战役\footnote[3]{ 石家庄战役是一九四七年十一月六日至十二日人民解放军晋察冀野战军和地方武装一部在石家庄向国民党军发起的一次攻坚战役。这次战役共歼灭国民党军二万四千余人,解放了石家庄,并使晋察冀和晋冀鲁豫两大解放区连成一片,开创了人民解放军夺取大城市的先例。}中,都实行了,收到了极大效果。证明只有好处,毫无害处。\\
  应当使士兵群众对于干部中的坏分子有揭发其错误和罪恶的权利。应当相信,士兵对于一切好的和较好的干部是不会不加爱护的。同时,应当使士兵在必要时,有从士兵群众中推选他们相信的下级干部候选人员、以待上级委任的权利。在下级干部极端缺乏的时候,这种推选很有用处。但是这种推选不是普遍的推选,而是某些必要时的推选。\\
\newpage\section*{\myformat{在不同地区实施土地法的不同策略}\\\myformat{(一九四八年二月三日)}}\addcontentsline{toc}{section}{在不同地区实施土地法的不同策略}
\begin{introduction}\item  这是毛泽东给刘少奇的一封电报。\end{introduction}
关于土地法\footnote[1]{ 土地法,指《中国土地法大纲》。见本卷《目前形势和我们的任务》注〔7〕。}的实施,应当分三种地区,采取不同策略。\\
  一、日本投降以前的老解放区。这种地区大体上早已分配土地,只须调整一部分土地。这种地区的工作中心,应当是按照平山经验\footnote[2]{ 平山县在河北省西部,当时属于晋察冀解放区。这里说的平山经验,是指该县在土地改革运动中,采取邀请党外群众列席党支部会议的方式,整顿党在农村的基层组织的经验。},用党内党外结合的方法整理党的队伍,解决党同群众间的矛盾。在这种老区,不是照土地法再来分配一次土地,也不是人为地、勉强地组织贫农团去领导农会,而是在农会中组织贫农小组。这种小组的积极分子,可以担负农会和农村政权中的领导工作,但是不应当排斥中农,规定一定由贫农做领导工作。这种地区的农会和农村政权的领导工作,应当由贫农和中农中思想正确、办事公道的积极分子去做。在这种地区,过去的贫农大多数已升为中农,中农已占乡村人口的大多数,所以必须吸收中农中的积极分子参加农村的领导工作。\\
  二、日本投降至大反攻,即一九四五年九月至一九四七年八月两年内所解放的地区。这种地区,占现在解放区的绝大部分,可称为半老区。在这种地区,经过两年清算斗争,经过执行《五四指示》\footnote[3]{ 即中共中央一九四六年五月四日发出的《关于土地问题的指示》,见本卷《三个月总结》注〔5〕。},群众的觉悟程度和组织程度已经相当提高,土地问题已经初步解决。但群众觉悟程度和组织程度尚不是很高,土地问题尚未彻底解决。这种地区,完全适用土地法,普遍地彻底地分配土地,并且应当准备一次分不好再分第二次,还要复查一、二次。这种地区,中农占少数,并且是观望的。贫农占大多数,积极要求土地。因此,必须组织贫农团,必须确定贫农团在农会中、在农村政权中的领导地位。\\
  三、大反攻后新解放的地区。这种地区,群众尚未发动,国民党和地主、富农的势力还很大,我们一切尚无基础。因此,不应当企图一下实行土地法,而应当分两个阶段实行土地法。第一阶段,中立富农,专门打击地主。在这个阶段中,又要分为宣传,做初步组织工作,分大地主浮财\footnote[4]{ 浮财,指粮食、金钱、衣服、什物等动产。},分大、中地主土地和照顾小地主等项步骤,然后进到分配地主阶级的土地。在这个阶段中,应当组织贫农团,作为领导骨干,还可组织以贫农为主体的农会(可称为农民协会)。第二阶段,将富农出租和多余的土地及其一部分财产拿来分配,并对前一阶段中分配地主土地尚不彻底的部分进行分配。第一阶段,大约须有两年时间;第二阶段,须有一年时间。太急了,必办不好。老区和半老区的土地改革和整党,也须有三年时间(从今年一月算起),太急了,也办不好。\\
\newpage\section*{\myformat{纠正土地改革宣传中的“左”倾错误}\\\myformat{(一九四八年二月十一日)}}\addcontentsline{toc}{section}{纠正土地改革宣传中的“左”倾错误}
\begin{introduction}\item  这是毛泽东为中共中央起草的对党内的指示。\end{introduction}
最近几个月中,许多地方的通讯社和报纸,不加选择地没有分析地传播了许多包含“左”倾错误偏向的不健全的通讯或文章。例如:\\
  一、不是宣传依靠贫雇农,巩固地联合中农,消灭封建制度的路线,而是孤立地宣传贫雇农路线。不是宣传无产阶级联合一切劳动人民、受压迫的民族资产阶级、知识分子和其它爱国分子(其中包括不反对土地改革的开明绅士),推翻帝国主义、封建主义和官僚资本主义的统治,建立中华人民共和国和人民民主政府,而是孤立地宣传所谓贫雇农打江山坐江山,或者说民主政府只是农民的政府,或者说民主政府只应该听工人和贫雇农的意见,而对中农,对独立劳动者,对民族资产阶级,对知识分子等,则一概不提。这是严重的原则性的错误。而我们的通讯社、报纸或广播电台竟将这类通讯发表。各地党委宣传部,对于此类错误竟没有任何的反映。此类宣传,在过去几个月中虽然不是普遍的,但是相当多,以致造成了一种空气,使人们误认为似乎这是正确的领导思想。甚至因为陕北广播电台播发了某些不正确的新闻,人们竟误认为这是被中央认可的意见。\\
  二、在整党问题上,关于既反对忽视成分、又反对唯成分论的宣传,有些地区不够有力,甚至有唯成分论的错误宣传。\\
  三、在土地改革问题上,关于既反对观望不前、又反对急性病的宣传,有些地区是抓紧了;但在许多地区却助长急性病,甚至发表赞扬急性病的东西。在领导者和群众的关系问题上,关于既反对命令主义、又反对尾巴主义的宣传,有些地区是注意了;但在许多地区却错误地强调所谓“群众要怎样办就怎样办”,迁就群众中的错误意见。甚至对于并非群众的、而只是少数人的错误意见,也无批判地接受。否定了党的领导作用,助长了尾巴主义。\\
  四、在工商业和工人运动的方针上,对于某些解放区存在着的严重的“左”的倾向,或者加以赞扬,或者熟视无睹。\\
  总之,过去几个月的宣传工作,正确地反映和指导了战争、土地改革、整党、生产、支援前线这些伟大斗争,帮助了这些斗争取得了伟大成绩,并且在宣传工作中占着主要成分,这是必须首先承认的。但是也必须看到一些错误缺点。其特点就是过左。其中有些是完全违背马克思列宁主义原则立场和完全脱离中央路线的。望各中央局、中央分局及其宣传部,新华总社和各地总分社,以及各地报纸的工作同志们,根据马克思列宁主义原则和中央路线,对过去几个月的宣传工作,加以检查,发扬成绩,纠正错误,务使对于战争、土地改革、整党、工人运动这些伟大的斗争,对于这一整个反帝反封建的革命,保障其获得胜利。这种检查,责成各地党委宣传部和新华总社负主要责任,并将结果用自己的名义写一个政策性的报告给中央宣传部。\\
\newpage\section*{\myformat{新解放区土地改革要点}\\\myformat{(一九四八年二月十五日)}}\addcontentsline{toc}{section}{新解放区土地改革要点}
\begin{introduction}\item  这是毛泽东为中共中央起草的对党内的指示。\end{introduction}
一、不要性急,应依环境、群众觉悟程度和领导干部强弱决定土地改革工作进行的速度。不要企图在几个月内完成土地改革,而应准备在两三年内完成全区的土地改革。这点在老区和半老区亦是如此。\\
  二、新区土地改革应分两个阶段。第一阶段,打击地主,中立富农。又要分几个步骤:首先打击大地主,然后打击其它地主。对于恶霸和非恶霸,对于大、中、小地主,在待遇上要有区别。第二阶段,平分土地,包括富农出租和多余的土地在内。但在待遇上,对待富农应同对待地主有所区别。总的打击面,一般不能超过户数百分之八,人口百分之十。在区别待遇和总的打击面上,半老区亦是如此。老区一般只是填平补齐\footnote[1]{ 见本卷《迎接中国革命的新高潮》注〔10〕。}工作,不发生此项问题。\\
  三、先组织贫农团,几个月后,再组织农民协会。严禁地主富农分子混入农民协会和贫农团。贫农团积极分子应作为农民协会的领导骨干,但必须吸引一部分中农积极分子参加农民协会的委员会。在土地改革斗争中,必须吸引中农参加,并照顾中农利益。\\
  四、不要全面动手,而应选择强的干部在若干地点先做,取得经验,逐步推广,波浪式地向前发展。在整个战略区是如此,在一个县内也是如此。这在老区、半老区都应如此。\\
  五、分别巩固区和游击区。在巩固区逐步进行土地改革。在游击区只作宣传工作和荫蔽的组织工作,分发若干浮财。不要公开成立群众团体,不要进行土地改革,以防敌人摧残群众。\\
  六、反动的地主武装组织和特务组织,必须消灭,不能利用。\\
  七、反动分子必须镇压,但是必须严禁乱杀,杀人愈少愈好。死刑案件应由县一级组织委员会审查批准。政治嫌疑案件的审判处理权,属于区党委一级的委员会。此点老区半老区都适用。\\
  八、应当利用地主富农家庭出身但是赞成土地改革的本地的革命的知识分子和半知识分子,参加建立根据地的工作。但要加紧对于他们的教育,防止他们把持权力,妨碍土地改革。一般不宜要他们在本区本乡办事。着重任用农民家庭出身的知识分子和半知识分子。\\
  九、严格注意保护工商业。从长期观点筹划经济和财政。军队和区乡政府都要防止浪费。\\
\newpage\section*{\myformat{关于工商业政策}\\\myformat{(一九四八年二月二十七日)}}\addcontentsline{toc}{section}{关于工商业政策}
\begin{introduction}\item  这是毛泽东为中共中央起草的对党内的指示。\end{introduction}
一、某些地方的党组织违反党中央的工商业政策,造成严重破坏工商业的现象。对于这种错误,必须迅速加以纠正。这些地方的党委,在纠正这种错误的时候,必须从领导方针和领导方法两方面认真地进行检查。\\
  二、在领导方针上。应当预先防止将农村中斗争地主富农、消灭封建势力的办法错误地应用于城市,将消灭地主富农的封建剥削和保护地主富农经营的工商业严格地加以区别,将发展生产、繁荣经济、公私兼顾、劳资两利的正确方针同片面的、狭隘的、实际上破坏工商业的、损害人民革命事业的所谓拥护工人福利的救济方针严格地加以区别。应当向工会同志和工人群众进行教育,使他们懂得,决不可只看到眼前的片面的福利而忘记了工人阶级的远大利益。应当引导工人和资本家在当地政府领导下,共同组织生产管理委员会,尽一切努力降低成本,增加生产,便利推销,达到公私兼顾、劳资两利、支持战争的目的。许多地方所犯的错误就是由于全部、大部或一部没有掌握上述方针而发生的。各中央局、分局应当明确提出此一问题,加以分析检查,定出正确方针,并分别发布党内指示和政府法令。\\
  三、在领导方法上。方针决定了,指示发出了,中央局、分局必须同区党委、地委或自己派出的工作团,以电报、电话、车骑通讯、口头谈话等方法密切联系,并且利用报纸做为自己组织和领导工作的极为重要的工具。必须随时掌握工作进程,交流经验,纠正错误,不要等数月、半年以至一年后,才开总结会,算总账,总的纠正。这样损失太大,而随时纠正,损失较少。在通常情况下,各中央局和下面的联系必须力求密切,经常注意明确划清许做和不许做的事情的界限,随时提醒下面,使之少犯错误。这都是领导方法问题。\\
  四、全党同志须知,现在敌人已经彻底孤立了,但是敌人的孤立并不就等于我们的胜利。我们如果在政策上犯了错误,还是不能取得胜利。具体说来,在战争、整党、土地改革、工商业和镇压反革命五个政策问题中,任何一个问题犯了原则的错误,不加改正,我们就会失败。政策是革命政党一切实际行动的出发点,并且表现于行动的过程和归宿。一个革命政党的任何行动都是实行政策。不是实行正确的政策,就是实行错误的政策;不是自觉地,就是盲目地实行某种政策。所谓经验,就是实行政策的过程和归宿。政策必须在人民实践中,也就是经验中,才能证明其正确与否,才能确定其正确和错误的程度。但是,人们的实践,特别是革命政党和革命群众的实践,没有不同这种或那种政策相联系的。因此,在每一行动之前,必须向党员和群众讲明我们按情况规定的政策。否则,党员和群众就会脱离我们政策的领导而盲目行动,执行错误的政策。\\
\newpage\section*{\myformat{关于民族资产阶级和开明绅士问题}\\\myformat{(一九四八年三月一日)}}\addcontentsline{toc}{section}{关于民族资产阶级和开明绅士问题}
\begin{introduction}\item  这是毛泽东为中共中央起草的对党内的指示。\end{introduction}
中国现阶段革命的性质,是无产阶级领导的、人民大众的、反对帝国主义、反对封建主义和反对官僚资本主义的革命。所谓人民大众,是指一切被帝国主义、封建主义、官僚资本主义所压迫、损害或限制的人们,也即是一九四七年十月中国人民解放军宣言上明确地指出的工、农、兵、学、商和其它一切爱国人士\footnote[1]{ 见本卷《中国人民解放军宣言》中八项政策的第一项。}。在宣言上所说的“学”,即是指一切受迫害、受限制的知识分子。所说的“商”,即是指一切受迫害、受限制的民族资产阶级,即中小资产阶级。所说的“其它爱国人士”,则主要地是指的开明绅士。现阶段的中国革命,即是由这些人们团结起来,组成反帝、反封建、反官僚资本主义的统一战线,而又以劳动人民为主体的革命。所谓劳动人民,是指一切体力劳动者(如工人、农民、手工业者等)以及和体力劳动者相近的、不剥削人而又受人剥削的脑力劳动者。中国现阶段革命的目的,是在推翻帝国主义、封建主义、官僚资本主义的统治,建立一个以劳动者为主体的、人民大众的新民主主义共和国,不是一般地消灭资本主义。\\
  我们不要抛弃那些过去和我们合作过、现在也还同我们合作、赞成反美蒋和土地改革的开明绅士。例如晋绥边区的刘少白、陕甘宁边区的李鼎铭\footnote[2]{ 刘少白,一九四二年十一月被选为晋绥边区临时参议会副议长。李鼎铭,一九四一年十一月被选为陕甘宁边区政府副主席。}等人,在抗日战争和抗日战争以后的困难时期内,曾经给我们以相当的帮助,而在我们实行土地改革的时候,他们又并不妨碍和反对土地改革,因此对他们仍应采取团结的政策。但是团结他们,并不是说将他们当作决定中国革命性质的力量来看。决定革命性质的力量,是主要的敌人和主要的革命者两方面。我们今天的主要敌人是帝国主义、封建主义和官僚资本主义,我们今天同敌人作斗争的主要力量是占全国人口百分之九十的一切从事体力劳动和脑力劳动的人民。这就决定了我们现阶段革命的性质是新民主主义的人民民主革命,而不同于十月革命那样的社会主义革命。\\
  依附帝国主义、封建主义、官僚资本主义,反对人民民主革命的民族资产阶级的少数右翼分子,他们也是革命的敌人;依附劳动人民反对反动派的民族资产阶级左翼分子以及从封建阶级分裂出来的少数开明绅士,他们也是革命者。但是这两者都不是敌人或革命者的主体,两者都不是可以决定革命性质的力量。民族资产阶级是一个在政治上非常软弱的和动摇的阶级。但是他们中间的大多数,由于也受着帝国主义、封建主义、官僚资本主义的迫害和限制,他们又可以参加人民民主革命,或者对革命守中立。他们是人民大众的一部分,但不是人民大众的主体,也不是决定革命性质的力量。但是因为他们在经济上具有重要性,又因为他们可以参加反对美蒋,或者在反对美蒋的斗争中采取中立的态度,因之我们便有可能和必要去团结他们。在中国共产党未产生以前,以孙中山为领导的国民党,曾经代表民族资产阶级,充当过当时中国革命(不彻底的旧民主主义革命)的领导者。但是,中国共产党一经产生,并且表现出自己的能力以后,他们就已经不能是中国革命(新民主主义革命)的领导者了。这个阶级曾经参加了一九二四年到一九二七年的革命运动,而在一九二七年到一九三一年(九一八事变以前),他们中间的不少分子,曾经附和了蒋介石的反动。但是,决不能因为这一点,就认为那个时期我们在政治上不应该争取他们,在经济上不应该保护他们;就认为我们在那个时期内对民族资产阶级的过左的政策不是冒险主义的政策。相反地,那时我们的政策,仍然应当是保护和争取他们,以便我们能够集中力量去反对主要敌人。在抗日时期,民族资产阶级是动摇于国民党和共产党两党之间的抗日的参加者。在现阶段,民族资产阶级的多数是增长了对美蒋的仇恨,他们中间的左翼分子依附于共产党,右翼分子则依附于国民党,其中间派则在国共两党之间采取犹豫和观望的态度。这种情况,使得我们有必要和可能争取其大多数,孤立其少数。为了达到这一目的,对这个阶级的经济地位必须慎重地加以处理,必须在原则上采取一律保护的政策。否则,我们便要在政治上犯错误。\\
  开明绅士是地主和富农阶级中带有民主色彩的个别人士。这些人士,同官僚资本主义和帝国主义有矛盾,同封建的地主、富农也有某种矛盾。我们团结他们,并不是因为他们在政治上有什么大的力量,也不是因为他们在经济上有什么重要性(他们根据封建制度占有的土地,应当在取得他们同意之后交给农民分配),而是因为他们在抗日战争时期,在反美蒋斗争时期,在政治上曾经给我们以相当的帮助。在土地改革时期,如果有少数开明绅士表示赞成我们的土地改革,对于全国土地改革的工作也是有益的。特别是对于争取全国的知识分子(中国的知识分子大部分是地主富农的家庭出身),对于争取全国的民族资产阶级(中国的民族资产阶级大部分同土地有联系),对于争取全国的开明绅士(大约有几十万人),以及对于孤立中国革命的主要敌人蒋介石反动派,都是有益的。正因为开明绅士有这些作用,他们也是反帝反封建反官僚资本主义革命统一战线中的一分子,所以,团结他们也是一个必须注意的问题。我们对于开明绅士的要求,在抗日时期是赞成抗日,赞成民主(不反共),赞成减租减息;在现阶段是赞成反美、反蒋,赞成民主(不反共),赞成土地改革。只要他们能够这样做,我们就应该毫无例外地去团结他们,并且在团结中教育他们。\\
\newpage\section*{\myformat{评西北大捷兼论解放军的新式整军运动}\\\myformat{(一九四八年三月七日)}}\addcontentsline{toc}{section}{评西北大捷兼论解放军的新式整军运动}
\begin{introduction}\item  这是毛泽东为中国人民解放军总部发言人起草的评论。这时西北战场国民党军的进攻已被粉碎,人民解放军已经转入进攻。这篇评论分析了西北战场的形势,也扼要地介绍了全国其它战场的概况。这篇评论的更重要方面,是在着重地说明了用“诉苦”和“三查”方法进行的新式整军运动的伟大意义。这个新式整军运动是人民解放军政治工作和民主运动的一个重要发展,是当时全解放区轰轰烈烈的土地改革运动和整党运动在军队中的反映。这个运动大大提高了全军官兵的政治觉悟、纪律性和战斗力,同时也极其有效地加速了把大批被俘国民党军队士兵改造为解放军战士的过程,对于人民解放军的巩固扩大和作战胜利起了重大的作用。关于这个新式整军运动的意义,参看本卷《军队内部的民主运动》、《在晋绥干部会议上的讲话》、《中共中央关于九月会议的通知》等文。\end{introduction}
人民解放军总部发言人评西北人民解放军最近一次大捷称:这次胜利改变了西北的形势,并将影响中原的形势。这次胜利,证明人民解放军用诉苦和三查方法进行了新式整军运动,将使自己无敌于天下。\\
  发言人说:这次西北人民解放军突然包围宜川敌军一个旅,胡宗南令其二十九军军长刘戡,率领两个整编师的四个旅,即整编二十七师之三十一旅、四十七旅,整编九十师之五十三旅、六十一旅共约二万四千余人,由洛川、宜君一线向宜川驰援,于二十八日到达宜川西南地区。西北人民解放军发起歼灭战,经过二十九日至三月一日三十小时的战斗,即将该部援军全部歼灭,无一漏网。计生俘一万八千余人,毙伤五千余人,刘戡本人和九十师师长严明等人,亦被击毙。接着于三日攻克宜川,又歼守敌整编七十六师的二十四旅五千余人。此役共歼敌一个军部、两个师部、五个旅,共三万人。在西北战场上,这是第一个大胜仗。\\
  发言人分析西北战场的形势说:胡宗南直接指挥的所谓“中央军”二十八个旅中,有八个旅属于三个主力师,即整编第一师、整编三十六师和整编九十师,其中整编第一师之第一旅,前年九月在晋南浮山被我歼灭一次,其一六七旅主力,去年五月在陕北蟠龙镇被我歼灭一次,整编三十六师之一二三旅、一六五旅,于去年八月在陕北米脂沙家店被我歼灭一次,这次整编九十师又被全歼,剩下的胡军主力,就只有整编第一师的七十八旅和整编三十六师的二十八旅,还没有受到过歼灭。因此,整个胡宗南军队,可以说已经没有什么精锐骨干了。经过此次宜川歼灭战,胡宗南过去直接指挥的正规兵力二十八个旅,现在只剩下二十三个旅,这二十三个旅分布在下列地区:晋南临汾一个旅,已成死棋;陕豫边境和洛阳、潼关线有九个旅,对付我陈、谢野战军;陕南有一个旅,任汉中一带守备。此外,分布在潼关到宝鸡、咸阳到延安“丁”字形交通线上的有十二个旅。其中三个是“后调旅”\footnote[1]{ “后调旅”,指国民党军队中那些在战场上被人民解放军大部歼灭后,其残余部分调到后方补充而不改变番号的旅。},全系新兵;被我军全歼新近补充起来的有两个旅;曾被我军给以歼灭性打击的有两个旅;受我军打击较少的五个旅。可以想见,这些部队不但很弱,而且极大部分分任守备。胡军以外还有邓宝珊两个旅防守榆林;宁夏马鸿逵和青海马步芳的九个旅分布在三边和陇东。以上胡、邓、马各部,全部正规军包括过去被歼一次至两次但又补充起来的部队在内,目前总共三十四个旅。\\
  以上是就西北敌军态势而言。再说所谓“丁”字形交通线上受我军打击较少的五个旅,其中两个旅困守延安,三个旅在大关中;其它多数是新补充的部队,少数是受过歼灭性打击的部队。这就是说,整个大关中特别是甘肃方面,敌军异常空虚,无法阻止人民解放军的攻势。这种形势,势必牵动蒋军在南线的一部分部署,首先是牵动其豫陕边境对付我陈、谢野战军的部署。我西北人民解放军在此次向南进攻中,旗开得胜,声威大震,改变了西北敌我对比的形势,今后将比过去更有效力地同南线各战场的人民解放军配合作战。\\
  发言人说:我刘邓、陈粟、陈谢三路野战大军,从去年夏秋起渡河南进,纵横驰骋于江淮河汉之间,歼灭大量敌人,调动和吸引蒋军南线全部兵力一百六十多个旅中约九十个旅左右于自己的周围,迫使蒋军处于被动地位,起了决定性的战略作用,获得全国人民的称赞\footnote[2]{ 晋冀鲁豫野战军司令员刘伯承、政治委员邓小平指挥的六个纵队,于一九四七年六月三十日起强渡黄河,向大别山进军,从而揭开了人民解放军战略进攻的序幕,先后建立了鄂豫、皖西、桐柏、江汉等根据地。华东野战军司令员兼政治委员陈毅、副司令员粟裕指挥的主力部队,在打破了国民党军对山东重点进攻后,于一九四七年八月初挺进鲁西南,九月下旬进军豫皖苏边区,发展了豫皖苏解放区。晋冀鲁豫野战军第四纵队司令员陈赓、政治委员谢富治指挥的两个纵队和一个军,一九四七年八月下旬经晋南强渡黄河,挺进豫西,建立了豫陕鄂、陕南等根据地。截至一九四八年三月底,转战中原的三路大军共歼敌二十余万人,胜利地完成了开创中原新解放区的战略任务。}。我东北野战军在冬季攻势中,冒零下三十度的严寒,歼灭大部敌人,迭克名城,威震全国\footnote[3]{ 东北野战军司令员兼政治委员林彪、副政治委员罗荣桓指挥野战军主力,于一九四七年十二月十五日至一九四八年三月十五日,在四平街至大石桥的中长路沿线和山海关至沈阳的北宁路沿线,发动了空前规模的冬季攻势,连续作战九十天,歼灭国民党军十五万六千余人(其中营口守敌一个师起义),攻克敌军坚固设防的战略要点四平街和其它城市共十八座及重镇多处。吉林守敌弃城逃往长春。这样,就使敌军在东北的占领区,缩小到只占东北面积的百分之二左右,东北敌军占据的长春、沈阳、锦州等城市陷于孤立。}。我晋察冀、山东、苏北和晋冀鲁豫各路野战军,都在去年英勇作战大量歼敌\footnote[4]{ 晋察冀军区司令员聂荣臻所领导的晋察冀野战军三个纵队和军区部队,于一九四七年九月初至十一月中旬,先后举行了大清河北、清风店和石家庄等战役,共歼敌四万六千余人,使晋察冀和晋冀鲁豫两解放区连成一片。华东野战军所属四个纵队和地方武装在一九四七年九月至十二月间,在内线兵团司令员许世友、华东野战军副政治委员兼内线兵团政治委员谭震林等指挥下,进行了胶东战役,歼敌六万三千余人,收复了十余座县城,改变了整个山东战场的局面。在苏北,华东野战军一部,在一九四七年八月至十二月间先后进行了盐(城)东(台)、李(堡)栟(茶)、盐(城)南等战役,共歼敌两万四千余人,收复了苏北广大地区。晋冀鲁豫军区第一副司令员徐向前指挥晋冀鲁豫野战军一部,于一九四七年十二月协同西北野战军一部攻克运城,歼敌一万三千余人,使临汾守敌陷于孤立。}之后,完成了冬季整训,不日又将展开春季攻势作战\footnote[5]{ 一九四八年春季,人民解放军各路野战军,在冬季整训之后,相继发起了春季攻势。晋察冀野战军以及晋冀鲁豫野战军和晋绥军区各一部,在三月至五月间,先后进行了察南绥东战役、临汾战役,共歼敌四万三千余人,收复了广大地区。中原野战军、中原军区和华东野战军各一部,在三月八日至六月三日,先后发起了洛阳战役、宋河战役、宛西战役、宛东战役,共歼敌五万六千余人,粉碎了国民党军中原防御体系,发展和巩固了中原解放区。华东野战军的山东兵团于三月十一日至五月八日,先后进行了胶济路西段战役和胶济路中段战役,共歼敌八万四千余人。至此,山东省除济南、青岛、烟台、临沂和津浦路济南至徐州段沿线的部分城镇外,全部解放。苏北兵团在三月间胜利地进行了益林战役,歼敌七千余人。}。总观全局,说明了一个真理,就是只要坚决反对保守主义,反对惧怕敌人,反对惧怕困难,依照党中央的战略总方针及其十大军事原则的指示\footnote[6]{ 十大军事原则,见本卷《目前形势和我们的任务》第三节。},我们就能展开进攻,大量歼灭敌人;打得蒋介石匪帮,或者只有暂时招架之功,并无还手之力;或者连招架都没有,只有被我一个一个地歼灭干净。\\
  发言人着重指出:西北野战军的战斗力,比之去年是空前地提高了\footnote[7]{ 西北战场的人民解放军,是由彭德怀、贺龙、习仲勋等领导的陕甘宁解放区和晋绥解放区的人民解放军所组成的。一九四七年春,彭德怀、习仲勋领导的参加陕北作战的西北野战兵团,为两个纵队又两个旅,共二万七千余人。同年七月三十一日,中央军委决定西北野战兵团正式命名为西北野战军。到一九四八年春,参加陕北作战的主力部队增加到五个纵队,七万五千余人,经过一年作战的锻炼和一九四七年冬至一九四八年初进行的新式整军运动以后,广大官兵的政治觉悟和部队的战斗力,也空前提高。这样,就为一九四八年三月西北野战军转入外线作战,创造了必要的条件。继宜川大捷之后,西北野战军于四月中旬发起西府(西安以西泾渭两河之间地区)、陇东战役,挺进泾水渭水间的广大地区,截断西兰公路,并且在四月二十一日收复延安。}。西北野战军在去年作战中,还只能一次最多歼灭敌人两个旅,此次宜川战役,则已能一次歼灭敌人五个旅。此次胜利如此显着,原因甚多,前线领导同志们的坚决的、灵活的指挥,后方领导同志们和广大人民的努力协助,以及敌军比较孤立,地形有利于我等项,都是应当指出的。但是最值得注意的,是在冬季两个多月中用诉苦和三查方法进行了新式的整军运动。由于诉苦(诉旧社会和反动派所给予劳动人民之苦)和三查(查阶级、查工作、查斗志)运动的正确进行,大大提高了全军指战员为解放被剥削的劳动大众,为全国的土地改革,为消灭人民公敌蒋介石匪帮而战的觉悟性;同时就大大加强了全体指战员在共产党领导之下的坚强的团结。在这个基础上,部队的纯洁性提高了,纪律整顿了,群众性的练兵运动开展了,完全有领导地有秩序地在部队中进行的政治、经济、军事三方面的民主发扬了。这样就使部队万众一心,大家想办法,大家出力量,不怕牺牲,克服物质条件的困难,群威群胆,英勇杀敌。这样的军队,将是无敌于天下的。\\
  发言人说:这种新式的整军运动,不但在西北方面实行了,在全国人民解放军中都已实行,或者正在实行着。这种整军运动,是在作战的间隙中进行的,并不妨碍作战。这种整军运动,同我党正确地进行着的整党运动、土地改革运动相结合,同我党缩小打击面,只反对帝国主义、封建主义和官僚资本主义,严禁乱打乱杀(杀人愈少愈好),坚决团结全国百分之九十以上人民大众的正确方针相结合,同我党实行正确的城市政策,坚决地保护和发展民族工商业的方针相结合,这样就必然会使人民解放军的威力无敌于天下。任凭蒋介石匪帮及其主子美国帝国主义在中国人民民主革命的伟大斗争面前如何拚命挣扎,胜利总是属于我们的。\\
\newpage\section*{\myformat{关于情况的通报}\\\myformat{(一九四八年三月二十日)}}\addcontentsline{toc}{section}{关于情况的通报}
\begin{introduction}\item  这是毛泽东为中共中央写的对党内的通报。在这以后,中共中央就离开陕甘宁边区,经晋绥解放区进入晋察冀解放区,在一九四八年五月到达河北省西部平山县的西柏坡村。\end{introduction}
一、最近几个月,中央集中全力解决在新形势下面关于土地改革方面、关于工商业方面、关于统一战线方面、关于整党方面、关于新区工作方面的各项具体的政策和策略的问题,反对党内右的和“左”的偏向,而主要是“左”的偏向。我们党的历史情况表明,在我党和国民党结成统一战线时期,党内容易发生右的偏向,而在我党和国民党分裂时期,党内容易发生“左”的偏向。现在的“左”的偏向,主要的是侵犯中农,侵犯民族资产阶级,职工运动中片面强调工人眼前福利,对待地主和对待富农没有区别,对待地主的大中小、恶霸非恶霸没有区别,不按平分原则给地主留下必要的生活出路,在镇压反革命斗争中越出了某些政策界限,以及不要代表民族资产阶级的党派,不要开明绅士,在新解放区忽视缩小打击面(即忽视中立富农和小地主)在策略上的重要性,工作步骤上的急性病等。这些“左”的偏向,在过去大约两年的时间内,各解放区都或多或少地发生过,有时成了严重的冒险主义倾向。好在纠正这类偏向并不甚困难,几个月内已经大体上纠正过来了,或者正在纠正着。但须各级领导者着重用力才能彻底纠正此类偏向。右的偏向主要是过高地估计敌人的力量,惧怕美国大量援蒋,对长期战争有些厌倦,对国际民主力量的强大的程度有些怀疑,不敢放手发动群众消灭封建制度,对党内成分不纯和作风不纯熟视无睹等。但这类偏向现在不是主要的,改正亦不困难。最近几个月,我党在战争、土地改革、整党整军、发展新区和争取民主党派等方面均有成绩,在这些工作中所发生的偏向有了着重的纠正,或正在纠正中,这样就可以使整个中国革命运动走上健全发展的轨道。只有党的政策和策略全部走上正轨,中国革命才有胜利的可能。政策和策略是党的生命,各级领导同志务必充分注意,万万不可粗心大意。\\
  二、由于对美国和蒋介石存着某种幻想,对我党和人民具有足以战胜一切内外敌人的力量表示怀疑,并因此认为所谓第三条道路\footnote[1]{ 见本卷《目前形势和我们的任务》注〔12〕。}尚有存在可能、将自己处于国共两党之间的中间地位的某些民主人士,在国民党的突然的攻势之下,使自己处于被动地位,最后终于在一九四八年一月间采用我党的口号,声明反蒋反美,联共联苏\footnote[2]{ 一九四七年十月,国民党反动政府宣布中国民主同盟为“非法团体”,强制民盟总部宣布解散。当时,其它民主党派也都遭到国民党反动派的迫害,不能在国民党统治区公开活动。一九四八年一月一日,李济深等在香港成立中国国民党革命委员会;一月五日,沈钧儒等在香港召开中国民主同盟一届三中全会,宣布恢复民盟总部,继续进行政治斗争。这两个党派当时都接受了中国共产党关于时局的主张,发表宣言,主张联合中共和其它民主党派,推翻蒋介石独裁政权,反对美国武装干涉中国内政。}。对于这些人,我们应当对他们采取团结的政策,对他们的某些错误观点则作适当的批评。在将来成立中央人民政府时,邀请他们一部分人参加政府工作是必要的和有益的。这些人的特点是从来不愿意接近劳动群众,又习惯于大城市的生活,不愿轻易到解放区来。虽然如此,他们所代表的社会基础,即民族资产阶级,却有其重要性,不可忽视。因此,应当争取他们。估计要待我们有更大的胜利,夺取几个例如沈阳、北平、天津那样的城市,共产党胜、国民党败的形势业已完全判明以后,邀请他们参加中央人民政府,他们可能愿意来解放区和我们共事。\\
  三、本年内,我们不准备成立中央人民政府,因为时机还未成熟。在本年蒋介石的伪国大开会选举蒋介石当了总统\footnote[3]{ 一九四八年三月二十九日至五月一日,国民党反动派在南京召集“国民大会”,选举蒋介石为“总统”,李宗仁为“副总统”。},他的威信更加破产之后,在我们取得更大胜利,扩大更多地方,并且最好在取得一二个头等大城市之后,在东北、华北、山东、苏北、河南、湖北、安徽等区连成一片之后,便有完全的必要成立中央人民政府。其时机大约在一九四九年。目前我们正将晋察冀区、晋冀鲁豫区和山东的渤海区统一在一个党委(华北局)、一个政府、一个军事机构的指挥之下(渤海区也许迟一点合并),这三区包括陇海路以北、津浦路和渤海以西、同蒲路以东、平绥路以南的广大地区\footnote[4]{ 一九四八年五月,晋察冀解放区和晋冀鲁豫解放区实行合并,改称华北解放区,同时成立了中共华北中央局、华北军区。晋察冀边区行政委员会和晋冀鲁豫边区政府也联合办公。同年八月,晋察冀和晋冀鲁豫两边区政府正式合并,成立华北人民政府。山东的渤海区仍属中共华东中央局管辖。}。这三区业已连成一片,共有人口五千万,大约短期内即可完成合并任务。这样做,可以有力地支持南线作战,可以抽出许多干部输往新解放区。该区的领导中心设在石家庄。中央亦准备移至华北,同中央工作委员会\footnote[5]{ 见本卷《中共中央关于暂时放弃延安和保卫陕甘宁边区的两个文件》注〔4〕。}合并。\\
  四、我南线各军,即山东兵团九个旅,苏北兵团七个旅,河淮间兵团二十一个旅,豫鄂陕兵团十个旅,江淮汉水间兵团十九个旅,西北兵团十二个旅,晋南豫北兵团十二个旅,除江淮汉水间刘邓兵团的主力因白崇禧集中兵力向大别山进攻\footnote[6]{ 白崇禧开始进攻大别山地区的时间是一九四七年十二月,进攻的兵力共三十三个旅。},未获休整,到二月底才抽出一部到淮河以北休整外,其余各兵团均在十二月至二月间作了休整。这是过去二十个月作战中的第一次大休整。这次休整,采取群众诉苦(诉旧社会和反动派所给予劳动人民之苦)、三查(查阶级成分,查工作,查斗志)和群众性练兵(官教兵,兵教官,兵教兵)的方法,发动了全军指挥员战斗员的高度的革命积极性,教好了或清除了一部分军队中的地主、富农分子或坏分子,提高了纪律,讲明了土地改革中的各项政策、对待工商业和知识分子的政策,发扬了军队中的民主作风,提高了军事技术和战术。这样就使得我军极大地增长了战斗力。现在除刘邓兵团的一部尚在休整外,各兵团均已于二月底三月初先后开始新的作战行动,并在两星期内歼敌九个旅。北线各军,即东北兵团四十六个旅、晋察冀兵团十八个旅、晋绥兵团两个旅,在冬季则大部作战,一部休整。东北兵团,利用辽河结冰,举行了三个月作战,歼敌八个旅,争取敌一个旅起义,攻占彰武、法库、新立屯、辽阳、鞍山、营口和四平街,并收复吉林。该兵团现已开始休整。俟休整完毕,或打长春,或打北宁路上之敌。晋察冀兵团休整一个多月,现已向平绥线行动。晋绥兵团数量较小,其主要任务是对阎锡山起钳制作用。总计我军现有南北两线大小十个兵团,正规兵力已达五十个纵队(等于国民党的整编师),一百五十六个旅(等于国民党的整编旅),一百三十二万二千余人,平均每旅(三个团)人数八千左右。此外,尚有非正规军,包括地方兵团、部队、游击队、后方军事机关、军事学校等在内,一百一十六万八千余人(其中作战部队占八十万人),全军总计为二百四十九万一千余人。而在一九四六年七月以前,我们只有正规军二十八个纵队,一百一十八个旅,六十一万二千余人,平均每旅(三个团)人数不足五千;加上非正规军六十六万五千余人,总计一百二十七万八千余人。可以看出,我们的军队现在是壮大了。旅的数目增加不多,每旅的人数却大为增加。经过二十个月作战,战斗力亦大为增加。\\
  五、国民党的正规军,从一九四六年七月至去年夏季,是九十三个师,二百四十八个旅,现在则有一百零四个师,二百七十九个旅的番号。其分布是:北线二十九个师,九十三个旅(沈阳卫立煌十三个师,四十五个旅;北平傅作义十一个师,三十三个旅;太原阎锡山五个师,十五个旅),约五十五万人。南线六十六个师,一百五十八个旅(郑州顾祝同三十八个师,八十六个旅;九江白崇禧十四个师,三十三个旅;西安胡宗南十四个师,三十九个旅),约一百零六万人。第二线九个师,二十八个旅(西北区,包括兰州以西地区,四个师,八个旅;西南区,包括川、康、滇、黔,四个师,十个旅;东南区,包括长江以南诸省,八个旅;台湾,一个师,两个旅),约十九万六千人。国民党正规军番号增加的原因,是因为国民党军大量被我军歼灭,并由战略攻势转入战略守势之后,甚感兵力不足,因此将大量地方部队和伪军升级或编组为正规军,计北线卫立煌系统增加三个师,十四个旅;傅作义系统增加两个师,六个旅;南线顾祝同系统,增加六个师,九个旅;胡宗南系统,增加两个旅;共计增加十一个师,三十一个旅。因此,国民党军现在不是九十三个师,而是一百零四个师,不是二百四十八个旅,而是二百七十九个旅。但是第一,最近几个月(至三月二十日为止)被我歼灭的六个师,二十九个旅,只有空番号,尚未来得及重建或补充,也许有一部分永远无法重建或补充了,因此,国民党军在实际上现在只有九十八个师,二百五十个旅,比之去年夏季以前只多了五个师的番号和二个实际的旅。第二,现在实有的二百五十个旅中,只有一百十八个旅未受过我军歼灭性的打击,其余一百三十二个旅,或者被我军歼灭过一次、二次,甚至三次,然后补充起来的;或者是受过我军一次、二次,甚至三次歼灭性打击的(以旅为单位,全体被消灭,或大部被消灭者,称为被歼灭;一个团以上被消灭,但其主力未受损失者,称为受歼灭性打击),其士气和战斗力甚为低落。在未受歼灭性打击的一百十八个旅中,有一部分是在第二线训练的新兵,有一部分是从地方部队和伪军升级或编组的,战斗力很弱。第三,国民党军队的数量也减少了。一九四六年七月以前,其正规部队二百万人,非正规部队七十三万八千人,特种部队三十六万七千人,海空部队十九万人,后勤机关、学校一百零一万人,总计四百三十万五千人。而在一九四八年二月,他的正规部队一百八十一万人,非正规部队五十六万人,特种部队二十八万人,海空部队十九万人,后勤机关、学校八十一万人,总计三百六十五万人,即是说,减少了六十五万五千人。一九四六年七月至一九四八年一月的十九个月中,我军共消灭国民党军队一百九十七万七千人(二月和三月上半月尚未统计好,大约有十八万人左右),即是说,国民党不但将其在过去作战期间所动员参军的一百余万新兵消耗了,而且大量消耗了它原有的兵力。在此种形势下,国民党采取和我们相反的方针,不是充实各旅人员的数目,而是减少旅的人员,增加旅的番号。国民党军在一九四六年平均每旅差不多有八千人,而在现在则平均每旅只有六千五百人左右。今后我军占地日广,国民党军兵源粮源日益缩小,估计再打一个整年,即至明年春季的时候,敌我两军在数量上可能达到大体上平衡的程度。我们的方针是稳扎稳打,不求速效,只求平均每个月消灭国民党正规军八个旅左右,每年消灭敌军约一百个旅左右。事实上,从去年秋季以后,超过了这个数目;今后可能有更大的超过。五年左右(一九四六年七月算起)消灭国民党全军的可能性是存在的\footnote[7]{ 五年左右消灭国民党全军,这是当时的预计。后来,这个时间缩短为三年半左右。参见本卷《中国军事形势的重大变化》。}。\\
  六、目前南北两线敌军在两个地区尚有较大的机动兵力,可以举行战役性的进攻,使那里的我军暂时处于困难地位。其一,即大别山,有约十四个机动旅。其二,淮河以北地区,有约十二个机动旅。在这两区,国民党军还有主动权(淮河以北地区,由于我抽出九个主力旅开至黄河以北休整,准备使用于其它方面,故国民党军有了主动权)。其余一切战场的敌军,全是被动挨打。具有对我特别有利形势的战场是东北、山东、西北、苏北、晋察冀、晋冀鲁豫和郑汉路\footnote[8]{ 指平汉路郑州至武汉段。}以西、长江以北、黄河以南的广大地区。\\
\newpage\section*{\myformat{在晋绥干部会议上的讲话}\\\myformat{(一九四八年四月一日)}}\addcontentsline{toc}{section}{在晋绥干部会议上的讲话}
同志们,今天我想讲的,主要地是一些和晋绥工作有关的问题,然后讲到一些和全国工作有关的问题。\\
\subsubsection*{\myformat{一}}
我认为,在过去一年内,在中共中央晋绥分局领导的区域内的土地改革工作和整党工作,是成功的。\\
  这是从两方面来看的。一方面,晋绥的党组织反对了右的偏向,发动了群众斗争,在全区三百多万人口的二百几十万人口中,完成了或者正在完成着土地改革工作和整党工作。另一方面,晋绥的党组织又纠正了在运动中发生的几个“左”的偏向,因而使全部工作走上了健全发展的轨道。从这两方面来看,晋绥解放区的土地改革工作和整党工作,我认为是成功的。\\
  “从此以后,再也不敢封建了,再也不敢厉害了,再也不敢贪污了。”这是晋绥人民的话。这是晋绥人民对于我们的土地改革工作和整党工作所做的结论。他们说“再也不敢封建了”,就是说,我们领导他们发动了斗争,消灭了或者正在消灭着新区的封建剥削制度和老区半老区的封建剥削制度的残余。他们说“再也不敢厉害了,再也不敢贪污了”,就是说,在我们的党和政府的组织内,过去存在着某种程度上的成分不纯或者作风不纯的严重现象,许多坏分子混入了党和政府的组织内,许多人发展了官僚主义的作风,仗势欺人,用强迫命令的方法去完成工作任务,因而引起群众不满,或者犯了贪污罪,或者侵占了群众的利益,这些情况,经过过去一年的土地改革工作和整党工作,已经从根本上改变了。\\
  “过去对于我们是致命的东西,现在去掉了。过去没有的东西,现在有了。”这是在座同志们中有一位同志对我说的。他所说的致命的东西,就是指的存在于党和政府组织内的成分不纯或作风不纯并因而引起群众不满的严重现象。这种现象,现在是根本上去掉了。他所说的过去没有而现在有了的东西,就是指的贫农团、新农会、区村人民代表会议,以及由于土地改革工作和整党工作所造成的农村中面目一新的气象。\\
  这些反映,我以为是合乎实际的。\\
  这就是晋绥解放区的土地改革工作和整党工作的伟大的成功。这是成功的第一个方面。在这个基础上,晋绥的党组织才能够在过去一年内完成巨大的军事勤务,支持伟大的人民解放战争。假使没有成功的土地改革工作和整党工作,要完成这样大的军事任务,那是困难的。\\
  另一方面,晋绥的党组织纠正了在工作中发生的几个“左”的偏向。这主要地是三个偏向。第一,在划分阶级成分中,在许多地方把许多并无封建剥削或者只有轻微剥削的劳动人民错误地划到地主富农的圈子里去,错误地扩大了打击面,忘记了我们在土地改革工作中可能和必须团结农村中户数百分之九十二左右,人数百分之九十左右,即全体农村劳动人民,建立反对封建制度的统一战线这样一个极端重要的战略方针。现在,这项偏向已经纠正了。这样,就大大地安定了人心,巩固了革命统一战线。第二,在土地改革工作中侵犯了属于地主富农所有的工商业;在清查经济反革命的斗争中,超出了应当清查的范围;以及在税收政策中,打击了工商业。这些,都是属于对待工商业方面的“左”的偏向。现在,这些偏向也已纠正,使工商业获得了恢复和发展的可能。第三,在过去一年的激烈的土地改革斗争中,晋绥的党组织没有能够明确地坚持我党严禁乱打乱杀的方针,以致在某些地方的土地改革中不必要地处死了一些地主富农分子,并给农村中的坏分子以乘机报复的可能,由他们罪恶地杀死了若干劳动人民。我们认为,经过人民法庭和民主政府,对于那些积极地并严重地反对人民民主革命和破坏土地改革工作的重要的犯罪分子,即那些罪大恶极的反革命分子和恶霸分子,判处死刑,是完全必要和正当的。不如此,就不能建立民主秩序。但是,对于一切站在国民党方面的普通人员,一般的地主富农分子,或犯罪较轻的分子,则必须禁止乱杀。同时,在人民法庭和民主政府进行对于犯罪分子的审讯工作时,必须禁止使用肉刑。过去一年中,晋绥在这方面曾经发生的偏向,现在也已纠正了。\\
  在认真地纠正了上述一切偏向之后,我们可以有证据地来说,在晋绥中央分局领导下面的全部工作,现在已经走上了健全发展的轨道。\\
  按照实际情况决定工作方针,这是一切共产党员所必须牢牢记住的最基本的工作方法。我们所犯的错误,研究其发生的原因,都是由于我们离开了当时当地的实际情况,主观地决定自己的工作方针。这一点,应当引为全体同志的教训。\\
  关于整理党的基层组织的工作,你们已经根据中央关于在老区半老区进行土地改革工作和整党工作的指示\footnote[1]{ 中共中央的这个指示,是一九四八年二月二十二日发出的。这个指示总结了解放区土地改革和整党工作的经验,规定了老区半老区土地改革和整党工作的一系列的政策和方法,着重地纠正了某些地区在这两项工作中曾经发生的“左”的偏向。},采用晋察冀解放区平山县的整党经验,即是邀集党外群众中的积极分子参加党的支部会议,展开批评和自我批评,借以改变党的组织的成分不纯或者作风不纯的现象,使党和人民群众密切地联系起来。你们这样做,将使你们有可能健全地完成对于党的组织的全部整理工作。\\
  对于那些犯了错误但是还可以教育的、同那些不可救药的分子有区别的党员和干部,不论其出身如何,都应当加以教育,而不是抛弃他们。你们已经执行了或者正在执行着这个方针,这也是对的。\\
  在反对封建制度的斗争中,在贫农团和农会的基础上建立起来的区村(乡)两级人民代表会议,是一项极可宝贵的经验。只有基于真正广大群众的意志建立起来的人民代表会议,才是真正的人民代表会议。这样的人民代表会议,现在已有可能在一切解放区出现。这样的人民代表会议一经建立,就应当成为当地的人民的权力机关,一切应有的权力必须归于代表会议及其选出的政府委员会。到了那时,贫农团和农会就成为它们的助手。我们曾经打算在各地农村中,在其土地改革任务大致完成以后再去建立人民代表会议。现在你们的经验以及其它解放区的经验,既已证明就在土地改革斗争当中建立区村两级人民代表会议及其选出的政府委员会,是可能的和必要的,那末,你们就应当这样做。在一切解放区,也就应当这样做。在区村两级人民代表会议普遍地建立起来的时候,就可以建立县一级的人民代表会议。有了县和县以下的各级人民代表会议,县以上的各级人民代表会议就容易建立起来了。在各级人民代表会议中,必须使一切民主阶层,包括工人、农民、独立劳动者、自由职业者、知识分子、民族工商业者以及开明绅士,尽可能地都有他们的代表参加进去。当然不是勉强凑数,而是要分别有市镇的农村和没有市镇的农村,分别市镇的大小,分别城市和农村,自然地而不是勉强地实现这个联合一切民主阶层的任务。\\
  在土地改革和整党的伟大的群众斗争中,教育了和产生了成万的积极分子和工作干部。他们是联系群众的,他们是中华人民共和国的极可宝贵的财富。今后应当加强对于他们的教育,使他们在工作中不断地获得进步。同时,应当向他们提出警告,决不可以因为成功,因为受到奖励,而骄傲自满。\\
  由于这一切,由于上述各方面的成功,应当说,晋绥解放区现在是比过去任何时候更加巩固了。在其它解放区,凡是这样做了的,也就同样地巩固了。\\
\subsubsection*{\myformat{二}}
晋绥解放区获得上述成功的原因,就领导方面来说,主要的是:(甲)在去年春季刘少奇同志的当面指示和去年春夏康生同志在临县郝家坡行政村的工作的帮助之下,晋绥分局在去年六月召开了地委书记会议。在这个会议上,批判了过去工作中存在着的右的偏向,彻底地揭发了各种离开党的路线的严重现象,决定了认真地发动土地改革工作和整党工作的方针。这个会议是基本上成功的。假如没有这个会议,这样大的土地改革工作和整党工作的成功是不可能的。这个会议的缺点是:没有按照老区半老区和新区的不同的情况决定不同的工作方针;在划分阶级成分的问题上采取了过左的方针;在如何消灭封建制度的问题上太注重了清查地主的地财;以及在对待群众要求的问题上缺乏清醒的分析,笼统地提出了“群众要怎样办就怎样办”的口号。关于这后一个问题,即党和群众的关系的问题,应当是:凡属人民群众的正确的意见,党必须依据情况,领导群众,加以实现;而对于人民群众中发生的不正确的意见,则必须教育群众,加以改正。地委书记会议仅仅强调了党应当执行群众意见的方面,而忽视了党应当教育群众和领导群众的方面,以致给了后来某些地区的工作同志以不正确的影响,助长了他们的尾巴主义错误。(乙)晋绥分局在今年一月采取了纠正“左”的偏向的适当的步骤。这个步骤是在分局同志参加中央十二月会议\footnote[2]{ 十二月会议,见本卷《目前形势和我们的任务》一文题解。}回来以后实行的。分局为此发出了五项指示\footnote[3]{ 指一九四八年一月十三日中共中央晋绥分局发出的《关于改正错订成分与团结中农的指示》。指示内容共分五项。其要点是:(一)由于划分阶级成分的标准不明确,在农民自发的要求下,将不少人错订为破产地主和富农,特别是将富裕中农错订为富农,影响了对中农的团结,这是错误的。(二)对上述错误应采取适当步骤,坚决地说服农民加以改正。对于已取出的财物,应作适当的退还。(三)向农民和干部说明,划分阶级成分,应以剥削关系为唯一标准。成分错订者应该改正。(四)掌握依靠贫雇农、团结中农的原则。在农民代表会议中,在农会领导机关中,使中农占有三分之一左右的比例,并在税收中、土地改革中,照顾他们的利益。(五)对于党在农村中的阶级政策,负责干部应认真研究。按照党对中农的政策,凡属错误,都要改正;同时必须通过群众去进行改正。在发出上述五项指示的同时,晋绥分局又发出了《关于保护工商业的指示》,纠正在土地改革中侵犯工商业的偏向。}。这一纠正偏向的步骤,如此适合群众的要求,又执行得如此迅速和彻底,在短时期内,几乎一切“左”的偏向都已纠正过来了。\\
\subsubsection*{\myformat{三}}
晋绥的党组织在抗日时期的领导路线,是基本上正确的。这表现在实行了减租减息,相当地恢复和发展了农业生产和家庭纺织业、军事工业和一部分轻工业,建立了党的基础,建立了民主政府,建立了近十万人的人民军队,因而就能依据这些工作作基础,进行了胜利的抗日战争,并打退了阎锡山等反动派的进攻。当然,这个时期的党和政府是有缺点的,这就是现在我们已经完全明白的它们在某种程度上的成分不纯或者作风不纯,以及由此产生的许多工作上的不良现象。但是,就总的情形说来,抗日时期的工作是有成绩的。这就给了我们在日本投降以后能够据以打败蒋介石的反革命进攻的有利条件。抗日时期,晋绥党组织的领导方面的缺点或错误,主要地是未能依靠最广大的群众克服党内和政府内在某种程度上的成分不纯或者作风不纯,以及由此产生的工作中的不良现象;这个任务,留给了你们到现在来完成。那时的晋绥的某些领导同志,缺乏对于党和群众的许多真实情况的了解,是造成上述现象的原因之一。这一点,也是同志们应当引为教训的。\\
\subsubsection*{\myformat{四}}
今后晋绥党组织的任务,是用极大的努力,继续完成土地改革工作和整党工作,继续发展和支持人民解放战争,不再加重人民负担,并酌量减轻人民负担,恢复和发展生产。你们现在正在开生产会议。在目前数年内,恢复和发展生产的目的是一方面改善人民的生活,一方面支持人民解放战争。你们有广大的农业和手工业,也有一部分使用机器的轻工业和重工业。希望你们好好地领导这些生产事业,否则就不能算作一个好的马克思主义者。在农业方面,过去被官僚主义分子所把持的、对于人民群众有害无益的那些变工队和合作社\footnote[4]{ 这里是指供销合作社。}都垮台了,这是完全可以理解的,并且是毫不可惜的。你们的任务,是在于细心地保存和发展那些为人民群众所拥护的变工队、合作社和其它必要的经济组织,并推广这样的组织于各地。\\
\subsubsection*{\myformat{五}}
全国的形势,是同志们所关心的。自从去年党的全国土地会议\footnote[5]{ 参见本卷《目前形势和我们的任务》注〔7〕。}决定采取新的方针,展开土地改革工作和整党工作以后,差不多在一切解放区都召开了有关整党和土地改革的大的干部会议,批判了存在于党内的右倾思想,揭发了党内在某种程度上存在着的成分不纯或者作风不纯的严重现象。而在以后,在许多地区,又采取适当的步骤,纠正了或者正在纠正着“左”的偏向。这样,就使我党在全国的工作,在新的政治形势和政治任务之下,走上了健全发展的轨道。差不多一切人民解放军的部队,在最近几个月内,都利用了战争的空隙,实行了大规模的整训。这种整训,是完全有领导地和有秩序地采用民主方法进行的。由此,激发了广大的指挥员和战斗员群众的革命热情,明确地认识了战争的目的,清除了存在于军队中的若干不正确的思想上的倾向和不良现象,教育了干部和战士,极大地提高了战斗力。这种民主的群众性的新式的整军运动,今后必须继续进行。你们可以清楚地看见,我们所实行的具有伟大历史意义的整党、整军和土地改革工作,我们的敌人国民党是一样也不能实行的。在我们方面,是如此认真地纠正自己的缺点,把我们的全党全军团结得差不多像一个人一样,使全党全军和人民群众密切地结合起来,有效地执行着我党中央所规定的一切政策和策略,胜利地进行着人民的解放战争。在我们的敌人方面,则一切相反。他们是那样腐化,那样充满日益增多的无法解决的内部争吵,那样被人民唾弃而陷于完全的孤立,打了那样多的败仗,因此他们就必不可免地走向灭亡。这就是中国革命和反革命的互相对比的全部形势。\\
  在这种形势下面,全党同志必须紧紧地掌握党的总路线,这就是新民主主义革命的路线。新民主主义的革命,不是任何别的革命,它只能是和必须是无产阶级领导的,人民大众的,反对帝国主义、封建主义和官僚资本主义的革命。这就是说,这个革命不能由任何别的阶级和任何别的政党充当领导者,只能和必须由无产阶级和中国共产党充当领导者。这就是说,由参加这个革命的人们所组成的统一战线是十分广大的,这里包括了工人、农民、独立劳动者、自由职业者、知识分子、民族资产阶级以及从地主阶级分裂出来的一部分开明绅士,这就是我们所说的人民大众。由这个人民大众所建立的国家和政府,就是中华人民共和国和无产阶级领导的各民主阶级联盟的民主联合政府。这个革命所要推翻的敌人,只是和必须是帝国主义、封建主义和官僚资本主义。这些敌人的集中表现,就是蒋介石国民党的反动统治。\\
  封建主义是帝国主义和官僚资本主义的同盟者及其统治的基础。因此,土地制度的改革,是中国新民主主义革命的主要内容。土地改革的总路线,是依靠贫农,团结中农,有步骤地、有分别地消灭封建剥削制度,发展农业生产。土地改革所依靠的基本力量,只能和必须是贫农。这个贫农阶层,和雇农在一起,占了中国农村人口的百分之七十左右。土地改革的主要的和直接的任务,就是满足贫雇农群众的要求。土地改革必须团结中农,贫雇农必须和占农村人口百分之二十左右的中农结成巩固的统一战线。不这样做,贫雇农就会陷于孤立,土地改革就会失败。土地改革的一个任务,是满足某些中农的要求。必须容许一部分中农保有比较一般贫农所得土地的平均水平为高的土地量。我们赞助农民平分土地的要求,是为了便于发动广大的农民群众迅速地消灭封建地主阶级的土地所有制度,并非提倡绝对的平均主义。谁要是提倡绝对的平均主义,那就是错误的。现在农村中流行的一种破坏工商业、在分配土地问题上主张绝对平均主义的思想,它的性质是反动的、落后的、倒退的。我们必须批判这种思想。土地改革的对象,只是和必须是地主阶级和旧式富农的封建剥削制度,不能侵犯民族资产阶级,也不要侵犯地主富农所经营的工商业,特别注意不要侵犯没有剥削或者只有轻微剥削的中农、独立劳动者、自由职业者和新式富农。土地改革的目的是消灭封建剥削制度,即消灭封建地主之为阶级,而不是消灭地主个人。因此,对地主必须分给和农民同样的土地财产,并使他们学会劳动生产,参加国民经济生活的行列。除了可以和应当惩办那些为广大人民群众所痛恨的查有实据的罪大恶极的反革命分子和恶霸分子以外,必须实行对一切人的宽大政策,禁止任何的乱打乱杀。消灭封建剥削制度应当是有步骤的,即是说,有策略的。必须依据环境所许可的情况,农民群众的觉悟程度和组织程度,决定发动斗争的策略,不要企图在一个早上消灭全部的封建剥削制度。土地改革的总的打击面,根据中国农村封建剥削制度的实际情况,一般地不能超过农村户数百分之八左右,人数百分之十左右。而在老的和半老的解放区内,此项数目还要减少。离开实际情况,错误地扩大打击面,是危险的。在新区,还必须分地区,分阶段。分地区,是说应当集中力量在那些可以巩固地占领的区域进行适当的合乎当地群众要求的土地改革工作;而在那些暂时尚难巩固地占领的区域,则不要忙于进行土地改革,而只做一些可以做的按照当前情况有利于群众的工作,以待情况的变化。分阶段,是说在人民解放军刚才占领的区域,应当提出和实行中立富农和中立中小地主的策略,将打击面缩小到只消灭国民党的反动武装和打击豪绅恶霸分子。应当集中一切力量去完成这个任务,作为新区工作的第一个阶段。然后,依据群众的觉悟程度和组织程度被提高了的情况,逐步地发展到消灭全部封建制度的阶段。在新区,分浮财和分土地,均必须在环境比较安定和绝大多数群众充分发动之后,否则就是冒险的,靠不住的,有害无益的。在新区,必须充分地利用抗日时期的经验。所谓有分别地消灭封建制度,就是说,必须分别地主和富农,分别地主的大中小,分别地主富农中的恶霸分子和非恶霸分子,在平分土地、消灭封建制度的大原则下面,不是一律地而是有所分别地决定和实行给予这些不同情况的人们以不同的待遇。在我们这样做了的时候,人们就会感觉到,我们的工作是完全合乎情理的。发展农业生产,是土地改革的直接目的。只有消灭封建制度,才能取得发展农业生产的条件。在任何地区,一经消灭了封建制度,完成了土地改革任务,党和民主政府就必须立即提出恢复和发展农业生产的任务,将农村中的一切可能的力量转移到恢复和发展农业生产的方面去,组织合作互助,改良农业技术,提倡选种,兴办水利,务使增产成为可能。农村党的精力的最大部分,必须放在恢复和发展农业生产和市镇上的工业生产上面。为了迅速地恢复和发展农业生产和市镇上的工业生产,在消灭封建制度的斗争中,必须注意尽一切努力最大限度地保存一切可用的生产资料和生活资料,采取办法坚决地反对任何人对于生产资料和生活资料的破坏和浪费,反对大吃大喝,注意节约。为了发展农业生产,必须劝告农民在自愿原则下逐步地组织为现时经济条件所许可的以私有制为基础的各种生产的和消费的合作团体。消灭封建制度,发展农业生产,就给发展工业生产,变农业国为工业国的任务奠定了基础,这就是新民主主义革命的最后目的。\\
  同志们知道,我党规定了中国革命的总路线和总政策,又规定了各项具体的工作路线和各项具体的政策。但是,许多同志往往记住了我党的具体的各别的工作路线和政策,忘记了我党的总路线和总政策。而如果真正忘记了我党的总路线和总政策,我们就将是一个盲目的不完全的不清醒的革命者,在我们执行具体工作路线和具体政策的时候,就会迷失方向,就会左右摇摆,就会贻误我们的工作。\\
  让我再说一遍:\\
  无产阶级领导的,人民大众的,反对帝国主义、封建主义和官僚资本主义的革命,这就是中国的新民主主义的革命,这就是中国共产党在当前历史阶段的总路线和总政策。\\
  依靠贫农,团结中农,有步骤地、有分别地消灭封建剥削制度,发展农业生产,这就是中国共产党在新民主主义的革命时期,在土地改革工作中的总路线和总政策。\\
\newpage\section*{\myformat{对晋绥日报编辑人员的谈话}\\\myformat{(一九四八年四月二日)}}\addcontentsline{toc}{section}{对晋绥日报编辑人员的谈话}
我们的政策,不光要使领导者知道,干部知道,还要使广大的群众知道。有关政策的问题,一般地都应当在党的报纸上或者刊物上进行宣传。我们正在进行土地制度的改革。有关土地改革的各项政策,都应当在报上发表,在电台广播,使广大群众都能知道。群众知道了真理,有了共同的目的,就会齐心来做。这和打仗一样,要打好仗,不光要干部齐心,还要战士齐心。陕北的部队经过整训诉苦以后,战士们的觉悟提高了,明了了为什么打仗,怎样打法,个个磨拳擦掌,士气很高,一出马就打了胜仗。群众齐心了,一切事情就好办了。马克思列宁主义的基本原则,就是要使群众认识自己的利益,并且团结起来,为自己的利益而奋斗。报纸的作用和力量,就在它能使党的纲领路线,方针政策,工作任务和工作方法,最迅速最广泛地同群众见面。\\
  在我们一些地方的领导机关中,有的人认为,党的政策只要领导人知道就行,不需要让群众知道。这是我们的有些工作不能做好的基本原因之一。我党二十几年来,天天做群众工作,近十几年来,天天讲群众路线。我们历来主张革命要依靠人民群众,大家动手,反对只依靠少数人发号施令。但是在有些同志的工作中间,群众路线仍然不能贯彻,他们还是只靠少数人冷冷清清地做工作。其原因之一,就是他们做一件事情,总不愿意向被领导的人讲清楚,不懂得发挥被领导者的积极性和创造力。他们主观上也要大家动手动脚去做,但是不让大家知道要做的是怎么一回事,应当怎样做法,这样,大家怎么能动起来,事情怎么能够办好?要解决这个问题,根本上当然要从思想上进行群众路线的教育,同时也要教给同志们许多具体办法。办法之一,就是要充分地利用报纸。办好报纸,把报纸办得引人入胜,在报纸上正确地宣传党的方针政策,通过报纸加强党和群众的联系,这是党的工作中的一项不可小看的、有重大原则意义的问题。\\
  同志们是办报的。你们的工作,就是教育群众,让群众知道自己的利益,自己的任务,和党的方针政策。办报和办别的事一样,都要认真地办,才能办好,才能有生气。我们的报纸也要靠大家来办,靠全体人民群众来办,靠全党来办,而不能只靠少数人关起门来办。我们的报上天天讲群众路线,可是报社自己的工作却往往没有实行群众路线。例如,报上常有错字,就是因为没有把消灭错字认真地当做一件事情来办。如果采取群众路线的方法,报上有了错字,就把全报社的人员集合起来,不讲别的,专讲这件事,讲清楚错误的情况,发生错误的原因,消灭错误的办法,要大家认真注意。这样讲上三次五次,一定能使错误得到纠正。小事如此,大事也是如此。\\
  善于把党的政策变为群众的行动,善于使我们的每一个运动,每一个斗争,不但领导干部懂得,而且广大的群众都能懂得,都能掌握,这是一项马克思列宁主义的领导艺术。我们的工作犯不犯错误,其界限也在这里。当着群众还不觉悟的时候,我们要进攻,那是冒险主义。群众不愿干的事,我们硬要领导他们去干,其结果必然失败。当着群众要求前进的时候,我们不前进,那是右倾机会主义。陈独秀机会主义\footnote[1]{ 见本书第一卷《中国革命战争的战略问题》注〔4〕。}的错误,就是落后于群众的觉悟程度,不能领导群众前进,而且反对群众前进。这些问题有许多同志还不懂得。我们的报纸要好好地宣传这些观点,使大家都能明白。\\
  报纸工作人员为了教育群众,首先要向群众学习。同志们都是知识分子。知识分子往往不懂事,对于实际事物往往没有经历,或者经历很少。你们对于一九三三年制订的《怎样分析农村阶级》的小册子,就看不大懂;这一点,农民比你们强,只要给他们一说就都懂得了。崞县\footnote[2]{ 崞县,今山西省原平县。}两个区的农民一百八十多人,开了五天会,解决了分配土地中的许多问题。假如你们的编辑部来讨论那些问题,恐怕两个星期也解决不了。原因很简单,那些问题你们不懂得。要使不懂得变成懂得,就要去做去看,这就是学习。报社的同志应当轮流出去参加一个时期的群众工作,参加一个时期的土地改革工作,这是很必要的。在没有出去参加群众工作的时候,也应当多听多看关于群众运动的材料,并且下工夫研究这些材料。我们练兵的口号是:“官教兵,兵教官,兵教兵。”战士们有很多打仗的实际经验。当官的要向战士学习,把别人的经验变成自己的,他的本领就大了。报社的同志也要经常向下边反映上来的材料学习,慢慢地使自己的实际知识丰富起来,使自己成为有经验的人。这样,你们的工作才能够做好,你们才能担负起教育群众的任务。\\
  《晋绥日报》在去年六月的地委书记会议以后,有很大进步。内容丰富,尖锐泼辣,有朝气,反映了伟大的群众斗争,为群众讲了话。我很愿意看它。但是从今年一月开始纠正“左”的偏向以后的这一时期,你们的报纸却有点泄气的样子,不够明确,不够泼辣,材料也少了,使人不大想看。你们现在正在检查工作,总结经验,这样很好。总结了反右反“左”的经验,使头脑清醒起来,你们的工作就会有改进。\\
  《晋绥日报》在去年六月以后进行的反对右倾的斗争,是完全正确的。在反右倾的斗争中,你们作得很认真,充分地反映了群众运动的实际情况。对于你们认为错误的观点和材料,你们采用编者按语的形式加以批注。你们的批注后来也有缺点,但是那种认真的精神是好的。你们的缺点主要是把弓弦拉得太紧了。拉得太紧,弓弦就会断。古人说:“文武之道,一张一弛。”\footnote[3]{ 参见《礼记•杂记下》。原文是:“张而不弛,文武弗能也。弛而不张,文武弗为也。一张一弛,文武之道也。”文武,指周文王、周武王。}现在“弛”一下,同志们会清醒起来。过去的工作有成绩,但也有缺点,主要是“左”的偏向。现在作一次全面的总结,纠正了“左”的偏向,就会做出更大的成绩来。\\
  在我们纠正偏差的时候,有的人把过去的工作看得毫无成绩,认为完全错了。这是不对的。这些人没有看到,党领导了那么多的农民得到土地,打倒了封建主义,整顿了党的组织,改进了干部的作风,现在又纠正了“左”的偏向,教育了干部和群众。这不是很大的成绩吗?对于我们的工作,对于群众的事业,应当采取分析的态度,不应当否定一切。过去发生“左”的偏向,是因为大家没有经验。没有经验,就难免要犯错误。从没有经验到有经验,要有一个过程。去年六月到现在的短短时期内,经过反右和反“左”的斗争,使大家都知道了反右、反“左”是怎么一回事。没有这样一个过程,大家是不会知道的。\\
  经过检查工作、总结经验以后,我相信,你们的报纸会办得更好。应当保持你们报纸的过去的优点,要尖锐、泼辣、鲜明,要认真地办。我们必须坚持真理,而真理必须旗帜鲜明。我们共产党人从来认为隐瞒自己的观点是可耻的。我们党所办的报纸,我们党所进行的一切宣传工作,都应当是生动的,鲜明的,尖锐的,毫不吞吞吐吐。这是我们革命无产阶级应有的战斗风格。我们要教育人民认识真理,要动员人民起来为解放自己而斗争,就需要这种战斗的风格。用钝刀子割肉,是半天也割不出血来的。\\
\newpage\section*{\myformat{再克洛阳后给洛阳前线指挥部的电报}\\\myformat{(一九四八年四月八日)}}\addcontentsline{toc}{section}{再克洛阳后给洛阳前线指挥部的电报}
\begin{introduction}\item  这是毛泽东为中共中央起草的电报。因为它的内容不但适用于洛阳,也基本上适用于一切新解放的城市,所以这个电报同时发给了其它前线和其它地区的领导同志。\end{introduction}
此次再克洛阳\footnote[1]{ 洛阳是当时国民党军队在河南西部的一个重要据点。人民解放军于一九四八年三月十四日首次攻克洛阳,以后为便于歼灭敌人有生力量,又主动撤出。同年四月五日,人民解放军再度攻克该城。},可能巩固。关于城市政策,应注意下列各点。\\
  一、极谨慎地清理国民党统治机构,只逮捕其中主要反动分子,不要牵连太广。\\
  二、对于官僚资本要有明确界限,不要将国民党人经营的工商业都叫作官僚资本而加以没收。对于那些查明确实是由国民党中央政府、省政府、县市政府经营的,即完全官办的工商业,应该确定归民主政府接管营业的原则。但如民主政府一时来不及接管或一时尚无能力接管,则应该暂时委托原管理人负责管理,照常开业,直至民主政府派人接管时为止。对于这些工商业,应该组织工人和技师参加管理,并且信任他们的管理能力。如国民党人已逃跑,企业处于停歇状态,则应该由工人和技师选出代表,组织管理委员会管理,然后由民主政府委任经理和厂长,同工人一起加以管理。对于著名的国民党大官僚所经营的企业,应该按照上述原则和办法处理。对于小官僚和地主所办的工商业,则不在没收之列。一切民族资产阶级经营的企业,严禁侵犯。\\
  三、禁止农民团体进城捉拿和斗争地主。对于土地在乡村家在城里的地主,由民主市政府依法处理。其罪大恶极者,可根据乡村农民团体的请求送到乡村处理。\\
  四、入城之初,不要轻易提出增加工资减少工时的口号。在战争时期,能够继续生产,能够不减工时,维持原有工资水平,就是好事。将来是否酌量减少工时增加工资,要依据经济情况即企业是否向上发展来决定。\\
  五、不要忙于组织城市人民进行民主改革和生活改善的斗争。要等市政管理有了头绪,人心已经安定,经过周密调查,弄清情况和筹有妥善解决办法的时候,才可以按情况酌量处理。\\
  六、大城市目前的中心问题是粮食和燃料问题,必须有计划地加以处理。城市一经由我们管理,就必须有计划地逐步解决贫民的生活问题。不要提“开仓济贫”的口号。不要使他们养成依赖政府救济的心理。\\
  七、国民党员和三青团员,必须妥善地予以清理和登记。\\
  八、一切作长期打算。严禁破坏任何公私生产资料和浪费生活资料,禁止大吃大喝,注意节约。\\
  九、市委书记和市长必须委派懂政策有能力的人担任。市委书记和市长应该对所属一切工作人员加以训练,讲明各项城市政策和策略。城市已经属于人民,一切应该以城市由人民自己负责管理的精神为出发点。如果应用对待国民党管理的城市的政策和策略,来对待人民自己管理的城市,那就是完全错误的。
\newpage\section*{\myformat{新解放区农村工作的策略问题}\\\myformat{(一九四八年五月二十四日)}}\addcontentsline{toc}{section}{新解放区农村工作的策略问题}
\begin{introduction}\item  这是毛泽东给邓小平的电报。\end{introduction}
新解放区农村工作的策略问题有全盘考虑之必要。新解放区必须充分利用抗日时期的经验,在解放后的相当时期内,实行减租减息和酌量调剂种子口粮的社会政策和合理负担的财政政策,把主要的打击对象限于政治上站在国民党方面坚决反对我党我军的重要反革命分子,如同抗日时期只逮捕汉奸分子和没收他们的财产一样,而不是立即实行分浮财、分土地的社会改革政策。因为过早地分浮财,只是少数勇敢分子欢迎,基本群众并未分得,因而会表示不满。而且,社会财富迅速分散,于军队亦不利。过早地分土地,使军需负担过早地全部落在农民身上,不是落在地主富农身上。不如不分浮财,不分土地,在社会改革上普遍实行减租减息,使农民得到实益;在财政政策上实行合理负担,使地主富农多出钱。这样,社会财富不分散,社会秩序较稳定,利于集中一切力量消灭国民党反动派。在一两年甚至三年以后,在大块根据地上,国民党反动派已被消灭,环境已经安定,群众已经觉悟和组织起来,战争已经向遥远地方推进,那时就可进入像华北那样的分浮财、分土地的土地改革阶段。这一个减租减息阶段是任何新解放地区所不能缺少的,缺少了这个阶段,我们就要犯错误。就是在华北、东北、西北各大解放区的接敌地区,亦须实行上述同样的策略。\\
\newpage\section*{\myformat{一九四八年的土地改革工作和整党工作}\\\myformat{(一九四八年五月二十五日)}}\addcontentsline{toc}{section}{一九四八年的土地改革工作和整党工作}
\begin{introduction}\item  这是毛泽东为中共中央起草的对党内的指示。\end{introduction}
\subsubsection*{\myformat{一}}
必须注意季节。必须利用今年整个秋季和冬季,即自今年九月至明年三月,共七个月时间,在各中央局和分局所划定的地区内,依次完成下列各项工作:(甲)乡村情况调查。(乙)按照正确政策实行初步整党。上级派到乡村的工作团或工作组,必须首先团结当地党的支部组织内的一切积极分子和较好分子,共同领导当地的土地改革工作。(丙)组织或改组或充实贫农团和农会,发动土地改革斗争。(丁)按照正确标准,划分阶级成分。(戊)按照正确政策,实行分配封建土地和封建财产。实行分配的最后结果,必须使一切主要阶层都感觉公道和合乎情理,地主阶级分子亦感觉生活有出路,有保障。(己)建立乡(村)、区、县三级人民代表会议,并选举三级政府委员会。(庚)发给土地证,确定地权。(辛)调整或改订农业税(公粮)负担的标准。这种标准,必须遵守公私兼顾的原则,这即是一方面利于支持战争,一方面使农民有恢复和发展生产的兴趣,利于改善农民的生活。(壬)按照正确政策,完成党的支部组织的整理工作。(癸)将工作方向由土地改革方面,转移到团结农村中一切劳动人民并组织地主富农的劳动力为共同恢复和发展农业生产而奋斗的方面去。开始组织在自愿和等价交换两项原则上的小规模的变工组织和其它合作团体;准备好种子、肥料和燃料;做好生产计划;发放必要的和可能的农业贷款(以贷给生产资料为主,必须有借有还,严格区别于救济性质的赈款);在可能的地点,做好兴修水利的计划。以上是由土地改革到生产的全部工作过程,必须使一切直接从事土地改革工作的同志了解这样的工作过程,避免工作的片面性,并不失时机地于秋冬两季全部完成上述工作。\\
\subsubsection*{\myformat{二}}
为达上述目的,今年六月至八月的三个月内,必须完成下列工作:(甲)划定土地改革工作范围。这种范围,必须是在下列三项条件下划定之:第一,当地一切敌人武装力量已经全部消灭,环境已经安定,而非动荡不定的游击区域。第二,当地基本群众(雇农、贫农、中农)的绝对大多数已经有了分配土地的要求,而不只是少数人有此要求。第三,党的工作干部在数量上和质量上,确能掌握当地的土地改革工作,而非听任群众的自发活动。如果某一地区,在上述三个条件中,有任何一个条件不具备,即不应当将该地区列入一九四八年进行土地改革的范围。例如,在华北、华东、东北、西北各解放区的接敌区域和中原局所属江淮河汉区域的绝大部分地区,因为尚不具备第一个条件,即不应当列入今年的土地改革计划内。明年是否列入,还要看情况才能决定。在这类地区,应当充分利用抗日时期的经验,实行减租减息和酌量调剂种子食粮的社会政策和合理负担的财政政策,以便联合或中立一切可能联合或中立的社会力量,帮助人民解放军消灭一切国民党武装力量和打击政治上最反动的恶霸分子。在这类地区,既不要分土地,也不要分浮财,因为这些都是在新区和接敌区的条件之下,不利于联合或中立一切可能联合或中立的社会力量、完成消灭国民党反动力量这一基本任务的。(乙)开好干部会议。在为着土地改革和整党工作召集的干部会议中,必须充分讲明关于这两项工作的全部正确政策,将许可做的事和不许可做的事,分清界限。必须将中央颁布的各项重要文件,责成一切从事土地改革工作和整党工作的干部,认真学习,完全了解,并责成他们全部遵守,不许擅自修改。如有不适合当地情况的部分,可以和应当提出修改的意见,但必须取得中央同意,方能实行修改。今年的各级干部会议,必须由各地高级领导机关,在开会之前,作充分而恰当的准备,这即是事前由少数人商量(由一个人负主责),提出问题和分析问题,写好成文的纲要,精心斟酌这个纲要的内容和文字(注意简明扼要,反对不着边际的长篇大论),然后向干部会议作报告,开展讨论,吸收讨论中的意见,加以补充和修改,作为定论;并将此项文件通知全党和尽可能地在报纸上公开发表。必须反对经验主义的方法,这即是事前毫无准备,不提出问题,不分析问题,不向干部会议作精心准备的、内容文字都有斟酌的报告,而听凭到会人员无目的地杂乱无章地议论,致使会议时间延长,得不到明确而周密的结论。各中央局、中央分局、区党委、省委和地委的领导工作中,如果存在着这种有害的经验主义方法,必须注意克服。讨论政策的会议,人数不可太多,只要事先有良好准备,会议的时间亦可缩短。按情况,大约以十几个人,或二三十人,或四五十人,开会一星期左右为适宜。传达政策的会议,人数可以多些,时间亦不可过长。只有整党性质的高级和中级的干部会议,人数可以多些,时间亦可以长些。(丙)九月上半月,至迟九月下半月,全部直接从事土地改革工作的干部必须到达乡村,并开始工作,否则就不能利用秋冬两季的全部时间,完成全部土地改革、整党建政和准备春耕的工作。\\
\subsubsection*{\myformat{三}}
在干部会议中和在工作中,必须教育干部善于分析具体情况,从不同地区、不同历史条件的具体情况出发,决定当地当时的工作任务和工作方法。必须区别城市和农村的不同,必须区别老区、半老区、接敌区和新区的不同,否则就要犯错误。\\
\subsubsection*{\myformat{四}}
凡属封建制度已经根本消灭,贫雇农已经得到大体上相当于平均数的土地,他们同中农所有的土地虽有差别(这种差别是许可的),但是相差不多者,即应认为土地问题已经解决,不要再提土地改革问题。在这类地区的中心任务,是恢复和发展生产,完成整党建政工作和支持前线的工作。在这类地区的部分乡村中,如果尚有土地须待分配或调剂,阶级成分须待改订,土地证须待发给者,自然应当按照实际情形完成这些工作。\\
\subsubsection*{\myformat{五}}
在一切解放区,不论是已经完成土地改革的地区,或者尚未完成土地改革的地区,都必须在今年秋季指导农民耕种麦地,并进行一部分土地的秋耕。在冬季,要号召农民积肥。所有这些,都对一九四九年解放区农业的生产和收成有极大重要性,必须用行政力量,配合群众工作,加以实现。\\
\subsubsection*{\myformat{六}}
必须坚决地克服许多地方存在着的某些无纪律状态或无政府状态,即擅自修改中央的或上级党委的政策和策略,执行他们自以为是的违背统一意志和统一纪律的极端有害的政策和策略;在工作繁忙的借口之下,采取事前不请示事后不报告的错误态度,将自己管理的地方,看成好像一个独立国。这种状态,给予革命利益的损害,极为巨大。各级党委必须对这一点进行反复讨论,认真克服这种无纪律状态或无政府状态,将一切可能和必须集中的权力,集中于中央和中央代表机关\footnote[1]{ 这里所说的中央代表机关,是指中央局和中央分局。}。\\
\subsubsection*{\myformat{七}}
中央、中央局(分局)、区党委(省委)、地委、县委、区委、直到支部,必须充分利用无线电、有线电、电话、邮递、专人送信等项通讯方法,小型会议(例如四五个人的),区域会议(例如几个县的),和个别谈话等项会谈方法,小型巡视团(例如三至五个人的)和个别有威信的委员的巡视方法,同时充分利用通讯社和报纸,密切地互相联系起来,以便掌握运动的动态,随时互通情报,交流经验,及时纠正错误,发扬成绩。不要等候几个月,或半年,甚至更长时间,下面才向上面作总结性的报告,上面才向下面作一般性的指示。这种报告和指示,往往过时,失去作用,或者减少了作用。犯错误的已经犯过,来不及纠正,损失太大。全党迫切需要的,是不失时机的生动的具体的报告和指示。\\
\subsubsection*{\myformat{八}}
必须将城市工作和农村工作,将工业生产任务和农业生产任务,放在各中央局、分局、区党委、省委、地委和市委的领导工作的适当位置。即是说,不要因为领导土地改革工作和农业生产工作,而忽视或放松对于城市工作和工业生产工作的领导。我们现在已经有了许多大中小城市和广大的工矿交通企业,如果各有关领导机关忽视或放松这一方面的工作,我们就要犯错误。\\
\newpage\section*{\myformat{关于辽沈战役\footnote[1]{ 辽沈战役,是一九四八年九月十二日至十一月二日中国人民解放军东北野战军在辽宁省西部和沈阳、长春地区同国民党军进行的一次具有决定意义的战役。战役前,国民党军在东北地区的总兵力有四个兵团,连同地方保安团队,共约五十五万人,分别收缩在长春、沈阳、锦州三个孤立地区。东北野战军集中主力十二个纵队、一个炮兵纵队和十七个独立师,连同地方武装共一百零三万人,在东北广大人民支援下,发起辽沈战役。北宁线上的锦州,是联结东北和华北的一个战略要点。防守锦州地区的国民党军开始时为东北“剿总”副总司令范汉杰指挥下的六个师,连同特种兵、后勤及地方部队约十万人,到九月下旬,国民党又空运一个师增援锦州。打下锦州是辽沈战役的关键。东北野战军根据中央军委的指示,以一个纵队和六个独立师、一个骑兵师、一个炮兵团继续围困长春;以五个纵队又一个师以及炮兵纵队主力、一个坦克营围攻锦州;以两个纵队配置于锦州西南的塔山、虹螺山一线,十一个师配置于彰武、新立屯以东地区,分别阻击由锦西、葫芦岛方向和沈阳方向救援锦州之敌;以一个纵队位于高桥地区为战役预备队。锦州地区的作战是从九月十二日开始的。正当人民解放军攻克义县,扫清锦州外围敌人时,蒋介石慌忙飞到东北亲自指挥,并急调北宁线华北“剿总”的五个师来援,连同原来在锦西的四个师,共九个师,于十月十日起开始向塔山阵地猛攻,但始终未能突破人民解放军阵地。廖耀湘兵团(国民党第九兵团)十一个师又三个骑兵旅由沈阳驰援锦州,被人民解放军阻击在彰武、新立屯地区。十月十四日人民解放军对锦州发起进攻,经过三十一小时激战,全歼该敌,俘虏敌东北“剿总”副总司令范汉杰、第六兵团司令卢浚泉以下约九万人。锦州的解放迫使长春敌人的一个军起义,其余全部投降。此时,东北国民党军全军覆灭的命运,已成定局。但蒋介石仍然梦想夺回锦州,打通关内外的联络,严令廖耀湘兵团继续向锦州前进。东北野战军在攻占锦州后,就立即向东北方面回师,从黑山、大虎山南北两翼合围廖耀湘兵团。十月二十六日将廖兵团包围于黑山、大虎山、新民地区,经两日一夜的激战,全部歼灭该敌,俘虏敌兵团司令廖耀湘,军长李涛、向凤武、郑庭笈以下十万余人。人民解放军乘胜追击,十一月二日解放沈阳、营口。至此,辽沈战役胜利结束,共歼灭敌人四十七万二千余人。锦西、葫芦岛地区的国民党军于十一月九日从海上撤向关内,东北全境获得解放。}的作战方针}\\\myformat{(一九四八年九月、十月)}}\addcontentsline{toc}{section}{关于辽沈战役\footnote[1]{ 辽沈战役,是一九四八年九月十二日至十一月二日中国人民解放军东北野战军在辽宁省西部和沈阳、长春地区同国民党军进行的一次具有决定意义的战役。战役前,国民党军在东北地区的总兵力有四个兵团,连同地方保安团队,共约五十五万人,分别收缩在长春、沈阳、锦州三个孤立地区。东北野战军集中主力十二个纵队、一个炮兵纵队和十七个独立师,连同地方武装共一百零三万人,在东北广大人民支援下,发起辽沈战役。北宁线上的锦州,是联结东北和华北的一个战略要点。防守锦州地区的国民党军开始时为东北“剿总”副总司令范汉杰指挥下的六个师,连同特种兵、后勤及地方部队约十万人,到九月下旬,国民党又空运一个师增援锦州。打下锦州是辽沈战役的关键。东北野战军根据中央军委的指示,以一个纵队和六个独立师、一个骑兵师、一个炮兵团继续围困长春;以五个纵队又一个师以及炮兵纵队主力、一个坦克营围攻锦州;以两个纵队配置于锦州西南的塔山、虹螺山一线,十一个师配置于彰武、新立屯以东地区,分别阻击由锦西、葫芦岛方向和沈阳方向救援锦州之敌;以一个纵队位于高桥地区为战役预备队。锦州地区的作战是从九月十二日开始的。正当人民解放军攻克义县,扫清锦州外围敌人时,蒋介石慌忙飞到东北亲自指挥,并急调北宁线华北“剿总”的五个师来援,连同原来在锦西的四个师,共九个师,于十月十日起开始向塔山阵地猛攻,但始终未能突破人民解放军阵地。廖耀湘兵团(国民党第九兵团)十一个师又三个骑兵旅由沈阳驰援锦州,被人民解放军阻击在彰武、新立屯地区。十月十四日人民解放军对锦州发起进攻,经过三十一小时激战,全歼该敌,俘虏敌东北“剿总”副总司令范汉杰、第六兵团司令卢浚泉以下约九万人。锦州的解放迫使长春敌人的一个军起义,其余全部投降。此时,东北国民党军全军覆灭的命运,已成定局。但蒋介石仍然梦想夺回锦州,打通关内外的联络,严令廖耀湘兵团继续向锦州前进。东北野战军在攻占锦州后,就立即向东北方面回师,从黑山、大虎山南北两翼合围廖耀湘兵团。十月二十六日将廖兵团包围于黑山、大虎山、新民地区,经两日一夜的激战,全部歼灭该敌,俘虏敌兵团司令廖耀湘,军长李涛、向凤武、郑庭笈以下十万余人。人民解放军乘胜追击,十一月二日解放沈阳、营口。至此,辽沈战役胜利结束,共歼灭敌人四十七万二千余人。锦西、葫芦岛地区的国民党军于十一月九日从海上撤向关内,东北全境获得解放。}的作战方针}
\begin{introduction}\item  这是毛泽东为中共中央军事委员会起草的给林彪、罗荣桓等的电报。毛泽东在这里提出的关于辽沈战役的作战方针,后来得到了完全的实现。辽沈战役的结果是:(一)歼敌四十七万人,加上当时人民解放军在其它各个战场上的胜利,就使人民解放军在数量上对于国民党军也转入了优势;(二)解放了东北全境,并为解放平津和全华北准备了前提;(三)人民解放军获得了进行大规模歼灭战的经验;(四)由于东北的解放,解放战争获得了战略上巩固的和具有一定工业基础的后方,党和人民获得了逐步转入经济恢复工作的有利条件。辽沈战役是中国人民解放战争中有决定意义的三个最大战役的第一个。其它两个是淮海战役和平津战役。这三大战役,共进行四个月零十九天,歼灭国民党军正规军一百四十四个师(旅),非正规军二十九个师,共一百五十四万余人。在这个期间,其它战场的人民解放军也都展开了进攻,歼灭了大量的敌人。在战争的头两年,人民解放军每月平均歼灭敌军八个旅左右。到了这时,人民解放军歼灭敌军的数目,已经不是平均每月八个旅,而是三十八个旅了。这三大战役,使国民党赖以发动反革命内战的精锐部队基本上归于消灭,大大加速了解放战争的全国胜利的到来。关于淮海战役和平津战役,见本卷《关于淮海战役的作战方针》、《关于平津战役的作战方针》两文。\end{introduction}
\subsection*{\myformat{一 九月七日的电报}}
我们准备五年左右(从一九四六年七月算起)根本上打倒国民党,这是具有可能性的\footnote[2]{ 五年左右根本上打倒国民党,这是当时的预计。后来,这个时间缩短为三年半左右。参见本卷《中国军事形势的重大变化》。}。只要我们每年歼灭国民党正规军一百个旅左右,五年歼敌五百个旅左右,就能达到此项目的。过去两年我军共歼敌正规军一百九十一个旅,平均每年九十五个半旅,每月八个旅弱。今后三年要求我军歼敌正规军三百个旅以上。今年七月至明年六月,我们希望能歼敌正规军一百十五个旅左右。此数分配于各野战军和各兵团\footnote[3]{ 一九四八年十一月一日中共中央军事委员会根据中共中央政治局九月会议的决定,把原各大战略区的部队划分为野战部队、地方部队和游击部队三类。将野战部队编为野战军。野战军以下辖兵团,兵团辖军(即原来的纵队),军辖师,师辖团。各野战军以其所在地区分为中国人民解放军西北野战军、中原野战军、华东野战军、东北野战军。各野战军所属兵团、军、师的数目,依各大战略区具体情况而定。后来,西北野战军改为第一野战军,辖两个兵团;中原野战军改为第二野战军,辖三个兵团;华东野战军改为第三野战军,辖四个兵团;东北野战军改为第四野战军,辖四个兵团。华北军区所辖的三个兵团,直属中国人民解放军总部。}。要求华东野战军担负歼灭四十个旅左右(他们七月歼灭的七个旅在内),并攻占济南和苏北、豫东、皖北若干大中小城市。要求中原野战军担负歼灭十四个旅左右(七月已歼两个旅在内),并攻占鄂豫皖三省若干城市。要求西北野战军担负歼灭十二个旅左右(八月已歼一个半旅在内)。要求华北徐向前、周士第兵团歼灭阎锡山十四个旅左右(七月已歼八个旅在内),并攻占太原。要求你们配合罗瑞卿、杨成武两兵团担负歼灭卫立煌、傅作义两军三十五个旅左右(七月杨成武已歼一个旅在内),并攻占北宁、平绥、平承、平保各线除北平、天津、沈阳三点以外的一切城市。欲达此目的,战役部署指挥的适当,作战休息调节的适当,是决定性关键。你们如果能在九十两月或再多一点时间内歼灭锦州至唐山一线之敌,并攻克锦州、榆关、唐山诸点,就可以达到歼敌十八个旅左右之目的。为了歼灭这些敌人,你们现在就应该准备使用主力于该线,而置长春、沈阳两敌于不顾,并准备在打锦州时歼灭可能由长、沈援锦之敌。因为锦、榆、唐三点及其附近之敌互相孤立,攻歼取胜比较确实可靠,攻锦打援亦较有希望。如果你们以主力位于新民及其以北地区准备打长、沈出来之敌,则该敌因受你们威胁太大,可能不敢出来。一方面长、沈之敌可能不出来,另一方面锦、榆、唐诸点及其附近之敌(十八个旅)则因你们去的兵力过小,可能收缩于锦、唐两点,变为不甚好打而又不得不打,费时费力,这样就有可能使自己陷入被动地位。不如置长、沈两敌于不顾,专顾锦、榆、唐一头为适宜。再则,今年九月至明年六月的十个月内,你们要准备进行三次大战役,每次准备费去两个月左右时间,共费去六个月左右时间,余四个月作为休息时间。如果在你们进行锦、榆、唐战役(第一个大战役)期间,长、沈之敌倾巢援锦(因为你们主力不是位于新民而是位于锦州附近,卫立煌才敢于来援),则你们便可以不离开锦、榆、唐线连续大举歼灭援敌,争取将卫立煌全军就地歼灭。这是最理想的情况。于此,你们应当注意:(一)确立攻占锦、榆、唐三点并全部控制该线的决心。(二)确立打你们前所未有的大歼灭战的决心,即在卫立煌全军来援的时候敢于同他作战。(三)为适应上述两项决心,重新考虑作战计划并筹办全军军需(粮食、弹药、新兵等)和处理俘虏事宜。以上意见望考虑电复。\\
\subsection*{\myformat{二 十月十日的电报}}
(一)从你们开始攻击锦州之日起,一个时期内是你们战局紧张期间,望你们每两日或每三日以敌情(锦州守敌之抵抗能力,葫芦岛、锦西援敌和沈阳援敌之进度,长春敌军之动态)我情(攻城进度,攻城和阻援之伤亡程度)电告我们一次。\\
  (二)这一时期的战局,很有可能如你们曾经说过的那样,发展成为极有利的形势,即不但能歼灭锦州守敌,而且能歼灭葫、锦援敌之一部,而且能歼灭长春逃敌之一部或大部。如果沈阳援敌进至大凌河以北地区,恰当你们业已攻克锦州、使你们有可能转移兵力将该敌加以包围的话,那就也可能歼灭沈阳援敌。这一切的关键是争取在一星期内外攻克锦州。\\
  (三)按照我军攻击锦州的进度和东西两路援敌的进度,决定阻援部署的方法。如果沈阳援敌进得较慢(如果长春之敌在你们攻锦过程中突围,并被我十二纵等部抓住歼击,则沈阳援敌可能被麻痹,进得较慢,或停止不进,或回头救援长春之敌),葫、锦援敌进得较快,则你们应准备以总预备队加入四纵、十一纵方面歼灭该敌一部,首先停止该敌之前进。如果葫、锦援敌被我四纵、十一纵等部所钳制和阻止而进得很慢或停止不进,长春之敌没有突围,沈阳援敌进得较快,而锦州之敌业已大部被歼,全城已接近于攻克,则你们应使沈敌深入大凌河以北,以便及时转移兵力包围该敌,然后徐图歼击。\\
  (四)你们的中心注意力必须放在锦州作战方面,求得尽可能迅速地攻克该城。即使一切其它目的都未达到,只要攻克了锦州,你们就有了主动权,就是一个伟大的胜利。前面所说各点,只是希望你们予以相当的注意。尤其在锦州作战的头几天内,东西援敌不会大动,你们要用全部精力注于锦州方面之作战。\\
\newpage\section*{\myformat{关于健全党委制}\\\myformat{(一九四八年九月二十日)}}\addcontentsline{toc}{section}{关于健全党委制}
\begin{introduction}\item  这是毛泽东为中共中央起草的决定。关于这个文件的意义,邓小平一九五六年九月十六日在中国共产党第八次全国代表大会上所作的《关于修改党的章程的报告》说:“在我们党内,从长时期以来,由党的集体而不由个人决定重大的问题,已经形成一个传统。违背集体领导原则的现象虽然在党内经常发生,但是这种现象一经发现,就受到党中央的批判和纠正。中央在一九四八年九月关于健全党委制的决定,对于加强党的集体领导,尤其起了重大的作用。……这个决定在全党实行了,并且直到现在仍然保持着它的效力。……这个决定的重要意义,在于它总结了党内认真实行集体领导的成功的经验,促使那些把集体领导变为有名无实的组织纠正自己的错误,并且扩大了实行集体领导的范围。”\end{introduction}
党委制是保证集体领导、防止个人包办的党的重要制度。近查有些(当然不是一切)领导机关,个人包办和个人解决重要问题的习气甚为浓厚。重要问题的解决,不是由党委会议做决定,而是由个人做决定,党委委员等于虚设。委员间意见分歧的事亦无由解决,并且听任这些分歧长期地不加解决。党委委员间所保持的只是形式上的一致,而不是实质上的一致。此种情形必须加以改变。今后从中央局至地委,从前委至旅委以及军区(军分会或领导小组)、政府党组、民众团体党组、通讯社和报社党组,都必须建立健全的党委会议制度,一切重要问题(当然不是无关重要的小问题或者已经会议讨论解决只待执行的问题)均须交委员会讨论,由到会委员充分发表意见,做出明确决定,然后分别执行。地委、旅委以下的党委亦应如此。高级领导机关的部(例如宣传部、组织部)、委(例如工委、妇委、青委)、校(例如党校)、室(例如研究室),亦应有领导分子的集体会议。当然必须注意每次会议时间不可太长,会议次数不可太频繁,不可沉溺于细小问题的讨论,以免妨碍工作。在会议之前,对于复杂的和有分歧意见的重要问题,又须有个人商谈,使委员们有思想准备,以免会议决定流于形式或不能做出决定。委员会又须分别为常委会和全体会两种,不可混在一起。此外,还须注意,集体领导和个人负责,二者不可偏废。军队在作战时和情况需要时,首长有临机处置之权。\\
\newpage\section*{\myformat{中共中央关于九月会议的通知}\\\myformat{(一九四八年十月十日)}}\addcontentsline{toc}{section}{中共中央关于九月会议的通知}
\begin{introduction}\item  这是毛泽东为中共中央起草的对党内的通知。一九四八年九月会议是在河北省平山县西柏坡村召集的。它是日本投降以来到会人数最多的一次中央会议,因为在这以前,绝大多数中央委员都分散在各个解放区从事紧张的解放战争,交通十分困难,不可能举行这样大的会议。\end{introduction}
(一)一九四八年九月,中央召开了一次政治局会议,到政治局委员七人,有中央委员和候补中央委员十四人、重要工作人员十人参加,其中有华北、华东、中原、西北的党和军队的主要负责同志。这是从日本投降以来到会人数最多的一次中央会议。会议检查了过去时期的工作,规定了今后时期的工作任务。\\
  (二)一九四五年四月党的第七次全国代表大会以后,中央委员会和全党领导骨干,表现了比较抗日时期更为良好的团结。这种团结,使得我党能够应付日本投降以后整三年内国际国内所发生的许多重大事变,并在这些事变中使中国革命向前推进了一大步,摧毁了美帝国主义在中国广大人民中的政治影响,抵抗了国民党的再一次叛变\footnote[1]{ 国民党的第一次叛变是在一九二七年。这里所说的“再一次叛变”,是指抗日战争结束后国民党发动反革命的全面内战。},打退了它的军事进攻,使人民解放军由防御转到了进攻。\\
  在一九四六年七月至一九四八年六月的两年作战中,人民解放军歼敌二百六十四万人,其中俘敌一百六十三万人。两年主要缴获,计有步枪近九十万枝,重轻机枪六万四千余挺,小炮八千余门,步兵炮五千余门,山野重炮一千一百余门。两年中人民解放军由一百二十余万人增加到了二百八十万人。其中正规军由一百一十八个旅增加到了一百七十六个旅,正规军人数由六十一万增加到了一百四十九万。解放区现有面积二百三十五万平方公里,占全国面积九百五十九万七千平方公里的百分之二十四点五;现已有人口一亿六千八百万,占全国人口四亿七千五百万的百分之三十五点三;现有县城以上大中小城市五百八十六座,占全国城市二千零九座的百分之二十九。\\
  由于我党坚决领导农民实现了土地制度的改革,现已在大约一万万人口的区域彻底解决了土地问题,地主阶级和旧式富农的土地大致平均地分配给了农村人民,首先是贫雇农。\\
  我党党员由一九四五年五月的一百二十一万,增加到了现在的三百万(我党党员一九二七年国民党叛变以前为五万人,一九二七年国民党叛变以后降为大约一万人左右,一九三四年因土地革命顺利发展升至三十万人,一九三七年因南方革命失败\footnote[2]{ 南方革命失败,指以陈绍禹为代表的“左”倾冒险主义错误造成的一九三四年红军第五次反“围剿”战争的失败,和红军主力退出南方各革命根据地。}降为大约四万人左右,一九四五年因抗日战争顺利发展增至一百二十一万人,现在因反蒋战争和土地革命顺利发展又增至三百万人)。党在最近一年内,一方面基本上克服了并正在继续克服着党内在某种程度上存在着的成分不纯(地主富农分子)、思想不纯(地主富农思想)和作风不纯(官僚主义和命令主义)的不良现象,另一方面又克服了和正在继续克服着跟着大规模发动农民群众解决土地问题的斗争而产生的,部分地但是相当多地侵犯了中农,破坏了某些私人工商业,以及某些地方越出了镇压反革命的某些政策界限等项“左”的错误。经过过去三年、特别是最近一年的伟大的激烈的革命斗争,和对于自己错误的认真的纠正,全党的政治成熟程度是大进一步了。\\
  党在国民党区域的工作,有了很大的成绩,这表现在各大城市中争取了广大的工人、学生、教员、教授、文化人、市民和民族资本家站在我党方面,争取了一切民主党派、人民团体站在我党方面,抗拒了国民党的压迫,使国民党完全陷于孤立。在南方几个大区域内(闽粤赣边区,湘粤赣边区,粤桂边区,桂滇边区,云南南部,皖浙赣边区和浙江东部南部)建立了游击战争根据地,使这些地区的游击部队发展到了三万余人。\\
  两年内,特别是最近一年内,在人民解放军中,实行了有秩序的、有领导的、由全体战斗员和指挥员一起参加的民主运动,开展了自我批评,克服了和正在继续克服着军队中的官僚主义,恢复了在一九二七年至一九三二年期间曾经实行有效、而在后来被取消了的军队中的各级党委制和连队中的战士委员会制,这样就使军队指战员的政治积极性和自觉性大为提高,战斗力和纪律性大为增强,溶化了大约八十万左右从国民党军队来的俘虏兵,使他们变为解放战士\footnote[3]{ 指被人民解放军俘虏而从国民党反动军队中解放出来、经过教育、参加人民解放军的原国民党军士兵。},掉转枪口打国民党。两年内,从解放区动员了大约一百六十万左右分得了土地的农民参加人民解放军。\\
  我们现在已经有了相当多的铁路、矿山和工业,我党正在大规模地学习管理工业和做生意。两年内,我们的军事工业,有了相当大的增长。但是还不足以应付战争的需要。我们缺乏若干重要的原料和机器,我们基本上还不能炼钢。\\
  我们已在华北四千四百万人口的区域建立了统一的党和党外民主人士合作的人民政府,并决定由这个政府将华北、华东(有人口四千三百万)和西北(有人口七百万)三区的经济、财政、贸易、金融、交通和军事工业的领导和管理工作统一起来,以利支持前线,并且准备在不久的将来,将东北和中原两区的上述工作也统一起来。\\
  (三)中央会议,根据过去两年作战的成绩和整个敌我形势,认为建设五百万人民解放军,在大约五年左右的时间内(从一九四六年七月算起)歼敌正规军共五百个旅(师)左右(平均每年一百个旅左右),歼敌正规军、非正规军和特种部队共七百五十万人左右(平均每年一百五十万人左右),从根本上打倒国民党的反动统治,是有充分可能性的。\\
  国民党的军事力量,在一九四六年七月为四百三十万人,两年被歼和逃亡三百零九万人,补充二百四十四万人,现有三百六十五万人。估计今后三年尚能补充三百万人,今后三年被歼和逃亡可能达到四百五十万人左右。这样,五年作战结果,国民党的军事力量可能只剩下二百万人左右了。我军现有二百八十万人,今后三年准备收容俘虏参加我军一百七十万人(以占俘虏全数百分之六十计算),动员农民参军二百万人,除去消耗,五年作战结果,我军可能接近五百万人。如果五年作战出现了这样的结果,就可以说国民党的反动统治已经从根本上被我们打倒了。\\
  为了实现这一任务,必须每年歼敌正规军一百个旅(师)左右,五年共歼敌正规军五百个旅(师)左右。这是解决一切问题的关键。我们第一年歼敌正规军折合成九十七个旅(师),第二年歼敌正规军折合成九十四个旅(师),根据这一情形看来,这样的目标是可能达到并且可能超过的。国民党现有全部军事力量三百六十五万人中的百分之七十是在第一线(长江和巴山山脉之线以北,兰州和贺兰山脉之线以东,承德和长春之线以南),在其后方者(包括长江和巴山山脉之线以南,兰州和贺兰山脉之线以西)仅有大约百分之三十。国民党现有全部正规军二百八十五个旅,一百九十八万人,其中在第一线者二百四十九个旅,一百七十四万二千人(北线九十九个旅,六十九万四千人,南线一百五十个旅,一百零四万八千人),在其后方者,仅有三十六个旅,二十三万八千人,并且大部分是新建立的部队,缺乏战斗力。因此中央决定人民解放军第三年仍然全部在长江以北和华北、东北作战。为着执行歼敌任务,除有计划地谨慎地从解放区动员人民参军外,必须大量利用俘虏。\\
  (四)由于我党我军在过去长时期内是处于被敌人分割的、游击战争的并且是农村的环境之下,我们曾经允许各地方党的和军事的领导机关保持着很大的自治权,这一种情况,曾经使得各地方的党组织和军队发挥了他们的自动性和积极性,渡过了长期的严重的困难局面,但在同时,也产生了某些无纪律状态和无政府状态,地方主义和游击主义,损害了革命事业。目前的形势,要求我党用最大的努力克服这些无纪律状态和无政府状态,克服地方主义和游击主义,将一切可能和必须集中的权力集中于中央和中央代表机关\footnote[4]{ 这里所说的中央代表机关,是指中央局和中央分局。}手里,使战争由游击战争的形式过渡到正规战争的形式。过去两年中,军队和作战的正规性是增长了一步,但是还不够,必须在第三年内再进一大步。为此目的,必须尽一切可能修理和掌握铁路、公路、轮船等近代交通工具,加强城市和工业的管理工作,使党的工作的重心逐步地由乡村转到城市。\\
  (五)夺取全国政权的任务,要求我党迅速地有计划地训练大批的能够管理军事、政治、经济、党务、文化教育等项工作的干部。战争的第三年内,必须准备好三万至四万下级、中级和高级干部,以便第四年内军队前进的时候,这些干部能够随军前进,能够有秩序地管理大约五千万至一万万人口的新开辟的解放区。中国地方甚大,人口甚多,革命战争发展甚快,而我们的干部供应甚感不足,这是一个很大的困难。第三年内干部的准备,虽然大部分应当依靠老的解放区,但是必须同时注意从国民党统治的大城市中去吸收。国民党区大城市中有许多任务人和知识分子能够参加我们的工作,他们的文化水准较之老解放区的工农分子的文化水准一般要高些。国民党经济、财政、文化、教育机构中的工作人员,除去反动分子外,我们应当大批地利用。解放区的学校教育工作,必须恢复和发展。\\
  (六)召集政治协商会议的口号\footnote[5]{ 召开政治协商会议的口号是毛泽东提出的。在一九四八年四月三十日中共中央发布的《纪念“五一”劳动节口号》中,根据毛泽东的提议,提出:“各民主党派、各人民团体、各社会贤达迅速召开政治协商会议,讨论并实现召集人民代表大会,成立民主联合政府。”这个口号立即得到各民主党派、人民团体和无党派民主人士的热烈响应。政治协商会议,后来称为新政治协商会议,以后又改称中国人民政治协商会议,参见本卷《在新政治协商会议筹备会上的讲话》注〔1〕。},团结了国民党区域一切民主党派、人民团体和无党派民主人士于我党周围。现在,我们正在组织国民党区域的这些党派和团体的代表人物来解放区,准备在一九四九年召集中国一切民主党派、人民团体和无党派民主人士的代表们开会,成立中华人民共和国临时中央政府。\\
  (七)恢复和发展解放区的工业生产和农业生产,是支援战争、战胜国民党反动派的重要环节。中央会议认为,必须一方面使人民解放军向国民党区域发展胜利的进攻,将战争所需要的人力资源和物力资源大量地从国民党方面和国民党区域去取给;另一方面,必须用一切努力恢复和发展老解放区的工业生产和农业生产,使之较现有的水平有若干的增长。只有这两方面的任务都完成了,才能够保证打倒国民党反动统治,否则是不可能的。\\
  执行这两方面的任务,我们有很多的困难。大军进入国民党区域执行无后方的或半有后方的作战,一切军事需要必须全部地或大部地就地自己解决。而恢复和发展工业生产和农业生产则需要有较好的组织工作,很好地领导解放区内部的市场和管制对外贸易,解决某些机器和原料缺乏的问题,首先是解决交通运输和修理铁路、公路、河道的问题。目前解放区的经济状况和财政状况,存在着很大的困难,虽然我们的困难比较国民党的困难要小得多,但是确实有困难。这主要是物资和兵员不足供应战争的需要,通货膨胀已到了相当大的程度,而我们的组织工作特别是财经方面的组织工作不够,则是形成这种困难的原因之一。我们相信这些困难是能够克服的,并且必须克服这些困难。在克服困难的斗争中,必须反对浪费,厉行节约:在前线注意缴获归公,爱护自己的有生力量,爱护武器,节省弹药,保护俘虏;在后方,减少国家机构的开支,减少不急需的人力和畜力的动员,减少开会时间,注意农业的季节,不违农时,节省工业生产的成本,提高劳动生产率,全党动员学习管理工业生产、农业生产和做生意,尽可能地将各解放区的经济加以适当的组织,克服市场上的盲目性,并同一切投机操纵的分子进行必要的斗争。从这一切着手,我们就必能克服自己面前的困难。\\
  (八)提高干部的理论水平,扩大党内的民主生活,成为完成上述任务的重要环节。中央会议已通过关于扩大党内民主生活的专门决议\footnote[6]{ 指《中共中央关于召开党的各级代表大会和代表会议的决议》。这个决议,对于建立和扩大党内的正常民主生活问题作了以下的规定:各级党委要按照党章定期召开党的各级代表大会和代表会议。对于这种会议要赋予党章所规定的一切权力,不许侵犯。开会要有充分的准备。党内有不同意见的争论要及时地、真实地向上级报告,其中重要的争论并须报告中央。此外,还规定要健全党委制,各级党委必须实行重要问题经党委集体讨论和作出决定的制度,不应当由个人决定重要问题,但是集体领导和个人负责不可偏废。}。关于提高干部理论水平的问题也进行了讨论,并引起了到会同志的注意。\\
  (九)全国第六次劳动大会已经胜利地召开,并成立了中华全国总工会\footnote[7]{ 全国第六次劳动大会,是一九四八年八月在哈尔滨召开的。在这次大会上,恢复了中国工人阶级统一的全国组织中华全国总工会。前五次全国劳动大会,先后举行于一九二二年、一九二五年、一九二六年、一九二七年、一九二九年。}。明年上半年,将召开全国妇女代表大会,成立全国民主妇女联合会\footnote[8]{ 中国妇女第一次全国代表大会,是一九四九年三月在北平召开的。在这次大会上,成立了全国妇女群众组织的领导机构中华全国民主妇女联合会。一九五七年九月中国妇女第三次全国代表大会决定,中华全国民主妇女联合会改称中华全国妇女联合会。};将召开全国青年代表大会,成立全国青年联合会\footnote[9]{ 中华全国青年第一次代表大会,是一九四九年五月在北平召开的。在这次大会上,成立了中华全国民主青年联合总会。一九五八年四月中华全国青年第三次代表大会决定,中华全国民主青年联合总会改称中华全国青年联合会。};并将建立新民主主义青年团\footnote[10]{ 中国新民主主义青年团,是一九四九年一月由中共中央正式决定建立的。它的第一次全国代表大会,于一九四九年四月在北平召开。一九五七年五月团的第三次全国代表大会决定,中国新民主主义青年团改名为中国共产主义青年团。}。
\newpage\section*{\myformat{关于淮海战役的的作战方针}\\\myformat{(一九四八年十月十一日)}}\addcontentsline{toc}{section}{关于淮海战役的的作战方针}
\begin{introduction}\item  这是毛泽东为中共中央军事委员会起草的给华东野战军的电报。这个电报同时告知华东局和中原局。此后,全国局势发生了对中国人民解放军有利的急剧变化,中原野战军奉命东进徐州、蚌埠地区,由华东野战军、中原野战军和华东、中原的地方部队共同作战。战役发起后,毛泽东又为中央军委起草致华东野战军并告中原野战军的电报,下达在徐州附近歼灭国民党军刘峙集团主力的战役决心。于是,淮海战役发展成为中国人民解放战争中三个有决定意义的最大战役之一。这个战役共歼灭国民党军五十五万五千多人。毛泽东在这个电报中所提出的战役方针得到完全的成功,只是战役的发展比预计的更为顺利,因而胜利也比预计的更快更大。在这个战役以后,国民党反动政府的首都南京就处在人民解放军的直接威胁之下了。淮海战役在一九四九年一月十日结束,一月二十一日蒋介石即宣告“引退”,南京国民党反动统治集团从此陷入土崩瓦解的状态。\end{introduction}
关于淮海战役\footnote[1]{ 毛泽东在起草这个电报的时候,对淮海战役确定的作战任务,主要是消灭国民党军刘峙集团主力的一部,开辟苏北战场,使山东、苏北打成一片。后来,由于局势发生变化,中原野战军东进与华东野战军会合,共同作战,淮海战役遂发展成为以徐州为中心,东起海州,西止商丘,北起临城(今薛城),南达淮河的广大地区同国民党军进行的一次决定性的战役。集结在上述地区的国民党军队有徐州“剿总”总司令刘峙、副总司令杜聿明指挥下的四个兵团和三个绥靖区部队,连同以后从华中增援的黄维兵团,共五个兵团和三个绥靖区部队。人民解放军参加这次战役的有华东野战军十六个纵队,中原野战军七个纵队,华东、中原军区和华北军区所属冀鲁豫军区的地方武装,共六十余万人。在淮海战役过程中,中央军委决定由刘伯承、陈毅、邓小平、粟裕、谭震林组成总前委,邓小平为书记,执行领导淮海前线军事和作战的职权。战役自一九四八年十一月六日开始到一九四九年一月十日结束,历时六十五天,共歼灭国民党军五十五万五千人,此外还击退了由南京方面来援的刘汝明、李延年两个兵团,基本上解放了长江以北的华东、中原地区。整个战役,共分三个阶段。第一阶段,从十一月六日到二十二日,人民解放军于徐州以东新安镇碾庄地区,围歼了黄百韬兵团(黄百韬毙命),解放了碾庄以东陇海路两侧和徐州以西以北广大地区,切断了津浦路徐(州)蚌(埠)段间国民党军的联系。国民党第三绥靖区所属三个半师,共二万三千人,在徐州东北的贾汪、台儿庄地区起义。第二阶段,从十一月二十三日到十二月十五日,人民解放军在宿县西南双堆集地区围歼了黄维兵团,生俘兵团司令官黄维、副司令官吴绍周。该兵团一个师起义。同时,华东野战军将由徐州西逃的杜聿明指挥下的邱清泉、李弥、孙元良三个兵团包围于永城东北的青龙集、陈官庄地区,随后歼灭了力图突围的孙元良兵团,孙元良只身潜逃。第三阶段,从一九四八年十二月十六日到一九四九年一月十日,人民解放军淮海前线部队首先进行了二十天的战场休整。从一九四九年一月六日起,对青龙集、陈官庄地区被围的国民党军发起总攻,全歼邱清泉、李弥两个兵团,生俘杜聿明,击毙邱清泉,只有李弥逃脱。至此,规模巨大的淮海战役胜利结束。}部署,现在提出几点意见,供你们考虑。\\
  (一)本战役第一阶段的重心,是集中兵力歼灭黄百韬兵团,完成中间突破,占领新安镇、运河车站、曹八集、峄县、枣庄、临城、韩庄、沭阳、邳县、郯城、台儿庄、临沂等地。为达到这一目的,应以两个纵队担任歼灭敌一个师的办法,共以六个至七个纵队,分割歼灭敌二十五师、六十三师、六十四师。以五个至六个纵队,担任阻援和打援。以一个至二个纵队,歼灭临城、韩庄地区李弥部一个旅,并力求占领临韩,从北面威胁徐州,使邱清泉、李弥两兵团不敢以全力东援。以一个纵队,加地方兵团,位于鲁西南,侧击徐州、商丘段,以牵制邱兵团一部(孙元良三个师现将东进,望刘伯承、陈毅、邓小平即速部署攻击郑徐线牵制孙兵团)。以一个至二个纵队,活动于宿迁、睢宁、灵壁地区,以牵制李兵团。以上部署,即是说要用一半以上兵力,牵制、阻击和歼敌一部,以对付邱李两兵团,才能达到歼灭黄兵团三个师的目的。这一部署,大体如同九月间攻济打援\footnote[2]{ 攻济打援,是指一九四八年九月人民解放军在济南战役中所采取的作战方法。济南是津浦、胶济两铁路的交会点和连结华东、华北地区的战略要地,国民党以第二绥靖区的十万余人守备济南。同时,准备以配置在徐州地区的主力三个兵团,约十七万人,随时北援。为了以多数兵力歼灭援敌,以保证夺取济南,华东野战军以七个纵队,约十四万人,组成攻城集团,担负攻克济南的任务;以八个纵队,约十八万人,组成打援集团,随时准备阻击和歼灭沿津浦路北援的敌人。人民解放军于一九四八年九月十六日晚,对济南守敌发起攻击。经八昼夜连续攻击,于二十四日全歼守敌(内有二万余人起义),生俘国民党第二绥靖区司令官王耀武。由于人民解放军打援力量的强大和迅速达到攻济目的,徐州之敌未敢北援。}的部署,否则不能达到歼灭黄兵团三个师的目的。第一阶段,力争在战役开始后两星期至三星期内结束。\\
  (二)第二阶段,以大约五个纵队,攻歼海州、新浦、连云港、灌云地区之敌,并占领各城。估计这时,青岛之五十四师、三十二师很有可能由海运增至海、新、连地区\footnote[3]{ 后来该敌没有敢来。}。该地区连原有一个师将共有三个师,故我须用五个纵队担任攻击,而以其余兵力(主力)担任钳制邱李两兵团,仍然是九月间攻济打援部署的那个原则。此阶段亦须争取于两个至三个星期内完结。\\
  (三)第三阶段,可设想在两淮方面\footnote[4]{ 指淮阴、淮安一带。}作战。那时敌将增加一个师左右的兵力(整八师正由烟台南运),故亦须准备以五个纵队左右的兵力去担任攻击,而以其余主力担任打援和钳制。此阶段,大约亦须有两个至三个星期。\\
  三个阶段大概共须有一个半月至两个月的时间。\\
  (四)你们以十一、十二两月完成淮海战役。明年一月休整\footnote[5]{ 一九六〇年出版本卷第一版时,此处删去“二月西兵团转移”一句。}。三至七月同刘邓协力作战,将敌打至江边各点固守。秋季你们主力大约可以举行渡江作战\footnote[6]{ 毛泽东起草的这个电报,原共五点,这一段文字是其中的第五点。一九六〇年出版本卷第一版时,删去了第四点,其原文为:“淮海战役的结果,将是开辟了苏北战场,山东苏北打成一片,邱李两兵团固守徐蚌一线及其周围,使我难于歼击。此时,你们仍应分为东西两兵团。以大约五个纵队组成东兵团,在苏北苏中作战。以其余主力为西兵团,出豫皖两省,协同刘邓,攻取菏泽、开封、郑州、确山、信阳、南阳、淮河流域及大别山各城。西兵团与刘邓协力作战的方法,亦是一部兵力打城,以主要兵力打援阻援,这样去各个歼敌。刘邓因为兵力不足,不能实现如像你们攻济打援战役及淮海战役那样的作战。你们西兵团去后,就可以实现那样的作战。六七两月开封睢杞战役就是西兵团与刘邓协力的结果。”}。\\
\newpage\section*{\myformat{全世界革命力量团结起来,反对帝国主义的侵略}\\\myformat{(一九四八年十一月)}}\addcontentsline{toc}{section}{全世界革命力量团结起来,反对帝国主义的侵略}
\begin{introduction}\item  这是毛泽东给欧洲共产党和工人党情报局机关刊物《争取持久和平,争取人民民主!》所写的纪念十月革命三十一周年的论文。这篇论文发表在该刊一九四八年第二十一期。\end{introduction}
现在,当着全世界觉悟的工人阶级和一切真诚革命的人们对于苏联伟大的十月社会主义革命第三十一个周年举行欢欣鼓舞的纪念的时候,我想起了斯大林在一九一八年,在十月革命第一个周年纪念的时候所写的著名的论文。斯大林在这篇论文中说:“十月革命的伟大的世界意义,主要的是:第一,它扩大了民族问题的范围,把它从欧洲反对民族压迫的斗争的局部问题,变为各被压迫民族、各殖民地及半殖民地从帝国主义之下解放出来的总问题;第二,它给这一解放开辟了广大的可能性和现实的道路,这就大大地促进了西方和东方的被压迫民族的解放事业,把他们吸引到胜利的反帝国主义斗争的巨流中去;第三,它从而在社会主义的西方和被奴役的东方之间架起了一道桥梁,建立了一条从西方无产者经过俄国革命到东方被压迫民族的新的反对世界帝国主义的革命战线。”\footnote[1]{ 见斯大林《十月革命和民族问题》(《斯大林选集》上卷,人民出版社1979年版,第126页)。}\\
  历史是按照斯大林所指出的方向发展的。十月革命给世界人民解放事业开辟了广大的可能性和现实的道路,十月革命建立了一条从西方无产者经过俄国革命到东方被压迫民族的新的反对世界帝国主义的革命战线。这条革命战线是在列宁,而在列宁死后是在斯大林的英明的指导之下建立起来和发展起来的。\\
  既要革命,就要有一个革命党。没有一个革命的党,没有一个按照马克思列宁主义的革命理论和革命风格建立起来的革命党,就不可能领导工人阶级和广大人民群众战胜帝国主义及其走狗。自从马克思主义产生以来的一百多年的时间内,只是在有了俄国布尔什维克领导十月革命、领导社会主义建设和战胜法西斯侵略的榜样的时候,才在世界范围内建立了和发展了新式的革命党。自从有了这样的革命党,世界革命的面目就起了变化了。这个变化是如此巨大,以至使老一辈的人们完全不能设想的变革,都轰轰烈烈地出现了。中国共产党就是依照苏联共产党的榜样建立起来和发展起来的一个党。自从有了中国共产党,中国革命的面目就焕然一新了。这个事实难道还不明显吗?\\
  以苏联为首的世界革命统一战线,战胜了法西斯主义的德意日。这是十月革命的结果。假如没有十月革命,假如没有苏联共产党,没有苏联,没有苏联领导的西方和东方的反对帝国主义的革命统一战线,还能设想战胜法西斯德意日及其走狗们吗?如果说,十月革命给全世界工人阶级和被压迫民族的解放事业开辟了广大的可能性和现实的道路,那末,反法西斯的第二次世界大战的胜利,就是给全世界工人阶级和被压迫民族的解放事业开辟了更加广大的可能性和更加现实的道路。对于第二次世界大战的胜利的意义估计不足,将是一个极大的错误。\\
  第二次世界大战胜利以后,代替法西斯德意日的地位而疯狂地准备着新的世界战争、威胁全世界的美国帝国主义及其在各国的走狗们,反映了资本主义世界的极端腐败及其濒于灭亡的恐怖情绪。这个敌人还是有力量的,因此,每一个国家内部的一切革命力量必须团结起来,一切国家的革命力量必须团结起来,必须组成以苏联为首的反对帝国主义的统一战线,并遵循正确的政策,否则就不能胜利。这个敌人的基础是虚弱的,它的内部分崩离析,它脱离人民,它有无法解脱的经济危机,因此,它是能够被战胜的。对于敌人力量的过高估计和对于革命力量的估计不足,将是一个极大的错误。\\
  在中国共产党领导之下的以反对美国帝国主义对于中国的疯狂侵略,反对卖国、独裁和以内战屠杀中国人民的国民党反动政府为目标的伟大的中国人民民主革命,现在已经取得了巨大的胜利。中国共产党领导的人民解放军,在一九四六年七月至一九四八年六月的两年时间内,已经打退了国民党反动政府的四百三十万军队的进攻,并使自己由防御转到了进攻。在两年作战中(一九四八年七月以后的发展,尚未计算在内),人民解放军俘虏和消灭了国民党军队二百六十四万人。中国解放区现有面积二百三十五万平方公里,占全国面积九百五十九万七千平方公里的百分之二十四点五;现有人口一亿六千八百万,占全国人口四亿七千五百万的百分之三十五点三;现有城市五百八十六座,占全国城市二千零九座的百分之二十九。由于我党坚决地领导农民实现了土地制度的改革,现已在大约一亿人口的区域彻底地解决了土地问题,地主阶级和旧式富农的土地大致平均地分配给了农民,首先是贫农和雇农。中国共产党的党员,由一九四五年的一百二十一万人,增加到了现在的三百万人。中国共产党的任务,是在全国范围内团结一切革命力量,驱逐美国帝国主义的侵略势力,打倒国民党的反动统治,建立统一的民主的人民共和国。我们知道,我们面前还有许多困难。但是,我们不怕这些困难。我们认为困难是必须克服,并且能够克服的。\\
  十月革命的光芒照耀着我们。苦难的中国人民必须求得解放,并且他们坚信是能够求得解放的。一向孤立的中国革命斗争,自从十月革命胜利以后,就不再感觉孤立了。我们有全世界的共产党和工人阶级的援助。这一点,中国革命的先行者孙中山先生是理解的,他确定了联合苏联反对帝国主义的政策。他在临终的时候,还写了一封给苏联的信,当作他的一份遗嘱。背叛孙中山的政策、站在帝国主义反革命战线方面、反对自己国家的人民的,是国民党的蒋介石匪帮。但是人们不要很久就可以看到,国民党的全部反动统治将被中国人民所彻底地打碎。中国人民是勇敢的,中国共产党也是勇敢的,他们一定要解放全中国。\\
\newpage\section*{\myformat{中国军事形势的重大变化}\\\myformat{(一九四八年十一月十四日)}}\addcontentsline{toc}{section}{中国军事形势的重大变化}
\begin{introduction}\item  这是毛泽东为新华社写的一篇评论。在这篇评论里,毛泽东根据辽沈战役以后敌我力量变化的新形势,对于人民解放战争胜利的时间重新作了估计,指出从一九四八年十一月起,再有一年左右的时间,就可以打倒国民党的反动统治。后来的中国军事形势的发展,完全证明了毛泽东的这个预见。\end{introduction}
中国的军事形势现已进入一个新的转折点,即战争双方力量对比已经发生了根本的变化。人民解放军不但在质量上早已占有优势,而且在数量上现在也已经占有优势。这是中国革命的成功和中国和平的实现已经迫近的标志。\\
  国民党军队在战争的第二年底,即今年六月底,总数约计尚有三百六十五万人。这个数目,对于一九四六年七月国民党开始发动全国性内战时期的四百三十万人来说,是少了六十五万人。这是由于国民党军队在两年战争中虽然被歼、被俘和逃亡了大约三百零九万人(其中被歼、被俘为二百六十四万人),但在此期内又补充了约二百四十四万人,故亏短数尚只有六十五万人。最近则起了一个突变。经过战争第三年度的头四个月,即今年七月一日至十一月二日沈阳解放时,国民党军队即丧失了一百万人。四个月内国民党军队的补充情形尚未查明,假定它能补充三十万人,亏短数为七十万人。这样国民党的全部军队包括陆海空军、正规军非正规军、作战部队和后勤机关在内,现在只有二百九十万左右的人数。人民解放军,则由一九四六年六月的一百二十万人,增至一九四八年六月的二百八十万人,现在又增至三百余万人。这种情况,就使国民党军队在数量上长期占有的优势,急速地转入了劣势。这是由于四个月内人民解放军在全国各个战场英勇作战的结果,而特别是南线的睢杞战役\footnote[1]{ 睢杞战役,亦称豫东战役,是人民解放军在河南省东部的开封和睢县、杞县地区对国民党军进行的一次大规模战役。华东野战军八个纵队、中原野战军两个纵队以及冀鲁豫和豫皖苏军区部分兵力,在华东野战军代司令员兼代政治委员粟裕统一指挥下,于一九四八年六月十七日至二十二日,全歼开封守敌,共歼灭国民党军约四万人,击毙国民党军整编第六十六师师长李仲辛。蒋介石为了挽回其不利的战局,亲临开封上空督战,调集邱清泉、区寿年、黄百韬三个兵团,分路进攻开封。人民解放军于六月二十七日至七月六日,将区寿年兵团部及整编第七十五师以及黄百韬兵团的三个多团,先后包围于睢县、杞县地区,经九昼夜激战,歼敌五万余人,生俘兵团司令官区寿年和整编第七十五师师长沈澄年。}、济南战役\footnote[2]{ 济南战役,见本卷《关于淮海战役的作战方针》注〔2〕。},北线的锦州、长春、辽西、沈阳诸战役\footnote[3]{ 这里所说的锦州、长春、辽西、沈阳诸战役,统称辽沈战役。见本卷《关于辽沈战役的作战方针》注〔1〕。}的结果。国民党的正规军,因为它拚命地将非正规军编入正规军内,至今年六月底,尚有二百八十五个师的番号。四个月内,即被人民解放军歼灭了营以上部队合计共八十三个师,其中包括六十三个整师。\\
  这样,就使我们原来预计的战争进程,大为缩短。原来预计,从一九四六年七月起,大约需要五年左右时间,便可能从根本上打倒国民党反动政府。现在看来,只需从现时起,再有一年左右的时间,就可能将国民党反动政府从根本上打倒了。至于在全国一切地方消灭反动势力,完成人民解放,则尚需较多的时间。\\
  敌人是正在迅速崩溃中,但尚需共产党人、人民解放军和全国各界人民团结一致,加紧努力,才能最后地完全地消灭反动势力,在全国范围内建立统一的民主的人民共和国。\\
\newpage\section*{\myformat{关于平津战役\footnote[1]{ 平津战役,是一九四八年十一月二十九日至一九四九年一月三十一日由中国人民解放军东北野战军和华北军区第二、第三兵团及地方部队,在西起张家口,东至塘沽、唐山,包括北平、天津在内的地区同国民党军进行的一次具有决定意义的战役。中共中央决定由林彪、罗荣桓、聂荣臻组成总前委,领导平津前线人民解放军的一切行动。东北野战军在胜利地完成了解放东北全境的任务后,根据中央军委和毛泽东的指示,迅即挥师入关,和华北的人民解放军连同东北、华北参战的地方部队,总兵力约一百万人,合力围歼国民党华北“剿总”总司令傅作义指挥下的五十多万国民党军队。一九四八年十一月二十五日,华北军区第三兵团根据中央军委指示,自集宁地区东进,二十九日向张家口外围守敌发起进攻,形成对张家口包围之势,切断敌人西逃的道路。十二月二日,华北军区第二兵团向涿鹿地区急进,割断怀来、宣化间敌军的联系;东北野战军先遣兵团向南口、怀来急进,切断北平、怀来敌军的联系。毛泽东十二月十一日为中央军委起草的这个电报下达后,十二月中旬,人民解放军将敌分割包围于北平、天津、张家口、新保安、塘沽五个据点,截断了敌军南逃西撤的通路。十二月二十二日,围歼了新保安傅作义集团主力三十五军军部和二个师。二十四日,解放了张家口,全歼守敌五万四千余人。一九四九年一月十四日,包围天津的人民解放军发起总攻,经二十九小时激战,全歼守敌十三万余人,俘虏天津警备司令陈长捷,解放天津。至此,北平二十余万守敌,在人民解放军严密包围下完全陷于绝境。由于中国共产党的努力争取,经过谈判,北平守敌在傅作义率领下接受和平改编。一月三十一日人民解放军进入北平,北平宣告和平解放。在整个战役中,除塘沽守敌五万余人由海上逃跑外,人民解放军共歼灭和改编了国民党军队五十二万余人。绥远国民党军也于一九四九年九月通电起义,接受改编。}的作战方针}\\\myformat{(一九四八年十二月十一日)}}\addcontentsline{toc}{section}{关于平津战役\footnote[1]{ 平津战役,是一九四八年十一月二十九日至一九四九年一月三十一日由中国人民解放军东北野战军和华北军区第二、第三兵团及地方部队,在西起张家口,东至塘沽、唐山,包括北平、天津在内的地区同国民党军进行的一次具有决定意义的战役。中共中央决定由林彪、罗荣桓、聂荣臻组成总前委,领导平津前线人民解放军的一切行动。东北野战军在胜利地完成了解放东北全境的任务后,根据中央军委和毛泽东的指示,迅即挥师入关,和华北的人民解放军连同东北、华北参战的地方部队,总兵力约一百万人,合力围歼国民党华北“剿总”总司令傅作义指挥下的五十多万国民党军队。一九四八年十一月二十五日,华北军区第三兵团根据中央军委指示,自集宁地区东进,二十九日向张家口外围守敌发起进攻,形成对张家口包围之势,切断敌人西逃的道路。十二月二日,华北军区第二兵团向涿鹿地区急进,割断怀来、宣化间敌军的联系;东北野战军先遣兵团向南口、怀来急进,切断北平、怀来敌军的联系。毛泽东十二月十一日为中央军委起草的这个电报下达后,十二月中旬,人民解放军将敌分割包围于北平、天津、张家口、新保安、塘沽五个据点,截断了敌军南逃西撤的通路。十二月二十二日,围歼了新保安傅作义集团主力三十五军军部和二个师。二十四日,解放了张家口,全歼守敌五万四千余人。一九四九年一月十四日,包围天津的人民解放军发起总攻,经二十九小时激战,全歼守敌十三万余人,俘虏天津警备司令陈长捷,解放天津。至此,北平二十余万守敌,在人民解放军严密包围下完全陷于绝境。由于中国共产党的努力争取,经过谈判,北平守敌在傅作义率领下接受和平改编。一月三十一日人民解放军进入北平,北平宣告和平解放。在整个战役中,除塘沽守敌五万余人由海上逃跑外,人民解放军共歼灭和改编了国民党军队五十二万余人。绥远国民党军也于一九四九年九月通电起义,接受改编。}的作战方针}
\begin{introduction}\item  这是毛泽东为中共中央军事委员会起草的给林彪、罗荣桓等的电报。平津战役是中国人民解放战争中三个有决定意义的最大战役的最后一个。这个战役歼灭和改编了五十二万多国民党军,解放了北平、天津、张家口等重要城市,基本上结束了解放华北的战争。毛泽东在这里提出的战役方针,得到了完全的实现。\end{introduction}
一、张家口、新保安、怀来和整个北平、天津、塘沽、唐山诸敌,除某几个部队例如三十五军、六十二军、九十四军中的若干个别的师,在依靠工事保守时尚有较强的战斗力外,攻击精神都是很差的,都已成惊弓之鸟,尤其你们入关后是如此。切不可过分估计敌人的战斗力。我们有些同志过去都吃了过分估计敌人战斗力的亏,经过批评后他们也已懂得了。现在张家口、新保安两敌确已被围,大体上很难突围逃走。十六军约有一半迅速被歼。怀来敌一〇四军慌忙南逃,估计今日或明日可能被歼。该敌被歼后,你们准备以四纵由西南\footnote[2]{ 这里说的“西南”,是指南口西南地区。}向东北切断南口和北平间联系。估计此着不易实现,不是九十四军和十六军残部迅速撤回北平,就是九十四军、十六军和九十二军一起集中南口、昌平、沙河镇区域集团防守。但四纵此举直接威胁北平西北郊和北郊,可以钳制这些敌人不敢动。若这些敌人再敢西进接援三十五军,则可以直接切断其后路或直接攻北平,因此,这些敌人大约不敢再西进。我华北杨罗耿\footnote[3]{ 杨罗耿,指杨得志、罗瑞卿、耿飚。}兵团以九个师包围三十五军三个师,是绝对优势。他们提出早日歼灭该敌,我们拟要他们暂时不要打,以便吸引平津之敌不好下从海上逃走的决心。他们此次以两个纵队围住三十五军,以一个纵队阻住一〇四军,两敌都被击退。\\
  二、我们现在同意你们以五纵立即去南口附近,从东北面威胁北平、南口、怀柔诸敌。将来该纵即位于该地,以便将来(大约在十天或十五天之后,即在华北杨罗耿兵团歼灭三十五军之后)腾出四纵使用于东面。如此,请令五纵本日仍继续西进。\\
  三、三纵决不要去南口,该纵可按我们九日电开至北平以东、通县以南地区,从东面威胁北平,同四纵、十一纵、五纵形成对北平的包围。\\
  四、但我们的真正目的不是首先包围北平,而是首先包围天津、塘沽、芦台、唐山诸点。\\
  五、据我们估计,大约十二月十五日左右你们的十纵、九纵、六纵、八纵、炮纵、七纵就可集中于玉田为中心的地区。我们提议,十二月二十日至十二月二十五日数日内即取神速动作,以三纵(由北平东郊东调)、六纵、七纵、八纵、九纵、十纵等六个纵队包围天津、塘沽、芦台、唐山诸点之敌,如果诸点之敌那时大体仍如现时状态的话。其办法是以两个纵队位于以武清为中心的地区,即廊房、河西务、杨村诸点,以五个纵队插入天津、塘沽、芦台、唐山、古冶诸点之间,隔断诸敌之联系。各纵均须构筑两面阻击阵地,务使敌人不能跑掉,然后休整部队,恢复疲劳,然后攻歼几部分较小之敌。此时,四纵应由平西北移至平东。我华北杨罗耿兵团应于四纵移动之前歼灭新保安之敌。东面则应依情况,力争先歼塘沽之敌,控制海口。只要塘沽(最重要)、新保安两点攻克,就全局皆活了。以上部署,实际上是将张家口、新保安、南口、北平、怀柔、顺义、通县、宛平(涿县、良乡已被我占领)、丰台、天津、塘沽、芦台、唐山、开平诸点之敌一概包围了。\\
  六、此项办法,大体上即是你们在义县、锦州、锦西、兴城、绥中、榆关、滦县线上作战时期用过的办法\footnote[4]{ 东北野战军一九四八年九月在北宁线作战时,为不使义县、锦州、锦西、兴城、绥中、榆关(山海关)、滦县、昌黎之敌收缩集中,采用了先以一部分兵力分别包围和切断上述各点之敌,然后逐一歼灭的办法。}。\\
  七、从本日起的两星期内(十二月十一日至十二月二十五日)基本原则是围而不打(例如对张家口、新保安),有些则是隔而不围(即只作战略包围,隔断诸敌联系,而不作战役包围,例如对平、津、通州),以待部署完成之后各个歼敌。尤其不可将张家口、新保安、南口诸敌都打掉,这将迫使南口以东诸敌迅速决策狂跑,此点务求你们体会。\\
  八、为着不使蒋介石迅速决策海运平津诸敌南下,我们准备令刘伯承、邓小平、陈毅、粟裕于歼灭黄维兵团之后,留下杜聿明指挥之邱清泉、李弥、孙元良诸兵团(已歼约一半左右)之余部,两星期内不作最后歼灭之部署。\\
  九、为着不使敌人向青岛逃跑,我们准备令山东方面集中若干兵力控制济南附近一段黄河,并在胶济线上预作准备。\\
  十、敌向徐州、郑州、西安、绥远\footnote[5]{ 绥远,原来是一个省,一九五四年撤销,原辖地区划归内蒙古自治区。}诸路逃跑,是没有可能或很少可能的。\\
  十一、唯一的或主要的是怕敌人从海上逃跑。因此,在目前两星期内一般应采围而不打或隔而不围的办法。\\
  十二、此种计划出敌意外,在你们最后完成部署以前,敌人是很难觉察出来的。敌人现时可能估计你们要打北平。\\
  十三、敌人对于我军的积极性总是估计不足的,对于自己力量总是估计过高,虽然他们同时又是惊弓之鸟。平津之敌决不料你们在十二月二十五日以前能够完成上列部署。\\
  十四、为着在十二月二十五日以前完成上列部署,你们应该鼓励部队在此两星期内不惜疲劳,不怕减员,不怕受冻受饥,在完成上列部署以后,再行休整,然后从容攻击。\\
  十五、攻击次序大约是:第一塘芦区,第二新保安,第三唐山区,第四天津、张家口两区,最后北平区。\\
  十六、你们对上述计划意见如何?这个计划有何缺点?执行有何困难?统望考虑电告。\\
\newpage\section*{\myformat{敦促杜聿明等投降书}\\\myformat{(一九四八年十二月十七日)}}\addcontentsline{toc}{section}{敦促杜聿明等投降书}
\begin{introduction}\item  这是毛泽东为中原、华东两人民解放军司令部写的一个广播稿。\end{introduction}\\~\\
杜聿明将军、邱清泉将军、李弥将军和邱李两兵团诸位军长师长团长:\\
  你们现在已经到了山穷水尽的地步。黄维兵团已在十五日晚全军覆没,李延年兵团已掉头南逃,你们想和他们靠拢是没有希望了。你们想突围吗?四面八方都是解放军,怎么突得出去呢?你们这几天试着突围,有什么结果呢?你们的飞机坦克也没有用。我们的飞机坦克比你们多,这就是大炮和炸药,人们叫这些做土飞机、土坦克,难道不是比较你们的洋飞机、洋坦克要厉害十倍吗?你们的孙元良兵团已经完了,剩下你们两个兵团,也已伤俘过半。你们虽然把徐州带来的许多机关闲杂人员和青年学生,强迫编入部队,这些人怎么能打仗呢?十几天来,在我们的层层包围和重重打击之下,你们的阵地大大地缩小了。你们只有那么一点地方,横直不过十几华里,这样多人挤在一起,我们一颗炮弹,就能打死你们一堆人。你们的伤兵和随军家属,跟着你们叫苦连天。你们的兵士和很多干部,大家很不想打了。你们当副总司令的,当兵团司令的,当军长师长团长的,应当体惜你们的部下和家属的心情,爱惜他们的生命,早一点替他们找一条生路,别再叫他们作无谓的牺牲了。\\
  现在黄维兵团已被全部歼灭,李延年兵团向蚌埠逃跑,我们可以集中几倍于你们的兵力来打你们。我们这次作战才四十天,你们方面已经丧失了黄百韬十个师,黄维十一个师,孙元良四个师,冯治安四个师,孙良诚两个师,刘汝明一个师,宿县一个师,灵璧一个师,你们总共丧失了三十四个整师。其中除何基沣、张克侠率三个半师起义,廖运周率一个师起义,孙良诚率一个师投诚,赵壁光、黄子华各率半个师投诚\footnote[1]{ 何基沣、张克侠,都是国民党第三绥靖区副司令官,在淮海战役第一阶段中,于一九四八年十一月八日率一个军部和三个师、一个团共二万余人,在徐州东北贾汪、台儿庄地区起义。廖运周,是国民党第八十五军第一一〇师师长,在淮海战役第二阶段中,于一九四八年十一月二十七日率该师师部和两个整团共五千五百人,在安徽省宿县西南罗集起义。孙良诚,是国民党第一绥靖区副司令官兼第一〇七军军长,在淮海战役第一阶段中,于一九四八年十一月十三日率该军军部和一个师共五千八百人,在江苏睢宁西北投诚。赵壁光,是国民党第四十四军第一五〇师师长,在淮海战役第一阶段中,于一九四八年十一月十八日率残部二千余人在江苏徐州东碾庄地区投诚。黄子华,是国民党第八十五军第二十三师师长,在淮海战役第二阶段中,于一九四八年十二月十日率该师师部和所属残部及第二一六师一部、第八十五军部分直属部队一万余人,在安徽省蒙城东北双堆集投诚。}以外,其余二十七个半师,都被本军全部歼灭了。黄百韬兵团、黄维兵团和孙元良兵团的下场,你们已经亲眼看到了。你们应当学习长春郑洞国将军的榜样\footnote[2]{ 长春自东北人民解放军一九四七年冬季攻势后,即被东北人民解放军包围。一九四八年十月十九日,在人民解放军攻克锦州、东北国民党军全部动摇的形势下,长春国民党军最高指挥官东北“剿总”副总司令郑洞国,率领所部第一兵团直属机关部队和新七军全体官兵放下武器。},学习这次孙良诚军长、赵壁光师长、黄子华师长的榜样,立即下令全军放下武器,停止抵抗,本军可以保证你们高级将领和全体官兵的生命安全。只有这样,才是你们的唯一生路。你们想一想吧!如果你们觉得这样好,就这样办。如果你们还想打一下,那就再打一下,总归你们是要被解决的\footnote[3]{ 杜聿明(国民党徐州“剿总”副总司令)、邱清泉(国民党第二兵团司令官)、李弥(国民党第十三兵团司令官)三人,在人民解放军发出敦促其投降书以后,仍然负隅顽抗,结果在人民解放军强大攻势下全军覆没,杜聿明被俘,邱清泉被击毙,只有李弥逃走。}。\\
\begin{flushright}
中原人民解放军司令部\\
华东人民解放军司令部
\end{flushright}

\newpage\section*{\myformat{将革命进行到底}\\\myformat{(一九四八年十二月三十日)}}\addcontentsline{toc}{section}{将革命进行到底}
\begin{introduction}\item  这是毛泽东为新华社写的一九四九年新年献词。\end{introduction}
中国人民将要在伟大的解放战争中获得最后胜利,这一点,现在甚至我们的敌人也不怀疑了。\\
  战争走过了曲折的道路。国民党反动政府在发动反革命战争的时候,他们军队的数量约等于人民解放军的三倍半,他们军队的装备和人力物力的资源,更是远远地超过了人民解放军,他们拥有人民解放军所缺乏的现代工业和现代交通工具,他们获得美国帝国主义在军事上、经济上的大量援助,并且他们是经过了长期的准备的。就是因为这样,战争的第一年(一九四六年七月至一九四七年六月)表现为国民党的进攻和人民解放军的防御。国民党在一九四六年,在东北占领了沈阳、四平、长春、吉林、安东等城市和辽宁、辽北、安东等省\footnote[1]{ 一九四五年日本投降后,国民党政府曾将原东北三省(辽宁、吉林、黑龙江)划分为辽宁、辽北、安东、吉林、合江、松江、黑龙江、嫩江、兴安等九个省。一九四九年中国共产党领导下的东北行政委员会曾将东北三省行政区划调整为辽东、辽西、吉林、黑龙江、松江五个省,加上热河,统称为东北六省。一九五四年中央人民政府委员会决定将辽东省、辽西省合并改称辽宁省,松江省和黑龙江省合并为黑龙江省,吉林省仍照旧。一九五五年撤销了热河省,原辖地区分别划归河北、辽宁两省和内蒙古自治区。}的大部,在黄河以南占领了淮阴、菏泽等城市和鄂豫皖、苏皖、豫皖苏、鲁西南等解放区的大部,在长城以北占领了承德、集宁、张家口等城市和热河、绥远、察哈尔\footnote[2]{ 参见本卷《国民党进攻的真相》注〔6〕。}的大部,声势汹汹,不可一世。人民解放军采取了以歼灭国民党有生力量为主而不是以保守地方为主的正确的战略方针,每个月平均歼灭国民党正规军的数目约为八个旅(等于现在的师),终于迫使国民党放弃其全面进攻计划,而于一九四七年上半年将进攻的重点限制在南线的两翼,即山东和陕北。战争在第二年(一九四七年七月至一九四八年六月)发生了一个根本的变化。已经消灭了大量国民党正规军的人民解放军,在南线和北线都由防御转入了进攻,国民党方面则不得不由进攻转入防御。人民解放军不但在东北、山东和陕北都恢复了绝大部分的失地,而且把战线伸到了长江和渭水以北的国民党统治区。同时,在攻克石家庄、运城、四平、洛阳、宜川、宝鸡、潍县、临汾、开封等城市的作战中学会了攻坚战术\footnote[3]{ 石家庄于一九四七年十一月十二日攻克,运城于一九四七年十二月二十八日攻克,四平于一九四八年三月十三日攻克,洛阳于一九四八年三月十四日和四月五日两次攻克,宜川于一九四八年三月三日攻克,宝鸡于一九四八年四月二十六日攻克,潍县于一九四八年四月二十七日攻克,临汾于一九四八年五月十七日攻克,开封于一九四八年六月二十二日攻克。这些城市,都有大量的碉堡群或兼有高大的城墙,并设有多层外壕、铁丝网、鹿砦等副防御设备。人民解放军在当时既无飞机又无坦克,没有或仅有少量炮兵。在攻克上述城市中,人民解放军学会了一套攻坚战术。这些战术是:(1)连续爆破——以炸药对敌人各种防御设施进行连续爆破;(2)坑道作业——秘密掘进到敌人碉堡或城墙底下,用炸药炸开,随即发起猛烈突击;(3)对壕作业,亦即近迫作业——对着敌人的坚固工事,挖掘壕沟,荫蔽接近敌人,突然发起冲击;(4)抛射炸药包——利用抛射筒或迫击炮发射炸药包,破坏敌人防御工事;(5)集中兵力、火力突破一点,实行穿插分割等“尖刀战法”。}。人民解放军组成了自己的炮兵和工兵。不要忘记,人民解放军是没有飞机和坦克的,但是自从人民解放军形成了超过国民党军的炮兵和工兵以后,国民党的防御体系,连同他的飞机和坦克就显得渺小了。人民解放军已经不但能打运动战,而且能打阵地战。战争第三年的头半年(一九四八年七月至十二月)发生了另一个根本的变化。人民解放军在数量上由长期的劣势转入了优势。人民解放军不但已经能够攻克国民党坚固设防的城市,而且能够一次包围和歼灭成十万人甚至几十万人的国民党的强大精锐兵团。人民解放军歼灭国民党兵力的速度大为增加了。试看歼敌营以上正规军的统计(包括起义的敌军在内):第一年,九十七个旅,内有四十六个整旅;第二年,九十四个旅,内有五十个整旅;第三年的头半年,根据不完全的统计,一百四十七个师,内有一百一十一个整师\footnote[4]{ 这里所说的“整旅”和“整师”,是说国民党军被整个歼灭了的旅和师。这里所说的旅,是指国民党军队整编以后的旅,相当于整编以前的师;所说的师,指整编以前的师。}。半年歼敌整师的数目比过去两年歼敌整师的总数多了十五个。敌人的战略上的战线已经全部瓦解。东北的敌人已经完全消灭,华北的敌人即将完全消灭,华东和中原的敌人只剩下少数。国民党的主力在长江以北被消灭的结果,大大地便利了人民解放军今后渡江南进解放全中国的作战。同军事战线上的胜利同时,中国人民在政治战线上和经济战线上也取得了伟大的胜利。因为这样,中国人民解放战争在全国范围内的胜利,现在在全世界的舆论界,包括一切帝国主义的报纸,都完全没有争论了。\\
  敌人是不会自行消灭的。无论是中国的反动派,或是美国帝国主义在中国的侵略势力,都不会自行退出历史舞台。正是因为他们看到了中国人民解放战争在全国范围内的胜利,已经不能用单纯的军事斗争的方法加以阻止,他们就一天比一天地重视政治斗争的方法。中国反动派和美国侵略者现在一方面正在利用现存的国民党政府来进行“和平”阴谋,另一方面则正在设计使用某些既同中国反动派和美国侵略者有联系,又同革命阵营有联系的人们,向他们进行挑拨和策动,叫他们好生工作,力求混入革命阵营,构成革命阵营中的所谓反对派,以便保存反动势力,破坏革命势力。根据确实的情报,美国政府已经决定了这样一项阴谋计划,并且已经开始在中国进行这项工作。美国政府的政策,已经由单纯地支持国民党的反革命战争转变为两种方式的斗争:第一种,组织国民党残余军事力量和所谓地方势力在长江以南和边远省份继续抵抗人民解放军;第二种,在革命阵营内部组织反对派,极力使革命就此止步;如果再要前进,则应带上温和的色彩,务必不要太多地侵犯帝国主义及其走狗的利益。英国和法国的帝国主义者,则是美国这一政策的拥护者。这种情形,现在许多人还没有看清楚,但是大约不要很久,人们就可以看得清楚了。\\
  现在摆在中国人民、各民主党派、各人民团体面前的问题,是将革命进行到底呢,还是使革命半途而废呢?如果要使革命进行到底,那就是用革命的方法,坚决彻底干净全部地消灭一切反动势力,不动摇地坚持打倒帝国主义,打倒封建主义,打倒官僚资本主义,在全国范围内推翻国民党的反动统治,在全国范围内建立无产阶级领导的以工农联盟为主体的人民民主专政的共和国。这样,就可以使中华民族来一个大翻身,由半殖民地变为真正的独立国,使中国人民来一个大解放,将自己头上的封建的压迫和官僚资本(即中国的垄断资本)的压迫一起掀掉,并由此造成统一的民主的和平局面,造成由农业国变为工业国的先决条件,造成由人剥削人的社会向着社会主义社会发展的可能性。如果要使革命半途而废,那就是违背人民的意志,接受外国侵略者和中国反动派的意志,使国民党赢得养好创伤的机会,然后在一个早上猛扑过来,将革命扼死,使全国回到黑暗世界。现在的问题就是一个这样明白地这样尖锐地摆着的问题。两条路究竟选择哪一条呢?中国每一个民主党派,每一个人民团体,都必须考虑这个问题,都必须选择自己要走的路,都必须表明自己的态度。中国各民主党派、各人民团体是否能够真诚地合作,而不致半途拆伙,就是要看它们在这个问题上是否采取一致的意见,是否能够为着推翻中国人民的共同敌人而采取一致的步骤。这里是要一致,要合作,而不是建立什么“反对派”,也不是走什么“中间路线”\footnote[5]{ “中间路线”,即所谓第三条道路,见本卷《目前形势和我们的任务》注〔12〕。}。\\
  以蒋介石等人为首的中国反动派,自一九二七年四月十二日反革命政变至现在的二十多年的漫长岁月中,难道还没有证明他们是一伙满身鲜血的杀人不眨眼的刽子手吗?难道还没有证明他们是一伙职业的帝国主义走狗和卖国贼吗?请大家想一想,从一九三六年十二月西安事变以来,从一九四五年十月重庆谈判和一九四六年一月政治协商会议以来,中国人民对于这伙盗匪曾经做得何等仁至义尽,希望同他们建立国内的和平。但是一切善良的愿望改变了他们的阶级本性的一分一厘一毫一丝没有呢?这些盗匪的历史,没有哪一个是可以和美国帝国主义分得开的。他们依靠美国帝国主义把四亿七千五百万同胞投入了空前残酷的大内战,他们用美国帝国主义所供给的轰炸机、战斗机、大炮、坦克、火箭筒、自动步枪、汽油弹、毒气弹等等杀人武器屠杀了成百万的男女老少,而美国帝国主义则依靠他们掠夺中国的领土权、领海权、领空权、内河航行权、商业特权、内政外交特权,直至打死人、压死人、强奸妇女而不受任何处罚的特权。难道被迫进行了如此长期血战的中国人民,还应该对于这些穷凶极恶的敌人表示亲爱温柔,而不加以彻底的消灭和驱逐吗?只有彻底地消灭了中国反动派,驱逐了美国帝国主义的侵略势力出中国,中国才能有独立,才能有民主,才能有和平,这个真理难道还不明白吗?\\
  值得注意的是,现在中国人民的敌人忽然竭力装作无害而且可怜的样子了(请读者记着,这种可怜相,今后还要装的)。最近做了国民党行政院长的孙科,在去年六月间,不是曾经宣布“在军事方面,只要打到底,终归可以解决”的吗?这次一上台却大谈其“光荣的和平”,说什么“政府曾努力追求和平,由于和平不能实现,不得已而用兵,用兵的最后目的仍在求得和平的恢复”。合众社上海十二月二十一日的电讯,马上就预料孙科的声明“在美国官方人士及国民党自由主义人士中,将遇到最广泛的赞扬”。美国官方人士现在不但热心于中国的“和平”,而且一再表示,从一九四五年十二月莫斯科苏美英三国外长会议以来,美国就遵守着“不干涉中国内政的政策”。应该怎样来对付这些君子国的先生们呢?这里用得着古代希腊的一段寓言:“一个农夫在冬天看见一条蛇冻僵着。他很可怜它,便拿来放在自己的胸口上。那蛇受了暖气就苏醒了,等到回复了它的天性,便把它的恩人咬了一口,使他受了致命的伤。农夫临死的时候说:我怜惜恶人,应该受这个恶报!”\footnote[6]{ 见《伊索寓言》中的《农夫与蛇》。}外国和中国的毒蛇们希望中国人民还像这个农夫一样地死去,希望中国共产党,中国的一切革命民主派,都像这个农夫一样地怀有对于毒蛇的好心肠。但是中国人民、中国共产党和中国真正的革命民主派,却听见了并且记住了这个劳动者的遗嘱。况且盘踞在大部分中国土地上的大蛇和小蛇,黑蛇和白蛇,露出毒牙的蛇和化成美女的蛇,虽然它们已经感觉到冬天的威胁,但是还没有冻僵呢!\\
  中国人民决不怜惜蛇一样的恶人,而且老老实实地认为:凡是耍着花腔,说什么要怜惜一下这类恶人呀,不然就不合国情、也不够伟大呀等等的人们,决不是中国人民的忠实朋友。像蛇一样的恶人为什么要怜惜呢?究竟是哪一个工人、哪一个农民、哪一个兵士主张怜惜这类恶人呢?确是有这么一种“国民党的自由主义人士”或非国民党的“自由主义人士”,他们劝告中国人民应该接受美国和国民党的“和平”,就是说,应该把帝国主义、封建主义和官僚资本主义的残余当作神物供养起来,以免这几种宝贝在世界上绝了种。但是他们决不是工人、农民、兵士,也不是工人、农民、兵士的朋友。\\
  我们认为中国人民革命阵营必须扩大,必须容纳一切愿意参加目前的革命事业的人们。中国人民的革命事业需要有主力军,也需要有同盟军,没有同盟军的军队是打不胜敌人的。正处在革命高潮中的中国人民需要有自己的朋友,应当记住自己的朋友,而不要忘记他们。忠实于人民革命事业的朋友,努力保护人民利益而反对保护敌人利益的朋友,在中国无疑是不少,无疑是一个也不应被忘记和被冷淡的。我们又认为中国人民革命阵营必须巩固,必须不容许坏人侵入,必须不容许错误的主张获得胜利。处在革命高潮中的中国人民除了记住自己的朋友以外,还应当牢牢地记住自己的敌人和敌人的朋友。如上所说,既然敌人正在阴谋地用“和平”的方法和混入革命阵营的方法以求保存和加强自己的阵地,而人民的根本利益则要求彻底消灭一切反动势力并驱逐美国帝国主义的侵略势力出中国,那末,凡是劝说人民怜惜敌人、保存反动势力的人们,就不是人民的朋友,而是敌人的朋友了。\\
  中国革命的怒潮正在迫使各社会阶层决定自己的态度。中国阶级力量的对比正在发生着新的变化。大群大群的人民正在脱离国民党的影响和控制而站到革命阵营一方面来,中国反动派完全陷入孤立无援的绝境。人民解放战争愈接近于最后胜利,一切革命的人民和一切人民的朋友将愈加巩固地团结一致,在中国共产党的领导之下,坚决地主张彻底消灭反动势力,彻底发展革命势力,一直达到在全中国范围内建立人民民主共和国,实现统一的民主的和平。与此相反,美国帝国主义者、中国反动派和他们的朋友,虽然不能够巩固地团结一致,虽然会发生无穷的互相争吵,互相恶骂,互相埋怨,互相抛弃,但是在有一点上却会互相合作,这就是用各种方法力图破坏革命势力而保存反动势力。他们将要用各种方法:公开的和秘密的,直接的和迂回的。但是可以断定,他们的政治阴谋将要和他们的军事进攻遭遇到同样的失败。已经有了充分经验的中国人民及其总参谋部中国共产党,一定会像粉碎敌人的军事进攻一样,粉碎敌人的政治阴谋,把伟大的人民解放战争进行到底。\\
  一九四九年中国人民解放军将向长江以南进军,将要获得比一九四八年更加伟大的胜利。\\
  一九四九年我们在经济战线上将要获得比一九四八年更加伟大的成就。我们的农业生产和工业生产将要比过去提高一步,铁路公路交通将要全部恢复。人民解放军主力兵团的作战将要摆脱现在还存在的某些游击性,进入更高程度的正规化。\\
  一九四九年将要召集没有反动分子参加的以完成人民革命任务为目标的政治协商会议,宣告中华人民共和国的成立,并组成共和国的中央政府。这个政府将是一个在中国共产党领导之下的、有各民主党派各人民团体的适当的代表人物参加的民主联合政府。\\
  这些就是中国人民、中国共产党、中国一切民主党派和人民团体在一九四九年所应努力求其实现的主要的具体的任务。我们将不怕任何困难团结一致地去实现这些任务。\\
  几千年以来的封建压迫,一百年以来的帝国主义压迫,将在我们的奋斗中彻底地推翻掉。一九四九年是极其重要的一年,我们应当加紧努力。\\
\newpage\section*{\myformat{评战犯求和}\\\myformat{(一九四九年一月四日)}}\addcontentsline{toc}{section}{评战犯求和}
\begin{introduction}\item  这是毛泽东为新华社写的揭露国民党利用和平谈判来保存反革命实力的一系列评论的第一篇。其它的评论是:《四分五裂的反动派为什么还要空喊“全面和平”?》、《国民党反动派由“呼吁和平”变为呼吁战争》、《评国民党对战争责任问题的几种答案》、《南京政府向何处去?》等。\end{introduction}
为了保存中国反动势力和美国在华侵略势力,中国第一号战争罪犯国民党匪帮首领蒋介石在今年元旦发表了一篇求和的声明。战犯蒋介石宣称:“只要和议无害于国家的独立完整,而有助于人民的休养生息,只要神圣的宪法不由我而违反,民主宪政不因此而破坏,中华民国的国体能够确保,中华民国的法统不致中断,军队有确实的保障,人民能够维持其自由的生活方式与目前最低生活水准,则我个人更无复他求。”“只要和平果能实现,则个人的进退出处,绝不萦怀,而一惟国民的公意是从。”人们不要以为战犯求和未免滑稽,也不要以为这样的求和声明实在可恶。须知由第一号战犯国民党匪首出面求和,并且发表这样的声明,对于中国人民认识国民党匪帮和美国帝国主义的阴谋计划,有一种显然的利益。中国人民可以由此知道:原来现在喧嚷着的所谓“和平”,就是蒋介石这一伙杀人凶犯及其美国主子所迫切地需要的东西。\\
  蒋介石供认了匪帮们的整个计划。这个计划的要点如下:\\
  “无害于国家的独立完整”——这是首先重要的。“和平”可以,“和平”而有害于四大家族和买办地主阶级的国家的“独立完整”,那就万万不可以。“和平”而有害于中美友好通商航海条约\footnote[1]{ 见本卷《迎接中国革命的新高潮》注〔5〕。}、中美空中运输协定\footnote[2]{ 国民党政府和美国政府的《中美空中运输协定》,于一九四六年十二月二十日在南京签订。蒋介石在这个协定里拍卖了中国的全部领空权。按照这个协定的规定,美国飞机可以在中国到处飞行、装卸和运转,完全控制中国的空运事业。美国飞机并在中国领土内享有“非营业性降落之权”,即军事着陆权。}、中美双边协定\footnote[3]{ 《中美双边协定》即所谓《中美关于经济援助之协定》,一九四八年七月三日由国民党政府和美国政府双方的代表在南京签订。这个协定规定:美国政府承允援助国民党政府,向国民党政府提供它所申请及美国政府所核准的援助;执行该协议的美方人员在中国享有与美驻华大使馆同等职员一样的优待及豁免;美国可以在中国取得它所需要的任何战略物资,国民党政府必须按时供给有关这些物资的情报;国民党政府保证美国的商品来华倾销。}等项条约,有害于美国在华驻扎海陆空军,建立军事基地,开发矿产和独占贸易等项特权,有害于将中国作为美国殖民地的地位,一句话,“和平”而有害于这一切保护蒋介石反动国家的“独立完整”的办法,那就一概不可以。\\
  “有助于人民的休养生息”——“和平”必须有助于已被击败但尚未消灭的中国反动派的休养生息,以便在休养好了之后,卷土重来,扑灭革命。“和平”就是为了这个。打了两年半了,“走狗不走”,美国人在生气,就是稍为休养一会儿也好。\\
  “神圣的宪法不由我而违反,民主宪政不因此而破坏,中华民国的国体能够确保,中华民国的法统不致中断”——确保中国反动阶级和反动政府的统治地位,确保这个阶级和这个政府的“法统不致中断”。这个“法统”是万万“中断”不得的,倘若“中断”了,那是很危险的,整个买办地主阶级将被消灭,国民党匪帮将告灭亡,一切大中小战争罪犯将被捉拿治罪。\\
  “军队有确实的保障”——这是买办地主阶级的命根,虽然已被可恶的人民解放军歼灭了几百万,但是现在还剩下一百几十万,务须“保障”而且“确实”。倘若“保障”而不“确实”,买办地主阶级就没有了本钱,“法统”还是要“中断”,国民党匪帮还是要灭亡,一切大中小战犯还是要被捉拿治罪。大观园里贾宝玉的命根是系在颈上的一块石头\footnote[4]{ 贾宝玉是公元十八世纪中国著名小说《红楼梦》中的人物,大观园是贾宝玉家里的花园。据说贾宝玉出生时口里含着一块玉石,这玉石是他的命根,系在颈上一时也不能离开,如果丢了便会“失魂丧魄”。},国民党的命根是它的军队,怎么好说不“保障”,或者虽有“保障”而不“确实”呢?\\
  “人民能够维持其自由的生活方式与目前最低生活水准”——中国买办地主阶级必须维持其向全国人民实行压迫剥削的自由和他们目前的骄奢淫逸的生活水准,中国劳动人民则必须维持其被人压迫剥削的自由和他们目前的饥寒交迫的生活水准。这是战犯求和的终极目的。倘若战犯们及其阶级不能维持其实行压迫剥削的自由和骄奢淫逸的生活水准,和平有什么用呢?而要这个,当然就要维持工人、农民、知识分子、公教人员目前这样饥寒交迫的“自由生活方式与最低生活水准”。这个条件一经我们的可爱的蒋总统提了出来,几千万的工人、手工工人和自由职业者,几万万的农民,几百万的知识分子和公教人员,惟有一齐拍掌,五体投地,口称万岁。倘若共产党还不许和,不能维持这样美好的生活方式和生活水准,那就罪该万死,“今后一切责任皆由共党负之”。\\
  上述一切,还没有包括一月一日战犯求和声明中的一切宝贝。还有另一个宝贝,这就是蒋介石在其新年致词中所说的“京沪决战”。哪里有这种“决战”的力量呢?蒋介石说:“要知道政府今天在军事、政治、经济无论哪一方面的力量,都要超过共党几倍乃至几十倍。”哎呀呀,这么大的力量怎样会不叫人们吓得要死呢?姑且把政治、经济两方面的力量放在一边不去说它们,单就“军事力量”一方面来说,人民解放军现在有三百多万人,“超过”这个数目一倍就是六百多万人,十倍就是三千多万人,“几十倍”是多少呢?姑且算作二十倍吧,就有六千多万人,无怪乎蒋总统要说“有决胜的把握”了。为什么求和呢?完全不是不能打,拿六千多万人压下去,世界上还有什么共产党或者什么别的党可以侥幸存在的呢?当然一概成了粉末。由此可见,求和决不是为了别的,完全是“为民请命”。\\
  难道万事皆好,一个缺点也没有吗?据说缺点是有的。什么缺点呢?蒋总统说:“现在所遗憾的,是我们政府里面一部分人员受了共党恶意宣传,因之心理动摇,几乎失了自信。因为他们在精神上受了共党的威胁,所以只看见敌人的力量,而就看不见自己还有比敌人超过几十倍的大力量存在。”新闻年年皆有,今年特别不同。拥有六千多万名军官和兵士的国民党人看不见自己的六千多万,倒看见了人民解放军的三百多万,这难道还不是一条特别新闻吗?\\
  要问:这样的新闻是否在市场上还有销路?是否还值得人们看上一眼?根据我们所得的北平城内的消息是:“元旦物价上午略跌,下午复原。”外国通讯社说:“上海对于蒋介石新年致词的反映是冷淡的。”这就答复了战犯蒋介石的销路问题。我们早就说过,蒋介石已经失了灵魂,只是一具僵尸,什么人也不相信他了。
\newpage\section*{\myformat{中共中央毛泽东主席关于时局的声明}\\\myformat{(一九四九年一月十四日)}}\addcontentsline{toc}{section}{中共中央毛泽东主席关于时局的声明}
自一九四六年七月,南京国民党反动政府在美国帝国主义者的帮助之下,违背人民意志,撕毁停战协定\footnote[1]{ 见本卷《以自卫战争粉碎蒋介石的进攻》注〔1〕。}和政治协商会议的决议\footnote[2]{ 见本卷《以自卫战争粉碎蒋介石的进攻》注〔2〕。},发动全国规模的反革命的国内战争以来,已经两年半了。在这两年半的战争中,南京国民党反动政府违背民意,召集了伪国民大会,颁布了伪宪法,选举了伪总统,颁布了所谓“动员戡乱”的伪令,出卖了大批的国家权利给美国政府,从美国政府获得了数十亿美元的外债,勾引了美国政府的海军和空军占据中国的领土、领海、领空,和美国政府订立了大批的卖国条约,接受美国军事顾问团参加中国的内战,从美国政府获得了大批的飞机、坦克、重炮、轻炮、机关枪、步枪、炮弹、子弹和其它军用物资,以为屠杀中国人民的武器。南京国民党反动政府在上述各项反动的卖国的内政外交基本政策的基础上,指挥它的数百万军队,向着中国人民解放区和中国人民解放军举行了残酷的进攻。所有华东、中原、华北、西北、东北各人民解放区,无一不受到国民党军队的蹂躏。解放区的中心城市延安、张家口、淮阴、菏泽、大名、临沂、烟台、承德、四平、长春、吉林、安东\footnote[3]{ 安东,今辽宁省丹东市。}等地,均曾被匪军占领。匪军所至,杀戮人民,奸淫妇女,焚毁村庄,掠夺财物,无所不用其极。在南京国民党反动政府的统治区域,则压迫工农兵学商各界广大人民群众出粮、出税、出力,敲骨吸髓,以供其所谓“戡乱剿匪”之用。南京国民党反动政府取消人民的一切自由权利;压迫一切民主党派和人民团体使其丧失合法的地位;压迫青年学生们的反内战、反饥饿、反迫害、反美国干涉中国内政和扶植日本侵略势力等项正义的运动;滥发伪法币和伪金圆券,破坏人民的经济生活,使广大人民陷于破产的地位;用各种搜括的方法,使国家最大的财富集中于蒋宋孔陈四大家族为首的官僚资本系统。总之,南京国民党反动政府,在其反动的卖国的内政外交基本政策的基础之上所举行的国内战争,业已陷全国人民于水深火热之中,南京国民党反动政府决不能逃脱自己应负的全部责任。同国民党相反,中国共产党自从日本投降以后,即尽一切努力向国民党政府要求防止和停止国内战争,实行国内和平。中国共产党根据此种方针,坚持奋斗,在全国人民的赞助之下,首先获得了一九四五年十月国共两党会谈纪要\footnote[4]{ 见本卷《关于重庆谈判》注〔1〕。}的签订。在一九四六年一月,又签订了国共两党的停战协定,并和各民主党派协作,在政治协商会议上迫使国民党接受了共同的决议。自此以后,中国共产党即和各民主党派各人民团体一道,为维护这些协定和决议而奋斗。但是可惜,所有这些维护国内和平和人民民主权利的行为,均不被国民党反动政府所尊重。相反地,被认为是软弱的表现,不值一顾。国民党反动政府认为人民可欺,认为停战协定和政治协商会议的决议可以随意撕毁,认为人民解放军不值一击,认为他们的数百万军队可以横行全国,认为美国政府对于他们的援助是无穷无尽的。以此种种,国民党反动政府就敢于违背全国人民的意志,发动了反革命战争。在此种情况下,中国共产党不得不坚决地起来反对国民党政府的反动政策,为着保卫国家的独立和人民的民主权利而奋斗。自一九四六年七月起,中国共产党领导英勇的人民解放军抵抗了国民党反动政府的四百三十万军队的进攻,然后又使自己转入了反攻,从而收复了解放区的一切失地,并且解放了石家庄、洛阳、济南、郑州、开封、沈阳、徐州、唐山诸大城市。中国人民解放军克服了无比的困难,壮大了自己,以美国政府送给国民党政府的大批武器装备了自己。在两年半的过程中,歼灭了国民党反动政府的主要军事力量和一切精锐师团。现在,人民解放军无论在数量上士气上和装备上均优于国民党反动政府的残余军事力量。至此,中国人民才开始吐了一口气。现在,情况已非常明显,只要人民解放军向着残余的国民党军再作若干次重大的攻击,全部国民党反动统治机构即将土崩瓦解,归于消灭。现在,国民党反动政府发动内战的政策,业已自食其果,众叛亲离,已至不能维持的境地。在此种形势下,为着保持国民党政府的残余力量,取得喘息时间,然后卷土重来扑灭革命力量的目的,中国第一名战争罪犯国民党匪帮首领南京政府伪总统蒋介石,于今年一月一日,提出了愿意和中国共产党进行和平谈判的建议。中国共产党认为这个建议是虚伪的。这是因为蒋介石在他的建议中提出了保存伪宪法、伪法统和反动军队等项为全国人民所不能同意的条件,以为和平谈判的基础。这是继续战争的条件,不是和平的条件。旬日以来,全国人民业已显示了自己的意志。人民渴望早日获得和平,但是不赞成战争罪犯们的所谓和平,不赞成他们的反动条件。在此种民意基础之上,中国共产党声明:虽然中国人民解放军具有充足的力量和充足的理由,确有把握,在不要很久的时间之内,全部地消灭国民党反动政府的残余军事力量;但是,为了迅速结束战争,实现真正的和平,减少人民的痛苦,中国共产党愿意和南京国民党反动政府及其它任何国民党地方政府和军事集团,在下列条件的基础之上进行和平谈判。这些条件是:(一)惩办战争罪犯;(二)废除伪宪法;(三)废除伪法统;(四)依据民主原则改编一切反动军队;(五)没收官僚资本;(六)改革土地制度;(七)废除卖国条约;(八)召开没有反动分子参加的政治协商会议,成立民主联合政府,接收南京国民党反动政府及其所属各级政府的一切权力\footnote[5]{ 毛泽东在这个声明里提出的八项和平条件,成为一九四九年四月中国共产党代表团和以张治中为首的国民党政府代表团进行和平谈判的基础。在这次谈判中拟定的国内和平协定,对八项和平条件作了具体规定。见本卷《向全国进军的命令》注〔1〕。}。中国共产党认为,上述各项条件反映了全国人民的公意,只有在上述各项条件之下所建立的和平,才是真正的民主的和平。如果南京国民党反动政府中的人们,愿意实现真正的民主的和平,而不是虚伪的反动的和平,那末,他们就应当放弃其反动的条件,承认中国共产党提出的八个条件,以为双方从事和平谈判的基础。否则,就证明他们的所谓和平,不过是一个骗局。我们希望全国人民、各民主党派、各人民团体,大家起来争取真正的民主的和平,反对虚伪的反动的和平。南京国民党政府系统中的爱国人士,亦应当赞助这样的和平建议。中国人民解放军全体指挥员战斗员同志注意:在南京国民党反动政府接受并实现真正的民主的和平以前,你们丝毫也不应当松懈你们的战斗努力。对于任何敢于反抗的反动派,必须坚决、彻底、干净、全部地歼灭之。\\
\newpage\section*{\myformat{中共发言人评南京行政院的决议}\\\myformat{(一九四九年一月二十一日)}}\addcontentsline{toc}{section}{中共发言人评南京行政院的决议}
南京国民党反动政府的官方通讯社中央社十九日电称:十九日上午九时行政院会议广泛讨论时局,决议如下:“政府为遵从全国人民之愿望,蕲求和平之早日实现,特慎重表示,愿与中共双方立即先行无条件停战,并各指定代表进行和平商谈。”中国共产党发言人称:南京行政院的这个决议没有提到一月一日南京伪总统蒋介石建议和平谈判的声明,也没有提到一月十四日中国共产党毛泽东主席建议和平谈判的声明,没有表示对于这两个建议究竟是拥护哪一个,反对哪一个,好像一月一日和一月十四日国共双方并没有提出过什么建议一样,却另外提出了自己的建议,这是完全令人不能理解的。在实际上,南京行政院不但完全忽视中共一月十四日的建议,而且直接推翻了伪总统蒋介石一月一日的建议。蒋介石在其一月一日的建议中说:“只要共党一有和平的诚意,能作确切的表示,政府必开诚相见,愿与商讨停止战事,恢复和平的具体方法。”过了十九天,同一个政府的一部分机构,即南京政府的“行政院”,却推翻了这个政府的“总统”的声明,不是“必开诚相见,愿与商讨停止战事恢复和平的具体方法”,而是“立即先行无条件停战,并各指定代表进行和平商谈”了。我们要问南京“行政院”的先生们,究竟是你们的建议为有效呢,还是你们的“总统”的建议为有效呢?你们的“总统”把“停止战事恢复和平”认为是一件事,声明必定开诚相见愿与中共商讨实现这件事的具体方法;你们则将战争与和平分割为两件事,不愿意派出代表和我们商讨停止战争的具体方法,而却异想天开地建议“立即先行无条件停战”,然后再派代表“进行和平商谈”,究竟是你们的建议对呢,还是你们“总统”的建议对呢?我们认为南京伪行政院是越出了自己的职权的,它没有资格推翻伪总统的建议而擅自作出自己的新建议。我们认为南京行政院的这个新建议是没有理由的,打了这么久这么大和这么残酷的战争,自应双方派人商讨和平的基本条件,并作出双方同意的停战协定,战争才能停得下来。不但人民有这种希望,就是国民党方面亦有不少人表示了这种希望。如果照南京行政院的毫无理由的“决议”,不先行停战就不愿意进行和平谈判,则国民党的和平诚意在什么地方呢?南京行政院的“决议”是做出来了,不先行停战就没有和平谈判的可能了,和平之门从此关死了,而如果要谈判,则只有取消这个毫无理由的“决议”,二者必居其一。如果南京行政院不愿意取消自己的“决议”,那就是表明南京国民党反动政府并无与其对方进行和平谈判的诚意。人们要问:南京方面果有诚意,为什么不愿意商讨和平的具体条件呢?南京的和平建议是虚伪的这样一个论断,难道不是已经证实了吗?中共发言人说:南京现在业已陷入无政府状态,伪总统有一个建议,伪行政院又有一个建议,这叫人们和谁去打交道呢?\\
\newpage\section*{\myformat{中共发言人关于命令国民党反动政府重新逮捕
前日本侵华军总司令冈村宁次和逮捕国民党内战罪犯的谈话}\\\myformat{(一九四九年一月二十八日)}}\addcontentsline{toc}{section}{中共发言人关于命令国民党反动政府重新逮捕
前日本侵华军总司令冈村宁次和逮捕国民党内战罪犯的谈话}
据南京国民党反动政府的中央通讯社一月二十六日电称:“政府发言人称:政府为提早结束战争,以减轻人民痛苦,一月以来已作种种措施与步骤。本月二十二日更正式派定和谈代表\footnote[1]{ 当时国民党反动政府派定的和平谈判代表是:邵力子、张治中、黄绍竑、彭昭贤、钟天心五人。}。日来只待中共方面指派代表,约定地点,以便进行商谈。惟据新华社陕北二十五日广播中共发言人谈话\footnote[2]{ 一九四九年一月二十五日中共发言人关于和平谈判的谈话指出:“我们允许南京反动政府派出代表和我们进行谈判,不是承认这个政府还有代表中国人民的资格,而是因为这个政府手里还有一部分反动的残余军事力量。如果这个政府感于自己已经完全丧失人民的信任,感于它手里的残余反动军事力量已经无法抵抗强大的人民解放军,而愿意接受中共的八个和平条件的话,那末,用谈判的方法去解决问题,使人民少受痛苦,当然是比较好的和有利于人民解放事业的。”关于谈判地点,谈话中说:“要待北平完全解放后才能确定,大约将在北平。”关于谈判代表,谈话中说:“彭昭贤是主战最力的国民党CC派的主要干部之一,人们认为是一个战争罪犯,中共方面不能接待这样的代表。”},一面虽声明愿与政府商谈和平解决,一面则肆意侮谩,语多乖戾。且谓谈判地点要待北平完全解放后才能确定。试问中共方面如不即时指派代表,约定地点,又不停止军事行动,而竟诿诸所谓北平完全解放以后,岂非拖延时间,延长战祸?须知全国人民希望消弭战祸,已属迫不及待。政府为表示绝大之诚意,仍盼中共认清:今日之事,应以拯救人民为前提,从速指派代表进行商谈,使和平得以早日实现。”又据南京中央社一月二十六日上海电称:“日本战犯前中国派遣军总司令官冈村宁次大将,二十六日由国防部审判战犯军事法庭举行复审后,于十六时由石美瑜庭长宣判无罪。当时庭上空气紧张。冈村肃立聆判后,微露笑容”等情。据此,中共发言人表示下列诸点:\\
  (一)日本战犯前中国派遣军总司令官冈村宁次大将,为日本侵华派遣军一切战争罪犯中的主要战争罪犯\footnote[3]{ 冈村宁次(一八八四——一九六六),是侵略中国历史最久、罪恶最大的日本战犯之一。一九二五年到一九二七年,担任北洋军阀孙传芳的军事顾问。一九二八年任日军步兵联队长,曾参加日军侵占济南的战争,是济南惨案的刽子手。一九三二年任日本上海派遣军副参谋长,参加侵占上海的战争。一九三三年曾代表日本政府和国民党反动政府签订《塘沽协定》。一九三七年至一九四五年期间,历任日军第十一军、华北方面军、第六方面军司令官和中国派遣军总司令官,在中国实行了极其残酷的烧光、杀光、抢光的“三光”政策。在一九四五年十二月延安公布的日本战犯名单中,被列为首要战争罪犯。在人民解放战争期间,他曾充当蒋介石的秘密军事顾问,为蒋介石策划向解放区的进攻。一九四九年一月被国民党反动政府宣判无罪,释放回国。一九五〇年又被蒋介石聘为“革命实践研究院”的高级教官。一九五五年以后,又纠集日本陆海军旧军人组织“战友联”(后改名为乡友联盟),积极参与复活日本军国主义的反动活动。},今被南京国民党反动政府的战犯军事法庭宣判无罪;中国共产党和中国人民解放军总部声明:这是不能容许的。中国人民在八年抗日战争中牺牲无数生命财产,幸而战胜,获此战犯,断不能容许南京国民党反动政府擅自宣判无罪。全国人民、一切民主党派、人民团体以及南京国民党反动政府系统中的爱国人士,必须立即起来反对南京反动政府方面此种出卖民族利益,勾结日本法西斯军阀的犯罪行为。我们现在向南京反动政府的先生们提出严重警告:你们必须立即将冈村宁次重新逮捕监禁,不得违误。此事与你们现在要求和我们进行谈判一事,有密切关系。我们认为你们现在的种种作为,是在企图以虚伪的和平谈判掩护你们重整战备,其中包括勾引日本反动派来华和你们一道屠杀中国人民一项阴谋在内;你们释放冈村宁次,就是为了这个目的。因此,我们决不许可你们这样做。我们有权命令你们重新逮捕冈村宁次,并依照我们将要通知你们的时间地点,由你们负责押送人民解放军。其它日本战争罪犯,暂由你们管押,听候处理,一概不得擅自释放或纵令逃逸,违者严惩不贷。\\
  (二)从南京国民党反动政府发言人一月二十六日的声明中,获知南京的先生们要求和平谈判是那样地紧张、热烈、殷勤、迫切,据说都是为了“缩短战争时间”,“减轻人民痛苦”,“以拯救人民为前提”;而感觉中共方面对于接受你们的愿望则是这样地不紧张,不热烈,不殷勤,不迫切,“又不停止军事行动”,实在是“拖延时间,延长战祸”。我们老实告诉南京的先生们:你们是战争罪犯,你们是要受审判的人们。你们口中的所谓“和平”、“民意”,我们是不相信的。你们依赖美国势力,违反人民意志,撕毁停战协定\footnote[4]{ 见本卷《以自卫战争粉碎蒋介石的进攻》注〔1〕。}和政治协商会议的决议\footnote[5]{ 见本卷《以自卫战争粉碎蒋介石的进攻》注〔2〕。},发动这次残酷无比的反人民反民主反革命的国内战争。那时你们是那样地紧张、热烈、殷勤、迫切,什么人的劝告也不听。你们召开伪国大,制定伪宪法,选举伪总统,颁发“动员戡乱”的伪令,又是那样地紧张、热烈、殷勤、迫切,又是什么人的劝告也不听。那时,上海、南京和各大都市的官办的或御用的所谓参议会、商会、工会、农会、妇女团体、文化团体一齐起哄,“拥护动员戡乱”,“消灭共匪”,又是那样地紧张、热烈、殷勤、迫切,又是什么人的劝告也不听。如今,过了两年半,被你们屠杀的人民何止数百万,被你们焚毁的村庄,奸淫的妇女,掠夺的财物,被你们的空军炸毁的有生无生力量,是数不清的,你们犯了滔天大罪,这笔账必得算一算。听说你们很有些反对清算斗争。但是这一次清算斗争是事出有因的,必得清一清,算一算,斗一斗,争一争。你们是打败了。你们激怒了人民。人民一齐起来和你们拚命。人民不欢喜你们,人民斥责你们,人民起来了,你们孤立了,因此你们打败了。你们提出了五条\footnote[6]{ 国民党反动政府提出的“五条”,指蒋介石在一九四九年元旦声明中所提出的关于和平谈判的五个条件,即:一、“无害于国家的独立完整”;二、“有助于人民的休养生息”;三、“神圣的宪法不由我而违反,民主宪政不因此而破坏,中华民国的国体能够确保,中华民国的法统不致中断”;四、“军队有确实的保障”;五、“人民能够维持其自由的生活方式与目前最低生活水准”。毛泽东当时即对这五条作了严正的驳斥,见本卷《评战犯求和》。},我们提出了八条\footnote[7]{ 中国共产党的“八条”,指一九四九年一月十四日毛泽东关于时局的声明中所提出的和平谈判的八个条件。见本卷《中共中央毛泽东主席关于时局的声明》。},人民立即拥护我们的八条,不拥护你们的五条。你们不敢批驳我们的八条,不敢坚持你们的五条。你们声明愿以我们的八条为谈判的基础。这样难道还不好吗?为什么还不快点谈呢?于是乎显得你们很紧张,很热烈,很殷勤,很迫切,很主张“无条件停战”,“缩短战争时间”,“减轻人民痛苦”,“以拯救人民为前提”。而我们呢?显然是不紧张,不热烈,不殷勤,不迫切,“拖延时间,延长战祸”。但是且慢,南京的先生们,我们会要紧张起来,热烈起来,殷勤起来,迫切起来的,战争时间一定可以缩短,人民的痛苦一定可以减轻。你们既然同意以我们的八个条件为双方谈判的基础,你们和我们会要一齐忙碌起来的。实行这八条,够得上你们,我们,一切民主党派,人民团体以及全国各界人民忙上几个月,半年,一年,几年,恐怕还忙不完呢!南京的先生们听着:八条不是抽象的条文,要有具体的内容,目前这一个短时期内还是大家想一想要紧,为此耽搁一段时间,人民也会原谅的。老实说,人民的意见是要好好地准备这一次谈判。谈是一定要谈的,谁要中途翻了不肯谈,那是决不许可的,因此你们的代表一定得准备来。但是我们还得一些时间做准备工作,不容许战争罪犯们替我们规定谈判的时间。我们和北平人民正在做一件重要工作,按照八个条件和平地解决北平问题。你们在北平的人例如傅作义将军等也参加了这件工作,经过你们的通讯社的公告,你们已经承认了这件工作是做得对的\footnote[8]{ 国民党中央通讯社于一九四九年一月二十七日发表南京政府国防部的文告称:“华北方面,为了缩短战争,获致和平,借以保全北平故都基础与文物古迹,傅总司令作义曾于二十二日发表文告,宣布自二十二日上午十时起休战。平市国军大部当即遵从总部指示,先后撤离市区,开入指定地点。”文告并称:“绥远大同两地亦将实施休战。”}。这就不但替和平谈判准备了地点,而且替解决南京、上海、武汉、西安、太原、归绥\footnote[9]{ 归绥,今呼和浩特市。}、兰州、迪化\footnote[10]{ 迪化,今乌鲁木齐市。}、成都、昆明、长沙、南昌、杭州、福州、广州、台湾、海南岛等地的和平问题树立了榜样。因此,这件工作是应当受到赞美的,南京的先生们对此不应当表示不够郑重的态度。我们正在同各民主党派、人民团体和无党派民主人士,包括在我们区域的和在你们区域的都在内,商量战争罪犯的名单问题,准备第一个条件的具体内容。这个名单,大约不要很久就可以正式公布出来。南京的先生们,你们知道,直到现在,我们和各民主党派、人民团体,都还没有来得及商量和正式公布这样一个名单,这是要请先生们原谅的。其原因,是你们的和谈要求来得稍为迟了些。如果早一点,也许我们已经准备好了。但是,你们也并不是没有事做。除了逮捕日本战犯冈村宁次以外,你们必须立即动手逮捕一批内战罪犯,首先逮捕去年十二月二十五日中共权威人士声明中所提四十三个战犯之在南京、上海、奉化、台湾等处者。其中最主要的,是蒋介石、宋子文、陈诚、何应钦、顾祝同、陈立夫、陈果夫、朱家骅、王世杰、吴国桢、戴传贤、汤恩伯、周至柔、王叔铭、桂永清\footnote[11]{ 宋子文,曾任国民党政府财政部长、行政院长、外交部长、驻美特使等职。陈诚,曾任国民党军参谋总长,当时任国民党政府台湾省主席。何应钦,曾任国民党军参谋总长和国防部长。顾祝同,当时任国民党军参谋总长。陈立夫、陈果夫、朱家骅都是国民党CC派的主要头目。王世杰,曾任国民党政府外交部长。吴国桢,当时任国民党政府上海市长。戴传贤,即戴季陶,长期充当蒋介石的谋士,当时是国民党中央常务委员。汤恩伯,当时任国民党京沪杭警备总司令。周至柔,当时任国民党军空军总司令。王叔铭,当时任国民党军空军副总司令兼参谋长。桂永清,当时任国民党军海军总司令。}等人。特别重要的是蒋介石,该犯现已逃至奉化,很有可能逃往外国,托庇于美国或英国帝国主义,因此,你们务必迅即逮捕该犯,毋令逃逸。此事你们要负完全责任,倘有逃逸情事,必以纵匪论处,决不姑宽,勿谓言之不预。我们认为只有逮捕这些战争罪犯,才是为了缩短战争时间,减轻人民痛苦,认真地做了一件工作。只要战争罪犯们还存在,就只会延长战争时间,加重人民痛苦。\\
  (三)以上二项,要求南京反动政府给予答复。\\
  (四)八条中其它各条双方应行准备的工作,另一次再通知南京。\\
\newpage\section*{\myformat{中共发言人关于和平条件必须包括惩办日本战犯和国民党战犯的声明}\\\myformat{(一九四九年二月五日)}}\addcontentsline{toc}{section}{中共发言人关于和平条件必须包括惩办日本战犯和国民党战犯的声明}
上月二十八日中国共产党发言人关于和平谈判问题的声明,到上月三十一日得到了国民党反动卖国政府发言人的答复。国民党反动卖国政府的发言人在这个答复里,对于中共发言人所提出的各项问题,提出了狡辩。对于中共要求国民党反动卖国政府负责重新逮捕日本侵华罪魁冈村宁次\footnote[1]{本卷《中共发言人关于命令国民党反动政府重新逮捕前日本侵华军总司令冈村宁次和逮捕国民党内战罪犯的谈话》注〔3〕。},准备押送人民解放军,并负责看管其它日本战犯勿令逃逸一节,该发言人说,这“是一个司法问题。这完全与和谈无关,更不能作为和谈的先决条件”。对于中共要求国民党反动卖国政府负责逮捕战争罪犯蒋介石等人一节,说是“真正的和平不应该有先决条件”。并且说中共发言人的声明“态度上似乎不够郑重”,而且是“节外生枝”。对此,中共发言人声明:在一月二十八日那种时候,我们还把国民党反动卖国政府说成是一个政府,在这点上说来,我们的态度确乎不够郑重。这个所谓“政府”究竟还存在不存在呢?它是存在于南京吗?南京没有行政机关。它是存在于广州吗?广州没有行政首脑。它是存在于上海吗?上海既没有行政机关,又没有行政首脑。它是存在于奉化吗?奉化只有一个宣布“退休”了的伪总统,别的什么都没有。因此郑重地说起来,已经不应当把它看成一个政府,它至多只是一个假定的或象征的政府了。但是我们仍然假定有那么一个象征的“政府”,并且假定有一个足以代表这个所谓“政府”发言的发言人。那末,该发言人应当知道,这个假定的象征的国民党反动卖国政府,现在不但对于和平谈判毫无建树,而且确确实实地是在不断地节外生枝。例如当着你们如此急切地要求谈判的时候,忽然判决冈村宁次无罪,这难道不是节外生枝吗?在中共要求予以重新逮捕以后,又把他送往日本,并且把其它二百六十名战犯也送往日本,这难道不是节外生枝吗?日本现在是什么人统治呢?难道是日本人民在统治而不是帝国主义分子在统治吗?日本是你们如此热爱的地方,以致使你们相信日本战犯们生活在你们统治的区域,还不如使他们生活在日本较为安全些,较为舒服些,较为能受到正当待遇些。这是一个司法问题吗?为什么发生这个司法问题呢?难道日本侵略者和我们打了八个整年这件事,你们也忘了吗?完全与和谈无关吗?一月十四日中共提出八条\footnote[2]{本卷《中共中央毛泽东主席关于时局的声明》。}的时候,并没有发生释放冈村宁次这件事情。一月二十六日这件事情发生了,就应提出了,就与和谈有关了。一月三十一日你们接受麦克阿瑟的命令,又将日本战犯二百六十名连同冈村宁次一起送往日本,就更与和谈有关了。为什么你们要求和谈呢?是因为你们打了败仗。你们为什么打败仗呢?是因为你们发动反人民的国内战争。你们在什么时候发动这次国内战争呢?是在日本投降以后。你们发动这次战争是打谁呢?是打在抗日战争中立了大功的人民解放军和人民解放区。用什么力量来打呢?除了美国援助之外,是在你们统治区域从人民方面捉来和刮来的力量。中国人民和日本侵略者一场大决斗刚刚完毕,一个对外战争刚刚完毕,你们就发动这次对内战争。你们打败了,要求谈判,忽然又宣告日本首要战犯冈村宁次无罪。我们刚刚向你们提出抗议,要求你们重新监禁冈村宁次并准备交给人民解放军,你们又慌忙将他和其它二百六十名日本战犯一齐送往日本。国民党反动卖国政府的先生们,你们这件事做得太无道理了,太违反人民意志了。我们现在特地在你们的头衔上加上卖国二字,你们应当承认了。你们的政府很久以来就是卖国政府,仅仅为了节省文字起见,有时我们省写了这两个字,现在不能省了。你们除去历次的卖国罪以外,现在又犯了一次卖国罪,而且这一次犯得很严重,和平谈判会议上必得谈这个问题。无论你们叫节外生枝也好,不叫节外生枝也好,这件事必得谈,因为这件事是发生在一月十四日以后的,没有包括在我们原来所提的八个条件以内,因此我们认为必须在第一个条件中增加惩办日本战犯一个项目。这样,这一条就有两个项目,即是:(甲)惩办日本战争罪犯;(乙)惩办国内战争罪犯。我们提出这个项目是有理由的,是反映全国人民意志的。全国人民都要惩办日本战犯。即在国民党内,也有许多人认为惩办冈村宁次等日本战争罪犯和惩办蒋介石等国内战争罪犯一样是理所当然。无论你们说我们是有和平诚意也好,没有和平诚意也好,这两类战犯问题都得谈判,这两类战犯都得惩办。关于叫你们在谈判之前逮捕一批内战罪犯和防止这些战犯逃跑的问题,你们认为“不应有先决条件”。国民党反动卖国政府的先生们,这不是先决条件,这是你们承认惩办战犯一条为谈判基础之后自然产生的要求。叫你们逮捕,是怕战犯们跑掉。当着我们在谈判的准备工作还没有做好的时候,你们如丧考妣地急着要谈判,你们闲得发慌,因此叫你们做一件合理的工作。这些战犯总是要逮捕的,任凭他们跑到天涯海角也是要逮捕的。你们是愿意“缩短战争时间”、“减轻人民痛苦”、“以拯救人民为前提”的大慈大悲救苦救难的人们,你们是有很多的好心眼儿的人们,你们对于这些屠杀几百万同胞的负责者应当没有什么爱惜,从你们愿意以惩办战犯作为一条谈判基础这一点来看,你们似乎也并不很爱惜这些东西。但是既然你们声明叫你们马上逮捕这些东西显得颇有为难之处,那末也罢,你们就防止他们逃跑吧,千万莫叫这些东西跑掉了。先生们,请想一想,当着你们辛辛苦苦地派出代表团和我们讨论惩办战犯问题的时候,战犯们已经跑了,那末,还谈什么呢?你们的代表团先生们的脸上还有什么光彩呢?你们那样多的“和平诚意”从何表现呢?怎么可以证明先生们是真的愿意“缩短战争时间”、“减轻人民痛苦”、“以拯救人民为前提”,而没有一点儿假呢?此外,该发言人还说了许多废话,这些废话是骗不了任何人的,我们认为没有答复的必要。南京或广州或奉化或上海的假定的象征的国民党反动卖国“政府”(注意,政府二字加上引号)的先生们,如果你们以为我们的这篇声明的态度又有些不够郑重的话,那末,请原谅,我们对你们只能取这种态度。\\
\newpage\section*{\myformat{把军队变为工作队}\\\myformat{(一九四九年二月八日)}}\addcontentsline{toc}{section}{把军队变为工作队}
\begin{introduction}
\item 这是毛泽东为中共中央军事委员会写的复第二野战军和第三野战军的电报。这个电报,同时发给其它有关的野战军和有关的中央局。这个电报估计到在辽沈、淮海、平津三大战役以后,严重的战争时期已经过去,因而及时地提出了人民解放军不但是一个战斗队,同时必须是一个工作队,而且在一定条件下主要地要担负工作队的任务。这个方针,对当时新解放区干部问题的解决和人民革命事业的顺利发展起了巨大的作用。关于人民解放军是战斗队又是工作队的性质,参看本卷《在中国共产党第七届中央委员会第二次全体会议上的报告》的第二部分。
\end{introduction}
四日电悉。你们加紧整训,准备提前一个月出动\footnote[1]{ 指第二野战军和第三野战军准备把渡江作战的行动由一九四九年四月提前到三月。后来由于同国民党反动政府进行和平谈判,人民解放军渡江作战的时间又延至四月下旬。},甚好。望照此去做,不要放松。但在实际上,三月仍须整训,并须着重学习政策,准备接收并管理大城市。今后将一反过去二十年先乡村后城市的方式,而改变为先城市后乡村的方式。军队不但是一个战斗队,而且主要地是一个工作队。军队干部应当全体学会接收城市和管理城市,懂得在城市中善于对付帝国主义和国民党反动派,善于对付资产阶级,善于领导工人和组织工会,善于动员和组织青年,善于团结和训练新区的干部,善于管理工业和商业,善于管理学校、报纸、通讯社和广播电台,善于处理外交事务,善于处理各民主党派、人民团体的问题,善于调剂城市和乡村的关系,解决粮食、煤炭和其它必需品的问题,善于处理金融和财政问题。总之,过去军队干部和战士们所不熟悉的一切城市问题,今后均应全部负担在自己的身上。你们前进,要占领四五个省的地区,除城市外,还有广大乡村的工作要你们去做。南方乡村,因为完全是新区,和北方老区的工作根本不同。头一年还不能实行减租减息政策,大体上只能照原样交租交息。要在此种条件下去进行乡村工作。因此,乡村工作,也得从新学习。但是,乡村工作和城市工作比较起来,是易于学习的。城市工作则较为困难,而又是目前学习的最主要方面。如果我们的干部不能迅速学会管理城市,则我们将会发生极大困难。因此,你们必须在二月处理其它一切问题,而在三月一个整月内,全部学习城市工作和新区工作。国民党只有一百几十万军队,散布在广大地方。当然还有许多仗要打,但是像淮海战役\footnote[2]{ 见本卷《关于淮海战役的作战方针》注〔1〕。}那样大规模作战的可能性就不多了,或者简直可以说是没有了,严重的战争时期已经过去了。军队还是一个战斗队,在这一点上决不能松气,如果松气,那就是错误的。但是,军队变为工作队,现在已经要求我们这样提出任务了。如果现在我们还不提出此种任务,并下决心去做,我们就会犯极大的错误。我们现在正在准备五万三千个干部随军南下,但是这个数目很小。占领八九个省、占领几十个大城市所需要的工作干部,数量极大,这主要依靠军队本身自己解决。军队就是一个学校,二百一十万野战军,等于几千个大学和中学,一切工作干部,主要地依靠军队本身来解决。此点,你们必须有明确的认识。既然严重的战争基本上已经过去,则军队人数和装备的补充,以达到适当程度为宜,决不可要求太多、太好、太完备,以至引起财政危机。这一点,你们亦必须严重考虑。上述方针,完全适用于第四野战军,请林彪、罗荣桓同志同样注意。我们已和康生同志谈了许多,请他于十二日赶到你们处,和你们会商。你们意见如何及如何处置,会商后请即电告。华东局华东军区机构,立即移至徐州同总前委\footnote[3]{ 为适应淮海战役作战的需要,中共中央军事委员会于一九四八年十一月十六日决定,由刘伯承、陈毅、邓小平、粟裕、谭震林组成总前委,邓小平任总前委书记,统一领导中原野战军和华东野战军,执行领导淮海前线军事和作战的职权。淮海战役后,中央军委又作出决定,在渡江作战中,总前委照旧行使领导军事和作战的职权,华东局和总前委均直属中央。}和第三野战军前委一同工作,集中精力布置南进。一切后方工作交山东分局负责。\\
\newpage\section*{\myformat{四分五裂的反动派为什么还要空喊“全面和平”?}\\\myformat{(一九四九年二月十五日)}}\addcontentsline{toc}{section}{四分五裂的反动派为什么还要空喊“全面和平”?}
国民党反动统治崩溃的速度,比人们预料的要快。现在距离解放军攻克济南只有四个多月,距离攻克沈阳只有三个多月,但是国民党在军事上、政治上、经济上、文化宣传上的一切残余力量,却已经陷于不可挽救的四分五裂、土崩瓦解的状态。国民党统治的总崩溃开始于北线的辽沈战役\footnote[1]{ 见本卷《关于辽沈战役的作战方针》注〔1〕。}、平津战役\footnote[2]{ 见本卷《关于平津战役的作战方针》注〔1〕。}和南线的淮海战役\footnote[3]{ 见本卷《关于淮海战役的作战方针》注〔1〕。}期间,这三个战役使国民党在去年十月初至今年一月底的不足四个月中丧失约一百五十四万多人,包括国民党正规军一百四十四个整师。国民党统治总崩溃是中国人民解放战争和中国人民革命运动伟大胜利的必然结果,但是国民党及其美国主人的“和平”叫嚣,对于促进国民党崩溃一事,也起了相当的作用。国民党反动派从今年一月一日开始搬起的一块名叫“和平攻势”的石头,原想用来打击中国人民的,现在是打在他们自己的脚上了。或者说得正确些,是把国民党自己从头到脚都打烂了。除了傅作义将军协助人民解放军已经和平地解决了北平问题以外,各地希望和平解决的还大有人在。美国人站在一旁发干急,深恨其儿子们不争气。其实,和平攻势这个法宝出产于美国工厂,还在大半年以前就由美国人送给了国民党。司徒雷登\footnote[4]{ 见本卷《别了,司徒雷登》注〔1〕。}本人曾经泄露了这个秘密。他在蒋介石发出所谓元旦文告以后,曾告中央社记者说,这是“我过去所一直亲自努力以求的东西”。据美国通讯社称,该记者因发表了这段“不得发表”的话而丢了饭碗。蒋介石集团长期地不敢接受美国人的这个命令,其理由,在国民党中央宣传部去年十二月二十七日的一项指示中说得很明白:“我如不能战,即亦不能和。我如能战,则言和又徒使士气人心解体。故无论我能战与否,言和皆有百害而无一利。”国民党当时发出这个指示,是因为国民党的其它派别已经在主张言和了。去年十二月二十五日,白崇禧及其指导下的湖北省参议会向蒋介石提出了“和平解决”的问题\footnote[5]{ 一九四八年十二月二十四日,国民党华中“剿总”总司令白崇禧,利用当时对蒋介石极为不利的形势,发出致蒋介石电,提出“恢复和平谈判”以解决时局的主张,目的是逼蒋下台,抬高桂系地位。二十五日,在白崇禧指导下,湖北省参议会通过致蒋介石电,指出“如战祸继续蔓延,不立谋改弦更张之道,则国将不国,民将不民”,要蒋介石“循政治解决之常轨,觅取途径,恢复和谈”。},迫使蒋介石不得不在今年一月一日发布在五个条件\footnote[6]{ 见本卷《中共发言人关于命令国民党反动政府重新逮捕前日本侵华军总司令冈村宁次和逮捕国民党内战罪犯的谈话》注〔6〕。}下进行和谈的声明。蒋介石希望从白崇禧手里夺回和平攻势的发明权,并在新的商标下继续其旧的统治。蒋介石于一月八日派张群到汉口去要求白崇禧的支持,同日向美英法苏四国政府要求干涉中国的内战\footnote[7]{ 国民党政府于一九四九年一月八日向美、英、法、苏四国政府要求干涉中国内战,遭到四国政府拒绝。美国政府在其一月十二日答复国民党政府的“备忘录”中,说明美国之所以拒绝国民党政府的要求,是因“殊难相信”“能达到任何有益的效果”。这就是说,美国当时已感到再也无力挽救它所扶植的蒋介石反动政权的灭亡。}。但是这些步骤全都失败了。中国共产党毛泽东主席在一月十四日的声明,致命地击破了蒋介石的假和平阴谋,使蒋介石在一个星期以后不得不“引退”到幕后去。虽然蒋介石、李宗仁和美国人对于这一手曾经作过各种布置,希望合演一出比较可看的双簧,但是结果却和他们的预期相反,不但台下的观众愈走愈稀,连台上的演员也陆续失踪。蒋介石在奉化仍然以“在野地位”继续指挥他的残余力量,但是他已丧失了合法地位,相信他的人已愈来愈少。孙科的“行政院”自动宣布“迁政府于广州”,它一面脱离了它的“总统”“代总统”,另一面也脱离了它的“立法院”“监察院”。孙科的“行政院”号召战争\footnote[8]{ 国民党政府行政院长孙科于一九四九年二月六七两日在广州两次发表谈话,反对李宗仁关于以中共所提八项和平条件作为谈判基础的声明,说是“现政府已迁穗办公,吾人应对过去重新检讨”,又说“共党所提出之惩治战犯一节,即系绝对不能接受者”。},但是进行战争的“国防部”却既不在广州,也不在南京,人们只知道它的发言人在上海。这样,李宗仁在石头城上所能看见的东西,就只剩下了“天低吴楚,眼空无物”\footnote[9]{ 这是借用公元十四世纪元朝人萨都剌所作《登石头城》调寄《念奴娇》词中的话。这首词的上阕是:“石头城上,望天低吴楚,眼空无物。指点六朝形胜地,惟有青山如壁。蔽日旌旗,连云樯橹,白骨纷如雪。一江南北,消磨多少豪杰。”南京古称石头城。吴楚泛指长江的中下游。}。李宗仁自上月二十一日登台到现在下过的命令,没有一项是实行了的。虽然国民党已经没有一个“全面”的“政府”,虽然许多地方都在进行着局部和平的活动,但是国民党死硬派却在反对局部和平而要求所谓“全面和平”,其目的就是取消和平,妄想再战;他们深怕局部和平的活动蔓延起来,至于不可收拾。以一个四分五裂土崩瓦解的国民党而要求所谓“全面和平”的滑稽剧,在本月九日上海伪国防部政工局长战争罪犯邓文仪的一篇声明中,达到了高峰。邓文仪和孙科一样,推翻了上月二十二日李宗仁关于以中共的八项和平条件\footnote[10]{ 见本卷《中共中央毛泽东主席关于时局的声明》。}为谈判基础的声明,而要求所谓“平等的和平,全面的和平”,否则“不惜牺牲一切,与共党周旋到底”。但是邓文仪没有说出在今天他的对方究竟应和什么人去谈判“平等的”“全面的”和平。似乎找邓文仪是不能解决问题的,似乎不找邓文仪或者其它张三李四也不能解决问题,这就未免叫人为难了。据中央社上海九日电称:“新闻记者问邓文仪:李代总统是否已同意邓局长所发表之四项意见\footnote[11]{ 国民党政府国防部政工局长邓文仪,于一九四九年二月九日在上海发表《和平与战争的发展》的书面谈话,提出所谓“四项意见”:一、“政府要和”,二、“中共要战”,三、“北平局部和平成了骗局”,四、“备战以言和,能战方能和”,“不惜牺牲一切,与共党周旋到底”。}?答:本人系在国防部立场发言,本日所发表之四项意见,事前并未呈经李代总统过目。”邓文仪在这里不但创造了一个伪国防部的局部立场以区别于伪国民党政府的全面立场,而且事实上还创造了一个伪国防部政工局的小局部立场以区别于伪国防部的大局部立场。因为邓文仪公开反对并污蔑北平的和平解决,而伪国防部则在一月二十七日称赞北平的和平解决,是“为了缩短战争,获致和平,借以保全北平故都基础与文物古迹”,并称其它地方例如大同绥远等处\footnote[12]{ 在天津、北平解放后,华北国民党军队只剩下几个残余的孤立据点,其中包括太原、大同、新乡、安阳、归绥(今呼和浩特市)等地。太原国民党军于一九四九年四月二十四日被人民解放军完全歼灭。大同国民党军于四月二十九日接受和平改编。新乡国民党军于五月五日向人民解放军投降。五月六日,安阳国民党军被人民解放军歼灭,归绥于九月十九日和平解放。}亦将依同样方法“实施休战”。由此可见,叫喊“全面和平”最起劲的反动派,原来就是最缺乏全面立场的反动派。一个国防部政工局可以和国防部互相矛盾,又可以和它的代总统互相矛盾。这些反动派是今天中国实现和平的最大障碍。他们梦想在“全面和平”的口号下鼓吹全面战争,即所谓“战要全面战,和要全面和”。但是,事实上他们既没有什么力量实行全面和平,也没有什么力量实行全面战争。全面的力量是在中国人民、中国人民解放军、中国共产党和其它民主党派这一方面,不在四分五裂土崩瓦解的国民党方面。一方面,握有全面的力量,另一方面,陷于四分五裂土崩瓦解的绝境,这种局面,是中国人民长期奋斗和国民党长期作孽的结果。任何郑重的人,都不能忽视今天中国政治形势中这个基本的事实。\\
\newpage\section*{\myformat{国民党反动派由“呼吁和平”变为呼吁战争}\\\myformat{(一九四九年二月十六日)}}\addcontentsline{toc}{section}{国民党反动派由“呼吁和平”变为呼吁战争}
自从一月一日蒋匪介石发动和平攻势以后,曾经连篇累牍地表示自己是愿意“缩短战争时间”,“减轻人民痛苦”,“以拯救人民为前提”的国民党反动派的英雄好汉们,一到二月上旬,和平的调子就突然低落下去,“和共党周旋到底”的老调忽又高弹起来。最近数日,更是如此。二月十三日,国民党中央宣传部发给“各党部各党报”的《特别宣传指示》上说:“叶剑英向我后方宣传中共对和平有诚意,而指责政府军事布置为无诚意谋和。各报对此,必须依据下列各点从正面与侧面力加驳斥。”这个《特别宣传指示》一连列举了好几点应当“驳斥”的理由。“政府与其无条件投降,不如作战到底。”“毛泽东一月十四日声明所提八点为亡国条件,政府原不应接受。”“中共应负破坏和平之责任。今日中共反而提出所谓战犯名单,将政府负责人士尽皆列入,更要求政府先行逮捕,其蛮横无理,显而易见。中共如不改变此种作风,则和平商谈之途径,势难寻觅。”两星期以前那种如丧考妣地急着要谈判的神气,再也不见了。所谓“缩短战争时间”,“减轻人民痛苦”,“以拯救人民为前提”这些传遍人间、沁人心脾的名句,再也不提了。假如中共不愿意改变自己的“作风”,一定要惩办战争罪犯,那就不能谈和平了。究竟是以拯救人民为前提呢,还是以拯救战争罪犯为前提呢?按照国民党英雄好汉的《特别宣传指示》,是选择了后者。战争罪犯的名单,中共方面尚在向各民主党派人民团体征求意见中,现在已经收到了好几方面的意见。根据这些已经收到的意见,都是不赞成去年十二月二十五日中共权威人士所提的那个名单。他们认为那个名单所列战犯只有四十三个,为数太少;他们认为要负发动反革命战争屠杀数百万人民的责任的人决不止四十三个,而应当是一百几十个。现在姑且假定战犯将确定为一百几十个。那末,请问国民党的英雄好汉们,你们为什么要反对惩办战犯呢?你们不是愿意“缩短战争时间”“减轻人民痛苦”的吗?假如因为你们这一反对,使得战争还要打下去,岂非拖延时间,延长战祸?“拖延时间,延长战祸”这八个字的罪名是你们在一九四九年一月二十六日以南京政府发言人的名义发出声明,加在共产党身上的,现在难道你们想收回去,写上招贴,挂在你们自己身上,以为荣耀吗?你们是“以拯救人民为前提”的大慈大悲的人们,为什么一下子又改成以拯救战犯为前提了呢?根据你们政府内政部的统计,中国人民的数目,不是四亿五千万,而是四亿七千五百万,这和一百几十个战犯相比,究竟大小如何呢?英雄们是学过算术的,请你们按照算术教科书好好地算一下再作结论吧。倘若你们不去算清楚就将你们那个原来很好、我们也同意、全国人民也同意的提法——“以拯救人民为前提”,急急忙忙地改成“以拯救一百几十个战犯为前提”,那你们可要仔细,你们就一定站不住脚。这些口口声声“以拯救人民为前提”的人们,在自己“呼吁和平”几个星期之后,又不再是“呼吁和平”,而是呼吁战争了。国民党死硬派就是这样倒霉的,他们坚决地反对人民,站在人民的头上横行霸道,因而把自己孤立在宝塔的尖顶上,而且至死也不悔悟。长江流域和南方的人民大众,包括工人,农民,知识分子,城市小资产阶级,民族资产阶级,开明绅士,有良心的国民党人,都请听着:站在你们头上横行霸道的国民党死硬派,没有几天活命的时间了,我们和你们是站在一个方面的,一小撮死硬派不要几天就会从宝塔尖上跌下去,一个人民的中国就要出现了。\\
\newpage\section*{\myformat{评国民党对战争责任问题的几种答案}\\\myformat{(一九四九年二月十八日)}}\addcontentsline{toc}{section}{评国民党对战争责任问题的几种答案}
“政府自抗战结束以后,即以和平建国方针力谋中共问题之和平解决。经过一年半之时间,一切协议皆为中共所破坏,故中共应负破坏和平之责任。今日中共反而提出所谓战犯名单,将政府负责人士尽皆列入,更要求政府先行逮捕,其蛮横无理,显而易见。中共如不改变此种作风,则和平商谈之途径,势难寻觅。”以上是一九四九年二月十三日国民党中央宣传部所发《特别宣传指示》中关于战争责任问题的全部论点。\\
  这个论点,不是别人的,是第一名战争罪犯蒋介石的。蒋介石在其元旦声明里说:“中正为三民主义的信徒,秉承国父的遗教,本不愿在对日作战之后再继之以剿匪的军事,来加重人民的痛苦。所以抗日战事甫告结束,我们政府立即揭举和平建国的方针,更进而以政治商谈、军事调处的方法解决共党问题。不意经过了一年有半的时间,共党对于一切协议和方案都横加梗阻,使其不能依预期的步骤见诸实施。而最后更发动其全面武装叛乱,危害国家的生存。我政府迫不得已,乃忍痛动员,从事戡乱。”\\
  在蒋介石发表这个声明的前七天,即一九四八年十二月二十五日,即有中共权威人士提出了四十三个战犯名单,赫然列在第一名的,就是这个蒋介石。战犯们又要求和,又要逃避责任,只有将责任推在共产党身上一个法子。可是这是不调和的。共产党既然应负发动战争的责任,那末,就应当惩办共产党。既然是“匪”,就应当“剿匪”。既然“发动其全面武装叛乱”,就应当“戡乱”。“剿匪”,“戡乱”,是百分之百的对,为什么可以不剿不戡了呢?为什么从一九四九年一月一日以后,一切国民党的公开文件一律将“共匪”改成了“共党”呢?\\
  孙科觉得有些不妥,他在蒋介石发表元旦声明的同一天的晚上,发表广播演说,关于战争责任问题,提出了一个不同的论点。孙科说:“回忆三年前,当抗战胜利的初期,由于人民需要休养生息,由于国家需要积极建设,由于各党派对国家和人民的需要尚有共同的认识,我们曾经集合各方代表和社会贤达于一堂,举行过政治协商会议。经过三星期的努力,更多谢杜鲁门总统的特使马歇尔先生的善意调协,我们也曾经商定了一个和平建国纲领和解决各种争端的具体方案。假如当时我们能将各种方案及时实行,试问今日的中国应该是如何的繁荣,今天的中国人民应该是如何的幸福啊!可惜当时各方既未能完全放弃小我的利害,全国人民亦未能用最大的努力去促进这个和平运动的成功,遂致战祸复发,生灵涂炭。”\\
  孙科比较蒋介石“公道”一点。你看,他不是如同蒋介石那样,将战争责任一塌括子推在共产党身上,而是采取了“平均地权”的办法,将责任平分给“各方”。这里也有国民党,也有共产党,也有民主同盟\footnote[1]{ 见本书第三卷《论联合政府》注〔16〕。},也有社会贤达。不宁唯是,而且有“全国人民”,四亿七千五百万同胞一个也逃不了责任。蒋介石是专打共产党的板子,孙科是给各党各派无党无派全国同胞每人一板子,连蒋介石,也许还有孙科,也得挨上一板子。你看,两个国民党人,孙科和蒋介石,在这里打架。\\
  第三个国民党人跑上来说:不然,照我的意见,责任应全归国民党。这个人的名字叫做李宗仁。一九四九年一月二十二日,李宗仁以“代总统”的身份,发表了一个声明。关于战争责任问题,他说:“在八年抗战之后,继之以三年之内战,不仅将抗战胜利后国家可能复兴之一线生机毁灭无遗,而战祸遍及黄河南北,田园庐舍悉遭摧毁荒废,无辜人民之死伤成千累万,妻离子散啼饥号寒者到处皆是。此一惨绝人寰之浩劫,实为我国内战史上空前所未有。”\\
  李宗仁在这里出的是无头告示,他也没有说国民党应负责任,也没有说共产党或者别的方面应负责任,但是他说出了一个事实,这个“惨绝人寰的浩劫”,不是出在别的地方,而是出在“黄河南北”。查黄河以南直至长江,黄河以北直至松花江,谁在这里造成这个“惨绝人寰的浩劫”呢?难道是这里的人民和人民的军队自己打自己造成的吗?李宗仁是做过北平行营主任的,桂系的军队是和蒋系军队一道打到过山东省的沂蒙山区的\footnote[2]{ 沂蒙山区,指山东省的沂山、蒙山一带地区。曾经和蒋系军队一道进攻这一地区的桂系军队是第四十六军。该军于一九四六年十月从海南岛由海上输送到青岛登陆,一九四七年二月在山东省的莱芜地区全部被歼灭。},所以他有确实的情报,知道这种“浩劫”的地点和情况。如果说,李宗仁别的什么都不好,那末,他说出了这句老实话,总算是好的。而且他对这场战争起的名称,不叫“戡乱”或“剿匪”,而叫“内战”,这在国民党方面来说,也算得颇为别致。\\
  根据李宗仁自己的逻辑,在同一个声明里,他说:“中共方面所提八条件,政府愿即开始商谈。”李宗仁知道八条的第一条,就是惩办战犯,而且也有他自己的大名在内。战犯的应当惩办,是“浩劫”的逻辑的结论。为了这一点,至今国民党死硬派还在吞吞吐吐地埋怨李宗仁,即所谓“毛泽东一月十四日声明所提八点为亡国条件,政府原不应接受”。\\
  死硬派的埋怨之所以只能是吞吞吐吐,而不敢明目张胆,是有原因的。当蒋介石还没有“引退”时,死硬派原来想批驳八条,后来蒋介石一想不妥,决定不驳,大概是认为驳了就绝了路了,这是一月十九日的事情。当着一月十九日早上,张君劢从南京回到上海,发表谈话,说了“关于中共所提八项条件,政府不久即可能发布另一文告,提出答复”这句话的时候,中央社即于晚间发出通报说:“顷播沪电张君劢谈话一稿,请于电文末加注按语如下:张氏谈话中所说政府不久即发布另一文告一点,中央社记者顷自有关方面探悉,政府并无发布另一文告之拟议。”一月二十一日蒋介石发表“引退”声明,并无只字批评八条,并且把他自己的五条\footnote[3]{ 见本卷《中共发言人关于命令国民党反动政府重新逮捕前日本侵华军总司令冈村宁次和逮捕国民党内战罪犯的谈话》注〔6〕。}也取消了,改变为“使领土主权克臻完整,历史文化与社会秩序不受摧残,人民生活与自由权利确有保障,在此原则之下,以致和平之功”。宪法、法统、军队等项都不敢再提了。因此,李宗仁在一月二十二日敢于承认以中共的八条为谈判基础,国民党死硬派也不敢明目张胆地出面反对,只能吞吞吐吐地说一声“政府原不应接受”。\\
  孙科的“平均地权”政策是否坚持不变呢?也不。一九四九年二月五日孙科“迁政府于广州”以后,二月七日发表演说,关于战争责任问题,他说:“半年以来,因战祸蔓延,大局发生严重变化,人民痛苦万状。凡此种种,均系过去所犯错误、失败及不合理现象种下前因,以致有今日局势严重之后果。吾人深知中国需要三民主义。三民主义一日不能实现,则中国之问题始终不能解决。追忆本党总理二十年以前以三民主义亲自遗交本党,冀其逐步得以实行。苟获实行,绝不致演至今日不可收拾之局面。”人们请看,国民党政府的行政院长在这里,不是平分责任给一切党派和全国同胞,而是由国民党自己担负起来了。孙科将一切板子都打在国民党的屁股上,使人们觉得甚为痛快。至于共产党呢?孙院长说:“吾人试观中共能以诱惑及麻醉人民,亦无非仅以实行三民主义之民生主义一部分,即平均地权一节为号召。吾人实应深感惭愧,而加强警惕,重新检讨过去之错误。”谢谢亲爱的院长,共产党虽然尚有“诱惑及麻醉人民”的罪名,总算没有别的滔天大罪,致邀免打,获保首领及屁股而归。\\
  孙院长的可爱,还不止此。他在同一演说里又说:“今日共党势力之蔓延,亦即系因吾人信仰之主义未能实行之故。本党在过去最大之错误,即系党内若干人士过分迷信武力,对内则争权倾轧,坐贻敌人分化离间之机会。及至八年抗战结束,本为实现和平统一千载难逢之时机,政府方面亦原有以政治方式解决国内纠纷之计划,不幸未能贯彻实施。人民于连年战乱之后,已亟待休养生息。刀兵再起,民不聊生,痛苦殊深,亦影响士气之消沉,以致军事步步失利。蒋总统俯顺民情,鉴于军事方法之未能解决问题,乃于元旦发表文告,号召和平。”好了,孙科这一名战争罪犯,没有被捕,也没有被打,即自动招供,而且忠实无误。谁是迷信武力,发动战争,及至军事方法未能解决问题,方始求和的呢?就是国民党,就是蒋介石。孙院长用字造句也很正确,他说过分迷信武力的是他们党内的“若干人士”。这一点,对于中共仅仅要求惩办若干国民党人,把他们称之为战争罪犯,而不要求惩办更多的更不是全体的国民党人,是互相一致的。\\
  我们和孙科之间,在这个数目字上并无争论。不同的是在结论上。我们认为,对于这些“迷信武力”,使得“刀兵再起,民不聊生”的国民党的“若干人士”,必须当作战犯加以惩办。孙科则不同意这样做。他说:“现共方之迟迟不行指派代表,一味拖延,显示共方亦正迷信武力,自以为目前业已羽毛丰满,可以凭借武力征服全国,故拒绝先行停战,其用心亦极显然。余兹须郑重提出者,即为求获得永久之和平,双方必须以平等资格进行商谈,条件则应公平合理,为全国人所能接受者。”这样看来,孙院长又有些不可爱了。他似乎认为惩办战争罪犯一项条件不算公平合理。他的这些话,和二月十三日国民党宣传部的《特别宣传指示》对于战犯问题所表示的态度,是一样地吞吞吐吐,不敢明目张胆地提出反对,较之李宗仁敢于承认以惩办战犯为谈判的基础条件之一,大不相同。\\
  但是孙院长仍旧有可爱的地方,这即是他说共产党“亦正迷信武力”,是表现在“迟迟不行指派代表”和“拒绝先行停战”这两点上,而不是如同国民党那样在一九四六年就迷信武力发动惨绝人寰的战争。夫“迟迟不行指派代表”者,是因为确定战犯名单是一件大事,要是“为全国人所能接受者”,少了,多了,都不合实际,“全国人”(但不包括战犯及其帮凶)不能接受,故须和各民主党派人民团体互相商量,以此“拖延”了一段时间,并且未能迅速指派代表,引起了孙科之流颇为不快。但是这也不能一口断定即为“亦正迷信武力”。大约不要很久,战犯名单就可公布,代表就可指派,谈判就可开始,孙院长就不能说我们“迷信武力”了。\\
  至于“拒绝先行停战”,这是服从蒋总统元旦文告而采取的正确的态度。蒋总统元旦文告说:“只要共党一有和平的诚意,能作确切的表示,政府必开诚相见,愿与商讨停止战事、恢复和平的具体方法。”孙科的行政院,于一月十九日,做出了一个违反蒋介石上述文告的决议,说什么“立即先行无条件停战,并各指定代表进行和平商谈”。中共发言人曾于一月二十一日给了这个不通的决议以严正的批评\footnote[4]{ 见本卷《中共发言人评南京行政院的决议》。}。不料该院长充耳不闻,又于二月七日乱说什么中共“拒绝先行停战”,就是表示中共“亦正迷信武力”。连蒋介石那样的战争罪犯,也知道停止战争,恢复和平,没有商谈是不可能的,孙科在这点上比蒋介石差远了。\\
  人们知道孙科之所以成为战犯,是因为他一向赞助蒋介石发动战争,并坚持战争。直到一九四七年六月二十二日他还说:“在军事方面,只要打到底,终归可以解决。”“目前已无和谈可言,政府必须打垮共党,否则即是共党推翻国民政府。”\footnote[5]{ 这是孙科一九四七年六月二十二日在南京接见美联社、国民党中央日报和新民报的记者时发表的谈话中说的,当时孙科任国民党政府副主席。}他就是国民党内迷信武力的“若干人士”之一。现在他站在一旁说风凉话,好像他并没有迷信过武力,三民主义没有实行他也不负责任。这是不忠实的。无论正国法,或者在国民党内正党法,孙科都逃不了挨板子。\\
\newpage\section*{\myformat{在中国共产党第七届中央委员会第二次全体会议上的报告}\\\myformat{(1949.3.5)}}\addcontentsline{toc}{section}{在中国共产党第七届中央委员会第二次全体会议上的报告}
\begin{introduction}
\item 中国共产党第七届中央委员会第二次全体会议,一九四九年三月五日至十三日举行于河北省平山县西柏坡村。出席的有中央委员三十四人,候补中央委员十九人。这次会议是在中国人民革命全国胜利的前夜召开的,是一次极其重要的会议。毛泽东在这次会议上所作的报告,提出了促进革命迅速取得全国胜利和组织这个胜利的各项方针;说明了在全国胜利的局面下,党的工作重心必须由乡村移到城市,城市工作必须以生产建设为中心;规定了党在全国胜利以后,在政治、经济、外交方面应当采取的基本政策,特别着重地分析了当时中国经济各种成分的状况和党所必须采取的正确政策,指出了中国由农业国转变为工业国、由新民主主义社会转变为社会主义社会的发展方向。毛泽东估计了中国人民民主革命胜利以后的国内外阶级斗争的新形势,及时地警告资产阶级的“糖衣炮弹”将成为对于无产阶级的主要危险。毛泽东的这个报告,和他在同年六月所写的《论人民民主专政》一文,构成了为中国人民政治协商会议第一届全体会议所通过的、在新中国成立初期曾经起了临时宪法作用的《共同纲领》的政策基础。党的第七届中央委员会第二次全体会议,根据毛泽东的报告,通过了相应的决议。在这次会议以后,中共中央就由河北省平山县西柏坡村迁往北平。
\end{introduction}
\subsection*{\myformat{一 }}
辽沈、淮海、平津三战役\footnote[1]{ 见本卷《关于辽沈战役的作战方针》注〔1〕、《关于淮海战役的作战方针》注〔1〕、《关于平津战役的作战方针》注〔1〕。}以后,国民党军队的主力已被消灭。国民党的作战部队仅仅剩下一百多万人,分布在新疆到台湾的广大的地区内和漫长的战线上。今后解决这一百多万国民党军队的方式,不外天津、北平、绥远三种。用战斗去解决敌人,例如解决天津的敌人那样,仍然是我们首先必须注意和必须准备的。人民解放军的全体指挥员、战斗员,绝对不可以稍微松懈自己的战斗意志,任何松懈战斗意志的思想和轻敌的思想,都是错误的。按照北平方式解决问题的可能性是增加了,这就是迫使敌军用和平方法,迅速地彻底地按照人民解放军的制度改编为人民解放军。用这种方法解决问题,对于反革命遗迹的迅速扫除和反革命政治影响的迅速肃清,比较用战争方法解决问题是要差一些的。但是,这种方法是在敌军主力被消灭以后必然地要出现的,是不可避免的;同时也是于我军于人民有利的,即是可以避免伤亡和破坏。因此,各野战军领导同志都应注意和学会这样一种斗争方式。这是一种斗争方式,是一种不流血的斗争方式,并不是不用斗争可以解决问题的。绥远方式,是有意地保存一部分国民党军队,让它原封不动,或者大体上不动,就是说向这一部分军队作暂时的让步,以利于争取这部分军队在政治上站在我们方面,或者保持中立,以便我们集中力量首先解决国民党残余力量中的主要部分,在一个相当的时间之后(例如在几个月,半年,或者一年之后),再去按照人民解放军制度将这部分军队改编为人民解放军\footnote[2]{ 一九四九年九月十九日,国民党绥远省政府主席董其武、兵团司令官孙兰峰等率部四万余人起义。起义部队自一九五〇年二月二十一日起,在人民解放军绥远军区领导之下进行整编,至四月十日改编成为人民解放军。}。这是又一种斗争方式。这种斗争方式对于反革命遗迹和反革命的政治影响,较之北平方式将要保留得较多些,保留的时间也将较长些。但是这种反革命遗迹和反革命政治影响,归根到底要被肃清,这是毫无疑问的。决不可以认为反革命力量顺从我们了,他们就成了革命党了,他们的反革命思想和反革命企图就不存在了。决不是这样。他们中的许多人将被改造,他们中的一部分人将被淘汰,某些坚决反革命分子将受到镇压。\\
\subsection*{\myformat{二 }}
人民解放军永远是一个战斗队。就是在全国胜利以后,在国内没有消灭阶级和世界上存在着帝国主义制度的历史时期内,我们的军队还是一个战斗队。对于这一点不能有任何的误解和动摇。人民解放军又是一个工作队,特别是在南方各地用北平方式或者绥远方式解决问题的时候是这样。随着战斗的逐步地减少,工作队的作用就增加了。有一种可能的情况,即在不要很久的时间之内,将要使人民解放军全部地转化为工作队,这种情况我们必须估计到。现在准备随军南下的五万三千个干部,对于不久将要被我们占领的极其广大的新地区来说,是很不够用的,我们必须准备把二百一十万野战军全部地化为工作队。这样,干部就够用了,广大地区的工作就可以展开了。我们必须把二百一十万野战军看成一个巨大的干部学校。\\
\subsection*{\myformat{三 }}
从一九二七年到现在,我们的工作重点是在乡村,在乡村聚集力量,用乡村包围城市,然后取得城市。采取这样一种工作方式的时期现在已经完结。从现在起,开始了由城市到乡村并由城市领导乡村的时期。党的工作重心由乡村移到了城市。在南方各地,人民解放军将是先占城市,后占乡村。城乡必须兼顾,必须使城市工作和乡村工作,使工人和农民,使工业和农业,紧密地联系起来。决不可以丢掉乡村,仅顾城市,如果这样想,那是完全错误的。但是党和军队的工作重心必须放在城市,必须用极大的努力去学会管理城市和建设城市。必须学会在城市中向帝国主义者、国民党、资产阶级作政治斗争、经济斗争和文化斗争,并向帝国主义者作外交斗争。既要学会同他们作公开的斗争,又要学会同他们作荫蔽的斗争。如果我们不去注意这些问题,不去学会同这些人作这些斗争,并在斗争中取得胜利,我们就不能维持政权,我们就会站不住脚,我们就会失败。在拿枪的敌人被消灭以后,不拿枪的敌人依然存在,他们必然地要和我们作拚死的斗争,我们决不可以轻视这些敌人。如果我们现在不是这样地提出问题和认识问题,我们就要犯极大的错误。\\
\subsection*{\myformat{四 }}
在城市斗争中,我们依靠谁呢?有些糊涂的同志认为不是依靠工人阶级,而是依靠贫民群众。有些更糊涂的同志认为是依靠资产阶级。在发展工业的方向上,有些糊涂的同志认为主要地不是帮助国营企业的发展,而是帮助私营企业的发展;或者反过来,认为只要注意国营企业就够了,私营企业是无足轻重的了。我们必须批判这些糊涂思想。我们必须全心全意地依靠工人阶级,团结其它劳动群众,争取知识分子,争取尽可能多的能够同我们合作的民族资产阶级分子及其代表人物站在我们方面,或者使他们保持中立,以便向帝国主义者、国民党、官僚资产阶级作坚决的斗争,一步一步地去战胜这些敌人。同时即开始着手我们的建设事业,一步一步地学会管理城市,恢复和发展城市中的生产事业。关于恢复和发展生产的问题,必须确定:第一是国营工业的生产,第二是私营工业的生产,第三是手工业生产。从我们接管城市的第一天起,我们的眼睛就要向着这个城市的生产事业的恢复和发展。务须避免盲目地乱抓乱碰,把中心任务忘记了,以至于占领一个城市好几个月,生产建设的工作还没有上轨道,甚至许多任务业陷于停顿状态,引起工人失业,工人生活降低,不满意共产党。这种状态是完全不能容许的。为了这一点,我们的同志必须用极大的努力去学习生产的技术和管理生产的方法,必须去学习同生产有密切联系的商业工作、银行工作和其它工作。只有将城市的生产恢复起来和发展起来了,将消费的城市变成生产的城市了,人民政权才能巩固起来。城市中其它的工作,例如党的组织工作,政权机关的工作,工会的工作,其它各种民众团体的工作,文化教育方面的工作,肃反工作,通讯社报纸广播电台的工作,都是围绕着生产建设这一个中心工作并为这个中心工作服务的。如果我们在生产工作上无知,不能很快地学会生产工作,不能使生产事业尽可能迅速地恢复和发展,获得确实的成绩,首先使工人生活有所改善,并使一般人民的生活有所改善,那我们就不能维持政权,我们就会站不住脚,我们就会要失败。\\
\subsection*{\myformat{五 }}
南方和北方的情况是不同的,党的工作任务也就必须有所区别。南方现时还是被国民党统治的区域。在这里,党和人民解放军的任务是在城市和乡村中消灭国民党的反动武装力量,建立党的组织,建立政权,发动民众,建立工会、农会和其它民众团体,建立人民武装力量,肃清国民党残余势力,恢复和发展生产事业。在乡村中,则是首先有步骤地展开清剿土匪和反对恶霸即地主阶级当权派的斗争,完成减租减息的准备工作,以便在人民解放军到达那个地区大约一年或者两年以后,就能实现减租减息的任务,造成分配土地的先决条件;同时必须注意尽可能地维持农业生产的现有水平不使降低。北方则除少数新解放区以外,是完全另外一种情况。在这里,已经推翻了国民党的统治,建立了人民的统治,并且根本上解决了土地问题。党在这里的中心任务,是动员一切力量恢复和发展生产事业,这是一切工作的重点所在。同时必须恢复和发展文化教育事业,肃清残余的反动力量,巩固整个北方,支持人民解放军。\\
\subsection*{\myformat{六 }}
我们已经进行了广泛的经济建设工作,党的经济政策已经在实际工作中实施,并且收到了显着的成效。但是,在为什么应当采取这样的经济政策而不应当采取别样的经济政策这个问题上,在理论和原则性的问题上,党内是存在着许多糊涂思想的。这个问题应当怎样来回答呢?我们认为应当这样地来回答。中国的工业和农业在国民经济中的比重,就全国范围来说,在抗日战争以前,大约是现代性的工业占百分之十左右,农业和手工业占百分之九十左右。这是帝国主义制度和封建制度压迫中国的结果,这是旧中国半殖民地和半封建社会性质在经济上的表现,这也是在中国革命的时期内和在革命胜利以后一个相当长的时期内一切问题的基本出发点。从这一点出发,产生了我党一系列的战略上、策略上和政策上的问题。对于这些问题的进一步的明确的认识和解决,是我党当前的重要任务。这就是说:\\
  第一,中国已经有大约百分之十左右的现代性的工业经济,这是进步的,这是和古代不同的。由于这一点,中国已经有了新的阶级和新的政党——无产阶级和资产阶级,无产阶级政党和资产阶级政党。无产阶级及其政党,由于受到几重敌人的压迫,得到了锻炼,具有了领导中国人民革命的资格。谁要是忽视或轻视了这一点,谁就要犯右倾机会主义的错误。\\
  第二,中国还有大约百分之九十左右的分散的个体的农业经济和手工业经济,这是落后的,这是和古代没有多大区别的,我们还有百分之九十左右的经济生活停留在古代。古代有封建的土地所有制,现在被我们废除了,或者即将被废除,在这点上,我们已经或者即将区别于古代,取得了或者即将取得使我们的农业和手工业逐步地向着现代化发展的可能性。但是,在今天,在今后一个相当长的时期内,我们的农业和手工业,就其基本形态说来,还是和还将是分散的和个体的,即是说,同古代近似的。谁要是忽视或轻视了这一点,谁就要犯“左”倾机会主义的错误。\\
  第三,中国的现代性工业的产值虽然还只占国民经济总产值的百分之十左右,但是它却极为集中,最大的和最主要的资本是集中在帝国主义者及其走狗中国官僚资产阶级的手里。没收这些资本归无产阶级领导的人民共和国所有,就使人民共和国掌握了国家的经济命脉,使国营经济成为整个国民经济的领导成分。这一部分经济,是社会主义性质的经济,不是资本主义性质的经济。谁要是忽视或轻视了这一点,谁就要犯右倾机会主义的错误。\\
  第四,中国的私人资本主义工业,占了现代性工业中的第二位,它是一个不可忽视的力量。中国的民族资产阶级及其代表人物,由于受了帝国主义、封建主义和官僚资本主义的压迫或限制,在人民民主革命斗争中常常采取参加或者保持中立的立场。由于这些,并由于中国经济现在还处在落后状态,在革命胜利以后一个相当长的时期内,还需要尽可能地利用城乡私人资本主义的积极性,以利于国民经济的向前发展。在这个时期内,一切不是于国民经济有害而是于国民经济有利的城乡资本主义成分,都应当容许其存在和发展。这不但是不可避免的,而且是经济上必要的。但是中国资本主义的存在和发展,不是如同资本主义国家那样不受限制任其泛滥的。它将从几个方面被限制——在活动范围方面,在税收政策方面,在市场价格方面,在劳动条件方面。我们要从各方面,按照各地、各业和各个时期的具体情况,对于资本主义采取恰如其分的有伸缩性的限制政策。孙中山的节制资本的口号,我们依然必须用和用得着。但是为了整个国民经济的利益,为了工人阶级和劳动人民现在和将来的利益,决不可以对私人资本主义经济限制得太大太死,必须容许它们在人民共和国的经济政策和经济计划的轨道内有存在和发展的余地。对于私人资本主义采取限制政策,是必然要受到资产阶级在各种程度和各种方式上的反抗的,特别是私人企业中的大企业主,即大资本家。限制和反限制,将是新民主主义国家内部阶级斗争的主要形式。如果认为我们现在不要限制资本主义,认为可以抛弃“节制资本”的口号,这是完全错误的,这就是右倾机会主义的观点。但是反过来,如果认为应当对私人资本限制得太大太死,或者认为简直可以很快地消灭私人资本,这也是完全错误的,这就是“左”倾机会主义或冒险主义的观点。\\
  第五,占国民经济总产值百分之九十的分散的个体的农业经济和手工业经济,是可能和必须谨慎地、逐步地而又积极地引导它们向着现代化和集体化的方向发展的,任其自流的观点是错误的。必须组织生产的、消费的和信用的合作社,和中央、省、市、县、区的合作社的领导机关。这种合作社是以私有制为基础的在无产阶级领导的国家政权管理之下的劳动人民群众的集体经济组织。中国人民的文化落后和没有合作社传统,可能使得我们遇到困难;但是可以组织,必须组织,必须推广和发展。单有国营经济而没有合作社经济,我们就不可能领导劳动人民的个体经济逐步地走向集体化,就不可能由新民主主义社会发展到将来的社会主义社会,就不可能巩固无产阶级在国家政权中的领导权。谁要是忽视或轻视了这一点,谁也就要犯绝大的错误。国营经济是社会主义性质的,合作社经济是半社会主义性质的,加上私人资本主义,加上个体经济,加上国家和私人合作的国家资本主义经济,这些就是人民共和国的几种主要的经济成分,这些就构成新民主主义的经济形态。\\
  第六,人民共和国的国民经济的恢复和发展,没有对外贸易的统制政策是不可能的。从中国境内肃清了帝国主义、封建主义、官僚资本主义和国民党的统治(这是帝国主义、封建主义和官僚资本主义三者的集中表现),还没有解决建立独立的完整的工业体系问题,只有待经济上获得了广大的发展,由落后的农业国变成了先进的工业国,才算最后地解决了这个问题。而欲达此目的,没有对外贸易的统制是不可能的。中国革命在全国胜利,并且解决了土地问题以后,中国还存在着两种基本的矛盾。第一种是国内的,即工人阶级和资产阶级的矛盾。第二种是国外的,即中国和帝国主义国家的矛盾。因为这样,工人阶级领导的人民共和国的国家政权,在人民民主革命胜利以后,不是可以削弱,而是必须强化。对内的节制资本和对外的统制贸易,是这个国家在经济斗争中的两个基本政策。谁要是忽视或轻视了这一点,谁就将要犯绝大的错误。\\
  第七,中国的经济遗产是落后的,但是中国人民是勇敢而勤劳的,中国人民革命的胜利和人民共和国的建立,中国共产党的领导,加上世界各国工人阶级的援助,其中主要地是苏联的援助,中国经济建设的速度将不是很慢而可能是相当地快的,中国的兴盛是可以计日程功的。对于中国经济复兴的悲观论点,没有任何的根据。\\
\subsection*{\myformat{七 }}
旧中国是一个被帝国主义所控制的半殖民地国家。中国人民民主革命的彻底的反帝国主义的性质,使得帝国主义者极为仇视这个革命,竭尽全力地帮助国民党。这就更加激起了中国人民对于帝国主义者的深刻的愤怒,并使帝国主义者丧失了自己在中国人民中的最后一点威信。同时,整个帝国主义制度在第二次世界大战以后是大大地削弱了,以苏联为首的世界反帝国主义阵线的力量是空前地增长了。所有这些情形,使得我们可以采取和应当采取有步骤地彻底地摧毁帝国主义在中国的控制权的方针。帝国主义者的这种控制权,表现在政治、经济和文化等方面。在国民党军队被消灭、国民党政府被打倒的每一个城市和每一个地方,帝国主义者在政治上的控制权即随之被打倒,他们在经济上和文化上的控制权也被打倒。但帝国主义者直接经营的经济事业和文化事业依然存在,被国民党承认的外交人员和新闻记者依然存在。对于这些,我们必须分别先后缓急,给以正当的解决。不承认国民党时代的任何外国外交机关和外交人员的合法地位,不承认国民党时代的一切卖国条约的继续存在,取消一切帝国主义在中国开办的宣传机关,立即统制对外贸易,改革海关制度,这些都是我们进入大城市的时候所必须首先采取的步骤。在做了这些以后,中国人民就在帝国主义面前站立起来了。剩下的帝国主义的经济事业和文化事业,可以让它们暂时存在,由我们加以监督和管制,以待我们在全国胜利以后再去解决。对于普通外侨,则保护其合法的利益,不加侵犯。关于帝国主义对我国的承认问题,不但现在不应急于去解决,而且就是在全国胜利以后的一个相当时期内也不必急于去解决。我们是愿意按照平等原则同一切国家建立外交关系的,但是从来敌视中国人民的帝国主义,决不能很快地就以平等的态度对待我们,只要一天它们不改变敌视的态度,我们就一天不给帝国主义国家在中国以合法的地位。关于同外国人做生意,那是没有问题的,有生意就得做,并且现在已经开始做,几个资本主义国家的商人正在互相竞争。我们必须尽可能地首先同社会主义国家和人民民主国家做生意,同时也要同资本主义国家做生意。\\
\subsection*{\myformat{八 }}
召集政治协商会议和成立民主联合政府的一切条件,均已成熟。一切民主党派、人民团体和无党派民主人士都站在我们方面。上海和长江流域的资产阶级,正在同我们拉关系。南北通航通邮业已开始。陷于四分五裂的国民党,已经脱离了一切群众。我们正在准备和南京反动政府进行谈判\footnote[3]{ 关于和南京国民党反动政府举行和平谈判事宜,中共中央于一九四九年三月二十六日决定:“(一)谈判开始时间:四月一日。(二)谈判地点:北平。(三)派周恩来、林伯渠、林彪、叶剑英、李维汉为代表(四月一日中共中央决定加派聂荣臻为代表),周恩来为首席代表,与南京方面的代表团举行谈判,按照一月十四日毛泽东主席对时局的声明及其所提八项条件以为双方谈判的基础。(四)将上列各项经广播电台即日通知南京国民党反动政府,按照上述时间地点,派遣其代表团,携带为八项条件所需的必要材料,以利举行谈判。”}。南京反动政府方面在这个谈判中的推动力量是桂系军阀,国民党主和派和上海资产阶级。他们的目的是使联合政府中有他们一份,尽可能地保存较多的军队,保存上海和南方资产阶级的利益,力求使革命带上温和的色彩。这一派人承认以我们的八条\footnote[4]{ 见本卷《中共中央毛泽东主席关于时局的声明》。}为谈判基础,但是希望讨价还价,使他们的损失不要太大。企图破坏这一谈判的是蒋介石及其死党。蒋介石还有六十个师位于江南一带,他们仍在准备作战。我们的方针是不拒绝谈判,要求对方完全承认八条,不许讨价还价。其交换条件是不打桂系和其它国民党主和派;一年左右也不去改编他们的军队;南京政府中的一部分人员允许其加入政治协商会议和联合政府;对上海和南方资产阶级的某些利益允许给以保护。这个谈判是全面性的,如能成功,对于我们向南方进军和占领南方各大城市将要减少许多阻碍,是有很大利益的。不能成功,则待进军以后各个地进行地方性的谈判。谈判的时间拟在三月下旬。我们希望四月或五月占领南京,然后在北平召集政治协商会议,成立联合政府,并定都北平。我们既然允许谈判,就要准备在谈判成功以后许多麻烦事情的到来,就要准备一副清醒的头脑去对付对方采用孙行者钻进铁扇公主肚子里兴妖作怪\footnote[5]{ 孙行者钻进铁扇公主的肚子里从而战败铁扇公主的故事,见明朝吴承恩着的神话小说《西游记》第五十九回。 }的政策。只要我们精神上有了充分的准备,我们就可以战胜任何兴妖作怪的孙行者。不论是全面的和平谈判,或者局部的和平谈判,我们都应当这样去准备。我们不应当怕麻烦、图清静而不去接受这些谈判,我们也不应当糊里糊涂地去接受这些谈判。我们的原则性必须是坚定的,我们也要有为了实现原则性的一切许可的和必需的灵活性。\\
\subsection*{\myformat{九 }}
无产阶级领导的以工农联盟为基础的人民民主专政,要求我们党去认真地团结全体工人阶级、全体农民阶级和广大的革命知识分子,这些是这个专政的领导力量和基础力量。没有这种团结,这个专政就不能巩固。同时也要求我们党去团结尽可能多的能够同我们合作的城市小资产阶级和民族资产阶级的代表人物,它们的知识分子和政治派别,以便在革命时期使反革命势力陷于孤立,彻底地打倒国内的反革命势力和帝国主义势力;在革命胜利以后,迅速地恢复和发展生产,对付国外的帝国主义,使中国稳步地由农业国转变为工业国,把中国建设成一个伟大的社会主义国家。因为这样,我党同党外民主人士长期合作的政策,必须在全党思想上和工作上确定下来。我们必须把党外大多数民主人士看成和自己的干部一样,同他们诚恳地坦白地商量和解决那些必须商量和解决的问题,给他们工作做,使他们在工作岗位上有职有权,使他们在工作上做出成绩来。从团结他们出发,对他们的错误和缺点进行认真的和适当的批评或斗争,达到团结他们的目的。对他们的错误或缺点采取迁就态度,是不对的。对他们采取关门态度或敷衍态度,也是不对的。每一个大城市和每一个中等城市,每一个战略性区域和每一个省,都应当培养一批能够同我们合作的有威信的党外民主人士。我们党内由土地革命战争时期的关门主义作风所养成的对待党外民主人士的不正确态度,在抗日时期并没有完全克服,在一九四七年各根据地土地改革高潮时期又曾出现过。这种态度只会使我党陷于孤立,使人民民主专政不能巩固,使敌人获得同盟者。现在中国第一次在我党领导之下的政治协商会议即将召开,民主联合政府即将成立,革命即将在全国胜利,全党对于这个问题必须有认真的检讨和正确的认识,必须反对右的迁就主义和“左”的关门主义或敷衍主义两种倾向,而采取完全正确的态度。\\
\subsection*{\myformat{十 }}
我们很快就要在全国胜利了。这个胜利将冲破帝国主义的东方战线,具有伟大的国际意义。夺取这个胜利,已经是不要很久的时间和不要花费很大的气力了;巩固这个胜利,则是需要很久的时间和要花费很大的气力的事情。资产阶级怀疑我们的建设能力。帝国主义者估计我们终久会要向他们讨乞才能活下去。因为胜利,党内的骄傲情绪,以功臣自居的情绪,停顿起来不求进步的情绪,贪图享乐不愿再过艰苦生活的情绪,可能生长。因为胜利,人民感谢我们,资产阶级也会出来捧场。敌人的武力是不能征服我们的,这点已经得到证明了。资产阶级的捧场则可能征服我们队伍中的意志薄弱者。可能有这样一些共产党人,他们是不曾被拿枪的敌人征服过的,他们在这些敌人面前不愧英雄的称号;但是经不起人们用糖衣裹着的炮弹的攻击,他们在糖弹面前要打败仗。我们必须预防这种情况。夺取全国胜利,这只是万里长征走完了第一步。如果这一步也值得骄傲,那是比较渺小的,更值得骄傲的还在后头。在过了几十年之后来看中国人民民主革命的胜利,就会使人们感觉那好像只是一出长剧的一个短小的序幕。剧是必须从序幕开始的,但序幕还不是高潮。中国的革命是伟大的,但革命以后的路程更长,工作更伟大,更艰苦。这一点现在就必须向党内讲明白,务必使同志们继续地保持谦虚、谨慎、不骄、不躁的作风,务必使同志们继续地保持艰苦奋斗的作风。我们有批评和自我批评这个马克思列宁主义的武器。我们能够去掉不良作风,保持优良作风。我们能够学会我们原来不懂的东西。我们不但善于破坏一个旧世界,我们还将善于建设一个新世界。中国人民不但可以不要向帝国主义者讨乞也能活下去,而且还将活得比帝国主义国家要好些。\\
\newpage\section*{\myformat{党委会的工作方法}\\\myformat{(一九四九年三月十三日)}}\addcontentsline{toc}{section}{党委会的工作方法}
\begin{introduction}
\item  这是毛泽东在中国共产党第七届中央委员会第二次全体会议上所作的结论的一部分。
\end{introduction}
一、党委书记要善于当“班长”。党的委员会有一二十个人,像军队的一个班,书记好比是“班长”。要把这个班带好,的确不容易。目前各中央局、分局都领导很大的地区,担负很繁重的任务。领导工作不仅要决定方针政策,还要制定正确的工作方法。有了正确的方针政策,如果在工作方法上疏忽了,还是要发生问题。党委要完成自己的领导任务,就必须依靠党委这“一班人”,充分发挥他们的作用。书记要当好“班长”,就应该很好地学习和研究。书记、副书记如果不注意向自己的“一班人”作宣传工作和组织工作,不善于处理自己和委员之间的关系,不去研究怎样把会议开好,就很难把这“一班人”指挥好。如果这“一班人”动作不整齐,就休想带领千百万人去作战,去建设。当然,书记和委员之间的关系是少数服从多数,这同班长和战士之间的关系是不一样的。这里不过是一个比方。\\
  二、要把问题摆到桌面上来。不仅“班长”要这样做,委员也要这样做。不要在背后议论。有了问题就开会,摆到桌面上来讨论,规定它几条,问题就解决了。有问题而不摆到桌面上来,就会长期不得解决,甚至一拖几年。“班长”和委员还要能互相谅解。书记和委员,中央和各中央局,各中央局和区党委之间的谅解、支持和友谊,比什么都重要。这一点过去大家不注意,七次代表大会以来,在这方面大有进步,友好团结关系大大增进了。今后仍然应该不断注意。\\
  三、“互通情报”。就是说,党委各委员之间要把彼此知道的情况互相通知、互相交流。这对于取得共同的语言是很重要的。有些人不是这样做,而是像老子说的“鸡犬之声相闻,老死不相往来”\footnote[1]{ 见《老子》第八十章。原文是:“邻国相望,鸡犬之声相闻,民至老死不相往来。”},结果彼此之间就缺乏共同的语言。我们有些高级干部,在马克思列宁主义的基本理论问题上也有不同的语言,原因是学习还不够。现在党内的语言比较一致了,但是,问题还没有完全解决。例如,在土地改革中,对什么是“中农”和什么是“富农”,就还有不同的了解。\\
  四、不懂得和不了解的东西要问下级,不要轻易表示赞成或反对。有些文件起草出来压下暂时不发,就是因为其中还有些问题没有弄清楚,需要先征求下级的意见。我们切不可强不知以为知,要“不耻下问”\footnote[2]{ 见《论语·公冶长》。原文是:“敏而好学,不耻下问。”},要善于倾听下面干部的意见。先做学生,然后再做先生;先向下面干部请教,然后再下命令。各中央局、各前委处理问题的时候,除军事情况紧急和事情已经弄清楚者外,都应该这样办。这不会影响自己的威信,而只会增加自己的威信。我们做出的决定包括了下面干部提出的正确意见,他们当然拥护。下面干部的话,有正确的,也有不正确的,听了以后要加以分析。对正确的意见,必须听,并且照它做。中央领导之所以正确,主要是由于综合了各地供给的材料、报告和正确的意见。如果各地不来材料,不提意见,中央就很难正确地发号施令。对下面来的错误意见也要听,根本不听是不对的;不过听了而不照它做,并且要给以批评。\\
  五、学会“弹钢琴”。弹钢琴要十个指头都动作,不能有的动,有的不动。但是,十个指头同时都按下去,那也不成调子。要产生好的音乐,十个指头的动作要有节奏,要互相配合。党委要抓紧中心工作,又要围绕中心工作而同时开展其它方面的工作。我们现在管的方面很多,各地、各军、各部门的工作,都要照顾到,不能只注意一部分问题而把别的丢掉。凡是有问题的地方都要点一下,这个方法我们一定要学会。钢琴有人弹得好,有人弹得不好,这两种人弹出来的调子差别很大。党委的同志必须学好“弹钢琴”。\\
  六、要“抓紧”。就是说,党委对主要工作不但一定要“抓”,而且一定要“抓紧”。什么东西只有抓得很紧,毫不放松,才能抓住。抓而不紧,等于不抓。伸着巴掌,当然什么也抓不住。就是把手握起来,但是不握紧,样子像抓,还是抓不住东西。我们有些同志,也抓主要工作,但是抓而不紧,所以工作还是不能做好。不抓不行,抓而不紧也不行。\\
  七、胸中有“数”。这是说,对情况和问题一定要注意到它们的数量方面,要有基本的数量的分析。任何质量都表现为一定的数量,没有数量也就没有质量。我们有许多同志至今不懂得注意事物的数量方面,不懂得注意基本的统计、主要的百分比,不懂得注意决定事物质量的数量界限,一切都是胸中无“数”,结果就不能不犯错误。例如,要进行土地改革,对于地主、富农、中农、贫农各占人口多少,各有多少土地,这些数字就必须了解,才能据以定出正确的政策。对于何谓富农,何谓富裕中农,有多少剥削收入才算富农,否则就算富裕中农,这也必须找出一个数量的界限。在任何群众运动中,群众积极拥护的有多少,反对的有多少,处于中间状态的有多少,这些都必须有个基本的调查,基本的分析,不可无根据地、主观地决定问题。\\
  八、“安民告示”。开会要事先通知,像出安民告示一样,让大家知道要讨论什么问题,解决什么问题,并且早作准备。有些地方开干部会,事前不准备好报告和决议草案,等开会的人到了才临时凑合,好像“兵马已到,粮草未备”,这是不好的。如果没有准备,就不要急于开会。\\
  九、“精兵简政”。讲话、演说、写文章和写决议案,都应当简明扼要。会议也不要开得太长。\\
  十、注意团结那些和自己意见不同的同志一道工作。不论在地方上或部队里,都应该注意这一条。对党外人士也是一样。我们都是从五湖四海汇集拢来的,我们不仅要善于团结和自己意见相同的同志,而且要善于团结和自己意见不同的同志一道工作。我们当中还有犯过很大错误的人,不要嫌这些人,要准备和他们一道工作。\\
  十一、力戒骄傲。这对领导者是一个原则问题,也是保持团结的一个重要条件。就是没有犯过大错误,而且工作有了很大成绩的人,也不要骄傲。禁止给党的领导者祝寿,禁止用党的领导者的名字作地名、街名和企业的名字,保持艰苦奋斗作风,制止歌功颂德现象。\\
  十二、划清两种界限。首先,是革命还是反革命?是延安还是西安\footnote[3]{ 延安是一九三七年一月至一九四七年三月中共中央的所在地,西安则是国民党反动派在西北的统治中心。毛泽东以此来比喻革命和反革命。}?有些人不懂得要划清这种界限。例如,他们反对官僚主义,就把延安说得好似“一无是处”,而没有把延安的官僚主义同西安的官僚主义比较一下,区别一下。这就从根本上犯了错误。其次,在革命的队伍中,要划清正确和错误、成绩和缺点的界限,还要弄清它们中间什么是主要的,什么是次要的。例如,成绩究竟是三分还是七分?说少了不行,说多了也不行。一个人的工作,究竟是三分成绩七分错误,还是七分成绩三分错误,必须有个根本的估计。如果是七分成绩,那末就应该对他的工作基本上加以肯定。把成绩为主说成错误为主,那就完全错了。我们看问题一定不要忘记划清这两种界限:革命和反革命的界限,成绩和缺点的界限。记着这两条界限,事情就好办,否则就会把问题的性质弄混淆了。自然,要把界限划好,必须经过细致的研究和分析。我们对于每一个人和每一件事,都应该采取分析研究的态度。\\
  我和政治局的同志觉得,要有以上这些方法,才能把党委的工作搞好。除了开好代表大会以外,党的各级委员会把自己的领导工作做好,是极为重要的。我们一定要讲究工作方法,把党委的领导工作提高一步。\\
\newpage\section*{\myformat{南京政府向何处去?}\\\myformat{(一九四九年四月四日)}}\addcontentsline{toc}{section}{南京政府向何处去?}
两条路摆在南京国民党政府及其军政人员的面前:一条是向蒋介石战犯集团及其主人美国帝国主义靠拢,这就是继续与人民为敌,而在人民解放战争中和蒋介石战犯集团同归于尽;一条是向人民靠拢,这就是与蒋介石战犯集团和美国帝国主义决裂,而在人民解放战争中立功赎罪,以求得人民的宽恕和谅解。第三条路是没有的。\\
  在南京的李宗仁何应钦政府\footnote[1]{ 一九四九年三月十二日,李宗仁在孙科辞职后,任命何应钦继任行政院长。}中,存在着三部分人。一部分人坚持地走第一条路。无论他们在口头上怎样说得好听,在行动上他们是继续备战,继续卖国,继续压迫和屠杀要求真和平的人民。他们是蒋介石的死党。一部分人愿意走第二条路,但是他们还不能作出有决定性的行动。第三部分是一些徘徊歧路、动向不明的人们。他们既不想得罪蒋介石和美国政府,又想得到人民民主阵营的谅解和容纳。但这是幻想,是不可能的。\\
  南京的李宗仁何应钦政府,基本上是第一部分人和第三部分人的混合物,第二部分人为数甚少。这个政府到今天为止,仍然是蒋介石和美国政府的工具。\\
  四月一日发生于南京的惨案\footnote[2]{ 一九四九年四月一日,南京十一个专科以上学校的学生六千余人举行游行示威,要求国民党反动政府接受中国共产党的八项和平谈判条件。国民党南京卫戍总司令张耀明在蒋介石授意下,指使军警特务凶殴示威学生,死学生二人,伤一百余人。},不是什么偶然的事件。这是李宗仁何应钦政府保护蒋介石、保护蒋介石死党、保护美国侵略势力的必然结果。这是李宗仁何应钦政府和蒋介石死党一同荒谬地鼓吹所谓“平等的光荣的和平”,借以抵抗中共八项和平条件\footnote[3]{ 见本卷《中共中央毛泽东主席关于时局的声明》。},特别是抵抗惩办战争罪犯的结果。李宗仁何应钦政府既然派出和谈代表团前来北平同中国共产党谈判和平,并表示愿意接受中国共产党的八项条件以为谈判的基础,那末,如果这个政府是有最低限度的诚意,就应当以处理南京惨案为起点,逮捕并严惩主凶蒋介石、汤恩伯、张耀明,逮捕并严惩在南京上海的特务暴徒,逮捕并严惩那些坚决反对和平、积极破坏和谈、积极准备抵抗人民解放军向长江以南推进的反革命首要。庆父不死,鲁难未已\footnote[4]{ 事见《左传·闵公元年》。庆父是春秋时鲁国的公子,曾经一再制造鲁国的内乱,先后杀死两个国君。当时的人有“不去庆父,鲁难未已”的说法。后人常常把制造内乱的人比之为庆父。}。战犯不除,国无宁日。这个真理,难道现在还不明白吗?\\
  我们愿意正告南京政府:如果你们没有能力办这件事,那末,你们也应协助即将渡江南进的人民解放军去办这件事。时至今日,一切空话不必说了,还是做件切实的工作,借以立功自赎为好。免得逃难,免得再受蒋介石死党的气,免得永远被人民所唾弃。只有这一次机会了,不要失掉这个机会。人民解放军就要向江南进军了。这不是拿空话吓你们,无论你们签订接受八项条件的协定也好,不签这个协定也好,人民解放军总是要前进的。签一个协定而后前进,对几方面都有利——对人民有利,对人民解放军有利,对国民党政府系统中一切愿意立功自赎的人们有利,对国民党军队的广大官兵有利,只对蒋介石,对蒋介石死党,对帝国主义者不利。不签这个协定,情况也差不多,可以用局部谈判的方法去解决。可能还有些战斗,但是不会有很多的战斗了。从新疆到台湾这样广大的地区内和漫长的战线上,国民党只有一百一十万左右的作战部队了,没有很多的仗可打了。无论签订一个全面性的协定也好,不签这个协定而签许多局部性的协定也好,对于蒋介石,对于蒋介石死党,对于美国帝国主义,一句话,对于一切至死不变的反动派,情况都是一样的,他们将决定地要灭亡。也许签订一个全面性协定对于南京方面和我们方面,都比较不签这个协定,来得稍微有利一些,所以我们还是争取签订这个协定。但是签订这个全面性协定,我们须得准备应付许多拖泥带水的事情。不签这个协定而去签订许多局部协定,对于我们要爽快得多。虽然如此,我们还是准备签订这个协定。南京政府及其代表团如果也愿意这样做,那末,就得在这几天下决心,一切幻想和一切空话都应当抛弃了。我们并不强迫你们下这个决心。南京政府及其代表团是否下这个决心,有你们自己的自由。就是说,你们或者听蒋介石和司徒雷登\footnote[5]{ 见本卷《别了,司徒雷登》注〔1〕。}的话,并和他们永远站在一起,或者听我们的话,和我们站在一起,对于这二者的选择,有你们自己的自由。但是选择的时间没有很多了,人民解放军就要进军了,一点游移的余地也没有了。\\
\newpage\section*{\myformat{向全国进军的命令}\\\myformat{(一九四九年四月二十一日)}}\addcontentsline{toc}{section}{向全国进军的命令}
\begin{introduction}
\item 这个命令是毛泽东起草的。在国民党反动政府拒绝签订国内和平协定以后,人民解放军遵照毛泽东主席和朱德总司令的这个命令,向尚未解放的广大地区,举行了规模空前的全面大进军。刘伯承、邓小平等领导的第二野战军和陈毅、粟裕、谭震林等领导的第三野战军,于一九四九年四月二十日夜起,至二十一日,在西起九江东北的湖口,东至江阴,长达五百余公里的战线上,强渡长江,彻底摧毁国民党军苦心经营了三个半月的长江防线。四月二十三日,解放了国民党二十二年来的反革命统治中心南京,宣告了国民党反动统治的覆灭。接着,又分路向南挺进,于五月三日解放杭州,五月二十二日解放南昌,五月二十七日攻占中国最大的城市上海。七月,开始进军福建。八月十七日解放福州,十月十七日解放厦门。林彪、罗荣桓等领导的第四野战军,于五月十四日,在武汉以东团风至武穴间一百余公里的地段上,强渡长江。五月十六、十七两日,解放华中的重镇武昌、汉阳和汉口。接着,又南下湖南。国民党湖南省主席程潜、第一兵团司令陈明仁,于八月四日宣布起义,湖南省和平解放。第四野战军在九、十月间进行衡(阳)宝(庆)战役,歼灭了国民党白崇禧军的主力以后,又向广东、广西进军。十月十四日解放广州,十一月二十二日解放桂林,十二月四日解放南宁。和第二、第三野战军进行渡江作战的同时,聂荣臻、徐向前等领导的华北各兵团,四月二十四日攻克太原。彭德怀、贺龙等领导的第一野战军,在五月二十日解放西安之后,同第十九、第二十兵团继续向西北国民党统治区进军,八月二十六日攻克兰州,九月五日解放西宁,九月二十三日解放银川,全部歼灭了国民党马步芳、马鸿逵军。九月下旬,国民党新疆省警备总司令陶峙岳、省主席鲍尔汉宣布起义,新疆省和平解放。刘伯承、邓小平等领导的第二野战军同贺龙、李井泉等领导的第十八兵团和第一野战军一部,于十一月初开始向西南进军。十一月十五日解放贵阳,十一月三十日解放重庆。十二月九日,国民党云南省主席卢汉,西康省主席刘文辉,西南军政长官公署副长官邓锡侯、潘文华等人宣布起义,云南、西康两省和平解放。十二月下旬,进入西南的人民解放军进行了成都战役,全部歼灭国民党胡宗南军,二十七日解放成都。到一九四九年十二月底,人民解放军已经全部歼灭了中国大陆上的国民党军队,解放了除西藏以外的全部中国大陆。
\end{introduction}\\~\\
各野战军全体指挥员战斗员同志们,南方各游击区人民解放军同志们:\\
  由中国共产党的代表团和南京国民党政府的代表团经过长时间的谈判所拟定的国内和平协定,已被南京国民党政府所拒绝\footnote[1]{ 一九四九年四月一日,以张治中为首的国民党政府和平谈判代表团到达北平,和中国共产党代表团进行谈判。经过半个月谈判,拟定了国内和平协定。四月十五日,中国共产党代表团将国内和平协定(最后修正案),提交南京政府代表团,四月二十日,被南京政府所拒绝。国内和平协定(最后修正案)全文如下:}。南京国民党政府的负责人员之所以拒绝这个国内和平协定,是因为他们仍然服从美国帝国主义和国民党匪首蒋介石的命令,企图阻止中国人民解放事业的推进,阻止用和平方法解决国内问题。经过双方代表团的谈判所拟定的国内和平协定八条二十四款,表示了对于战犯问题的宽大处理,对于国民党军队的官兵和国民党政府的工作人员的宽大处理,对于其它各项问题亦无不是从民族利益和人民利益出发作了适宜的解决。拒绝这个协定,就是表示国民党反动派决心将他们发动的反革命战争打到底。拒绝这个协定,就是表示国民党反动派在今年一月一日所提议的和平谈判,不过是企图阻止人民解放军向前推进,以便反动派获得喘息时间,然后卷土重来,扑灭革命势力。拒绝这个协定,就是表示南京李宗仁政府所谓承认中共八个和平条件\footnote[2]{ 见本卷《中共中央毛泽东主席关于时局的声明》。}以为谈判基础是完全虚伪的。因为,既然承认惩办战争罪犯,用民主原则改编一切国民党反动军队,接收南京政府及其所属各级政府的一切权力以及其它各项基础条件,就没有理由拒绝根据这些基础条件所拟定的而且是极为宽大的各项具体办法。在此种情况下,我们命令你们:\\
  (一)奋勇前进,坚决、彻底、干净、全部地歼灭中国境内一切敢于抵抗的国民党反动派,解放全国人民,保卫中国领土主权的独立和完整。\\
  (二)奋勇前进,逮捕一切怙恶不悛的战争罪犯。不管他们逃至何处,均须缉拿归案,依法惩办。特别注意缉拿匪首蒋介石。\\
  (三)向任何国民党地方政府和地方军事集团宣布国内和平协定的最后修正案。对于凡愿停止战争、用和平方法解决问题者,你们即可照此最后修正案的大意和他们签订地方性的协定。\\
  (四)在人民解放军包围南京之后,如果南京李宗仁政府尚未逃散,并愿意于国内和平协定上签字,我们愿意再一次给该政府以签字的机会。\\
  \begin{flushright}
  中国人民革命军事委员会主席 毛泽东\\
中国人民解放军总司令 朱 德
  \end{flushright}

\newpage\section*{\myformat{中国人民解放军布告}\\\myformat{(一九四九年四月二十五日)}}\addcontentsline{toc}{section}{中国人民解放军布告}
国民党反动派业已拒绝接受和平条件\footnote[1]{ 见本卷《向全国进军的命令》注〔1〕。},坚持其反民族反人民的罪恶的战争立场。全国人民希望人民解放军迅速消灭国民党反动派。我们已命令人民解放军奋勇前进,消灭一切敢于抵抗的国民党反动军队,逮捕一切怙恶不悛的战争罪犯,解放全国人民,保卫中国领土主权的独立和完整,实现全国人民所渴望的真正的统一。人民解放军所到之处,深望各界人民予以协助。兹特宣布约法八章,愿与我全体人民共同遵守之。\\
  (一)保护全体人民的生命财产。各界人民,不分阶级、信仰和职业,均望保持秩序,采取和人民解放军合作的态度。人民解放军则采取和各界人民合作的态度。如有反革命分子或其它破坏分子,乘机捣乱、抢劫或破坏者,定予严办。\\
  (二)保护民族工商农牧业。凡属私人经营的工厂、商店、银行、仓库、船舶、码头、农场、牧场等,一律保护,不受侵犯。希望各业员工照常生产,各行商店照常营业。\\
  (三)没收官僚资本。凡属国民党反动政府和大官僚分子所经营的工厂、商店、银行、仓库、船舶、码头、铁路、邮政、电报、电灯、电话、自来水和农场、牧场等,均由人民政府接管。其中,如有民族工商农牧业家私人股份经调查属实者,当承认其所有权。所有在官僚资本企业中供职的人员,在人民政府接管以前,均须照旧供职,并负责保护资财、机器、图表、账册、档案等,听候清点和接管。保护有功者奖,怠工破坏者罚。凡愿继续服务者,在人民政府接管后,准予量才录用,不使流离失所。\\
  (四)保护一切公私学校、医院、文化教育机关、体育场所,和其它一切公益事业。凡在这些机关供职的人员,均望照常供职,人民解放军一律保护,不受侵犯。\\
  (五)除怙恶不悛的战争罪犯和罪大恶极的反革命分子外,凡属国民党中央、省、市、县各级政府的大小官员,“国大”代表,立法、监察委员,参议员,警察人员,区镇乡保甲人员,凡不持枪抵抗、不阴谋破坏者,人民解放军和人民政府一律不加俘虏,不加逮捕,不加侮辱。责成上述人员各安职守,服从人民解放军和人民政府的命令,负责保护各机关资财、档案等,听候接收处理。这些人员中,凡有一技之长而无严重的反动行为或严重的劣迹者,人民政府准予分别录用。如有乘机破坏,偷盗,舞弊,携带公款、公物、档案潜逃,或拒不交代者,则须予以惩办。\\
  (六)为着确保城乡治安、安定社会秩序的目的,一切散兵游勇,均应向当地人民解放军或人民政府投诚报到。凡自动投诚报到,并将所有武器交出者,概不追究。其有抗不报到,或隐藏武器者,即予逮捕查究。窝藏不报者,须受相当的处分。\\
  (七)农村中的封建的土地所有权制度,是不合理的,应当废除。但是废除这种制度,必须是有准备和有步骤的。一般地说来,应当先行减租减息,后行分配土地,并且需要人民解放军到达和工作一个相当长的时期之后,方才谈得到认真地解决土地问题。农民群众应当组织起来,协助人民解放军进行各项初步的改革工作。同时,努力耕种,使现有的农业生产水平不致降低,然后逐步加以提高,借以改善农民生活,并供给城市人民以商品粮食。城市的土地房屋,不能和农村土地问题一样处理。\\
  (八)保护外国侨民生命财产的安全。希望一切外国侨民各安生业,保持秩序。一切外国侨民,必须遵守人民解放军和人民政府的法令,不得进行间谍活动,不得有反对中国民族独立事业和人民解放事业的行为,不得包庇中国战争罪犯、反革命分子及其它罪犯。否则,当受人民解放军和人民政府的法律制裁。\\
  人民解放军纪律严明,公买公卖,不许妄取民间一针一线。希望我全体人民,一律安居乐业,切勿轻信谣言,自相惊扰。切切此布。\\
  \begin{flushright}
  中国人民革命军事委员会主席 毛泽东\\
中国人民解放军总司令 朱 德
  \end{flushright}

\newpage\section*{\myformat{中国人民解放军总部发言人为英国军舰暴行\footnote[1]{ 一九四九年四月二十日至二十一日,当人民解放军渡江作战的时候,侵入中国内河长江的紫石英号等四艘英国军舰先后驶向人民解放军防区,妨碍渡江,中英双方发生了军事冲突。英舰开炮打死打伤人民解放军二百五十二人。紫石英号也被人民解放军击伤被迫停于镇江附近江中,其它三艘英舰逃走。英国当局曾由其远东舰队司令布朗特经过紫石英号舰长同人民解放军代表进行多次谈判,要求将紫石英号放行。当谈判还在进行之际,紫石英号军舰于七月三十日夜趁江陵解放号客轮经过镇江下驶,强行靠近该轮与之并行,借以逃跑。当人民解放军警告其停驶时,紫石英号军舰竟开炮射击,并撞沉木船多只,逃出长江。}发表的声明}\\\myformat{(一九四九年四月三十日)}}\addcontentsline{toc}{section}{中国人民解放军总部发言人为英国军舰暴行发表的声明}
\begin{introduction}
\item 这是毛泽东为中国人民解放军总部发言人起草的声明。在这个声明里,表明了中国人民不怕任何威胁、坚决反对帝国主义侵略的严正立场,并且表明了即将成立的新中国的对外政策。
\end{introduction}
我们斥责战争贩子丘吉尔的狂妄声明\footnote[2]{ 一九四九年四月二十六日,英国保守党首领丘吉尔在下院发言,把中国人民解放军炮击驶入人民解放军防区的英舰说成是“暴行”,并且要英国政府“派一两艘航空母舰到中国海上去……实行武力的报复”。}。四月二十六日,丘吉尔在英国下院,要求英国政府派两艘航空母舰去远东,“实行武力的报复”。丘吉尔先生,你“报复”什么?英国的军舰和国民党的军舰一道,闯入中国人民解放军的防区,并向人民解放军开炮,致使人民解放军的忠勇战士伤亡二百五十二人之多。英国人跑进中国境内做出这样大的犯罪行为,中国人民解放军有理由要求英国政府承认错误,并执行道歉和赔偿。难道你们今后应当做的不是这些,反而是开动军队到中国来向中国人民解放军进行“报复”吗?艾德礼首相的话也是错误的\footnote[3]{ 一九四九年四月二十六日,英国首相艾德礼在议会中宣称:“英国军舰有合法权利在长江行驶,执行和平使命,因为它们得到国民党政府的许可。”同时,他在谈到英国方面代表和中国人民解放军代表交涉的情况时,又造谣说:中国人民解放军“准备让英舰紫石英号开往南京,但要有一个条件,就是该舰要协助人民解放军渡江”。}。他说英国有权开动军舰进入中国的长江。长江是中国的内河,你们英国人有什么权利将军舰开进来?没有这种权利。中国的领土主权,中国人民必须保卫,绝对不允许外国政府来侵犯。艾德礼说:人民解放军“准备让英舰紫石英号开往南京,但要有一个条件,就是该舰要协助人民解放军渡江”。艾德礼是在撒谎,人民解放军并没有允许紫石英号开往南京。人民解放军不希望任何外国武装力量帮助渡江,或做任何别的什么事情。相反,人民解放军要求英国、美国、法国在长江黄浦江和在中国其它各处的军舰、军用飞机、陆战队等项武装力量,迅速撤离中国的领水、领海、领土、领空,不要帮助中国人民的敌人打内战。中国人民革命军事委员会和人民政府直到现在还没有同任何外国政府建立外交关系。中国人民革命军事委员会和人民政府愿意保护从事正常业务的在华外国侨民。中国人民革命军事委员会和人民政府愿意考虑同各外国建立外交关系,这种关系必须建立在平等、互利、互相尊重主权和领土完整的基础上,首先是不能帮助国民党反动派。中国人民革命军事委员会和人民政府不愿意接受任何外国政府所给予的任何带威胁性的行动。外国政府如果愿意考虑同我们建立外交关系,它就必须断绝同国民党残余力量的关系,并且把它在中国的武装力量撤回去。艾德礼埋怨中国共产党因为没有同外国建立外交关系而不愿意同外国政府的旧外交人员(国民党承认的领事)发生关系,这种埋怨是没有理由的。过去数年内,美国、英国、加拿大等国政府是帮助国民党反对我们的,难道艾德礼先生也忘记了?被击沉不久的重庆号重巡洋舰\footnote[4]{ 重庆号巡洋舰,是英国政府于一九四八年五月十九日赠给国民党反动政府的,是国民党政府海军中最大的巡洋舰。一九四九年二月二十五日,这艘军舰的官兵在上海吴淞口举行起义,脱离国民党政府,加入中国人民海军。同年三月十九日,美帝国主义和国民党政府出动重轰炸机多架,将其炸沉于中国东北辽东湾的葫芦岛附近。}是什么国家赠给国民党的,艾德礼先生难道也不知道吗?\\
\newpage\section*{\myformat{在新政治协商会议筹备会上的讲话}\\\myformat{(一九四九年六月十五日)}}\addcontentsline{toc}{section}{在新政治协商会议筹备会上的讲话}\\~\\
诸位代表先生:\\
  我们的新的政治协商会议的筹备会\footnote[1]{ 新政治协商会议筹备会,于一九四九年六月十五日至十九日在北平召开第一次全体会议。参加这次会议的,包括中国共产党和各民主党派、各人民团体、各界民主人士、国内少数民族、海外华侨等二十三个单位,一百三十四人。会议通过了《新政治协商会议筹备会组织条例》和《关于参加新政治协商会议的单位及其代表名额的规定》,选出了以毛泽东为主任的常务委员会。当时所以叫做新政治协商会议,是为了区别于一九四六年一月十日至三十一日在重庆举行的政治协商会议。以后,在一九四九年九月十七日新政协筹备会第二次全体会议上,决定将新政治协商会议改称中国人民政治协商会议。},今天开幕了。这个筹备会的任务,就是:完成各项必要的准备工作,迅速召开新的政治协商会议,成立民主联合政府,以便领导全国人民,以最快的速度肃清国民党反动派的残余力量,统一全中国,有系统地和有步骤地在全国范围内进行政治的、经济的、文化的和国防的建设工作。全国人民希望我们这样做,我们就应当这样做。\\
  新的政治协商会议,是中国共产党在一九四八年五月一日向全国人民提议召开的\footnote[2]{ 见本卷《中共中央关于九月会议的通知》注〔5〕。}。这个提议,迅速地得到了全国各民主党派、各人民团体、各界民主人士、国内少数民族和海外华侨的响应。中国共产党、各民主党派、各人民团体、各界民主人士、国内少数民族和海外华侨都认为:必须打倒帝国主义、封建主义、官僚资本主义和国民党反动派的统治,必须召集一个包含各民主党派、各人民团体、各界民主人士、国内少数民族和海外华侨的代表人物的政治协商会议,宣告中华人民共和国的成立,并选举代表这个共和国的民主联合政府,才能使我们的伟大的祖国脱离半殖民地的和半封建的命运,走上独立、自由、和平、统一和强盛的道路。这是一个共同的政治基础。这是中国共产党、各民主党派、各人民团体、各界民主人士、国内少数民族和海外华侨团结奋斗的共同的政治基础,这也是全国人民团结奋斗的共同的政治基础。这个政治基础是如此巩固,以至于没有一个认真的民主党派、人民团体和民主人士提出任何不同的意见,大家认为只有这一条道路,才是解决中国一切问题的正确的方向。\\
  全国人民拥护自己的人民解放军,取得了战争的胜利。这一次伟大的人民解放战争,从一九四六年七月开始,到现在,业已三年了。这一次战争是由国民党反动派在获得外国帝国主义的援助之下发动的。国民党反动派背信弃义,撕毁了一九四六年一月的停战协定\footnote[3]{ 见本卷《以自卫战争粉碎蒋介石的进攻》注〔1〕。}和政治协商会议的决议\footnote[4]{ 见本卷《以自卫战争粉碎蒋介石的进攻》注〔2〕。},发动了这一次反人民的国内战争。可是,仅仅三年时间,即已被英勇的人民解放军所打败。不久以前,在国民党反动派的和平阴谋被揭穿以后,人民解放军即已奋勇前进,横渡长江。国民党反动派的都城南京,已被夺取。上海、杭州、南昌、武汉、西安,已被解放。现在,人民解放军的各路野战军,正在向南方和西北各省,举行着自有中国历史以来未曾有过的大进军。三个年头中,人民解放军共已消灭反动的国民党军五百五十九万人。截至现时为止,残余的国民党军,包括它的正规部队、非正规部队和后方军事机关军事学校等在内,只有一百五十万人左右了。肃清这一部分残余敌军,还需要一些时间,但已为期不远了。\\
  这是全中国人民的胜利,也是全世界人民的胜利。整个世界,除了帝国主义者和各国反动派,对于中国人民的这个伟大的胜利,没有不欢欣鼓舞的。中国人民反对自己的敌人的斗争和世界人民反对自己的敌人的斗争,其意义是同一的。全中国人民和全世界人民一齐看见了这样的事实:帝国主义者指挥中国反动派用反革命战争残酷地反对中国人民,中国人民用革命战争胜利地打倒了反动派。\\
  在这里,我认为有必要唤起人们的注意,这即是:帝国主义者及其走狗中国反动派对于他们在中国这块土地上的失败,是不会甘心的。他们还会要互相勾结在一起,用各种可能的方法,反对中国人民。例如,派遣他们的走狗钻进中国内部来进行分化工作和捣乱工作。这是必然的,他们决不会忘记这一项工作。例如,唆使中国反动派,甚至加上他们自己的力量,封锁中国的海港。只要还有可能,他们就会这样做。再则,假如他们还想冒险的话,派出一部分兵力侵扰中国的边境,也不是不可能的。所有这些,我们都必须充分地估计到。我们决不可因为胜利,而放松对于帝国主义分子及其走狗们的疯狂的报复阴谋的警惕性,谁要是放松这一项警惕性,谁就将在政治上解除武装,而使自己处于被动的地位。在这种情况下,全国人民必须团结起来,坚决、彻底、干净、全部地粉碎帝国主义者及其走狗中国反动派的任何一项反对中国人民的阴谋计划。中国必须独立,中国必须解放,中国的事情必须由中国人民自己作主张,自己来处理,不容许任何帝国主义国家再有一丝一毫的干涉。\\
  中国的革命是全民族人民大众的革命,除了帝国主义者、封建主义者、官僚资产阶级分子、国民党反动派及其帮凶们而外,其余的一切人都是我们的朋友,我们有一个广大的和巩固的革命统一战线。这个统一战线是如此广大,它包含了工人阶级、农民阶级、城市小资产阶级和民族资产阶级。这个统一战线是如此巩固,它具备了战胜任何敌人和克服任何困难的坚强的意志和源源不竭的能力。我们现在所处的时代是帝国主义制度走向全部崩溃的时代,帝国主义者业已陷入不可解脱的危机之中,不论他们还要如何继续反对中国人民,中国人民总是有办法取得最后胜利的。\\
  同时,我们向全世界声明:我们所反对的只是帝国主义制度及其反对中国人民的阴谋计划。任何外国政府,只要它愿意断绝对于中国反动派的关系,不再勾结或援助中国反动派,并向人民的中国采取真正的而不是虚伪的友好态度,我们就愿意同它在平等、互利和互相尊重领土主权的原则的基础之上,谈判建立外交关系的问题。中国人民愿意同世界各国人民实行友好合作,恢复和发展国际间的通商事业,以利发展生产和繁荣经济。\\
  诸位代表先生:我们召集新的政治协商会议成立民主联合政府的一切条件,均已成熟。全中国人民是如此热烈地盼望我们召开会议和成立政府。我相信,我们现在开始的工作,是能够满足这个希望的,并且不需要多久的时间就能满足这个希望。\\
  中国民主联合政府一经成立,它的工作重点将是:(一)肃清反动派的残余,镇压反动派的捣乱;(二)尽一切可能用极大力量从事人民经济事业的恢复和发展,同时恢复和发展人民的文化教育事业。\\
  中国人民将会看见,中国的命运一经操在人民自己的手里,中国就将如太阳升起在东方那样,以自己的辉煌的光焰普照大地,迅速地荡涤反动政府留下来的污泥浊水,治好战争的创伤,建设起一个崭新的强盛的名副其实的人民共和国。\\
  中华人民共和国万岁!\\
  民主联合政府万岁!\\
  全国人民大团结万岁!\\
\newpage\section*{\myformat{论人民民主专政}\\\myformat{纪念中国共产党二十八周年}\\\myformat{(一九四九年六月三十日)}}\addcontentsline{toc}{section}{论人民民主专政}
一九四九年的七月一日这一个日子表示,中国共产党已经走过二十八年了。像一个人一样,有他的幼年、青年、壮年和老年。中国共产党已经不是小孩子,也不是十几岁的年青小伙子,而是一个大人了。人到老年就要死亡,党也是这样。阶级消灭了,作为阶级斗争的工具的一切东西,政党和国家机器,将因其丧失作用,没有需要,逐步地衰亡下去,完结自己的历史使命,而走到更高级的人类社会。我们和资产阶级政党相反。他们怕说阶级的消灭,国家权力的消灭和党的消灭。我们则公开声明,恰是为着促使这些东西的消灭而创设条件,而努力奋斗。共产党的领导和人民专政的国家权力,就是这样的条件。不承认这一条真理,就不是共产主义者。没有读过马克思列宁主义的刚才进党的青年同志们,也许还不懂得这一条真理。他们必须懂得这一条真理,才有正确的宇宙观。他们必须懂得,消灭阶级,消灭国家权力,消灭党,全人类都要走这一条路的,问题只是时间和条件。全世界共产主义者比资产阶级高明,他们懂得事物的生存和发展的规律,他们懂得辩证法,他们看得远些。资产阶级所以不欢迎这一条真理,是因为他们不愿意被人们推翻。被推翻,例如眼前国民党反动派被我们所推翻,过去日本帝国主义被我们和各国人民所推翻,对于被推翻者来说,这是痛苦的,不堪设想的。对于工人阶级、劳动人民和共产党,则不是什么被推翻的问题,而是努力工作,创设条件,使阶级、国家权力和政党很自然地归于消灭,使人类进到大同境域。为着说清我们在下面所要说的问题,在这里顺便提一下这个人类进步的远景的问题。\\
  我们党走过二十八年了,大家知道,不是和平地走过的,而是在困难的环境中走过的,我们要和国内外党内外的敌人作战。谢谢马克思、恩格斯、列宁和斯大林,他们给了我们以武器。这武器不是机关枪,而是马克思列宁主义。\\
  列宁在一九二〇年在《共产主义运动中的“左派”幼稚病》一书中,描写过俄国人寻找革命理论的经过\footnote[1]{ 见《共产主义运动中的“左派”幼稚病》第二章。列宁在那里说:“在将近半个世纪里,大约从上一世纪四十年代至九十年代,俄国进步的思想界在空前野蛮和反动的沙皇制度的压迫之下,曾如饥似渴地寻求正确的革命理论,专心致志地、密切地注视着欧美在这方面的每一种‘最新成就’。俄国在半个世纪里,经受了闻所未闻的痛苦和牺牲,表现了空前未有的革命英雄气概,以难以置信的毅力和舍身忘我的精神去探索、学习和实验,经受了失望,进行了验证,参照了欧洲的经验,真是饱经苦难找到了马克思主义这个唯一正确的革命理论。”(《列宁全集》第39卷,人民出版社1986年版,第5—6页)}。俄国人曾经在几十个年头内,经历艰难困苦,方才找到了马克思主义。中国有许多事情和十月革命以前的俄国相同,或者近似。封建主义的压迫,这是相同的。经济和文化落后,这是近似的。两个国家都落后,中国则更落后。先进的人们,为了使国家复兴,不惜艰苦奋斗,寻找革命真理,这是相同的。\\
  自从一八四〇年鸦片战争\footnote[2]{ 见本书第一卷《论反对日本帝国主义的策略》注〔35〕。}失败那时起,先进的中国人,经过千辛万苦,向西方国家寻找真理。洪秀全\footnote[3]{ 洪秀全(一八一四——一八六四),广东花县人,十九世纪中叶太平天国农民革命运动的领袖。他早年深感清朝腐败,外侮日深,而吸取西方基督教教义的平等思想,开始宣传“拜上帝教”。一八五一年,他和杨秀清等领导群众在广西桂平县的金田村起义,建号太平天国。一八五九年,洪仁写的主张向西方国家学习的《资政新篇》,经洪秀全批准颁行。}、康有为\footnote[4]{ 康有为(一八五八——一九二七),广东南海人。一八九五年中国在甲午战争中被日本打败后,他联合一千三百多名在北京参加科举考试的举人联名向光绪皇帝上“万言书”,要求“变法维新”,主张改君主专制制度为君主立宪制度。一八九八年光绪皇帝任用他和谭嗣同、梁启超等人参预政事,试图变法。后来顽固派的代表慈禧太后重握政权,维新运动遂告失败。康、梁逃亡海外,组织保皇会,同孙中山所代表的资产阶级、小资产阶级的革命派相对立,成为反动的政治派别。康有为的著作有《新学伪经考》、《孔子改制考》、《大同书》等。}、严复\footnote[5]{ 严复(一八五四——一九二一),福建闽侯人。曾在英国海军学校留学。中日甲午战争后,主张君主立宪,变法维新。曾译赫胥黎《天演论》、亚当·斯密《原富》、穆勒《名学》和孟德斯鸠《法意》等书,传播了欧洲资产阶级的思想。}和孙中山,代表了在中国共产党出世以前向西方寻找真理的一派人物。那时,求进步的中国人,只要是西方的新道理,什么书也看。向日本、英国、美国、法国、德国派遣留学生之多,达到了惊人的程度。国内废科举,兴学校\footnote[6]{ 参见本书第二卷《新民主主义论》注〔27〕。},好像雨后春笋,努力学习西方。我自己在青年时期,学的也是这些东西。这些是西方资产阶级民主主义的文化,即所谓新学,包括那时的社会学说和自然科学,和中国封建主义的文化即所谓旧学是对立的。学了这些新学的人们,在很长的时期内产生了一种信心,认为这些很可以救中国,除了旧学派,新学派自己表示怀疑的很少。要救国,只有维新,要维新,只有学外国。那时的外国只有西方资本主义国家是进步的,它们成功地建设了资产阶级的现代国家。日本人向西方学习有成效,中国人也想向日本人学。在那时的中国人看来,俄国是落后的,很少人想学俄国。这就是十九世纪四十年代至二十世纪初期中国人学习外国的情形。\\
  帝国主义的侵略打破了中国人学西方的迷梦。很奇怪,为什么先生老是侵略学生呢?中国人向西方学得很不少,但是行不通,理想总是不能实现。多次奋斗,包括辛亥革命\footnote[7]{ 见本书第一卷《湖南农民运动考察报告》注〔3〕。}那样全国规模的运动,都失败了。国家的情况一天一天坏,环境迫使人们活不下去。怀疑产生了,增长了,发展了。第一次世界大战震动了全世界。俄国人举行了十月革命,创立了世界上第一个社会主义国家。过去蕴藏在地下为外国人所看不见的伟大的俄国无产阶级和劳动人民的革命精力,在列宁、斯大林领导之下,像火山一样突然爆发出来了,中国人和全人类对俄国人都另眼相看了。这时,也只是在这时,中国人从思想到生活,才出现了一个崭新的时期。中国人找到了马克思列宁主义这个放之四海而皆准的普遍真理,中国的面目就起了变化了。\\
  中国人找到马克思主义,是经过俄国人介绍的。在十月革命以前,中国人不但不知道列宁、斯大林,也不知道马克思、恩格斯。十月革命一声炮响,给我们送来了马克思列宁主义。十月革命帮助了全世界的也帮助了中国的先进分子,用无产阶级的宇宙观作为观察国家命运的工具,重新考虑自己的问题。走俄国人的路——这就是结论。一九一九年,中国发生了五四运动\footnote[8]{ 见本书第一卷《实践论》注〔6〕。}。一九二一年,中国共产党成立。孙中山在绝望里,遇到了十月革命和中国共产党。孙中山欢迎十月革命,欢迎俄国人对中国人的帮助,欢迎中国共产党同他合作。孙中山死了,蒋介石起来。在二十二年的长时间内,蒋介石把中国拖到了绝境。在这个时期中,以苏联为主力军的反法西斯的第二次世界大战,打倒了三个帝国主义大国,两个帝国主义大国在战争中被削弱了,世界上只剩下一个帝国主义大国即美国没有受损失。而美国的国内危机是很深重的。它要奴役全世界,它用武器帮助蒋介石杀戮了几百万中国人。中国人民在中国共产党领导之下,在驱逐日本帝国主义之后,进行了三年的人民解放战争,取得了基本的胜利。\\
  就是这样,西方资产阶级的文明,资产阶级的民主主义,资产阶级共和国的方案,在中国人民的心目中,一齐破了产。资产阶级的民主主义让位给工人阶级领导的人民民主主义,资产阶级共和国让位给人民共和国。这样就造成了一种可能性:经过人民共和国到达社会主义和共产主义,到达阶级的消灭和世界的大同。康有为写了《大同书》,他没有也不可能找到一条到达大同的路。资产阶级的共和国,外国有过的,中国不能有,因为中国是受帝国主义压迫的国家。唯一的路是经过工人阶级领导的人民共和国。\\
  一切别的东西都试过了,都失败了。曾经留恋过别的东西的人们,有些人倒下去了,有些人觉悟过来了,有些人正在换脑筋。事变是发展得这样快,以至使很多人感到突然,感到要重新学习。人们的这种心情是可以理解的,我们欢迎这种善良的要求重新学习的态度。\\
  中国无产阶级的先锋队,在十月革命以后学了马克思列宁主义,建立了中国共产党。接着就进入政治斗争,经过曲折的道路,走了二十八年,方才取得了基本的胜利。积二十八年的经验,如同孙中山在其临终遗嘱里所说“积四十年之经验”一样,得到了一个相同的结论,即是:深知欲达到胜利,“必须唤起民众,及联合世界上以平等待我之民族,共同奋斗”。孙中山和我们具有各不相同的宇宙观,从不同的阶级立场出发去观察和处理问题,但在二十世纪二十年代,在怎样和帝国主义作斗争的问题上,却和我们达到了这样一个基本上一致的结论。\\
  孙中山死去二十四年了,中国革命的理论和实践,在中国共产党领导之下,都大大地向前发展了,根本上变换了中国的面目。到现在为止,中国人民已经取得的主要的和基本的经验,就是这两件事:(一)在国内,唤起民众。这就是团结工人阶级、农民阶级、城市小资产阶级和民族资产阶级,在工人阶级领导之下,结成国内的统一战线,并由此发展到建立工人阶级领导的以工农联盟为基础的人民民主专政的国家;(二)在国外,联合世界上以平等待我的民族和各国人民,共同奋斗。这就是联合苏联,联合各人民民主国家,联合其它各国的无产阶级和广大人民,结成国际的统一战线。\\
  “你们一边倒。”正是这样。一边倒,是孙中山的四十年经验和共产党的二十八年经验教给我们的,深知欲达到胜利和巩固胜利,必须一边倒。积四十年和二十八年的经验,中国人不是倒向帝国主义一边,就是倒向社会主义一边,绝无例外。骑墙是不行的,第三条道路是没有的。我们反对倒向帝国主义一边的蒋介石反动派,我们也反对第三条道路\footnote[9]{ 见本卷《目前形势和我们的任务》注〔12〕。}的幻想。\\
  “你们太刺激了。”我们讲的是对付国内外反动派即帝国主义者及其走狗们,不是讲对付任何别的人。对于这些人,并不发生刺激与否的问题,刺激也是那样,不刺激也是那样,因为他们是反动派。划清反动派和革命派的界限,揭露反动派的阴谋诡计,引起革命派内部的警觉和注意,长自己的志气,灭敌人的威风,才能孤立反动派,战而胜之,或取而代之。在野兽面前,不可以表示丝毫的怯懦。我们要学景阳冈上的武松\footnote[10]{ 武松是中国著名小说《水浒传》中的一个英雄,他在景阳冈打虎的故事,在民间流传极广。}。在武松看来,景阳冈上的老虎,刺激它也是那样,不刺激它也是那样,总之是要吃人的。或者把老虎打死,或者被老虎吃掉,二者必居其一。\\
  “我们要做生意。”完全正确,生意总是要做的。我们只反对妨碍我们做生意的内外反动派,此外并不反对任何人。大家须知,妨碍我们和外国做生意以至妨碍我们和外国建立外交关系的,不是别人,正是帝国主义者及其走狗蒋介石反动派。团结国内国际的一切力量击破内外反动派,我们就有生意可做了,我们就有可能在平等、互利和互相尊重领土主权的基础之上和一切国家建立外交关系了。\\
  “不要国际援助也可以胜利。”这是错误的想法。在帝国主义存在的时代,任何国家的真正的人民革命,如果没有国际革命力量在各种不同方式上的援助,要取得自己的胜利是不可能的。胜利了,要巩固,也是不可能的。伟大的十月革命的胜利和巩固,就是这样的,列宁和斯大林早已告诉我们了。第二次世界大战打倒三个帝国主义国家并建立各人民民主国家,也是这样。人民中国的现在和将来,也是这样。请大家想一想,假如没有苏联的存在,假如没有反法西斯的第二次世界大战的胜利,假如没有打倒日本帝国主义,假如没有各人民民主国家的出现,假如没有东方各被压迫民族正在起来斗争,假如没有美国、英国、法国、德国、意大利、日本等等资本主义国家内部的人民大众和统治他们的反动派之间的斗争,假如没有这一切的综合,那末,堆在我们头上的国际反动势力必定比现在不知要大多少倍。在这种情形下,我们能够胜利吗?显然是不能的。胜利了,要巩固,也不可能。这件事,中国人民的经验是太多了。孙中山临终时讲的那句必须联合国际革命力量的话,早已反映了这一种经验。\\
  “我们需要英美政府的援助。”在现时,这也是幼稚的想法。现时英美的统治者还是帝国主义者,他们会给人民国家以援助吗?我们同这些国家做生意以及假设这些国家在将来愿意在互利的条件之下借钱给我们,这是因为什么呢?这是因为这些国家的资本家要赚钱,银行家要赚利息,借以解救他们自己的危机,并不是什么对中国人民的援助。这些国家的共产党和进步党派,正在促使它们的政府和我们做生意以至建立外交关系,这是善意的,这就是援助,这和这些国家的资产阶级的行为,不能相提并论。孙中山的一生中,曾经无数次地向资本主义国家呼吁过援助,结果一切落空,反而遭到了无情的打击。在孙中山一生中,只得过一次国际的援助,这就是苏联的援助。请读者们看一看孙先生的遗嘱吧,他在那里谆谆嘱咐人们的,不是叫人们把眼光向着帝国主义国家的援助,而是叫人们“联合世界上以平等待我之民族”。孙先生有了经验了,他吃过亏,上过当。我们要记得他的话,不要再上当。我们在国际上是属于以苏联为首的反帝国主义战线一方面的,真正的友谊的援助只能向这一方面去找,而不能向帝国主义战线一方面去找。\\
  “你们独裁。”可爱的先生们,你们讲对了,我们正是这样。中国人民在几十年中积累起来的一切经验,都叫我们实行人民民主专政,或曰人民民主独裁,总之是一样,就是剥夺反动派的发言权,只让人民有发言权。\\
  人民是什么?在中国,在现阶段,是工人阶级,农民阶级,城市小资产阶级和民族资产阶级。这些阶级在工人阶级和共产党的领导之下,团结起来,组成自己的国家,选举自己的政府,向着帝国主义的走狗即地主阶级和官僚资产阶级以及代表这些阶级的国民党反动派及其帮凶们实行专政,实行独裁,压迫这些人,只许他们规规矩矩,不许他们乱说乱动。如要乱说乱动,立即取缔,予以制裁。对于人民内部,则实行民主制度,人民有言论集会结社等项的自由权。选举权,只给人民,不给反动派。这两方面,对人民内部的民主方面和对反动派的专政方面,互相结合起来,就是人民民主专政。\\
  为什么理由要这样做?大家很清楚。不这样,革命就要失败,人民就要遭殃,国家就要灭亡。\\
  “你们不是要消灭国家权力吗?”我们要,但是我们现在还不要,我们现在还不能要。为什么?帝国主义还存在,国内反动派还存在,国内阶级还存在。我们现在的任务是要强化人民的国家机器,这主要地是指人民的军队、人民的警察和人民的法庭,借以巩固国防和保护人民利益。以此作为条件,使中国有可能在工人阶级和共产党的领导之下稳步地由农业国进到工业国,由新民主主义社会进到社会主义社会和共产主义社会,消灭阶级和实现大同。军队、警察、法庭等项国家机器,是阶级压迫阶级的工具。对于敌对的阶级,它是压迫的工具,它是暴力,并不是什么“仁慈”的东西。“你们不仁。”正是这样。我们对于反动派和反动阶级的反动行为,决不施仁政。我们仅仅施仁政于人民内部,而不施于人民外部的反动派和反动阶级的反动行为。\\
  人民的国家是保护人民的。有了人民的国家,人民才有可能在全国范围内和全体规模上,用民主的方法,教育自己和改造自己,使自己脱离内外反动派的影响(这个影响现在还是很大的,并将在长时期内存在着,不能很快地消灭),改造自己从旧社会得来的坏习惯和坏思想,不使自己走入反动派指引的错误路上去,并继续前进,向着社会主义社会和共产主义社会前进。\\
  我们在这方面使用的方法,是民主的即说服的方法,而不是强迫的方法。人民犯了法,也要受处罚,也要坐班房,也有死刑,但这是若干个别的情形,和对于反动阶级当作一个阶级的专政来说,有原则的区别。\\
  对于反动阶级和反动派的人们,在他们的政权被推翻以后,只要他们不造反,不破坏,不捣乱,也给土地,给工作,让他们活下去,让他们在劳动中改造自己,成为新人。他们如果不愿意劳动,人民的国家就要强迫他们劳动。也对他们做宣传教育工作,并且做得很用心,很充分,像我们对俘虏军官们已经做过的那样。这也可以说是“施仁政”吧,但这是我们对于原来是敌对阶级的人们所强迫地施行的,和我们对于革命人民内部的自我教育工作,不能相提并论。\\
  这种对于反动阶级的改造工作,只有共产党领导的人民民主专政的国家才能做到。这件工作做好了,中国的主要的剥削阶级——地主阶级和官僚资产阶级即垄断资产阶级,就最后地消灭了。剩下一个民族资产阶级,在现阶段就可以向他们中间的许多人进行许多适当的教育工作。等到将来实行社会主义即实行私营企业国有化的时候,再进一步对他们进行教育和改造的工作。人民手里有强大的国家机器,不怕民族资产阶级造反。\\
  严重的问题是教育农民。农民的经济是分散的,根据苏联的经验,需要很长的时间和细心的工作,才能做到农业社会化。没有农业社会化,就没有全部的巩固的社会主义。农业社会化的步骤,必须和以国有企业为主体的强大的工业的发展相适应。人民民主专政的国家,必须有步骤地解决国家工业化的问题。本文不打算多谈经济问题,这里不来详说。\\
  一九二四年,孙中山亲自领导的有共产党人参加的国民党第一次全国代表大会,通过了一个著名的宣言。这个宣言上说:“近世各国所谓民权制度,往往为资产阶级所专有,适成为压迫平民之工具。若国民党之民权主义,则为一般平民所共有,非少数人所得而私也。”除了谁领导谁这一个问题以外,当作一般的政治纲领来说,这里所说的民权主义,是和我们所说的人民民主主义或新民主主义相符合的。只许为一般平民所共有、不许为资产阶级所私有的国家制度,如果加上工人阶级的领导,就是人民民主专政的国家制度了。\\
  蒋介石背叛孙中山,拿了官僚资产阶级和地主阶级的专政作为压迫中国平民的工具。这个反革命专政,实行了二十二年,到现在才为我们领导的中国平民所推翻。\\
  骂我们实行“独裁”或“极权主义”的外国反动派,就是实行独裁或极权主义的人们。他们实行了资产阶级对无产阶级和其它人民的一个阶级的独裁制度,一个阶级的极权主义。孙中山所说压迫平民的近世各国的资产阶级,正是指的这些人。蒋介石的反革命独裁,就是从这些反动家伙学来的。\\
  宋朝的哲学家朱熹,写了许多书,说了许多话,大家都忘记了,但有一句话还没有忘记:“即以其人之道,还治其人之身。”\footnote[11]{ 这是朱熹在《中庸》第十三章注文中所说的话。}我们就是这样做的,即以帝国主义及其走狗蒋介石反动派之道,还治帝国主义及其走狗蒋介石反动派之身。如此而已,岂有他哉!\\
  革命的专政和反革命的专政,性质是相反的,而前者是从后者学来的。这个学习很要紧。革命的人民如果不学会这一项对待反革命阶级的统治方法,他们就不能维持政权,他们的政权就会被内外反动派所推翻,内外反动派就会在中国复辟,革命的人民就会遭殃。\\
  人民民主专政的基础是工人阶级、农民阶级和城市小资产阶级的联盟,而主要是工人和农民的联盟,因为这两个阶级占了中国人口的百分之八十到九十。推翻帝国主义和国民党反动派,主要是这两个阶级的力量。由新民主主义到社会主义,主要依靠这两个阶级的联盟。\\
  人民民主专政需要工人阶级的领导。因为只有工人阶级最有远见,大公无私,最富于革命的彻底性。整个革命历史证明,没有工人阶级的领导,革命就要失败,有了工人阶级的领导,革命就胜利了。在帝国主义时代,任何国家的任何别的阶级,都不能领导任何真正的革命达到胜利。中国的小资产阶级和民族资产阶级曾经多次领导过革命,都失败了,就是明证。\\
  民族资产阶级在现阶段上,有其很大的重要性。我们还有帝国主义站在旁边,这个敌人是很凶恶的。中国的现代工业在整个国民经济上的比重还很小。现在没有可靠的数目字,根据某些材料来估计,在抗日战争以前,现代工业产值不过只占全国国民经济总产值的百分之十左右。为了对付帝国主义的压迫,为了使落后的经济地位提高一步,中国必须利用一切于国计民生有利而不是有害的城乡资本主义因素,团结民族资产阶级,共同奋斗。我们现在的方针是节制资本主义,而不是消灭资本主义。但是民族资产阶级不能充当革命的领导者,也不应当在国家政权中占主要的地位。民族资产阶级之所以不能充当革命的领导者和所以不应当在国家政权中占主要地位,是因为民族资产阶级的社会经济地位规定了他们的软弱性,他们缺乏远见,缺乏足够的勇气,并且有不少人害怕民众。\\
  孙中山主张“唤起民众”,或“扶助农工”。谁去“唤起”和“扶助”呢?孙中山的意思是说小资产阶级和民族资产阶级。但这在事实上是办不到的。孙中山的四十年革命是失败了,这是什么原因呢?在帝国主义时代,小资产阶级和民族资产阶级不可能领导任何真正的革命到胜利,原因就在此。\\
  我们的二十八年,就大不相同。我们有许多宝贵的经验。一个有纪律的,有马克思列宁主义的理论武装的,采取自我批评方法的,联系人民群众的党。一个由这样的党领导的军队。一个由这样的党领导的各革命阶级各革命派别的统一战线。这三件是我们战胜敌人的主要武器。这些都是我们区别于前人的。依靠这三件,使我们取得了基本的胜利。我们走过了曲折的道路。我们曾和党内的机会主义倾向作斗争,右的和“左”的。凡在这三件事上犯了严重错误的时候,革命就受挫折。错误和挫折教训了我们,使我们比较地聪明起来了,我们的事情就办得好一些。任何政党,任何个人,错误总是难免的,我们要求犯得少一点。犯了错误则要求改正,改正得越迅速,越彻底,越好。\\
  总结我们的经验,集中到一点,就是工人阶级(经过共产党)领导的以工农联盟为基础的人民民主专政。这个专政必须和国际革命力量团结一致。这就是我们的公式,这就是我们的主要经验,这就是我们的主要纲领。\\
  党的二十八年是一个长时期,我们仅仅做了一件事,这就是取得了革命战争的基本胜利。这是值得庆祝的,因为这是人民的胜利,因为这是在中国这样一个大国的胜利。但是我们的事情还很多,比如走路,过去的工作只不过是像万里长征走完了第一步。残余的敌人尚待我们扫灭。严重的经济建设任务摆在我们面前。我们熟习的东西有些快要闲起来了,我们不熟习的东西正在强迫我们去做。这就是困难。帝国主义者算定我们办不好经济,他们站在一旁看,等待我们的失败。\\
  我们必须克服困难,我们必须学会自己不懂的东西。我们必须向一切内行的人们(不管什么人)学经济工作。拜他们做老师,恭恭敬敬地学,老老实实地学。不懂就是不懂,不要装懂。不要摆官僚架子。钻进去,几个月,一年两年,三年五年,总可以学会的。苏联共产党人开头也有一些人不大会办经济,帝国主义者也曾等待过他们的失败。但是苏联共产党是胜利了,在列宁和斯大林领导之下,他们不但会革命,也会建设。他们已经建设起来了一个伟大的光辉灿烂的社会主义国家。苏联共产党就是我们的最好的先生,我们必须向他们学习。国际和国内的形势都对我们有利,我们完全可以依靠人民民主专政这个武器,团结全国除了反动派以外的一切人,稳步地走到目的地。\\
\newpage\section*{\myformat{丢掉幻想,准备斗争}\\\myformat{(一九四九年八月十四日)}}\addcontentsline{toc}{section}{丢掉幻想,准备斗争}
\begin{introduction}
\item 本文和下面的《别了,司徒雷登》、《为什么要讨论白皮书?》、《“友谊”,还是侵略?》、《唯心历史观的破产》四篇文章,都是毛泽东为新华社写的对于美国国务院白皮书和艾奇逊信件的评论。这些评论揭露了美国对华政策的帝国主义本质,批评了国内一部分资产阶级知识分子对于美国帝国主义的幻想,并且对中国革命的发生和胜利的原因作了理论上的说明。
\end{introduction}
美国国务院关于中美关系的白皮书以及艾奇逊国务卿给杜鲁门总统的信\footnote[1]{ 这里所说的美国的白皮书,是指美国国务院在一九四九年八月五日发表的题为《美国与中国的关系》的白皮书。艾奇逊致杜鲁门的信,是指艾奇逊在美国国务院编好白皮书之后,于一九四九年七月三十日写给杜鲁门的一封信。白皮书的正文分为八章,叙述从一八四四年美国强迫中国签订《望厦条约》以来,直至一九四九年中国人民革命在全国范围内取得基本胜利时为止的中美关系。白皮书特别详细地叙述了抗日战争末期至一九四九年的五年中间,美国实施扶蒋反共政策,千方百计地反对中国人民,结果遭到失败的经过。},在现在这个时候发表,不是偶然的。这些文件的发表,反映了中国人民的胜利和帝国主义的失败,反映了整个帝国主义世界制度的衰落。帝国主义制度内部的矛盾重重,无法克服,使帝国主义者陷入了极大的苦闷中。\\
  帝国主义给自己准备了灭亡的条件。殖民地半殖民地的人民大众和帝国主义自己国家内的人民大众的觉悟,就是这样的条件。帝国主义驱使全世界的人民大众走上消灭帝国主义的伟大斗争的历史时代。\\
  帝国主义替这些人民大众准备了物质条件,也准备了精神条件。\\
  工厂、铁道、枪炮等等,这些是物质条件。中国人民解放军的强大的物质装备,大部分是从美国帝国主义得来的,一部分是从日本帝国主义得来的,一部分是自己制造的。\\
  自从一八四〇年英国人侵略中国\footnote[2]{ 见本书第一卷《论反对日本帝国主义的策略》注〔35〕。}以来,接着就是英法联军进攻中国的战争\footnote[3]{ 见本书第二卷《中国革命和中国共产党》注〔18〕。},法国进攻中国的战争\footnote[4]{ 见本书第二卷《中国革命和中国共产党》注〔19〕。},日本进攻中国的战争\footnote[5]{ 见本书第一卷《矛盾论》注〔22〕。},英国、法国、日本、沙皇俄国、德国、美国、意大利、奥地利等八国联军进攻中国的战争\footnote[6]{ 见本书第二卷《中国革命和中国共产党》注〔21〕。},日本和沙皇俄国在中国领土内进行的战争\footnote[7]{ 指一九〇四年至一九〇五年日本同沙皇俄国之间为争夺在中国东北和朝鲜的权益而进行的一次帝国主义战争。战场主要在中国东北境内的奉天(今沈阳市)、辽阳地区和旅顺口一带,使中国人民遭受巨大的损失。沙皇俄国在战争中遭到失败,经美国调停,同日本订立《朴次茅斯和约》。日俄战争后,日本取代了沙皇俄国在中国东三省南部的支配地位;日本对于朝鲜的独占地位,也在《朴次茅斯和约》中得到沙皇俄国的承认。},一九三一年开始的日本进攻中国东北的战争\footnote[8]{ 见本书第一卷《论反对日本帝国主义的策略》注〔4〕。},一九三七年开始继续了八年之久的日本进攻中国全境的战争,最后是最近三年来表面上是蒋介石实际上是美国进攻中国人民的战争。这最后一次战争,艾奇逊的信上说,美国对国民党政府的物质帮助占国民党政府的“货币支出的百分之五十以上”,“美国供给了中国军队(指国民党军队)的军需品”。这就是美国出钱出枪蒋介石出人替美国打仗杀中国人的战争。所有这一切侵略战争,加上政治上、经济上、文化上的侵略和压迫,造成了中国人对于帝国主义的仇恨,使中国人想一想,这究竟是怎么一回事,迫使中国人的革命精神发扬起来,从斗争中团结起来。斗争,失败,再斗争,再失败,再斗争,积一百零九年的经验,积几百次大小斗争的经验,军事的和政治的、经济的和文化的、流血的和不流血的经验,方才获得今天这样的基本上的成功。这就是精神条件,没有这个精神条件,革命是不能胜利的。\\
  为了侵略的必要,帝国主义给中国造成了买办制度,造成了官僚资本。帝国主义的侵略刺激了中国的社会经济,使它发生了变化,造成了帝国主义的对立物——造成了中国的民族工业,造成了中国的民族资产阶级,而特别是造成了在帝国主义直接经营的企业中、在官僚资本的企业中、在民族资产阶级的企业中做工的中国的无产阶级。为了侵略的必要,帝国主义以不等价交换的方法剥削中国的农民,使农民破产,给中国造成了数以万万计的广大的贫农群众,贫农占了农村人口的百分之七十。为了侵略的必要,帝国主义给中国造成了数百万区别于旧式文人或士大夫的新式的大小知识分子。对于这些人,帝国主义及其走狗中国的反动政府只能控制其中的一部分人,到了后来,只能控制其中的极少数人,例如胡适、傅斯年、钱穆之类,其它都不能控制了,他们走到了它的反面。学生、教员、教授、技师、工程师、医生、科学家、文学家、艺术家、公务人员,都造反了,或者不愿意再跟国民党走了。共产党是一个穷党,又是被国民党广泛地无孔不入地宣传为杀人放火,奸淫抢掠,不要历史,不要文化,不要祖国,不孝父母,不敬师长,不讲道理,共产公妻,人海战术,总之是一群青面獠牙,十恶不赦的人。可是,事情是这样地奇怪,就是这样的一群,获得了数万万人民群众的拥护,其中,也获得了大多数知识分子尤其是青年学生们的拥护。\\
  有一部分知识分子还要看一看。他们想,国民党是不好的,共产党也不见得好,看一看再说。其中有些人口头上说拥护,骨子里是看。正是这些人,他们对美国存着幻想。他们不愿意将当权的美国帝国主义分子和不当权的美国人民加以区别。他们容易被美国帝国主义分子的某些甜言蜜语所欺骗,似乎不经过严重的长期的斗争,这些帝国主义分子也会和人民的中国讲平等,讲互利。他们的头脑中还残留着许多反动的即反人民的思想,但他们不是国民党反动派,他们是人民中国的中间派,或右派。他们就是艾奇逊所说的“民主个人主义”的拥护者。艾奇逊们的欺骗做法在中国还有一层薄薄的社会基础。\\
  艾奇逊的白皮书表示,美国帝国主义者对于中国的目前这个局面是毫无办法了。国民党是那样的不行,无论帮它多少总是命定地完蛋了,他们不能控制了,他们无可奈何了。艾奇逊在他的信中说:“中国内战不祥的结局超出美国政府控制的能力,这是不幸的事,却也是无可避免的。在我国能力所及的合理的范围之内,我们所做的以及可能做的一切事情,都无法改变这种结局;这种结局之所以终于发生,也并不是因为我们少做了某些事情。这是中国内部各种力量的产物,我国曾经设法去左右这些力量,但是没有效果。”\\
  按照逻辑,艾奇逊的结论应该是,照着中国某些思想糊涂的知识分子的想法或说法,“放下屠刀,立地成佛”,“强盗收心做好人”,给人民的中国以平等和互利的待遇,再也不要做捣乱工作了。但是不,艾奇逊说,还是要捣乱的,并且确定地要捣乱。效果呢?据说是会有的。依靠一批什么人物呢?就是“民主个人主义”的拥护者。艾奇逊说:“……中国悠久的文明和她的民主个人主义终于会再显身手,中国终于会摆脱外国的羁绊。对于中国目前和将来一切朝着这个目标的发展,我认为都应当得到我们的鼓励。”\\
  帝国主义者的逻辑和人民的逻辑是这样的不同。捣乱,失败,再捣乱,再失败,直至灭亡——这就是帝国主义和世界上一切反动派对待人民事业的逻辑,他们决不会违背这个逻辑的。这是一条马克思主义的定律。我们说“帝国主义是很凶恶的”,就是说它的本性是不能改变的,帝国主义分子决不肯放下屠刀,他们也决不能成佛,直至他们的灭亡。\\
  斗争,失败,再斗争,再失败,再斗争,直至胜利——这就是人民的逻辑,他们也是决不会违背这个逻辑的。这是马克思主义的又一条定律。俄国人民的革命曾经是依照了这条定律,中国人民的革命也是依照这条定律。\\
  阶级斗争,一些阶级胜利了,一些阶级消灭了。这就是历史,这就是几千年的文明史。拿这个观点解释历史的就叫做历史的唯物主义,站在这个观点的反面的是历史的唯心主义。\\
  自我批评的方法只能用于人民的内部,希望劝说帝国主义者和中国反动派发出善心,回头是岸,是不可能的。唯一的办法是组织力量和他们斗争,例如我们的人民解放战争,土地革命,揭露帝国主义,“刺激”他们,把他们打倒,制裁他们的犯法行为,“只许他们规规矩矩,不许他们乱说乱动”\footnote[9]{ 见《论人民民主专政》(本卷第1475页)。}。然后,才有希望在平等和互利的条件下和外国帝国主义国家打交道。然后,才有希望把已经缴械了和投降了的地主阶级分子、官僚资产阶级分子和国民党反动集团的成员及其帮凶们给以由坏人变好人的教育,并尽可能地把他们变成好人。中国的许多自由主义分子,亦即旧民主主义分子,亦即杜鲁门、马歇尔、艾奇逊、司徒雷登们所瞩望的和经常企图争取的所谓“民主个人主义”的拥护者们之所以往往陷入被动地位,对问题的观察往往不正确——对美国统治者的观察往往不正确,对国民党的观察往往不正确,对苏联的观察往往不正确,对中国共产党的观察也往往不正确,就是因为他们没有或不赞成用历史唯物主义的观点去看问题的缘故。\\
  先进的人们,共产党人,各民主党派,觉悟了的工人,青年学生,进步的知识分子,有责任去团结人民中国内部的中间阶层、中间派、各阶层中的落后分子、一切还在动摇犹豫的人们(这些人们还要长期地动摇着,坚定了又动摇,一遇困难就要动摇的),用善意去帮助他们,批评他们的动摇性,教育他们,争取他们站到人民大众方面来,不让帝国主义把他们拉过去,叫他们丢掉幻想,准备斗争。不要以为胜利了,就不要做工作了。还要做工作,还要做很多的耐心的工作,才能真正地争取这些人。争取了他们,帝国主义就完全孤立了,艾奇逊的一套就无所施其伎了。\\
  “准备斗争”的口号,是对于在中国和帝国主义国家的关系的问题上,特别是在中国和美国的关系的问题上,还抱有某些幻想的人们说的。他们在这个问题上还是被动的,还没有下决心,还没有和美国帝国主义(以及英国帝国主义)作长期斗争的决心,因为他们对美国还有幻想。在这个问题上,他们和我们还有一个很大的或者相当大的距离。\\
  美国白皮书和艾奇逊信件的发表是值得庆祝的,因为它给了中国怀有旧民主主义思想亦即民主个人主义思想,而对人民民主主义,或民主集体主义,或民主集中主义,或集体英雄主义,或国际主义的爱国主义,不赞成,或不甚赞成,不满,或有某些不满,甚至抱有反感,但是还有爱国心,并非国民党反动派的人们,浇了一瓢冷水,丢了他们的脸。特别是对那些相信美国什么都好,希望中国学美国的人们,浇了一瓢冷水。\\
  艾奇逊公开说,要“鼓励”中国的民主个人主义者摆脱所谓“外国的羁绊”。这就是说,要推翻马克思列宁主义,推翻中国共产党领导的人民民主专政的制度。因为,据说,这个主义和这个制度是“外国的”,在中国没有根的,是德国的马克思(此人已死了六十六年),俄国的列宁(此人已死了二十五年)和斯大林(此人还活着)强加于中国人的,而且这个主义和这个制度是坏透了,提倡什么阶级斗争,打倒帝国主义等等,因此,必须推翻。这件事,经过杜鲁门总统,马歇尔幕后总司令,艾奇逊国务卿(即经手发布白皮书的一位可爱的洋大人)和司徒雷登滚蛋大使们一“鼓励”,据说中国的“民主个人主义终于会再显身手”。艾奇逊们认为这是在做“鼓励”工作,但是很可能被中国的那些虽然相信美国但是尚有爱国心的民主个人主义者认为是一瓢冷水,使他们感觉丢脸:不和中国的人民民主专政的当局好好地打交道,却要干这些混账工作,而且公开地发表出来,丢脸,丢脸!对于有爱国心的人们说来,艾奇逊的话不是一种“鼓励”,而是一种侮辱。\\
  中国是处在大革命中,全中国热气腾腾,有良好的条件去争取和团结一切对人民革命事业尚无深仇大恨,但有错误思想的人。先进的人们应当利用白皮书,向一切这样的人进行说服工作。\\
\newpage\section*{\myformat{别了,司徒雷登}\\\myformat{(一九四九年八月十八日)}}\addcontentsline{toc}{section}{别了,司徒雷登}
美国的白皮书,选择在司徒雷登\footnote[1]{ 司徒雷登(一八七六——一九六二),美国人,生于中国杭州。一九〇五年开始在中国传教,一九一九年起任美国在中国兴办的燕京大学的校长。一九四六年七月十一日,出任美国驻中国大使,积极支持国民党反动政府进行反人民内战。一九四九年四月南京解放后,司徒雷登留在南京观望。同年八月二日,由于美帝国主义阻挠中国人民革命胜利的一切努力都已彻底失败,司徒雷登不得不悄然离开中国。}业已离开南京、快到华盛顿、但是尚未到达的日子——八月五日发表,是可以理解的,因为他是美国侵略政策彻底失败的象征。司徒雷登是一个在中国出生的美国人,在中国有相当广泛的社会联系,在中国办过多年的教会学校,在抗日时期坐过日本人的监狱,平素装着爱美国也爱中国,颇能迷惑一部分中国人,因此被马歇尔看中,做了驻华大使,成为马歇尔系统中的风云人物之一。在马歇尔系统看来,他只有一个缺点,就是在他代表马歇尔系统的政策在中国当大使的整个时期,恰恰就是这个政策彻底地被中国人民打败了的时期,这个责任可不小。以脱卸责任为目的的白皮书,当然应该在司徒雷登将到未到的日子发表为适宜。\\
  美国出钱出枪,蒋介石出人,替美国打仗杀中国人,借以变中国为美国殖民地的战争,组成了美国帝国主义在第二次世界大战以后的世界侵略政策的一个重大的部分。美国侵略政策的对象有好几个部分。欧洲部分,亚洲部分,美洲部分,这三个是主要的部分。中国是亚洲的重心,是一个具有四亿七千五百万人口的大国,夺取了中国,整个亚洲都是它的了。美帝国主义的亚洲战线巩固了,它就可以集中力量向欧洲进攻。美帝国主义在美洲的战线,它是认为比较地巩固的。这些就是美国侵略者的整个如意算盘。\\
  可是,一则美国的和全世界的人民都不要战争;二则欧洲人民的觉悟,东欧各人民民主国家的兴起,特别是苏联这个空前强大的和平堡垒耸立在欧亚两洲之间,顽强地抵抗着美国的侵略政策,使美国的注意力大部分被吸引住了;三则,这是主要的,中国人民的觉悟,中国共产党领导的武装力量和民众组织力量已经空前地强大起来了。这样,就迫使美帝国主义的当权集团不能采取大规模地直接地武装进攻中国的政策,而采取了帮助蒋介石打内战的政策。\\
  美国的海陆空军已经在中国参加了战争。青岛、上海和台湾,有美国的海军基地。北平、天津、唐山、秦皇岛、青岛、上海、南京都驻过美国的军队。美国的空军控制了全中国,并从空中拍摄了全中国战略要地的军用地图。在北平附近的安平镇,在长春附近的九台,在唐山,在胶东半岛,美国的军队或军事人员曾经和人民解放军接触过,被人民解放军俘虏过多次\footnote[2]{ 一九四五年日本投降以后,以侵略中国领土主权和干涉中国内政为目的的美国军队即在中国登陆,侵驻北平、上海、南京、天津、唐山、开平、秦皇岛、静海、青岛等地区,并不断地向解放区进犯。本文中所举的安平镇事件,是一九四六年七月二十九日驻天津美军配合国民党军队进攻河北省香河县安平镇的事件。九台事件,是一九四七年三月一日美军向长春和九台间的和气堡人民解放军阵地进行军事侦察的事件。唐山事件,是指一九四六年六月十六日驻唐山美军向宋家营等地侵扰,和同年七月间,在唐山附近的滦县三河庄子、昌黎县西河南村的侵扰。美军对胶东半岛的侵犯,前后发生多次,著名的有两次,一次是一九四七年八月二十八日美国的飞机和军舰向牟平县浪暖口、小里岛侵犯;一次是同年十二月二十五日美军配合国民党军队进攻即墨县北的王疃院。对于上述美军进犯解放区的侵略行为,中国人民解放军或地方人民武装,都曾采取了严正的自卫行动。}。陈纳德航空队曾经广泛地参战\footnote[3]{ 陈纳德,美国人。抗日战争时期,曾任国民党政府空军顾问,并组织“美国志愿航空队”(又称“飞虎队”,后改为第十四航空队),支持中国抗战。日本投降后,他率领美国第十四航空队一部分人员,组织空运队,帮助国民党进行内战。}。美国的空军除替蒋介石运兵外,又炸沉了起义的重庆号巡洋舰\footnote[4]{ 见本卷《中国人民解放军总部发言人为英国军舰暴行发表的声明》注〔4〕。}。所有这些,都是直接参战的行动,只是还没有公开宣布作战,并且规模还不算大,而以大规模地出钱出枪出顾问人员帮助蒋介石打内战为主要的侵略方式。\\
  美国之所以采取这种方式,是被中国和全世界的客观形势所决定的,并不是美帝国主义的当权派——杜鲁门、马歇尔系统不想直接侵略中国。在助蒋作战的开头,又曾演过一出美国出面调处国共两党争端的文明戏,企图软化中国共产党和欺骗中国人民,不战而控制全中国。和谈失败了,欺骗不行了,战争揭幕了。\\
  对于美国怀着幻想的善忘的自由主义者或所谓“民主个人主义”者们,请你们看一看艾奇逊的话:“和平来到的时候,美国在中国碰到了三种可能的选择:(一)它可以一干二净地撤退;(二)它可以实行大规模的军事干涉,帮助国民党毁灭共产党;(三)它可以帮助国民党把他们的权力在中国最大可能的地区里面建立起来,同时却努力促成双方的妥协来避免内战。”\\
  为什么不采取第一个政策呢?艾奇逊说:“我相信当时的美国民意认为,第一种选择等于叫我们不要坚决努力地先做一番补救工作,就把我们的国际责任,把我们对华友好的传统政策,统统放弃。”原来美国的所谓“国际责任”和“对华友好的传统政策”,就是干涉中国。干涉就叫做担负国际责任,干涉就叫做对华友好,不干涉是不行的。艾奇逊在这里强奸了美国的民意,这是华尔街的“民意”,不是美国的民意。\\
  为什么不采取第二个政策呢?艾奇逊说:“第二种供选择的政策,从理论上来看,以及回顾起来,虽然都似乎是令人神往,却是完全行不通的。战前的十年里,国民党已经毁灭不了共产党。现在是战后了,国民党是削弱了,意志消沉了,失去了民心,这在前文已经有了说明。在那些从日本手里收复过来的地区里,国民党文武官员的行为一下子就断送了人民对国民党的支持,断送了它的威信。可是共产党却比以往无论什么时候都强盛,整个华北差不多都被他们控制了。从国民党军队后来所表现的不中用的惨况看来,也许只有靠美国的武力才可以把共产党打跑。对于这样庞大的责任,无论是叫我们的军队在一九四五年来承担,或者是在以后来承担,美国人民显然都不会批准。我们因此采取了第三种供选择的政策……”\\
  好办法,美国出钱出枪,蒋介石出人,替美国打仗杀中国人,“毁灭共产党”,变中国为美国的殖民地,完成美国的“国际责任”,实现“对华友好的传统政策”。\\
  国民党腐败无能,“意志消沉了,失去了民心”,还是要出钱出枪叫它打仗。直接出兵干涉,在“理论上”是妥当的。单就美国统治者来说,“回顾起来”,也是妥当的。因为这样做起来实在有兴趣,“似乎是令人神往”。但是在事实上是不行的,“美国人民显然都不会批准”。不是我们——杜鲁门、马歇尔、艾奇逊等人的帝国主义系统——不想干,干是很想的,只是因为中国的形势,美国的形势,还有整个国际的形势(这点艾奇逊没有说)不许可,不得已而求其次,采取了第三条路。\\
  那些认为“不要国际援助也可以胜利”的中国人听着,艾奇逊在给你们上课了。艾奇逊是不拿薪水上义务课的好教员,他是如此诲人不倦地毫无隐讳地说出了全篇的真理。美国之所以没有大量出兵进攻中国,不是因为美国政府不愿意,而是因为美国政府有顾虑。第一顾虑中国人民反对它,它怕陷在泥潭里拔不出去。第二顾虑美国人民反对它,因此不敢下动员令。第三顾虑苏联和欧洲的人民以及各国的人民反对它,它将冒天下之大不韪。艾奇逊的可爱的坦白性是有限度的,这第三个顾虑他不愿意说。这是因为他怕在苏联面前丢脸,他怕已经失败了但是还要装做好像没有失败的样子的欧洲马歇尔计划\footnote[5]{ 第二次世界大战结束后,西欧由于战争破坏和自然灾害的影响,政治动荡,经济衰退。为了控制西欧和扩大国外市场,美国国务卿马歇尔在一九四七年六月五日的一次演说中,建议欧洲国家共同拟订一个“复兴”计划,由美国予以“援助”。七月,英、法、意等十六国在巴黎开会,决定接受马歇尔建议,成立欧洲经济合作委员会(后改为欧洲经济合作组织),提出“欧洲复兴方案”。由于这个方案是根据马歇尔的建议制订的,故又被称为马歇尔计划。一九四八年四月杜鲁门总统签署美国《一九四八年经济合作法》(即“一九四八年对外援助法”)后,马歇尔计划正式执行。一九五一年底,美国宣布提前结束执行这个计划。}陷入全盘崩溃的惨境。\\
  那些近视的思想糊涂的自由主义或民主个人主义的中国人听着,艾奇逊在给你们上课了,艾奇逊是你们的好教员。你们所设想的美国的仁义道德,已被艾奇逊一扫而空。不是吗?你们能在白皮书和艾奇逊信件里找到一丝一毫的仁义道德吗?\\
  美国确实有科学,有技术,可惜抓在资本家手里,不抓在人民手里,其用处就是对内剥削和压迫,对外侵略和杀人。美国也有“民主政治”,可惜只是资产阶级一个阶级的独裁统治的别名。美国有很多钱,可惜只愿意送给极端腐败的蒋介石反动派。现在和将来据说很愿意送些给它在中国的第五纵队,但是不愿意送给一般的书生气十足的不识抬举的自由主义者,或民主个人主义者,当然更加不愿意送给共产党。送是可以的,要有条件。什么条件呢?就是跟我走。美国人在北平,在天津,在上海,都洒了些救济粉,看一看什么人愿意弯腰拾起来。太公钓鱼,愿者上钩。嗟来之食,吃下去肚子要痛的\footnote[6]{ “太公钓鱼,愿者上钩”,是一个民间传说。据传周朝姜太公曾在渭水河边用无饵的直钩在水面三尺上钓鱼,说:“负命者上钓来!”(见《武王伐纣平话》卷中)“嗟来之食”,是指一种带侮辱性的施舍。齐国的一个饥民因为不吃嗟来之食而饿死的故事,见《礼记·檀弓下》。}。\\
  我们中国人是有骨气的。许多曾经是自由主义者或民主个人主义者的人们,在美国帝国主义者及其走狗国民党反动派面前站起来了。闻一多拍案而起,横眉怒对国民党的手枪,宁可倒下去,不愿屈服\footnote[7]{ 闻一多(一八九九——一九四六),湖北浠水人,著名的诗人、学者和教授。一九四三年以后,由于痛恨国民党政府的反动和腐败,积极参加争取民主的斗争。抗日战争结束后,积极地反对国民党勾结美帝国主义发动反人民的内战。一九四六年七月十五日在昆明被国民党特务暗杀。}。朱自清一身重病,宁可饿死,不领美国的“救济粮”\footnote[8]{ 朱自清(一八九八——一九四八),原籍浙江绍兴,生于江苏东海,现代文学家、教授。抗日战争结束后,他积极支持反对蒋介石统治的学生运动。一九四八年六月签名于抗议美国扶植日本和拒绝领取“美援”面粉的宣言。当时他的生活非常困苦,这年八月十二日终因贫病在北平逝世。在他逝世以前,还嘱咐家人不要买国民党政府配售的平价美援面粉。}。唐朝的韩愈写过《伯夷颂》\footnote[9]{ 韩愈(七六八——八二四),中国唐代著名的大作家。《伯夷颂》是韩愈所写的一篇散文。伯夷,殷末人,周武王进军讨伐殷王朝,他曾经表示反对;武王灭殷后,他逃避到首阳山,不食周粟而死。},颂的是一个对自己国家的人民不负责任、开小差逃跑、又反对武王领导的当时的人民解放战争、颇有些“民主个人主义”思想的伯夷,那是颂错了。我们应当写闻一多颂,写朱自清颂,他们表现了我们民族的英雄气概。\\
  多少一点困难怕什么。封锁吧,封锁十年八年,中国的一切问题都解决了。中国人死都不怕,还怕困难吗?老子说过:“民不畏死,奈何以死惧之。”\footnote[10]{ 见《老子》第七十四章。}美帝国主义及其走狗蒋介石反动派,对于我们,不但“以死惧之”,而且实行叫我们死。闻一多等人之外,还在过去的三年内,用美国的卡宾枪、机关枪、迫击炮、火箭炮、榴弹炮、坦克和飞机炸弹,杀死了数百万中国人。现在这种情况已近尾声了,他们打了败仗了,不是他们杀过来而是我们杀过去了,他们快要完蛋了。留给我们多少一点困难,封锁、失业、灾荒、通货膨胀、物价上升之类,确实是困难,但是比起过去三年来已经松了一口气了。过去三年的一关也闯过了,难道不能克服现在这点困难吗?没有美国就不能活命吗?\\
  人民解放军横渡长江,南京的美国殖民政府如鸟兽散。司徒雷登大使老爷却坐着不动,睁起眼睛看着,希望开设新店,捞一把。司徒雷登看见了什么呢?除了看见人民解放军一队一队地走过,工人、农民、学生一群一群地起来之外,他还看见了一种现象,就是中国的自由主义者或民主个人主义者们也大群地和工农兵学生等人一道喊口号,讲革命。总之是没有人去理他,使得他“茕茕孑立,形影相吊”\footnote[11]{ 见李密《陈情表》。},没有什么事做了,只好挟起皮包走路。\\
  中国还有一部分知识分子和其它人等存有糊涂思想,对美国存有幻想,因此应当对他们进行说服、争取、教育和团结的工作,使他们站到人民方面来,不上帝国主义的当。但是整个美帝国主义在中国人民中的威信已经破产了,美国的白皮书,就是一部破产的记录。先进的人们,应当很好地利用白皮书对中国人民进行教育工作。\\
  司徒雷登走了,白皮书来了,很好,很好。这两件事都是值得庆祝的。\\
\newpage\section*{\myformat{“友谊”,还是侵略?}\\\myformat{(一九四九年八月三十日)}}\addcontentsline{toc}{section}{“友谊”,还是侵略?}
为了寻找侵略的根据,艾奇逊重复地说了一大堆“友谊”,加上一大堆“原则”。\\
  艾奇逊说:“从我们历史很早的时期起,美国人民和政府就关心中国了。虽然距离遥远,背景又大不相同,把中美两国隔离开了,可是那些在宗教、慈善事业和文化方面团结中美两国人民的纽带,一直在加深着美国对中国的友谊,许多年来种种善意措施便是证据,例如用庚子赔款来教育中国学生,在第二次世界大战期间废除治外法权,以及战时和战后对中国的大规模援助等等。美国始终维持并且现在依然维持对华外交政策的各项基本原则,包括门户开放主义,尊重中国行政和领土的完整,以及反对任何外国控制中国等等,这是有案可稽的。”\\
  艾奇逊当面撒谎,将侵略写成了“友谊”。\\
  美帝国主义侵略中国的历史,自从一八四〇年帮助英国人进行鸦片战争\footnote[1]{ 见本书第一卷《论反对日本帝国主义的策略》注〔35〕。}起,直到被中国人民轰出中国止,应当写一本简明扼要的教科书,教育中国的青年人。美国是最早强迫中国给予治外法权\footnote[2]{ 这里所说的“治外法权”,是指领事裁判权。见本书第二卷《中国革命和中国共产党》注〔22〕。}的国家之一,这即是白皮书上提到的中美两国有史以来第一次签订的一八四四年的望厦条约\footnote[3]{ 《望厦条约》是美国侵略中国的第一个不平等条约。一八四四年七月,美国利用中国在鸦片战争失败后的处境,强迫中国清朝政府在澳门附近的望厦村签订了《中美五口贸易章程》,即《望厦条约》。这个条约共三十四款,其中规定美国享受英国在南京条约及其附件中除割地、赔款外所获得的一切特权,包括美国在中国享有领事裁判权。}。就是在这个条约里,美国除了强迫中国接受五口通商等事而外,强迫中国接受美国人传教也是一条。美帝国主义比较其它帝国主义国家,在很长的时期内,更加注重精神侵略方面的活动,由宗教事业而推广到“慈善”事业和文化事业。据有人统计,美国教会、“慈善”机关在中国的投资,总额达四千一百九十万美元;在教会财产中,医药费占百分之十四点七,教育费占百分之三十八点二,宗教活动费占百分之四十七点一\footnote[4]{ 材料引自美国雷麦着《外人在华投资》一书第十五章。}。我国许多有名的学校如燕京、协和、汇文、圣约翰、金陵、东吴、之江、湘雅、华西、岭南等,都是美国人设立的\footnote[5]{ 指北平的燕京大学、协和医学院,北平和南京的汇文中学,上海的圣约翰大学,南京的金陵大学,苏州的东吴大学,杭州的之江大学,长沙的湘雅医学院,成都的华西协合大学和广州的岭南大学。}。司徒雷登就是从事这些事业出了名,因而做了驻华大使的。艾奇逊们心中有数,所谓“那些在宗教、慈善事业和文化方面团结中美两国人民的纽带,一直在加深着美国对中国的友谊”,是有来历的。从一八四四年订约时算起,美国在这些事业上处心积虑地经营了一百零五年,据说都是为了“加深友谊”。\\
  参加八国联军打败中国\footnote[6]{ 见本书第二卷《中国革命和中国共产党》注〔21〕。},迫出庚子赔款,又用之于“教育中国学生”,从事精神侵略,也算一项“友谊”的表示。\\
  治外法权是“废除”了,强奸沈崇案的犯人回到美国,却被美国海军部宣布无罪释放\footnote[7]{ 一九四六年十二月二十四日,美国海军陆战队士兵皮尔逊等在北平强奸北京大学女生沈崇。这个事件,曾经激起全国人民的巨大愤怒,引起了国民党统治区广大学生抗议美军暴行、要求美军撤离中国的爱国民主运动。一九四七年一月,国民党政府不顾人民的抗议,竟将主犯皮尔逊交给美国方面单独处理。同年八月,美国海军部宣布皮尔逊无罪释放。},也算一项“友谊”的表示。\\
  “战时和战后的对华援助”,据白皮书说是四十五亿余美元,据我们统计是五十九亿一千四百余万美元,帮助蒋介石杀死几百万中国人,也算一项“友谊”的表示。\\
  所有一百零九年(从一八四〇年英美合作的鸦片战争算起)美帝国主义给予中国的“友谊”,特别是最近数年帮助蒋介石杀死几百万中国人这一项伟大的“友谊”,都是为着一个目的,就是“始终维持并且现在依然维持对华外交政策的各项基本原则,包括门户开放主义,尊重中国行政和领土的完整,以及反对任何外国控制中国等等”。\\
  杀死几百万中国人,不为别的,第一为了门户开放,第二为了尊重中国行政和领土的完整,第三为了反对任何外国控制中国。\\
  现在,只有广州、台湾等处一小片地方的门户,还向艾奇逊们开放着,第一个神圣的原则在那里“依然维持”着。其余的地方,比如上海吧,解放以后本来是开放的,现在却被人用美国的军舰和军舰上所装的大炮,实行了一条很不神圣的原则:门户封锁。\\
  现在,只有广州、台湾等处一小片地方的行政和领土,还算叨了艾奇逊第二个神圣原则的光,“依然维持”住了它们的“完整”。其余地方,一概倒运,行政和领土都是破碎得不像样子了。\\
  现在,只有广州、台湾等处地方,叨了第三个神圣原则的光,把“任何外国的控制”,连同美国的控制也在内,都给艾奇逊们“反对”掉了,因此还被中国人控制着。其余的国土,说来要掉眼泪,一概完了,都给外国人控制住了,中国人统统当了奴隶。至于是什么外国呢,艾奇逊老爷行文至此,还没有来得及点出,下文自明,无须多问。\\
  不干涉中国内政,是否也算一条原则呢,艾奇逊没有说,大概不能算吧。美国老爷的逻辑,就是这样。看完艾奇逊信件的全文,就可以证实这一项高明的逻辑。\\
\newpage\section*{\myformat{为什么要讨论白皮书?}\\\myformat{(一九四九年八月二十八日)}}\addcontentsline{toc}{section}{为什么要讨论白皮书?}
关于美国白皮书和艾奇逊的信件,我们业已在三篇文章(《无可奈何的供状》\footnote[1]{ 这是新华社编辑部写的一篇评论,发表于一九四九年八月十二日。}、《丢掉幻想,准备斗争》、《别了,司徒雷登》)中给了批评。这些批评,业已引起了全国各民主党派,各人民团体,各报社,各学校以及各界民主人士的广泛的注意和讨论,他们并发表了许多正确的和有益的声明、谈话或评论。各种讨论白皮书的座谈会正在开,整个的讨论还在发展。讨论的范围涉及中美关系,中苏关系,一百年来的中外关系,中国革命和世界革命力量的相互关系,国民党反动派和中国人民的关系,各民主党派各人民团体和各界民主人士在反帝国主义斗争中应取的态度,自由主义者或所谓民主个人主义者在整个对内对外关系中应取的态度,对于帝国主义的新阴谋如何对付,等等。这种现象是很好的,是很有教育作用的。\\
  现在全世界都在讨论中国革命和美国的白皮书,这件事不是偶然的,它表示了中国革命在整个世界历史上的伟大意义。就中国人来说,我们的革命是基本上胜利了,但是很久以来还没有获得一次机会来详尽地展开讨论这个革命和内外各方面的相互关系。这种讨论是必需的,现在并已找到了机会,这就是讨论美国的白皮书。过去关于这种讨论之所以没有获得机会,是因为革命还没有得到基本上的胜利,中外反动派将大城市和人民解放区隔绝了,再则革命的发展还没有使几个矛盾侧面充分暴露的缘故。现在不同了,大半个中国已被解放,各个内外矛盾的侧面都已充分地暴露出来,恰好美国发表了白皮书,这个讨论的机会就找到了。\\
  白皮书是一部反革命的书,它公开地表示美帝国主义对于中国的干涉。就这一点来说,表现了帝国主义已经脱出了常轨。伟大的胜利的中国革命,已经迫使美帝国主义集团内部的一个方面,一个派别,要用公开发表自己反对中国人民的若干真实材料,并作出反动的结论,去答复另一个方面,另一个派别的攻击,否则他们就混不下去了。公开暴露代替了遮藏掩盖,这就是帝国主义脱出常轨的表现。在几星期以前,在白皮书发表以前,帝国主义政府的反革命事业尽管每天都在做,但是在嘴上,在官方的文书上,却总是满篇的仁义道德,或者多少带一些仁义道德,从来不说实话。老奸巨猾的英帝国主义及其它几个小帝国主义国家,至今还是如此。后起的,暴发的,神经衰弱的,一方面遭受人民反对,另方面遭受其同伙中一派反对的美国杜鲁门、马歇尔、艾奇逊、司徒雷登等人的帝国主义系统,认为以公开暴露若干(不是一切)反革命真相的方法来和他们同伙中的对手辩论究竟哪一种反革命方法较为聪明的问题,是必要的和可行的。他们企图借此说服其对手,以便继续他们自认为较为聪明的反革命方法。两派反革命竞赛,一派说我们的法子最好,另一派说我们的法子最好。争得不得开交了,一派突然摊牌,将自己用过的许多法宝搬出来,名曰白皮书。\\
  这样一来,白皮书就变成了中国人民的教育材料。多少年来,在许多问题上,主要地是在帝国主义的本性问题和社会主义的本性问题上,我们共产党人所说的,在若干(曾经有一个时期是很多)中国人看来,总是将信将疑的,“怕未必吧”。这种情况,在一九四九年八月五日以后起了一个变化。艾奇逊上课了,艾奇逊以美国国务卿的资格说话了,他所说的和我们共产党人或其它先进人们所说的,就某些材料和某些结论来说,如出一辙。这一下,可不能不信了,使成群的人打开了眼界,原来是这么一回事。\\
  艾奇逊在其致杜鲁门的信的开头,提起他编纂白皮书的故事。他说他这本白皮书编得与众不同,很客观,很坦白。“这是关于一个伟大的国家生平最复杂、最苦恼的时期的坦白记录,这个国家早就和美国有着极亲密的友谊的联系。凡是找到了的材料都没有删略,尽管那里面有些话是批评我们政策的,尽管有些材料将来会成为批评的根据。我们政府对于有见识的和批评性的舆论能够感应,这便是我们的制度的固有力量。这种有见识的和批评性的舆论,正是右派和共产党的极权政府都不能忍受,都不肯宽容的。”\\
  中美两国人民间的某些联系是存在的。经过两国人民的努力,这种联系,将来可能发展到“极亲密的友谊的”那种程度。但是,因为中美两国反动派的阻隔,这种联系,过去和现在都受到了极大的阻碍。并且因为两国反动派向两国人民撒了许多谎,拆了许多烂污,就是说做了许多的坏宣传和坏事,使得两国人民的联系极不密切。艾奇逊所说的“极亲密的友谊的联系”,不是说的两国人民,而是说的两国反动派。在这里,艾奇逊既不客观,也不坦白,他混淆了两国人民和两国反动派的相互关系。对于两国人民,中国革命的胜利和中美两国反动派的失败,是一生中空前地愉快的事,目前的这个时期,是一生中空前地愉快的时期。只有杜鲁门、马歇尔、艾奇逊、司徒雷登和其它美国反动派,蒋介石、孔祥熙、宋子文、陈立夫、李宗仁、白崇禧和其它中国反动派与此相反,确是“生平最复杂、最苦恼的时期”。\\
  艾奇逊们对于舆论的看法,混淆了反动派的舆论和人民的舆论。对于人民的舆论,艾奇逊们什么也不能“感应”,他们都是瞎子和聋子。几年来,美国、中国和全世界的人民反对美国政府的反动的对外政策,他们是充耳不闻的。什么是艾奇逊所说的“有见识的和批评性的舆论”呢?就是被美国共和、民主两个反动政党所操纵的许许多多的报纸、通讯社、刊物、广播电台等项专门说谎和专门威胁人民的宣传机关。对于这些东西,艾奇逊说对了,共产党(不,还有人民)确是“都不能忍受,都不肯宽容的”。于是乎帝国主义的新闻处被我们封闭了,帝国主义的通讯社对中国报纸的发稿被我们禁止了,不允许它们自由自在地再在中国境内毒害中国人民的灵魂。\\
  共产党领导的政府是“极权政府”的话,也有一半是说得对的。这个政府是对于内外反动派实行专政或独裁的政府,不给任何内外反动派有任何反革命的自由活动的权利。反动派生气了,骂一句“极权政府”。其实,就人民政府关于镇压反动派的权力来说,千真万确地是这样的。这个权力,现在写在我们的纲领上,将来还要写在我们的宪法上。对于胜利了的人民,这是如同布帛菽粟一样地不可以须臾离开的东西。这是一个很好的东西,是一个护身的法宝,是一个传家的法宝,直到国外的帝国主义和国内的阶级被彻底地干净地消灭之日,这个法宝是万万不可以弃置不用的。越是反动派骂“极权政府”,就越显得是一个宝贝。但是艾奇逊的话有一半是说错了。共产党领导的人民民主专政的政府,对于人民内部来说,不是专政或独裁的,而是民主的。这个政府是人民自己的政府。这个政府的工作人员对于人民必须是恭恭敬敬地听话的。同时,他们又是人民的先生,用自我教育或自我批评的方法,教育人民。\\
  至于艾奇逊所说的“右派极权政府”,自从德意日三个法西斯政府倒了以后,在这个世界上,美国政府就是第一个这样的政府。一切资产阶级的政府,包括受帝国主义庇护的德意日三国的反动派政府在内,都是这样的政府。南斯拉夫的铁托政府现在也成了这一伙的帮手\footnote[2]{ 一九四八年六月,由保、罗、匈、波、苏、法、捷、意各国共产党、工人党参加的情报局会议,作出《关于南斯拉夫共产党状况》的决议,对南共进行公开的指责,并把南共开除出情报局。当时,中国共产党支持了这个决议。一九四九年情报局又通过《南斯拉夫共产党在杀人犯和间谍掌握中》的决议。对这个决议中国共产党没有表示态度。关于对待南斯拉夫的问题,一九五六年九月,毛泽东在同南共联盟参加中国共产党第八次全国代表大会代表团谈话时曾说:我们有对不起你们的地方。过去听了情报局的意见,我们虽然没有参加情报局,但对它也很难不支持。一九四九年情报局骂你们是刽子手、希特勒分子,对那个决议我们没有表示什么。一九四八年我们写过文章批评你们。其实也不应该采取这种方式,应该和你们商量。假如你们有些观点是错了,可以向你们谈,由你们自己来批评,不必那样急。反过来,你们对我们有意见,也可以采取这种办法,采取商量、说服的办法。在报纸上批评外国的党,成功的例子很少。这次事件对国际共产主义运动来说,是取得了一个深刻的历史教训。}。美国英国这一类型的政府是资产阶级一个阶级向人民实行专政的政府。它的一切都和人民政府相反,对于资产阶级内部是有所谓民主的,对于人民则是独裁的。希特勒、墨索里尼、东条、佛朗哥、蒋介石等人的政府取消了或者索性不用那片资产阶级内部民主的幕布,是因为国内阶级斗争紧张到了极点,取消或者索性不用那片布比较地有利些,免得人民也利用那片布去手舞足蹈。美国政府现在还有一片民主布,但是已被美国反动派剪得很小了,又大大地褪了颜色,比起华盛顿、杰斐逊、林肯\footnote[3]{ 华盛顿(一七三二——一七九九)、杰斐逊(一七四三——一八二六)、林肯(一八〇九——一八六五),都是美国早期著名的资产阶级政治家。华盛顿是美国独立战争时期(一七七五——一七八三)的殖民地起义军队总司令,美国的第一任总统。杰斐逊是美国《独立宣言》的起草者,曾任美国总统。林肯主张废除美国的黑奴制度,他在担任美国总统期间,领导了反对美国南部各州奴隶主的战争(一八六一——一八六五),并在一八六三年颁布了《解放黑奴宣言》。}的朝代来是差远了,这是阶级斗争迫紧了几步的缘故。再迫紧几步,美国的民主布必然要被抛到九霄云外去。\\
  大家可以看出,艾奇逊一开口就错了这许多。这是不可避免的,因为他是反动派。至于说,他的白皮书是怎样一个“坦白记录”这一点,我们认为坦白是有的,也是没有的。艾奇逊们主观上认为有利于他们一党一派的东西,他们是有坦白的。反之,则是没有的。装作坦白,是为了作战的目的。\\
\newpage\section*{\myformat{唯心历史观的破产}\\\myformat{(一九四九年九月十六日)}}\addcontentsline{toc}{section}{唯心历史观的破产}
中国人之所以应当感谢美国资产阶级发言人艾奇逊,不但是因为艾奇逊明确地供认了美国出钱出枪,蒋介石出人,替美国打仗杀中国人这样一种事实,使得中国的先进分子有证据地去说服落后分子。不是吗?你们看,艾奇逊自己招认了,最近数年的这一场使得几百万中国人丧失生命的大血战,是美国帝国主义有计划地组织成功的。中国人之所以应当感谢艾奇逊,又不但因为艾奇逊公开地宣称,他们要招收中国的所谓“民主个人主义”分子,组织美国的第五纵队,推翻中国共产党领导的人民政府,因此引起了中国人特别是那些带有自由主义色彩的中国人的注意,大家相约不要上美国人的当,到处警戒美帝国主义在暗地里进行的阴谋活动。中国人之所以应当感谢艾奇逊,还因为艾奇逊胡诌了一大篇中国近代史,而艾奇逊的历史观点正是中国知识分子中有一部分人所同具的观点,就是说资产阶级的唯心的历史观。驳斥了艾奇逊,就有可能使得广大的中国人获得打开眼界的益处。对于那些抱着和艾奇逊相同或者有某些相同的观点的人们,则可能是更加有益的。\\
  艾奇逊胡诌的中国近代史是什么呢?他首先试图从中国的经济状况和思想状况去说明中国革命的发生。在这里,他讲了很多的神话。\\
  艾奇逊说:“中国人口在十八、十九两个世纪里增加了一倍,因此使土地受到不堪负担的压力。人民的吃饭问题是每个中国政府必然碰到的第一个问题。一直到现在没有一个政府使这个问题得到了解决。国民党在法典里写上了许多土地改革法令,想这样来解决这个问题。这些法令有的失败了,有的被忽视。国民政府之所以有今天的窘况,很大的一个原因是它没有使中国有足够的东西吃。中共宣传的内容,一大部分是他们决心解决土地问题的诺言。”\\
  在不明事理的中国人看来,很有点像。人口太多了,饭少了,发生革命。国民党没有解决这个问题,共产党也不见得能解决这个问题,“一直到现在没有一个政府使这个问题得到了解决”。\\
  革命的发生是由于人口太多的缘故吗?古今中外有过很多的革命,都是由于人口太多吗?中国几千年以来的很多次的革命,也是由于人口太多吗?美国一百七十四年以前的反英革命\footnote[1]{ 指美国的独立战争。一七七五年起北美十三个殖民地的人民先后进行推翻英国殖民统治、争取独立的资产阶级革命战争。一七七六年七月四日发表了《独立宣言》,正式宣告脱离英国。英国战败后,被迫在一七八三年签订《巴黎和约》,承认美国独立。},也是由于人口太多吗?艾奇逊的历史知识等于零,他连美国独立宣言也没有读过。华盛顿杰斐逊\footnote[2]{ 参见本卷《为什么要讨论白皮书》注〔3〕。}们之所以举行反英革命,是因为英国人压迫和剥削美国人,而不是什么美国人口过剩。中国人民历次推翻自己的封建朝廷,是因为这些封建朝廷压迫和剥削人民,而不是什么人口过剩。俄国人所以举行二月革命和十月革命,是因为俄皇和俄国资产阶级的压迫和剥削,而不是什么人口过剩,俄国至今还是土地多过人口很远的。蒙古土地那么广大,人口那么稀少,照艾奇逊的道理是不能设想会发生革命的,但是却早已发生了\footnote[3]{ 指一九二一年至一九二四年外蒙古发生的人民革命。在这次革命中,蒙古人民革命党领导蒙古人民,在苏俄的支持和帮助下,驱逐日本帝国主义所支持的白俄军队,推翻外蒙古地方的封建统治,脱离当时中国的北洋政府,建立了蒙古人民共和国。}。\\
  按照艾奇逊的说法,中国是毫无出路的,人口有了四亿七千五百万,是一种“不堪负担的压力”,革命也好,不革命也好,总之是不得了。艾奇逊在这里寄予了很大的希望,这个希望他没有说出来,却被许多美国新闻记者经常地透露了出来,这就是所谓中国共产党解决不了自己的经济问题,中国将永远是天下大乱,只有靠美国的面粉,即是说变为美国的殖民地,才有出路。\\
  辛亥革命\footnote[4]{ 见本书第一卷《湖南农民运动考察报告》注〔3〕。}为什么没有成功,没有解决吃饭问题呢?是因为辛亥革命只推翻一个清朝政府,而没有推翻帝国主义和封建主义的压迫和剥削。\\
  北伐战争为什么没有成功,没有解决吃饭问题呢?是因为蒋介石背叛革命,投降帝国主义,成了压迫和剥削中国人的反革命首领。\\
  “一直到现在没有一个政府使这个问题得到了解决”吗?西北、华北、东北、华东各个解决了土地问题的老解放区,难道还有如同艾奇逊所说的那种“吃饭问题”存在吗?美国在中国的侦探或所谓观察家是不少的,为什么连这件事也没有探出来呢?上海等处的失业问题即吃饭问题,完全是帝国主义、封建主义、官僚资本主义和国民党反动政府的残酷无情的压迫和剥削的结果。在人民政府下,只消几年工夫,就可以和华北、东北等处一样完全地解决失业即吃饭的问题。\\
  中国人口众多是一件极大的好事。再增加多少倍人口也完全有办法,这办法就是生产。西方资产阶级经济学家如像马尔萨斯\footnote[5]{ 托马斯·罗伯特·马尔萨斯(一七六六——一八三四),英国国教会僧侣,经济学家。在一七九八年出版的《人口原理》一书中,他认为在没有任何限制的条件下,人类社会人口以几何数列增长,而生活资料以算术数列增长,因此人口必定发生过剩现象,只有贫困和罪恶(包括战争和瘟疫等)、“道德的抑制”(包括禁欲、无力赡养子女者不得结婚等)可使生活资料和人口之间恢复平衡。这一理论曾被资产阶级用来为他们对劳动人民的压迫和剥削及其侵略和战争政策辩护。}者流所谓食物增加赶不上人口增加的一套谬论,不但被马克思主义者早已从理论上驳斥得干干净净,而且已被革命后的苏联和中国解放区的事实所完全驳倒。根据革命加生产即能解决吃饭问题的真理,中共中央已命令全国各地的共产党组织和人民解放军,对于国民党的旧工作人员,只要有一技之长而不是反动有据或劣迹昭著的分子,一概予以维持,不要裁减。十分困难时,饭匀着吃,房子挤着住。已被裁减而生活无着者,收回成命,给以饭吃。国民党军起义的或被俘的,按此原则,一律收留。凡非首要的反动分子,只要悔罪,亦须给以生活出路。\\
  世间一切事物中,人是第一个可宝贵的。在共产党领导下,只要有了人,什么人间奇迹也可以造出来。我们是艾奇逊反革命理论的驳斥者,我们相信革命能改变一切,一个人口众多、物产丰盛、生活优裕、文化昌盛的新中国,不要很久就可以到来,一切悲观论调是完全没有根据的。\\
  “西方的影响”,这是艾奇逊解释中国革命所以发生的第二个原因。艾奇逊说:“中国自己的高度文化和文明,有了三千多年的发展,大体上不曾沾染外来的影响。中国人即是被武力征服,最后总是能够驯服和融化侵入者。他们自然会因此把自己当作世界的中心,把自己看成是文明人类的最高表现。到了十九世纪中叶,西方突破了中国孤立的墙壁,那在以前是不可逾越的。这些外来者带来了进取性,带来了发展得盖世无双的西方技术,带来了为以往的侵入者所从来不曾带入中国的高度文化。一部分由于这些品质,一部分由于清朝统治的衰落,西方人不但没有被中国融化,而且介绍了许多新思想进来,这些新思想发生了重要作用,激起了骚动和不安。”\\
  在不明事理的中国人看来,艾奇逊说得很有点像。西方的新观念输入了中国,引起了革命。\\
  革什么人的命呢?因为“清朝统治的衰落”,向弱点进攻,是革清朝的命了。艾奇逊在这里说得不恰当。辛亥革命是革帝国主义的命。中国人所以要革清朝的命,是因为清朝是帝国主义的走狗。反对英国鸦片侵略的战争\footnote[6]{ 见本书第一卷《论反对日本帝国主义的策略》注〔35〕。},反对英法联军侵略的战争\footnote[7]{ 见本书第二卷《中国革命和中国共产党》注〔18〕。},反对帝国主义走狗清朝的太平天国战争\footnote[8]{ 见本书第一卷《论反对日本帝国主义的策略》注〔36〕。},反对法国侵略的战争\footnote[9]{ 见本书第二卷《中国革命和中国共产党》注〔19〕。},反对日本侵略的战争\footnote[10]{ 见本书第一卷《矛盾论》注〔22〕。},反对八国联军侵略的战争\footnote[11]{ 见本书第二卷《中国革命和中国共产党》注〔21〕。},都失败了,于是再有反对帝国主义走狗清朝的辛亥革命,这就是到辛亥为止的近代中国史。艾奇逊所说的“西方的影响”是什么呢?就是马克思恩格斯在《共产党宣言》(一八四八年)中所说的西方资产阶级按照自己的面貌用恐怖的方法去改造世界\footnote[12]{ 《共产党宣言》第一章《资产者和无产者》中说:资产阶级“迫使一切民族——如果它们不想灭亡的话——采用资产阶级的生产方式;它迫使它们在自己那里推行所谓文明制度,即变成资产者。一句话,它按照自己的面貌为自己创造出一个世界。”(《马克思恩格斯选集》第1卷,人民出版社1972年版,第255页)}。在这个影响或改造过程中,西方资产阶级需要买办和熟习西方习惯的奴才,不得不允许中国这一类国家开办学校和派遣留学生,给中国“介绍了许多新思想进来”。随着也就产生了中国这类国家的民族资产阶级和无产阶级。同时并使农民破产,造成了广大的半无产阶级。这样,西方资产阶级就在东方造成了两类人,一类是少数人,这就是为帝国主义服务的洋奴;一类是多数人,这就是反抗帝国主义的工人阶级、农民阶级、城市小资产阶级、民族资产阶级和从这些阶级出身的知识分子,所有这些,都是帝国主义替自己造成的掘墓人,革命就是从这些人发生的。不是什么西方思想的输入引起了“骚动和不安”,而是帝国主义的侵略引起了反抗。\\
  在这个反抗运动中,在一个很长的时期内,即从一八四〇年的鸦片战争到一九一九年的五四运动\footnote[13]{ 见本书第一卷《实践论》注〔6〕。}的前夜,共计七十多年中,中国人没有什么思想武器可以抗御帝国主义。旧的顽固的封建主义的思想武器打了败仗了,抵不住,宣告破产了。不得已,中国人被迫从帝国主义的老家即西方资产阶级革命时代的武器库中学来了进化论、天赋人权论和资产阶级共和国等项思想武器和政治方案,组织过政党,举行过革命,以为可以外御列强,内建民国。但是这些东西也和封建主义的思想武器一样,软弱得很,又是抵不住,败下阵来,宣告破产了。\\
  一九一七年的俄国革命唤醒了中国人,中国人学得了一样新的东西,这就是马克思列宁主义。中国产生了共产党,这是开天辟地的大事变。孙中山也提倡“以俄为师”,主张“联俄联共”。总之是从此以后,中国改换了方向。\\
  艾奇逊是帝国主义政府的发言人,他当然一个字也不愿意提到帝国主义。他将帝国主义的侵略,说成“外来者带来了进取性”。看啊,多么美丽的名称——“进取性”。中国人学了这种“进取性”,不是进取到英国或美国去,只是在中国境内引起了“骚动和不安”,即是革帝国主义及其走狗的命。可惜没有一次成功,都给“进取性”的发明人即帝国主义者打败了。于是掉转头去学别的东西,很奇怪,果然一学就灵。\\
  “中国共产党是在二十年代初期,在俄罗斯革命的思想推动之下建立起来的”。艾奇逊说对了。这种思想不是别的,就是马克思列宁主义。这种思想,和艾奇逊所说的西方资产阶级的“为以往的侵入者所从来不曾带入中国的高度文化”相比较,不知要高出几多倍。其明效大验,就是和中国旧的封建主义文化相比较可以被艾奇逊们傲视为“高度文化”的那种西方资产阶级的文化,一遇见中国人民学会了的马克思列宁主义的新文化,即科学的宇宙观和社会革命论,就要打败仗。被中国人民学会了的科学的革命的新文化,第一仗打败了帝国主义的走狗北洋军阀,第二仗打败了帝国主义的又一名走狗蒋介石在二万五千里长征路上对于中国红军的拦阻,第三仗打败了日本帝国主义及其走狗汪精卫,第四仗最后地结束了美国和一切帝国主义在中国的统治及其走狗蒋介石等一切反动派的统治。\\
  马克思列宁主义来到中国之所以发生这样大的作用,是因为中国的社会条件有了这种需要,是因为同中国人民革命的实践发生了联系,是因为被中国人民所掌握了。任何思想,如果不和客观的实际的事物相联系,如果没有客观存在的需要,如果不为人民群众所掌握,即使是最好的东西,即使是马克思列宁主义,也是不起作用的。我们是反对历史唯心论的历史唯物论者。\\
  非常奇怪,“苏维埃的学说和实践,对于孙中山先生的思想和原则,尤其是在经济方面和党的组织方面,有相当的影响”。被艾奇逊们所傲视的西方的“高度文化”,对于孙先生的影响怎么样呢?艾奇逊没有说。孙先生以大半辈子的光阴从西方资产阶级文化中寻找救国真理,结果是失望,转而“以俄为师”,这是一个偶然的事件吗?显然不是。孙先生和他所代表的苦难的中国人民,一齐被“西方的影响”所激怒,下决心“联俄联共”,和帝国主义及其走狗奋斗和拚命,当然不是偶然的。在这里,艾奇逊不敢说苏联人是帝国主义侵略者,孙中山是向侵略者学习。那末,好了,孙中山可以向苏联人学习,而苏联人并非帝国主义侵略者,为什么孙中山的继承者,孙中山死后的中国人,就不可以向苏联人学习呢?为什么孙中山以外的中国人从马克思列宁主义学了科学的宇宙观和社会革命理论,并使之和中国的特点相结合,发动了中国的人民解放战争和人民大革命,创立了人民民主专政的共和国,就叫做“受苏联控制”,“共产国际的第五纵队”,“赤色帝国主义的走狗”呢?世上有这样高明的逻辑吗?\\
  自从中国人学会了马克思列宁主义以后,中国人在精神上就由被动转入主动。从这时起,近代世界历史上那种看不起中国人,看不起中国文化的时代应当完结了。伟大的胜利的中国人民解放战争和人民大革命,已经复兴了并正在复兴着伟大的中国人民的文化。这种中国人民的文化,就其精神方面来说,已经超过了整个资本主义的世界。比方美国的国务卿艾奇逊之流,他们对于现代中国和现代世界的认识水平,就在中国人民解放军的一个普通战士的水平之下。\\
  至此为止,艾奇逊以一个资产阶级大学教授讲述无聊课本的姿态,向人们表示他在寻求中国事变的因果关系。中国之所以发生革命,一因人口太多,二因西方思想的刺激。你们看,他好像是一个因果论者。接下去,他就连这点无聊的伪造的因果论也不见了,出现了一大堆莫名其妙的事变。中国人就是那样毫无原因地互相争权夺利和猜疑仇恨。斗争中的国民党和共产党,双方的精神力量的对比,发生了莫名其妙的变化,一方极度下降,降到零度以下,另一方极度上升,升到狂热的程度。什么原因呢?谁也不知道——这就是艾奇逊所代表的美国的“高度文化”中所固有的逻辑。\\
\newpage

















\end{document}
